\documentclass[a4paper,12pt]{article}
\usepackage[T2A]{fontenc}
\usepackage[utf8]{inputenc}
\usepackage[russian]{babel}
\usepackage{amsmath}
\usepackage{amssymb}
\usepackage{autobreak}
\usepackage{hyperref}
\setcounter{secnumdepth}{0}
\begin{document}
\title{Сокращение рода боблина}
\maketitle
\newpage
\begin{titlepage}
    \centering

    {\Large Уставший волшебник}\\[1cm]

    {\huge\bfseries «Завершение рода Боблина»}\\[0.5cm]

    \raggedright

    \textbf{Предыстория} \\[0.3cm]

    Жил-был самый обычный гоблин по имени Боблин и его очень большая семья. \\ 
    Как-то раз, одним жарким летом они все вместе решили отправиться на пикник.\\ 
    Они нашли великолепную полянку посреди болота: солнышко, зеленая трава, тенеко от непонятно башни, одним словом - благодать. \\ 
    Шел 5-ый час гоблинской пъянки, тут уже нервы волшебника живущего в башне не выдержали. \\ 
    и он решил обрушиить свой праведный гнев не семейство Боблина, истребив некоторую его часть. \\ 
    \vspace{0.8cm}
    \textbf{Боевой журнал}\\[0.3cm]

    В башне стоял особенный артефакт, который записывал ход сражения в виде странного набора символов.\\ 
    Которые лишь сам маг был способен понять, здесь и будет приведет этот боевой журнал. \\     \vfill

    \raggedleft
    \textit{«Если на странице стало больше знаков — значит, кто-то из клана Боблина опять что-то натворил.»}\\[0.3cm]

\end{titlepage}
\tableofcontents
\newpage
f(x) = \begin{align*}
\begin{autobreak}
x^{x}
\end{autobreak}
\end{align*}

\section{Прибывает Тейлор и куча дальних родственников}
Текущий ход событий: \begin{align*}
\begin{autobreak}
x^{x}
\end{autobreak}
\end{align*}

\subsection{Прибывает 1-ая волна родственников Тейлора-Боблина}
Текущий ход событий: \begin{align*}
\begin{autobreak}
x^{x}
\end{autobreak}
\end{align*}

\noindent\hrulefill\begin{center}
Родственники Боблина продолжают лезть к вам, держите посох крепче!
\end{center}
\begin{align*}
\begin{autobreak}
\frac{d}{dx}(x^{x}) = x^x \cdot (\frac{d}{dx}(x) \cdot \ln(x) + x \cdot \frac{\frac{d}{dx}(x)}{x})
\end{autobreak}
\end{align*}

\begin{center}
Ваше вошшебство откатило деверя Боблина до младенчества
\end{center}
\begin{align*}
\begin{autobreak}
\frac{d}{dx}(x) = 1
\end{autobreak}
\end{align*}

\begin{center}
Полиморф сработал отлично: зять Боблина теперь лягушка
\end{center}
\begin{align*}
\begin{autobreak}
\frac{d}{dx}(x) = 1
\end{autobreak}
\end{align*}

\subsection{Прибывает 2-ая волна родственников Тейлора-Боблина}
Текущий ход событий: \begin{align*}
\begin{autobreak}
x^{x} \cdot (\ln(x) + x \cdot \frac{1}{x})
\end{autobreak}
\end{align*}

\noindent\hrulefill\begin{center}
Битва продолжается, не теряйте духу, они когда-то, наверное, закончаться!
\end{center}
\begin{align*}
\begin{autobreak}
\frac{d}{dx}(x^{x} \cdot (\ln(x) + x \cdot \frac{1}{x})) = \frac{d}{dx}(x^{x}) \cdot \ln(x) + x \cdot \frac{1}{x} + x^{x} \cdot \frac{d}{dx}(\ln(x) + x \cdot \frac{1}{x})
\end{autobreak}
\end{align*}

\begin{center}
Их не становиться меньше, откуда они только лезут?!
\end{center}
\begin{align*}
\begin{autobreak}
\frac{d}{dx}(\ln(x) + x \cdot \frac{1}{x}) = \frac{d}{dx}(\ln(x)) + \frac{d}{dx}(x \cdot \frac{1}{x})
\end{autobreak}
\end{align*}

\begin{center}
Вот так и рождаются легенды о герое, истребившем половину рода Боблина.
\end{center}
\begin{align*}
\begin{autobreak}
\frac{d}{dx}(x \cdot \frac{1}{x}) = \frac{d}{dx}(x) \cdot \frac{1}{x} + x \cdot \frac{d}{dx}(\frac{1}{x})
\end{autobreak}
\end{align*}

\begin{center}
Небольшой взмах посохом — и план сражения выглядит куда приличнее.
\end{center}
\begin{align*}
\begin{autobreak}
\frac{d}{dx}(\frac{1}{x}) = \frac{\frac{d}{dx}(1) \cdot x - 1 \cdot \frac{d}{dx}(x)}{x^2}
\end{autobreak}
\end{align*}

\begin{center}
Поздравляю! От свояка Боблина осталась только полоыина
\end{center}
\begin{align*}
\begin{autobreak}
\frac{d}{dx}(x) = 1
\end{autobreak}
\end{align*}

\begin{center}
Вы разложили дядю Боблина на молекулы
\end{center}
\begin{align*}
\begin{autobreak}
\frac{d}{dx}(1) = 0
\end{autobreak}
\end{align*}

\begin{center}
Ваше заклинание свернуло невестку Боблина в шарик
\end{center}
\begin{align*}
\begin{autobreak}
\frac{d}{dx}(x) = 1
\end{autobreak}
\end{align*}

\begin{center}
Один из гоблинов упал
\end{center}
\begin{align*}
\begin{autobreak}
\frac{d}{dx}(\ln(x)) = \frac{\frac{d}{dx}(x)}{x}
\end{autobreak}
\end{align*}

\begin{center}
ААХХАХААХАХ Гоблин-Боблин
\end{center}
\begin{align*}
\begin{autobreak}
\frac{d}{dx}(x) = 1
\end{autobreak}
\end{align*}

\begin{center}
Родственники Боблина продолжают лезть к вам, держите посох крепче!
\end{center}
\begin{align*}
\begin{autobreak}
\frac{d}{dx}(x^{x}) = x^x \cdot (\frac{d}{dx}(x) \cdot \ln(x) + x \cdot \frac{\frac{d}{dx}(x)}{x})
\end{autobreak}
\end{align*}

\begin{center}
Ваше вошшебство откатило деверя Боблина до младенчества
\end{center}
\begin{align*}
\begin{autobreak}
\frac{d}{dx}(x) = 1
\end{autobreak}
\end{align*}

\begin{center}
Полиморф сработал отлично: зять Боблина теперь лягушка
\end{center}
\begin{align*}
\begin{autobreak}
\frac{d}{dx}(x) = 1
\end{autobreak}
\end{align*}

\subsection{Прибывает 3-ая волна родственников Тейлора-Боблина}
Текущий ход событий: \begin{align*}
\begin{autobreak}
x^{x} \cdot (\ln(x) + x \cdot \frac{1}{x}) \cdot (\ln(x) + x \cdot \frac{1}{x}) + x^{x} \cdot (\frac{1}{x} + \frac{1}{x} + x \cdot \frac{-1}{x \cdot x})
\end{autobreak}
\end{align*}

\noindent\hrulefill\begin{center}
Битва продолжается, не теряйте духу, они когда-то, наверное, закончаться!
\end{center}
\begin{align*}
\begin{autobreak}
\frac{d}{dx}(x^{x} \cdot (\ln(x) + x \cdot \frac{1}{x}) \cdot (\ln(x) + x \cdot \frac{1}{x}) + x^{x} \cdot (\frac{1}{x} + \frac{1}{x} + x \cdot \frac{-1}{x \cdot x})) = \frac{d}{dx}(x^{x} \cdot (\ln(x) + x \cdot \frac{1}{x}) \cdot (\ln(x) + x \cdot \frac{1}{x})) + \frac{d}{dx}(x^{x} \cdot (\frac{1}{x} + \frac{1}{x} + x \cdot \frac{-1}{x \cdot x}))
\end{autobreak}
\end{align*}

\begin{center}
Их не становиться меньше, откуда они только лезут?!
\end{center}
\begin{align*}
\begin{autobreak}
\frac{d}{dx}(x^{x} \cdot (\frac{1}{x} + \frac{1}{x} + x \cdot \frac{-1}{x \cdot x})) = \frac{d}{dx}(x^{x}) \cdot \frac{1}{x} + \frac{1}{x} + x \cdot \frac{-1}{x \cdot x} + x^{x} \cdot \frac{d}{dx}(\frac{1}{x} + \frac{1}{x} + x \cdot \frac{-1}{x \cdot x})
\end{autobreak}
\end{align*}

\begin{center}
Вот так и рождаются легенды о герое, истребившем половину рода Боблина.
\end{center}
\begin{align*}
\begin{autobreak}
\frac{d}{dx}(\frac{1}{x} + \frac{1}{x} + x \cdot \frac{-1}{x \cdot x}) = \frac{d}{dx}(\frac{1}{x}) + \frac{d}{dx}(\frac{1}{x} + x \cdot \frac{-1}{x \cdot x})
\end{autobreak}
\end{align*}

\begin{center}
Небольшой взмах посохом — и план сражения выглядит куда приличнее.
\end{center}
\begin{align*}
\begin{autobreak}
\frac{d}{dx}(\frac{1}{x} + x \cdot \frac{-1}{x \cdot x}) = \frac{d}{dx}(\frac{1}{x}) + \frac{d}{dx}(x \cdot \frac{-1}{x \cdot x})
\end{autobreak}
\end{align*}

\begin{center}
Один из гоблинов упал
\end{center}
\begin{align*}
\begin{autobreak}
\frac{d}{dx}(x \cdot \frac{-1}{x \cdot x}) = \frac{d}{dx}(x) \cdot \frac{-1}{x \cdot x} + x \cdot \frac{d}{dx}(\frac{-1}{x \cdot x})
\end{autobreak}
\end{align*}

\begin{center}
Родственники Боблина продолжают лезть к вам, держите посох крепче!
\end{center}
\begin{align*}
\begin{autobreak}
\frac{d}{dx}(\frac{-1}{x \cdot x}) = \frac{\frac{d}{dx}(-1) \cdot x \cdot x - -1 \cdot \frac{d}{dx}(x \cdot x)}{x \cdot x^2}
\end{autobreak}
\end{align*}

\begin{center}
Битва продолжается, не теряйте духу, они когда-то, наверное, закончаться!
\end{center}
\begin{align*}
\begin{autobreak}
\frac{d}{dx}(x \cdot x) = \frac{d}{dx}(x) \cdot x + x \cdot \frac{d}{dx}(x)
\end{autobreak}
\end{align*}

\begin{center}
Поздравляю! От свояка Боблина осталась только полоыина
\end{center}
\begin{align*}
\begin{autobreak}
\frac{d}{dx}(x) = 1
\end{autobreak}
\end{align*}

\begin{center}
Ваше заклинание свернуло невестку Боблина в шарик
\end{center}
\begin{align*}
\begin{autobreak}
\frac{d}{dx}(x) = 1
\end{autobreak}
\end{align*}

\begin{center}
Ваше волшебство оказалось не по зубам тёте Боблина, кстати, куда она делась?
\end{center}
\begin{align*}
\begin{autobreak}
\frac{d}{dx}(-1) = 0
\end{autobreak}
\end{align*}

\begin{center}
ААХХАХААХАХ Гоблин-Боблин
\end{center}
\begin{align*}
\begin{autobreak}
\frac{d}{dx}(x) = 1
\end{autobreak}
\end{align*}

\begin{center}
Их не становиться меньше, откуда они только лезут?!
\end{center}
\begin{align*}
\begin{autobreak}
\frac{d}{dx}(\frac{1}{x}) = \frac{\frac{d}{dx}(1) \cdot x - 1 \cdot \frac{d}{dx}(x)}{x^2}
\end{autobreak}
\end{align*}

\begin{center}
Ваше вошшебство откатило деверя Боблина до младенчества
\end{center}
\begin{align*}
\begin{autobreak}
\frac{d}{dx}(x) = 1
\end{autobreak}
\end{align*}

\begin{center}
Огненный шар испарил бабушку Боблина
\end{center}
\begin{align*}
\begin{autobreak}
\frac{d}{dx}(1) = 0
\end{autobreak}
\end{align*}

\begin{center}
Вот так и рождаются легенды о герое, истребившем половину рода Боблина.
\end{center}
\begin{align*}
\begin{autobreak}
\frac{d}{dx}(\frac{1}{x}) = \frac{\frac{d}{dx}(1) \cdot x - 1 \cdot \frac{d}{dx}(x)}{x^2}
\end{autobreak}
\end{align*}

\begin{center}
Полиморф сработал отлично: зять Боблина теперь лягушка
\end{center}
\begin{align*}
\begin{autobreak}
\frac{d}{dx}(x) = 1
\end{autobreak}
\end{align*}

\begin{center}
Заклинание хаоса раскидало части племянника Боблина по разным планам
\end{center}
\begin{align*}
\begin{autobreak}
\frac{d}{dx}(1) = 0
\end{autobreak}
\end{align*}

\begin{center}
Небольшой взмах посохом — и план сражения выглядит куда приличнее.
\end{center}
\begin{align*}
\begin{autobreak}
\frac{d}{dx}(x^{x}) = x^x \cdot (\frac{d}{dx}(x) \cdot \ln(x) + x \cdot \frac{\frac{d}{dx}(x)}{x})
\end{autobreak}
\end{align*}

\begin{center}
Поздравляю! От свояка Боблина осталась только полоыина
\end{center}
\begin{align*}
\begin{autobreak}
\frac{d}{dx}(x) = 1
\end{autobreak}
\end{align*}

\begin{center}
Ваше заклинание свернуло невестку Боблина в шарик
\end{center}
\begin{align*}
\begin{autobreak}
\frac{d}{dx}(x) = 1
\end{autobreak}
\end{align*}

\begin{center}
Один из гоблинов упал
\end{center}
\begin{align*}
\begin{autobreak}
\frac{d}{dx}(x^{x} \cdot (\ln(x) + x \cdot \frac{1}{x}) \cdot (\ln(x) + x \cdot \frac{1}{x})) = \frac{d}{dx}(x^{x} \cdot (\ln(x) + x \cdot \frac{1}{x})) \cdot \ln(x) + x \cdot \frac{1}{x} + x^{x} \cdot (\ln(x) + x \cdot \frac{1}{x}) \cdot \frac{d}{dx}(\ln(x) + x \cdot \frac{1}{x})
\end{autobreak}
\end{align*}

\begin{center}
Родственники Боблина продолжают лезть к вам, держите посох крепче!
\end{center}
\begin{align*}
\begin{autobreak}
\frac{d}{dx}(\ln(x) + x \cdot \frac{1}{x}) = \frac{d}{dx}(\ln(x)) + \frac{d}{dx}(x \cdot \frac{1}{x})
\end{autobreak}
\end{align*}

\begin{center}
Битва продолжается, не теряйте духу, они когда-то, наверное, закончаться!
\end{center}
\begin{align*}
\begin{autobreak}
\frac{d}{dx}(x \cdot \frac{1}{x}) = \frac{d}{dx}(x) \cdot \frac{1}{x} + x \cdot \frac{d}{dx}(\frac{1}{x})
\end{autobreak}
\end{align*}

\begin{center}
Их не становиться меньше, откуда они только лезут?!
\end{center}
\begin{align*}
\begin{autobreak}
\frac{d}{dx}(\frac{1}{x}) = \frac{\frac{d}{dx}(1) \cdot x - 1 \cdot \frac{d}{dx}(x)}{x^2}
\end{autobreak}
\end{align*}

\begin{center}
ААХХАХААХАХ Гоблин-Боблин
\end{center}
\begin{align*}
\begin{autobreak}
\frac{d}{dx}(x) = 1
\end{autobreak}
\end{align*}

\begin{center}
Ваш портал небытия вежливо удалил тещу Боблина из этого измерения
\end{center}
\begin{align*}
\begin{autobreak}
\frac{d}{dx}(1) = 0
\end{autobreak}
\end{align*}

\begin{center}
Ваше вошшебство откатило деверя Боблина до младенчества
\end{center}
\begin{align*}
\begin{autobreak}
\frac{d}{dx}(x) = 1
\end{autobreak}
\end{align*}

\begin{center}
Вот так и рождаются легенды о герое, истребившем половину рода Боблина.
\end{center}
\begin{align*}
\begin{autobreak}
\frac{d}{dx}(\ln(x)) = \frac{\frac{d}{dx}(x)}{x}
\end{autobreak}
\end{align*}

\begin{center}
Полиморф сработал отлично: зять Боблина теперь лягушка
\end{center}
\begin{align*}
\begin{autobreak}
\frac{d}{dx}(x) = 1
\end{autobreak}
\end{align*}

\begin{center}
Небольшой взмах посохом — и план сражения выглядит куда приличнее.
\end{center}
\begin{align*}
\begin{autobreak}
\frac{d}{dx}(x^{x} \cdot (\ln(x) + x \cdot \frac{1}{x})) = \frac{d}{dx}(x^{x}) \cdot \ln(x) + x \cdot \frac{1}{x} + x^{x} \cdot \frac{d}{dx}(\ln(x) + x \cdot \frac{1}{x})
\end{autobreak}
\end{align*}

\begin{center}
Один из гоблинов упал
\end{center}
\begin{align*}
\begin{autobreak}
\frac{d}{dx}(\ln(x) + x \cdot \frac{1}{x}) = \frac{d}{dx}(\ln(x)) + \frac{d}{dx}(x \cdot \frac{1}{x})
\end{autobreak}
\end{align*}

\begin{center}
Родственники Боблина продолжают лезть к вам, держите посох крепче!
\end{center}
\begin{align*}
\begin{autobreak}
\frac{d}{dx}(x \cdot \frac{1}{x}) = \frac{d}{dx}(x) \cdot \frac{1}{x} + x \cdot \frac{d}{dx}(\frac{1}{x})
\end{autobreak}
\end{align*}

\begin{center}
Битва продолжается, не теряйте духу, они когда-то, наверное, закончаться!
\end{center}
\begin{align*}
\begin{autobreak}
\frac{d}{dx}(\frac{1}{x}) = \frac{\frac{d}{dx}(1) \cdot x - 1 \cdot \frac{d}{dx}(x)}{x^2}
\end{autobreak}
\end{align*}

\begin{center}
Поздравляю! От свояка Боблина осталась только полоыина
\end{center}
\begin{align*}
\begin{autobreak}
\frac{d}{dx}(x) = 1
\end{autobreak}
\end{align*}

\begin{center}
Ваше заклинание дезинтегрировало брата Боблина
\end{center}
\begin{align*}
\begin{autobreak}
\frac{d}{dx}(1) = 0
\end{autobreak}
\end{align*}

\begin{center}
Ваше заклинание свернуло невестку Боблина в шарик
\end{center}
\begin{align*}
\begin{autobreak}
\frac{d}{dx}(x) = 1
\end{autobreak}
\end{align*}

\begin{center}
Их не становиться меньше, откуда они только лезут?!
\end{center}
\begin{align*}
\begin{autobreak}
\frac{d}{dx}(\ln(x)) = \frac{\frac{d}{dx}(x)}{x}
\end{autobreak}
\end{align*}

\begin{center}
ААХХАХААХАХ Гоблин-Боблин
\end{center}
\begin{align*}
\begin{autobreak}
\frac{d}{dx}(x) = 1
\end{autobreak}
\end{align*}

\begin{center}
Вот так и рождаются легенды о герое, истребившем половину рода Боблина.
\end{center}
\begin{align*}
\begin{autobreak}
\frac{d}{dx}(x^{x}) = x^x \cdot (\frac{d}{dx}(x) \cdot \ln(x) + x \cdot \frac{\frac{d}{dx}(x)}{x})
\end{autobreak}
\end{align*}

\begin{center}
Ваше вошшебство откатило деверя Боблина до младенчества
\end{center}
\begin{align*}
\begin{autobreak}
\frac{d}{dx}(x) = 1
\end{autobreak}
\end{align*}

\begin{center}
Полиморф сработал отлично: зять Боблина теперь лягушка
\end{center}
\begin{align*}
\begin{autobreak}
\frac{d}{dx}(x) = 1
\end{autobreak}
\end{align*}

teylor f(x)|dx = \begin{align*}
\begin{autobreak}
1 + \frac{1}{1} \cdot (x - 1)^{1} + \frac{2}{2} \cdot (x - 1)^{2} + \frac{3}{6} \cdot (x - 1)^{3}
\end{autobreak}
\end{align*}

\end{document}
