\documentclass[a4paper,12pt]{article}
\usepackage[T2A]{fontenc}
\usepackage[utf8]{inputenc}
\usepackage[russian]{babel}
\usepackage{amsmath}
\usepackage{amssymb}
\usepackage{autobreak}
\usepackage{hyperref}
\setcounter{secnumdepth}{0}
\begin{document}
\title{Сокращение рода боблина}
\maketitle
\newpage
\begin{titlepage}
    \centering

    {\Large Уставший волшебник}\\[1cm]

    {\huge\bfseries «Завершение рода Боблина»}\\[0.5cm]

    \raggedright

    \textbf{Предыстория} \\[0.3cm]

    Жил-был самый обычный гоблин по имени Боблин и его очень большая семья. \\ 
    Как-то раз, одним жарким летом они все вместе решили отправиться на пикник.\\ 
    Они нашли великолепную полянку посреди болота: солнышко, зеленая трава, тенеко от непонятно башни, одним словом - благодать. \\ 
    Шел 5-ый час гоблинской пъянки, тут уже нервы волшебника живущего в башне не выдержали. \\ 
    и он решил обрушиить свой праведный гнев не семейство Боблина, истребив некоторую его часть. \\ 
    \vspace{0.8cm}
    \textbf{Боевой журнал}\\[0.3cm]

    В башне стоял особенный артефакт, который записывал ход сражения в виде странного набора символов.\\ 
    Которые лишь сам маг был способен понять, здесь и будет приведет этот боевой журнал. \\     \vfill

    \raggedleft
    \textit{«Если на странице стало больше знаков — значит, кто-то из клана Боблина опять что-то натворил.»}\\[0.3cm]

\end{titlepage}
\tableofcontents
\newpage
f(x) = \begin{align*}
\begin{autobreak}
x^{x}
\end{autobreak}
\end{align*}

\section{Прибывает Тейлор и куча дальних родственников}
Текущий ход событий: \begin{align*}
\begin{autobreak}
x^{x}
\end{autobreak}
\end{align*}

\subsection{Прибывает 1-ая волна родственников Тейлора-Боблина}
Текущий ход событий: \begin{align*}
\begin{autobreak}
x^{x}
\end{autobreak}
\end{align*}

\noindent\hrulefill\begin{center}
Родственники Боблина продолжают лезть к вам, держите посох крепче!
\end{center}
\begin{align*}
\begin{autobreak}
\frac{d}{dx}(x^{x}) = x^{x} \cdot ( \frac{d}{dx}(x) \cdot \ln(x) + x \cdot \frac{d}{dx}(x) / x )
\end{autobreak}
\end{align*}

\begin{center}
Полиморф сработал отлично: зять Боблина теперь лягушка
\end{center}
\begin{align*}
\begin{autobreak}
\frac{d}{dx}(x) = 1
\end{autobreak}
\end{align*}

\begin{center}
Поздравляю! От свояка Боблина осталась только полоыина
\end{center}
\begin{align*}
\begin{autobreak}
\frac{d}{dx}(x) = 1
\end{autobreak}
\end{align*}

\subsection{Прибывает 2-ая волна родственников Тейлора-Боблина}
Текущий ход событий: \begin{align*}
\begin{autobreak}
x^{x} \cdot (\ln(x) + \frac{x}{x})
\end{autobreak}
\end{align*}

\noindent\hrulefill\begin{center}
Битва продолжается, не теряйте духу, они когда-то, наверное, закончаться!
\end{center}
\begin{align*}
\begin{autobreak}
\frac{d}{dx}(x^{x} \cdot (\ln(x) + \frac{x}{x})) = \frac{d}{dx}(x^{x}) \cdot \ln(x) + \frac{x}{x} + x^{x} \cdot \frac{d}{dx}(\ln(x) + \frac{x}{x})
\end{autobreak}
\end{align*}

\begin{center}
Их не становиться меньше, откуда они только лезут?!
\end{center}
\begin{align*}
\begin{autobreak}
\frac{d}{dx}(\ln(x) + \frac{x}{x}) = \frac{d}{dx}(\ln(x)) + \frac{d}{dx}(\frac{x}{x})
\end{autobreak}
\end{align*}

\begin{center}
Вот так и рождаются легенды о герое, истребившем половину рода Боблина.
\end{center}
\begin{align*}
\begin{autobreak}
\frac{d}{dx}(\frac{x}{x}) = \frac{\frac{d}{dx}(x) \cdot x - x \cdot \frac{d}{dx}(x)}{{x}^2}
\end{autobreak}
\end{align*}

\begin{center}
Ваше заклинание свернуло невестку Боблина в шарик
\end{center}
\begin{align*}
\begin{autobreak}
\frac{d}{dx}(x) = 1
\end{autobreak}
\end{align*}

\begin{center}
ААХХАХААХАХ Гоблин-Боблин
\end{center}
\begin{align*}
\begin{autobreak}
\frac{d}{dx}(x) = 1
\end{autobreak}
\end{align*}

\begin{center}
Небольшой взмах посохом — и план сражения выглядит куда приличнее.
\end{center}
\begin{align*}
\begin{autobreak}
\frac{d}{dx}(\ln(x)) = \frac{\frac{d}{dx}(x)}{x}
\end{autobreak}
\end{align*}

\begin{center}
Ваше вошшебство откатило деверя Боблина до младенчества
\end{center}
\begin{align*}
\begin{autobreak}
\frac{d}{dx}(x) = 1
\end{autobreak}
\end{align*}

\begin{center}
Один из гоблинов упал
\end{center}
\begin{align*}
\begin{autobreak}
\frac{d}{dx}(x^{x}) = x^{x} \cdot ( \frac{d}{dx}(x) \cdot \ln(x) + x \cdot \frac{d}{dx}(x) / x )
\end{autobreak}
\end{align*}

\begin{center}
Полиморф сработал отлично: зять Боблина теперь лягушка
\end{center}
\begin{align*}
\begin{autobreak}
\frac{d}{dx}(x) = 1
\end{autobreak}
\end{align*}

\begin{center}
Поздравляю! От свояка Боблина осталась только полоыина
\end{center}
\begin{align*}
\begin{autobreak}
\frac{d}{dx}(x) = 1
\end{autobreak}
\end{align*}

\subsection{Прибывает 3-ая волна родственников Тейлора-Боблина}
Текущий ход событий: \begin{align*}
\begin{autobreak}
x^{x} \cdot (\ln(x) + \frac{x}{x}) \cdot (\ln(x) + \frac{x}{x}) + x^{x} \cdot (\frac{1}{x} + \frac{x - x}{x \cdot x})
\end{autobreak}
\end{align*}

\noindent\hrulefill\begin{center}
Родственники Боблина продолжают лезть к вам, держите посох крепче!
\end{center}
A = \begin{align*}
\begin{autobreak}
x^{x} \cdot (\ln(x) + \frac{x}{x}) \cdot (\ln(x) + \frac{x}{x}) + x^{x} \cdot (\frac{1}{x} + \frac{x - x}{x \cdot x})
\end{autobreak}
\end{align*}

\begin{align*}
\begin{autobreak}
\frac{d}{dx}(A) = \frac{d}{dx}(x^{x} \cdot (\ln(x) + \frac{x}{x}) \cdot (\ln(x) + \frac{x}{x})) + \frac{d}{dx}(x^{x} \cdot (\frac{1}{x} + \frac{x - x}{x \cdot x}))
\end{autobreak}
\end{align*}

\begin{center}
Битва продолжается, не теряйте духу, они когда-то, наверное, закончаться!
\end{center}
\begin{align*}
\begin{autobreak}
\frac{d}{dx}(x^{x} \cdot (\frac{1}{x} + \frac{x - x}{x \cdot x})) = \frac{d}{dx}(x^{x}) \cdot \frac{1}{x} + \frac{x - x}{x \cdot x} + x^{x} \cdot \frac{d}{dx}(\frac{1}{x} + \frac{x - x}{x \cdot x})
\end{autobreak}
\end{align*}

\begin{center}
Их не становиться меньше, откуда они только лезут?!
\end{center}
\begin{align*}
\begin{autobreak}
\frac{d}{dx}(\frac{1}{x} + \frac{x - x}{x \cdot x}) = \frac{d}{dx}(\frac{1}{x}) + \frac{d}{dx}(\frac{x - x}{x \cdot x})
\end{autobreak}
\end{align*}

\begin{center}
Вот так и рождаются легенды о герое, истребившем половину рода Боблина.
\end{center}
\begin{align*}
\begin{autobreak}
\frac{d}{dx}(\frac{x - x}{x \cdot x}) = \frac{\frac{d}{dx}(x - x) \cdot x \cdot x - x - x \cdot \frac{d}{dx}(x \cdot x)}{{x \cdot x}^2}
\end{autobreak}
\end{align*}

\begin{center}
Небольшой взмах посохом — и план сражения выглядит куда приличнее.
\end{center}
\begin{align*}
\begin{autobreak}
\frac{d}{dx}(x \cdot x) = \frac{d}{dx}(x) \cdot x + x \cdot \frac{d}{dx}(x)
\end{autobreak}
\end{align*}

\begin{center}
Ваше заклинание свернуло невестку Боблина в шарик
\end{center}
\begin{align*}
\begin{autobreak}
\frac{d}{dx}(x) = 1
\end{autobreak}
\end{align*}

\begin{center}
ААХХАХААХАХ Гоблин-Боблин
\end{center}
\begin{align*}
\begin{autobreak}
\frac{d}{dx}(x) = 1
\end{autobreak}
\end{align*}

\begin{center}
Один из гоблинов упал
\end{center}
\begin{align*}
\begin{autobreak}
\frac{d}{dx}(x - x) = \frac{d}{dx}(x) - \frac{d}{dx}(x)
\end{autobreak}
\end{align*}

\begin{center}
Ваше вошшебство откатило деверя Боблина до младенчества
\end{center}
\begin{align*}
\begin{autobreak}
\frac{d}{dx}(x) = 1
\end{autobreak}
\end{align*}

\begin{center}
Полиморф сработал отлично: зять Боблина теперь лягушка
\end{center}
\begin{align*}
\begin{autobreak}
\frac{d}{dx}(x) = 1
\end{autobreak}
\end{align*}

\begin{center}
Родственники Боблина продолжают лезть к вам, держите посох крепче!
\end{center}
\begin{align*}
\begin{autobreak}
\frac{d}{dx}(\frac{1}{x}) = \frac{\frac{d}{dx}(1) \cdot x - 1 \cdot \frac{d}{dx}(x)}{{x}^2}
\end{autobreak}
\end{align*}

\begin{center}
Поздравляю! От свояка Боблина осталась только полоыина
\end{center}
\begin{align*}
\begin{autobreak}
\frac{d}{dx}(x) = 1
\end{autobreak}
\end{align*}

\begin{center}
Вы разложили дядю Боблина на молекулы
\end{center}
\begin{align*}
\begin{autobreak}
\frac{d}{dx}(1) = 0
\end{autobreak}
\end{align*}

\begin{center}
Битва продолжается, не теряйте духу, они когда-то, наверное, закончаться!
\end{center}
\begin{align*}
\begin{autobreak}
\frac{d}{dx}(x^{x}) = x^{x} \cdot ( \frac{d}{dx}(x) \cdot \ln(x) + x \cdot \frac{d}{dx}(x) / x )
\end{autobreak}
\end{align*}

\begin{center}
Ваше заклинание свернуло невестку Боблина в шарик
\end{center}
\begin{align*}
\begin{autobreak}
\frac{d}{dx}(x) = 1
\end{autobreak}
\end{align*}

\begin{center}
ААХХАХААХАХ Гоблин-Боблин
\end{center}
\begin{align*}
\begin{autobreak}
\frac{d}{dx}(x) = 1
\end{autobreak}
\end{align*}

\begin{center}
Их не становиться меньше, откуда они только лезут?!
\end{center}
\begin{align*}
\begin{autobreak}
\frac{d}{dx}(x^{x} \cdot (\ln(x) + \frac{x}{x}) \cdot (\ln(x) + \frac{x}{x})) = \frac{d}{dx}(x^{x} \cdot (\ln(x) + \frac{x}{x})) \cdot \ln(x) + \frac{x}{x} + x^{x} \cdot (\ln(x) + \frac{x}{x}) \cdot \frac{d}{dx}(\ln(x) + \frac{x}{x})
\end{autobreak}
\end{align*}

\begin{center}
Вот так и рождаются легенды о герое, истребившем половину рода Боблина.
\end{center}
\begin{align*}
\begin{autobreak}
\frac{d}{dx}(\ln(x) + \frac{x}{x}) = \frac{d}{dx}(\ln(x)) + \frac{d}{dx}(\frac{x}{x})
\end{autobreak}
\end{align*}

\begin{center}
Небольшой взмах посохом — и план сражения выглядит куда приличнее.
\end{center}
\begin{align*}
\begin{autobreak}
\frac{d}{dx}(\frac{x}{x}) = \frac{\frac{d}{dx}(x) \cdot x - x \cdot \frac{d}{dx}(x)}{{x}^2}
\end{autobreak}
\end{align*}

\begin{center}
Ваше вошшебство откатило деверя Боблина до младенчества
\end{center}
\begin{align*}
\begin{autobreak}
\frac{d}{dx}(x) = 1
\end{autobreak}
\end{align*}

\begin{center}
Полиморф сработал отлично: зять Боблина теперь лягушка
\end{center}
\begin{align*}
\begin{autobreak}
\frac{d}{dx}(x) = 1
\end{autobreak}
\end{align*}

\begin{center}
Один из гоблинов упал
\end{center}
\begin{align*}
\begin{autobreak}
\frac{d}{dx}(\ln(x)) = \frac{\frac{d}{dx}(x)}{x}
\end{autobreak}
\end{align*}

\begin{center}
Поздравляю! От свояка Боблина осталась только полоыина
\end{center}
\begin{align*}
\begin{autobreak}
\frac{d}{dx}(x) = 1
\end{autobreak}
\end{align*}

\begin{center}
Родственники Боблина продолжают лезть к вам, держите посох крепче!
\end{center}
\begin{align*}
\begin{autobreak}
\frac{d}{dx}(x^{x} \cdot (\ln(x) + \frac{x}{x})) = \frac{d}{dx}(x^{x}) \cdot \ln(x) + \frac{x}{x} + x^{x} \cdot \frac{d}{dx}(\ln(x) + \frac{x}{x})
\end{autobreak}
\end{align*}

\begin{center}
Битва продолжается, не теряйте духу, они когда-то, наверное, закончаться!
\end{center}
\begin{align*}
\begin{autobreak}
\frac{d}{dx}(\ln(x) + \frac{x}{x}) = \frac{d}{dx}(\ln(x)) + \frac{d}{dx}(\frac{x}{x})
\end{autobreak}
\end{align*}

\begin{center}
Их не становиться меньше, откуда они только лезут?!
\end{center}
\begin{align*}
\begin{autobreak}
\frac{d}{dx}(\frac{x}{x}) = \frac{\frac{d}{dx}(x) \cdot x - x \cdot \frac{d}{dx}(x)}{{x}^2}
\end{autobreak}
\end{align*}

\begin{center}
Ваше заклинание свернуло невестку Боблина в шарик
\end{center}
\begin{align*}
\begin{autobreak}
\frac{d}{dx}(x) = 1
\end{autobreak}
\end{align*}

\begin{center}
ААХХАХААХАХ Гоблин-Боблин
\end{center}
\begin{align*}
\begin{autobreak}
\frac{d}{dx}(x) = 1
\end{autobreak}
\end{align*}

\begin{center}
Вот так и рождаются легенды о герое, истребившем половину рода Боблина.
\end{center}
\begin{align*}
\begin{autobreak}
\frac{d}{dx}(\ln(x)) = \frac{\frac{d}{dx}(x)}{x}
\end{autobreak}
\end{align*}

\begin{center}
Ваше вошшебство откатило деверя Боблина до младенчества
\end{center}
\begin{align*}
\begin{autobreak}
\frac{d}{dx}(x) = 1
\end{autobreak}
\end{align*}

\begin{center}
Небольшой взмах посохом — и план сражения выглядит куда приличнее.
\end{center}
\begin{align*}
\begin{autobreak}
\frac{d}{dx}(x^{x}) = x^{x} \cdot ( \frac{d}{dx}(x) \cdot \ln(x) + x \cdot \frac{d}{dx}(x) / x )
\end{autobreak}
\end{align*}

\begin{center}
Полиморф сработал отлично: зять Боблина теперь лягушка
\end{center}
\begin{align*}
\begin{autobreak}
\frac{d}{dx}(x) = 1
\end{autobreak}
\end{align*}

\begin{center}
Поздравляю! От свояка Боблина осталась только полоыина
\end{center}
\begin{align*}
\begin{autobreak}
\frac{d}{dx}(x) = 1
\end{autobreak}
\end{align*}

\subsection{Прибывает 4-ая волна родственников Тейлора-Боблина}
A = \begin{align*}
\begin{autobreak}
x^{x} \cdot (\ln(x) + \frac{x}{x}) \cdot (\frac{1}{x} + \frac{x - x}{x \cdot x}) + x^{x} \cdot (\frac{-1}{x \cdot x} + \frac{0 - (x - x) \cdot (x + x)}{x \cdot x \cdot x \cdot x})
\end{autobreak}
\end{align*}

B = \begin{align*}
\begin{autobreak}
(x^{x} \cdot (\ln(x) + \frac{x}{x}) \cdot (\ln(x) + \frac{x}{x}) + x^{x} \cdot (\frac{1}{x} + \frac{x - x}{x \cdot x})) \cdot (\ln(x) + \frac{x}{x})
\end{autobreak}
\end{align*}

Текущий ход событий: \begin{align*}
\begin{autobreak}
B + x^{x} \cdot (\ln(x) + \frac{x}{x}) \cdot (\frac{1}{x} + \frac{x - x}{x \cdot x}) + A
\end{autobreak}
\end{align*}

\noindent\hrulefill\begin{center}
Один из гоблинов упал
\end{center}
A = \begin{align*}
\begin{autobreak}
x^{x} \cdot (\ln(x) + \frac{x}{x}) \cdot (\frac{1}{x} + \frac{x - x}{x \cdot x}) + x^{x} \cdot (\frac{-1}{x \cdot x} + \frac{0 - (x - x) \cdot (x + x)}{x \cdot x \cdot x \cdot x})
\end{autobreak}
\end{align*}

B = \begin{align*}
\begin{autobreak}
(x^{x} \cdot (\ln(x) + \frac{x}{x}) \cdot (\ln(x) + \frac{x}{x}) + x^{x} \cdot (\frac{1}{x} + \frac{x - x}{x \cdot x})) \cdot (\ln(x) + \frac{x}{x})
\end{autobreak}
\end{align*}

C = \begin{align*}
\begin{autobreak}
x^{x} \cdot (\ln(x) + \frac{x}{x}) \cdot (\frac{1}{x} + \frac{x - x}{x \cdot x})
\end{autobreak}
\end{align*}

\begin{align*}
\begin{autobreak}
\frac{d}{dx}(B + C + A) = \frac{d}{dx}(B + C) + \frac{d}{dx}(A)
\end{autobreak}
\end{align*}

\begin{center}
Родственники Боблина продолжают лезть к вам, держите посох крепче!
\end{center}
A = \begin{align*}
\begin{autobreak}
x^{x} \cdot (\ln(x) + \frac{x}{x}) \cdot (\frac{1}{x} + \frac{x - x}{x \cdot x}) + x^{x} \cdot (\frac{-1}{x \cdot x} + \frac{0 - (x - x) \cdot (x + x)}{x \cdot x \cdot x \cdot x})
\end{autobreak}
\end{align*}

\begin{align*}
\begin{autobreak}
\frac{d}{dx}(A) = \frac{d}{dx}(x^{x} \cdot (\ln(x) + \frac{x}{x}) \cdot (\frac{1}{x} + \frac{x - x}{x \cdot x})) + \frac{d}{dx}(x^{x} \cdot (\frac{-1}{x \cdot x} + \frac{0 - (x - x) \cdot (x + x)}{x \cdot x \cdot x \cdot x}))
\end{autobreak}
\end{align*}

\begin{center}
Битва продолжается, не теряйте духу, они когда-то, наверное, закончаться!
\end{center}
\begin{align*}
\begin{autobreak}
\frac{d}{dx}(x^{x} \cdot (\frac{-1}{x \cdot x} + \frac{0 - (x - x) \cdot (x + x)}{x \cdot x \cdot x \cdot x})) = \frac{d}{dx}(x^{x}) \cdot \frac{-1}{x \cdot x} + \frac{0 - (x - x) \cdot (x + x)}{x \cdot x \cdot x \cdot x} + x^{x} \cdot \frac{d}{dx}(\frac{-1}{x \cdot x} + \frac{0 - (x - x) \cdot (x + x)}{x \cdot x \cdot x \cdot x})
\end{autobreak}
\end{align*}

\begin{center}
Их не становиться меньше, откуда они только лезут?!
\end{center}
\begin{align*}
\begin{autobreak}
\frac{d}{dx}(\frac{-1}{x \cdot x} + \frac{0 - (x - x) \cdot (x + x)}{x \cdot x \cdot x \cdot x}) = \frac{d}{dx}(\frac{-1}{x \cdot x}) + \frac{d}{dx}(\frac{0 - (x - x) \cdot (x + x)}{x \cdot x \cdot x \cdot x})
\end{autobreak}
\end{align*}

\begin{center}
Вот так и рождаются легенды о герое, истребившем половину рода Боблина.
\end{center}
\begin{align*}
\begin{autobreak}
\frac{d}{dx}(\frac{0 - (x - x) \cdot (x + x)}{x \cdot x \cdot x \cdot x}) = \frac{\frac{d}{dx}(0 - (x - x) \cdot (x + x)) \cdot x \cdot x \cdot x \cdot x - 0 - (x - x) \cdot (x + x) \cdot \frac{d}{dx}(x \cdot x \cdot x \cdot x)}{{x \cdot x \cdot x \cdot x}^2}
\end{autobreak}
\end{align*}

\begin{center}
Небольшой взмах посохом — и план сражения выглядит куда приличнее.
\end{center}
\begin{align*}
\begin{autobreak}
\frac{d}{dx}(x \cdot x \cdot x \cdot x) = \frac{d}{dx}(x \cdot x) \cdot x \cdot x + x \cdot x \cdot \frac{d}{dx}(x \cdot x)
\end{autobreak}
\end{align*}

\begin{center}
Один из гоблинов упал
\end{center}
\begin{align*}
\begin{autobreak}
\frac{d}{dx}(x \cdot x) = \frac{d}{dx}(x) \cdot x + x \cdot \frac{d}{dx}(x)
\end{autobreak}
\end{align*}

\begin{center}
Ваше заклинание свернуло невестку Боблина в шарик
\end{center}
\begin{align*}
\begin{autobreak}
\frac{d}{dx}(x) = 1
\end{autobreak}
\end{align*}

\begin{center}
ААХХАХААХАХ Гоблин-Боблин
\end{center}
\begin{align*}
\begin{autobreak}
\frac{d}{dx}(x) = 1
\end{autobreak}
\end{align*}

\begin{center}
Родственники Боблина продолжают лезть к вам, держите посох крепче!
\end{center}
\begin{align*}
\begin{autobreak}
\frac{d}{dx}(x \cdot x) = \frac{d}{dx}(x) \cdot x + x \cdot \frac{d}{dx}(x)
\end{autobreak}
\end{align*}

\begin{center}
Ваше вошшебство откатило деверя Боблина до младенчества
\end{center}
\begin{align*}
\begin{autobreak}
\frac{d}{dx}(x) = 1
\end{autobreak}
\end{align*}

\begin{center}
Полиморф сработал отлично: зять Боблина теперь лягушка
\end{center}
\begin{align*}
\begin{autobreak}
\frac{d}{dx}(x) = 1
\end{autobreak}
\end{align*}

\begin{center}
Битва продолжается, не теряйте духу, они когда-то, наверное, закончаться!
\end{center}
\begin{align*}
\begin{autobreak}
\frac{d}{dx}(0 - (x - x) \cdot (x + x)) = \frac{d}{dx}(0) - \frac{d}{dx}((x - x) \cdot (x + x))
\end{autobreak}
\end{align*}

\begin{center}
Их не становиться меньше, откуда они только лезут?!
\end{center}
\begin{align*}
\begin{autobreak}
\frac{d}{dx}((x - x) \cdot (x + x)) = \frac{d}{dx}(x - x) \cdot x + x + x - x \cdot \frac{d}{dx}(x + x)
\end{autobreak}
\end{align*}

\begin{center}
Вот так и рождаются легенды о герое, истребившем половину рода Боблина.
\end{center}
\begin{align*}
\begin{autobreak}
\frac{d}{dx}(x + x) = \frac{d}{dx}(x) + \frac{d}{dx}(x)
\end{autobreak}
\end{align*}

\begin{center}
Поздравляю! От свояка Боблина осталась только полоыина
\end{center}
\begin{align*}
\begin{autobreak}
\frac{d}{dx}(x) = 1
\end{autobreak}
\end{align*}

\begin{center}
Ваше заклинание свернуло невестку Боблина в шарик
\end{center}
\begin{align*}
\begin{autobreak}
\frac{d}{dx}(x) = 1
\end{autobreak}
\end{align*}

\begin{center}
Небольшой взмах посохом — и план сражения выглядит куда приличнее.
\end{center}
\begin{align*}
\begin{autobreak}
\frac{d}{dx}(x - x) = \frac{d}{dx}(x) - \frac{d}{dx}(x)
\end{autobreak}
\end{align*}

\begin{center}
ААХХАХААХАХ Гоблин-Боблин
\end{center}
\begin{align*}
\begin{autobreak}
\frac{d}{dx}(x) = 1
\end{autobreak}
\end{align*}

\begin{center}
Ваше вошшебство откатило деверя Боблина до младенчества
\end{center}
\begin{align*}
\begin{autobreak}
\frac{d}{dx}(x) = 1
\end{autobreak}
\end{align*}

\begin{center}
Ваше волшебство оказалось не по зубам тёте Боблина, кстати, куда она делась?
\end{center}
\begin{align*}
\begin{autobreak}
\frac{d}{dx}(0) = 0
\end{autobreak}
\end{align*}

\begin{center}
Один из гоблинов упал
\end{center}
\begin{align*}
\begin{autobreak}
\frac{d}{dx}(\frac{-1}{x \cdot x}) = \frac{\frac{d}{dx}(-1) \cdot x \cdot x - -1 \cdot \frac{d}{dx}(x \cdot x)}{{x \cdot x}^2}
\end{autobreak}
\end{align*}

\begin{center}
Родственники Боблина продолжают лезть к вам, держите посох крепче!
\end{center}
\begin{align*}
\begin{autobreak}
\frac{d}{dx}(x \cdot x) = \frac{d}{dx}(x) \cdot x + x \cdot \frac{d}{dx}(x)
\end{autobreak}
\end{align*}

\begin{center}
Полиморф сработал отлично: зять Боблина теперь лягушка
\end{center}
\begin{align*}
\begin{autobreak}
\frac{d}{dx}(x) = 1
\end{autobreak}
\end{align*}

\begin{center}
Поздравляю! От свояка Боблина осталась только полоыина
\end{center}
\begin{align*}
\begin{autobreak}
\frac{d}{dx}(x) = 1
\end{autobreak}
\end{align*}

\begin{center}
Огненный шар испарил бабушку Боблина
\end{center}
\begin{align*}
\begin{autobreak}
\frac{d}{dx}(-1) = 0
\end{autobreak}
\end{align*}

\begin{center}
Битва продолжается, не теряйте духу, они когда-то, наверное, закончаться!
\end{center}
\begin{align*}
\begin{autobreak}
\frac{d}{dx}(x^{x}) = x^{x} \cdot ( \frac{d}{dx}(x) \cdot \ln(x) + x \cdot \frac{d}{dx}(x) / x )
\end{autobreak}
\end{align*}

\begin{center}
Ваше заклинание свернуло невестку Боблина в шарик
\end{center}
\begin{align*}
\begin{autobreak}
\frac{d}{dx}(x) = 1
\end{autobreak}
\end{align*}

\begin{center}
ААХХАХААХАХ Гоблин-Боблин
\end{center}
\begin{align*}
\begin{autobreak}
\frac{d}{dx}(x) = 1
\end{autobreak}
\end{align*}

\begin{center}
Их не становиться меньше, откуда они только лезут?!
\end{center}
\begin{align*}
\begin{autobreak}
\frac{d}{dx}(x^{x} \cdot (\ln(x) + \frac{x}{x}) \cdot (\frac{1}{x} + \frac{x - x}{x \cdot x})) = \frac{d}{dx}(x^{x} \cdot (\ln(x) + \frac{x}{x})) \cdot \frac{1}{x} + \frac{x - x}{x \cdot x} + x^{x} \cdot (\ln(x) + \frac{x}{x}) \cdot \frac{d}{dx}(\frac{1}{x} + \frac{x - x}{x \cdot x})
\end{autobreak}
\end{align*}

\begin{center}
Вот так и рождаются легенды о герое, истребившем половину рода Боблина.
\end{center}
\begin{align*}
\begin{autobreak}
\frac{d}{dx}(\frac{1}{x} + \frac{x - x}{x \cdot x}) = \frac{d}{dx}(\frac{1}{x}) + \frac{d}{dx}(\frac{x - x}{x \cdot x})
\end{autobreak}
\end{align*}

\begin{center}
Небольшой взмах посохом — и план сражения выглядит куда приличнее.
\end{center}
\begin{align*}
\begin{autobreak}
\frac{d}{dx}(\frac{x - x}{x \cdot x}) = \frac{\frac{d}{dx}(x - x) \cdot x \cdot x - x - x \cdot \frac{d}{dx}(x \cdot x)}{{x \cdot x}^2}
\end{autobreak}
\end{align*}

\begin{center}
Один из гоблинов упал
\end{center}
\begin{align*}
\begin{autobreak}
\frac{d}{dx}(x \cdot x) = \frac{d}{dx}(x) \cdot x + x \cdot \frac{d}{dx}(x)
\end{autobreak}
\end{align*}

\begin{center}
Ваше вошшебство откатило деверя Боблина до младенчества
\end{center}
\begin{align*}
\begin{autobreak}
\frac{d}{dx}(x) = 1
\end{autobreak}
\end{align*}

\begin{center}
Полиморф сработал отлично: зять Боблина теперь лягушка
\end{center}
\begin{align*}
\begin{autobreak}
\frac{d}{dx}(x) = 1
\end{autobreak}
\end{align*}

\begin{center}
Родственники Боблина продолжают лезть к вам, держите посох крепче!
\end{center}
\begin{align*}
\begin{autobreak}
\frac{d}{dx}(x - x) = \frac{d}{dx}(x) - \frac{d}{dx}(x)
\end{autobreak}
\end{align*}

\begin{center}
Поздравляю! От свояка Боблина осталась только полоыина
\end{center}
\begin{align*}
\begin{autobreak}
\frac{d}{dx}(x) = 1
\end{autobreak}
\end{align*}

\begin{center}
Ваше заклинание свернуло невестку Боблина в шарик
\end{center}
\begin{align*}
\begin{autobreak}
\frac{d}{dx}(x) = 1
\end{autobreak}
\end{align*}

\begin{center}
Битва продолжается, не теряйте духу, они когда-то, наверное, закончаться!
\end{center}
\begin{align*}
\begin{autobreak}
\frac{d}{dx}(\frac{1}{x}) = \frac{\frac{d}{dx}(1) \cdot x - 1 \cdot \frac{d}{dx}(x)}{{x}^2}
\end{autobreak}
\end{align*}

\begin{center}
ААХХАХААХАХ Гоблин-Боблин
\end{center}
\begin{align*}
\begin{autobreak}
\frac{d}{dx}(x) = 1
\end{autobreak}
\end{align*}

\begin{center}
Заклинание хаоса раскидало части племянника Боблина по разным планам
\end{center}
\begin{align*}
\begin{autobreak}
\frac{d}{dx}(1) = 0
\end{autobreak}
\end{align*}

\begin{center}
Их не становиться меньше, откуда они только лезут?!
\end{center}
\begin{align*}
\begin{autobreak}
\frac{d}{dx}(x^{x} \cdot (\ln(x) + \frac{x}{x})) = \frac{d}{dx}(x^{x}) \cdot \ln(x) + \frac{x}{x} + x^{x} \cdot \frac{d}{dx}(\ln(x) + \frac{x}{x})
\end{autobreak}
\end{align*}

\begin{center}
Вот так и рождаются легенды о герое, истребившем половину рода Боблина.
\end{center}
\begin{align*}
\begin{autobreak}
\frac{d}{dx}(\ln(x) + \frac{x}{x}) = \frac{d}{dx}(\ln(x)) + \frac{d}{dx}(\frac{x}{x})
\end{autobreak}
\end{align*}

\begin{center}
Небольшой взмах посохом — и план сражения выглядит куда приличнее.
\end{center}
\begin{align*}
\begin{autobreak}
\frac{d}{dx}(\frac{x}{x}) = \frac{\frac{d}{dx}(x) \cdot x - x \cdot \frac{d}{dx}(x)}{{x}^2}
\end{autobreak}
\end{align*}

\begin{center}
Ваше вошшебство откатило деверя Боблина до младенчества
\end{center}
\begin{align*}
\begin{autobreak}
\frac{d}{dx}(x) = 1
\end{autobreak}
\end{align*}

\begin{center}
Полиморф сработал отлично: зять Боблина теперь лягушка
\end{center}
\begin{align*}
\begin{autobreak}
\frac{d}{dx}(x) = 1
\end{autobreak}
\end{align*}

\begin{center}
Один из гоблинов упал
\end{center}
\begin{align*}
\begin{autobreak}
\frac{d}{dx}(\ln(x)) = \frac{\frac{d}{dx}(x)}{x}
\end{autobreak}
\end{align*}

\begin{center}
Поздравляю! От свояка Боблина осталась только полоыина
\end{center}
\begin{align*}
\begin{autobreak}
\frac{d}{dx}(x) = 1
\end{autobreak}
\end{align*}

\begin{center}
Родственники Боблина продолжают лезть к вам, держите посох крепче!
\end{center}
\begin{align*}
\begin{autobreak}
\frac{d}{dx}(x^{x}) = x^{x} \cdot ( \frac{d}{dx}(x) \cdot \ln(x) + x \cdot \frac{d}{dx}(x) / x )
\end{autobreak}
\end{align*}

\begin{center}
Ваше заклинание свернуло невестку Боблина в шарик
\end{center}
\begin{align*}
\begin{autobreak}
\frac{d}{dx}(x) = 1
\end{autobreak}
\end{align*}

\begin{center}
ААХХАХААХАХ Гоблин-Боблин
\end{center}
\begin{align*}
\begin{autobreak}
\frac{d}{dx}(x) = 1
\end{autobreak}
\end{align*}

\begin{center}
Битва продолжается, не теряйте духу, они когда-то, наверное, закончаться!
\end{center}
A = \begin{align*}
\begin{autobreak}
(x^{x} \cdot (\ln(x) + \frac{x}{x}) \cdot (\ln(x) + \frac{x}{x}) + x^{x} \cdot (\frac{1}{x} + \frac{x - x}{x \cdot x})) \cdot (\ln(x) + \frac{x}{x})
\end{autobreak}
\end{align*}

\begin{align*}
\begin{autobreak}
\frac{d}{dx}(A + x^{x} \cdot (\ln(x) + \frac{x}{x}) \cdot (\frac{1}{x} + \frac{x - x}{x \cdot x})) = \frac{d}{dx}(A) + \frac{d}{dx}(x^{x} \cdot (\ln(x) + \frac{x}{x}) \cdot (\frac{1}{x} + \frac{x - x}{x \cdot x}))
\end{autobreak}
\end{align*}

\begin{center}
Их не становиться меньше, откуда они только лезут?!
\end{center}
\begin{align*}
\begin{autobreak}
\frac{d}{dx}(x^{x} \cdot (\ln(x) + \frac{x}{x}) \cdot (\frac{1}{x} + \frac{x - x}{x \cdot x})) = \frac{d}{dx}(x^{x} \cdot (\ln(x) + \frac{x}{x})) \cdot \frac{1}{x} + \frac{x - x}{x \cdot x} + x^{x} \cdot (\ln(x) + \frac{x}{x}) \cdot \frac{d}{dx}(\frac{1}{x} + \frac{x - x}{x \cdot x})
\end{autobreak}
\end{align*}

\begin{center}
Вот так и рождаются легенды о герое, истребившем половину рода Боблина.
\end{center}
\begin{align*}
\begin{autobreak}
\frac{d}{dx}(\frac{1}{x} + \frac{x - x}{x \cdot x}) = \frac{d}{dx}(\frac{1}{x}) + \frac{d}{dx}(\frac{x - x}{x \cdot x})
\end{autobreak}
\end{align*}

\begin{center}
Небольшой взмах посохом — и план сражения выглядит куда приличнее.
\end{center}
\begin{align*}
\begin{autobreak}
\frac{d}{dx}(\frac{x - x}{x \cdot x}) = \frac{\frac{d}{dx}(x - x) \cdot x \cdot x - x - x \cdot \frac{d}{dx}(x \cdot x)}{{x \cdot x}^2}
\end{autobreak}
\end{align*}

\begin{center}
Один из гоблинов упал
\end{center}
\begin{align*}
\begin{autobreak}
\frac{d}{dx}(x \cdot x) = \frac{d}{dx}(x) \cdot x + x \cdot \frac{d}{dx}(x)
\end{autobreak}
\end{align*}

\begin{center}
Ваше вошшебство откатило деверя Боблина до младенчества
\end{center}
\begin{align*}
\begin{autobreak}
\frac{d}{dx}(x) = 1
\end{autobreak}
\end{align*}

\begin{center}
Полиморф сработал отлично: зять Боблина теперь лягушка
\end{center}
\begin{align*}
\begin{autobreak}
\frac{d}{dx}(x) = 1
\end{autobreak}
\end{align*}

\begin{center}
Родственники Боблина продолжают лезть к вам, держите посох крепче!
\end{center}
\begin{align*}
\begin{autobreak}
\frac{d}{dx}(x - x) = \frac{d}{dx}(x) - \frac{d}{dx}(x)
\end{autobreak}
\end{align*}

\begin{center}
Поздравляю! От свояка Боблина осталась только полоыина
\end{center}
\begin{align*}
\begin{autobreak}
\frac{d}{dx}(x) = 1
\end{autobreak}
\end{align*}

\begin{center}
Ваше заклинание свернуло невестку Боблина в шарик
\end{center}
\begin{align*}
\begin{autobreak}
\frac{d}{dx}(x) = 1
\end{autobreak}
\end{align*}

\begin{center}
Битва продолжается, не теряйте духу, они когда-то, наверное, закончаться!
\end{center}
\begin{align*}
\begin{autobreak}
\frac{d}{dx}(\frac{1}{x}) = \frac{\frac{d}{dx}(1) \cdot x - 1 \cdot \frac{d}{dx}(x)}{{x}^2}
\end{autobreak}
\end{align*}

\begin{center}
ААХХАХААХАХ Гоблин-Боблин
\end{center}
\begin{align*}
\begin{autobreak}
\frac{d}{dx}(x) = 1
\end{autobreak}
\end{align*}

\begin{center}
Ваш портал небытия вежливо удалил тещу Боблина из этого измерения
\end{center}
\begin{align*}
\begin{autobreak}
\frac{d}{dx}(1) = 0
\end{autobreak}
\end{align*}

\begin{center}
Их не становиться меньше, откуда они только лезут?!
\end{center}
\begin{align*}
\begin{autobreak}
\frac{d}{dx}(x^{x} \cdot (\ln(x) + \frac{x}{x})) = \frac{d}{dx}(x^{x}) \cdot \ln(x) + \frac{x}{x} + x^{x} \cdot \frac{d}{dx}(\ln(x) + \frac{x}{x})
\end{autobreak}
\end{align*}

\begin{center}
Вот так и рождаются легенды о герое, истребившем половину рода Боблина.
\end{center}
\begin{align*}
\begin{autobreak}
\frac{d}{dx}(\ln(x) + \frac{x}{x}) = \frac{d}{dx}(\ln(x)) + \frac{d}{dx}(\frac{x}{x})
\end{autobreak}
\end{align*}

\begin{center}
Небольшой взмах посохом — и план сражения выглядит куда приличнее.
\end{center}
\begin{align*}
\begin{autobreak}
\frac{d}{dx}(\frac{x}{x}) = \frac{\frac{d}{dx}(x) \cdot x - x \cdot \frac{d}{dx}(x)}{{x}^2}
\end{autobreak}
\end{align*}

\begin{center}
Ваше вошшебство откатило деверя Боблина до младенчества
\end{center}
\begin{align*}
\begin{autobreak}
\frac{d}{dx}(x) = 1
\end{autobreak}
\end{align*}

\begin{center}
Полиморф сработал отлично: зять Боблина теперь лягушка
\end{center}
\begin{align*}
\begin{autobreak}
\frac{d}{dx}(x) = 1
\end{autobreak}
\end{align*}

\begin{center}
Один из гоблинов упал
\end{center}
\begin{align*}
\begin{autobreak}
\frac{d}{dx}(\ln(x)) = \frac{\frac{d}{dx}(x)}{x}
\end{autobreak}
\end{align*}

\begin{center}
Поздравляю! От свояка Боблина осталась только полоыина
\end{center}
\begin{align*}
\begin{autobreak}
\frac{d}{dx}(x) = 1
\end{autobreak}
\end{align*}

\begin{center}
Родственники Боблина продолжают лезть к вам, держите посох крепче!
\end{center}
\begin{align*}
\begin{autobreak}
\frac{d}{dx}(x^{x}) = x^{x} \cdot ( \frac{d}{dx}(x) \cdot \ln(x) + x \cdot \frac{d}{dx}(x) / x )
\end{autobreak}
\end{align*}

\begin{center}
Ваше заклинание свернуло невестку Боблина в шарик
\end{center}
\begin{align*}
\begin{autobreak}
\frac{d}{dx}(x) = 1
\end{autobreak}
\end{align*}

\begin{center}
ААХХАХААХАХ Гоблин-Боблин
\end{center}
\begin{align*}
\begin{autobreak}
\frac{d}{dx}(x) = 1
\end{autobreak}
\end{align*}

\begin{center}
Битва продолжается, не теряйте духу, они когда-то, наверное, закончаться!
\end{center}
A = \begin{align*}
\begin{autobreak}
(x^{x} \cdot (\ln(x) + \frac{x}{x}) \cdot (\ln(x) + \frac{x}{x}) + x^{x} \cdot (\frac{1}{x} + \frac{x - x}{x \cdot x})) \cdot (\ln(x) + \frac{x}{x})
\end{autobreak}
\end{align*}

\begin{align*}
\begin{autobreak}
\frac{d}{dx}(A) = \frac{d}{dx}(x^{x} \cdot (\ln(x) + \frac{x}{x}) \cdot (\ln(x) + \frac{x}{x}) + x^{x} \cdot (\frac{1}{x} + \frac{x - x}{x \cdot x})) \cdot \ln(x) + \frac{x}{x} + x^{x} \cdot (\ln(x) + \frac{x}{x}) \cdot (\ln(x) + \frac{x}{x}) + x^{x} \cdot (\frac{1}{x} + \frac{x - x}{x \cdot x}) \cdot \frac{d}{dx}(\ln(x) + \frac{x}{x})
\end{autobreak}
\end{align*}

\begin{center}
Их не становиться меньше, откуда они только лезут?!
\end{center}
\begin{align*}
\begin{autobreak}
\frac{d}{dx}(\ln(x) + \frac{x}{x}) = \frac{d}{dx}(\ln(x)) + \frac{d}{dx}(\frac{x}{x})
\end{autobreak}
\end{align*}

\begin{center}
Вот так и рождаются легенды о герое, истребившем половину рода Боблина.
\end{center}
\begin{align*}
\begin{autobreak}
\frac{d}{dx}(\frac{x}{x}) = \frac{\frac{d}{dx}(x) \cdot x - x \cdot \frac{d}{dx}(x)}{{x}^2}
\end{autobreak}
\end{align*}

\begin{center}
Ваше вошшебство откатило деверя Боблина до младенчества
\end{center}
\begin{align*}
\begin{autobreak}
\frac{d}{dx}(x) = 1
\end{autobreak}
\end{align*}

\begin{center}
Полиморф сработал отлично: зять Боблина теперь лягушка
\end{center}
\begin{align*}
\begin{autobreak}
\frac{d}{dx}(x) = 1
\end{autobreak}
\end{align*}

\begin{center}
Небольшой взмах посохом — и план сражения выглядит куда приличнее.
\end{center}
\begin{align*}
\begin{autobreak}
\frac{d}{dx}(\ln(x)) = \frac{\frac{d}{dx}(x)}{x}
\end{autobreak}
\end{align*}

\begin{center}
Поздравляю! От свояка Боблина осталась только полоыина
\end{center}
\begin{align*}
\begin{autobreak}
\frac{d}{dx}(x) = 1
\end{autobreak}
\end{align*}

\begin{center}
Один из гоблинов упал
\end{center}
A = \begin{align*}
\begin{autobreak}
x^{x} \cdot (\ln(x) + \frac{x}{x}) \cdot (\ln(x) + \frac{x}{x}) + x^{x} \cdot (\frac{1}{x} + \frac{x - x}{x \cdot x})
\end{autobreak}
\end{align*}

\begin{align*}
\begin{autobreak}
\frac{d}{dx}(A) = \frac{d}{dx}(x^{x} \cdot (\ln(x) + \frac{x}{x}) \cdot (\ln(x) + \frac{x}{x})) + \frac{d}{dx}(x^{x} \cdot (\frac{1}{x} + \frac{x - x}{x \cdot x}))
\end{autobreak}
\end{align*}

\begin{center}
Родственники Боблина продолжают лезть к вам, держите посох крепче!
\end{center}
\begin{align*}
\begin{autobreak}
\frac{d}{dx}(x^{x} \cdot (\frac{1}{x} + \frac{x - x}{x \cdot x})) = \frac{d}{dx}(x^{x}) \cdot \frac{1}{x} + \frac{x - x}{x \cdot x} + x^{x} \cdot \frac{d}{dx}(\frac{1}{x} + \frac{x - x}{x \cdot x})
\end{autobreak}
\end{align*}

\begin{center}
Битва продолжается, не теряйте духу, они когда-то, наверное, закончаться!
\end{center}
\begin{align*}
\begin{autobreak}
\frac{d}{dx}(\frac{1}{x} + \frac{x - x}{x \cdot x}) = \frac{d}{dx}(\frac{1}{x}) + \frac{d}{dx}(\frac{x - x}{x \cdot x})
\end{autobreak}
\end{align*}

\begin{center}
Их не становиться меньше, откуда они только лезут?!
\end{center}
\begin{align*}
\begin{autobreak}
\frac{d}{dx}(\frac{x - x}{x \cdot x}) = \frac{\frac{d}{dx}(x - x) \cdot x \cdot x - x - x \cdot \frac{d}{dx}(x \cdot x)}{{x \cdot x}^2}
\end{autobreak}
\end{align*}

\begin{center}
Вот так и рождаются легенды о герое, истребившем половину рода Боблина.
\end{center}
\begin{align*}
\begin{autobreak}
\frac{d}{dx}(x \cdot x) = \frac{d}{dx}(x) \cdot x + x \cdot \frac{d}{dx}(x)
\end{autobreak}
\end{align*}

\begin{center}
Ваше заклинание свернуло невестку Боблина в шарик
\end{center}
\begin{align*}
\begin{autobreak}
\frac{d}{dx}(x) = 1
\end{autobreak}
\end{align*}

\begin{center}
ААХХАХААХАХ Гоблин-Боблин
\end{center}
\begin{align*}
\begin{autobreak}
\frac{d}{dx}(x) = 1
\end{autobreak}
\end{align*}

\begin{center}
Небольшой взмах посохом — и план сражения выглядит куда приличнее.
\end{center}
\begin{align*}
\begin{autobreak}
\frac{d}{dx}(x - x) = \frac{d}{dx}(x) - \frac{d}{dx}(x)
\end{autobreak}
\end{align*}

\begin{center}
Ваше вошшебство откатило деверя Боблина до младенчества
\end{center}
\begin{align*}
\begin{autobreak}
\frac{d}{dx}(x) = 1
\end{autobreak}
\end{align*}

\begin{center}
Полиморф сработал отлично: зять Боблина теперь лягушка
\end{center}
\begin{align*}
\begin{autobreak}
\frac{d}{dx}(x) = 1
\end{autobreak}
\end{align*}

\begin{center}
Один из гоблинов упал
\end{center}
\begin{align*}
\begin{autobreak}
\frac{d}{dx}(\frac{1}{x}) = \frac{\frac{d}{dx}(1) \cdot x - 1 \cdot \frac{d}{dx}(x)}{{x}^2}
\end{autobreak}
\end{align*}

\begin{center}
Поздравляю! От свояка Боблина осталась только полоыина
\end{center}
\begin{align*}
\begin{autobreak}
\frac{d}{dx}(x) = 1
\end{autobreak}
\end{align*}

\begin{center}
Ваше заклинание дезинтегрировало брата Боблина
\end{center}
\begin{align*}
\begin{autobreak}
\frac{d}{dx}(1) = 0
\end{autobreak}
\end{align*}

\begin{center}
Родственники Боблина продолжают лезть к вам, держите посох крепче!
\end{center}
\begin{align*}
\begin{autobreak}
\frac{d}{dx}(x^{x}) = x^{x} \cdot ( \frac{d}{dx}(x) \cdot \ln(x) + x \cdot \frac{d}{dx}(x) / x )
\end{autobreak}
\end{align*}

\begin{center}
Ваше заклинание свернуло невестку Боблина в шарик
\end{center}
\begin{align*}
\begin{autobreak}
\frac{d}{dx}(x) = 1
\end{autobreak}
\end{align*}

\begin{center}
ААХХАХААХАХ Гоблин-Боблин
\end{center}
\begin{align*}
\begin{autobreak}
\frac{d}{dx}(x) = 1
\end{autobreak}
\end{align*}

\begin{center}
Битва продолжается, не теряйте духу, они когда-то, наверное, закончаться!
\end{center}
\begin{align*}
\begin{autobreak}
\frac{d}{dx}(x^{x} \cdot (\ln(x) + \frac{x}{x}) \cdot (\ln(x) + \frac{x}{x})) = \frac{d}{dx}(x^{x} \cdot (\ln(x) + \frac{x}{x})) \cdot \ln(x) + \frac{x}{x} + x^{x} \cdot (\ln(x) + \frac{x}{x}) \cdot \frac{d}{dx}(\ln(x) + \frac{x}{x})
\end{autobreak}
\end{align*}

\begin{center}
Их не становиться меньше, откуда они только лезут?!
\end{center}
\begin{align*}
\begin{autobreak}
\frac{d}{dx}(\ln(x) + \frac{x}{x}) = \frac{d}{dx}(\ln(x)) + \frac{d}{dx}(\frac{x}{x})
\end{autobreak}
\end{align*}

\begin{center}
Вот так и рождаются легенды о герое, истребившем половину рода Боблина.
\end{center}
\begin{align*}
\begin{autobreak}
\frac{d}{dx}(\frac{x}{x}) = \frac{\frac{d}{dx}(x) \cdot x - x \cdot \frac{d}{dx}(x)}{{x}^2}
\end{autobreak}
\end{align*}

\begin{center}
Ваше вошшебство откатило деверя Боблина до младенчества
\end{center}
\begin{align*}
\begin{autobreak}
\frac{d}{dx}(x) = 1
\end{autobreak}
\end{align*}

\begin{center}
Полиморф сработал отлично: зять Боблина теперь лягушка
\end{center}
\begin{align*}
\begin{autobreak}
\frac{d}{dx}(x) = 1
\end{autobreak}
\end{align*}

\begin{center}
Небольшой взмах посохом — и план сражения выглядит куда приличнее.
\end{center}
\begin{align*}
\begin{autobreak}
\frac{d}{dx}(\ln(x)) = \frac{\frac{d}{dx}(x)}{x}
\end{autobreak}
\end{align*}

\begin{center}
Поздравляю! От свояка Боблина осталась только полоыина
\end{center}
\begin{align*}
\begin{autobreak}
\frac{d}{dx}(x) = 1
\end{autobreak}
\end{align*}

\begin{center}
Один из гоблинов упал
\end{center}
\begin{align*}
\begin{autobreak}
\frac{d}{dx}(x^{x} \cdot (\ln(x) + \frac{x}{x})) = \frac{d}{dx}(x^{x}) \cdot \ln(x) + \frac{x}{x} + x^{x} \cdot \frac{d}{dx}(\ln(x) + \frac{x}{x})
\end{autobreak}
\end{align*}

\begin{center}
Родственники Боблина продолжают лезть к вам, держите посох крепче!
\end{center}
\begin{align*}
\begin{autobreak}
\frac{d}{dx}(\ln(x) + \frac{x}{x}) = \frac{d}{dx}(\ln(x)) + \frac{d}{dx}(\frac{x}{x})
\end{autobreak}
\end{align*}

\begin{center}
Битва продолжается, не теряйте духу, они когда-то, наверное, закончаться!
\end{center}
\begin{align*}
\begin{autobreak}
\frac{d}{dx}(\frac{x}{x}) = \frac{\frac{d}{dx}(x) \cdot x - x \cdot \frac{d}{dx}(x)}{{x}^2}
\end{autobreak}
\end{align*}

\begin{center}
Ваше заклинание свернуло невестку Боблина в шарик
\end{center}
\begin{align*}
\begin{autobreak}
\frac{d}{dx}(x) = 1
\end{autobreak}
\end{align*}

\begin{center}
ААХХАХААХАХ Гоблин-Боблин
\end{center}
\begin{align*}
\begin{autobreak}
\frac{d}{dx}(x) = 1
\end{autobreak}
\end{align*}

\begin{center}
Их не становиться меньше, откуда они только лезут?!
\end{center}
\begin{align*}
\begin{autobreak}
\frac{d}{dx}(\ln(x)) = \frac{\frac{d}{dx}(x)}{x}
\end{autobreak}
\end{align*}

\begin{center}
Ваше вошшебство откатило деверя Боблина до младенчества
\end{center}
\begin{align*}
\begin{autobreak}
\frac{d}{dx}(x) = 1
\end{autobreak}
\end{align*}

\begin{center}
Вот так и рождаются легенды о герое, истребившем половину рода Боблина.
\end{center}
\begin{align*}
\begin{autobreak}
\frac{d}{dx}(x^{x}) = x^{x} \cdot ( \frac{d}{dx}(x) \cdot \ln(x) + x \cdot \frac{d}{dx}(x) / x )
\end{autobreak}
\end{align*}

\begin{center}
Полиморф сработал отлично: зять Боблина теперь лягушка
\end{center}
\begin{align*}
\begin{autobreak}
\frac{d}{dx}(x) = 1
\end{autobreak}
\end{align*}

\begin{center}
Поздравляю! От свояка Боблина осталась только полоыина
\end{center}
\begin{align*}
\begin{autobreak}
\frac{d}{dx}(x) = 1
\end{autobreak}
\end{align*}

\subsection{Прибывает 5-ая волна родственников Тейлора-Боблина}
A = \begin{align*}
\begin{autobreak}
x^{x} \cdot (\ln(x) + \frac{x}{x}) \cdot (\frac{1}{x} + \frac{x - x}{x \cdot x}) + x^{x} \cdot (\frac{-1}{x \cdot x} + \frac{0 - (x - x) \cdot (x + x)}{x \cdot x \cdot x \cdot x})
\end{autobreak}
\end{align*}

B = \begin{align*}
\begin{autobreak}
(x^{x} \cdot (\ln(x) + \frac{x}{x}) \cdot (\ln(x) + \frac{x}{x}) + x^{x} \cdot (\frac{1}{x} + \frac{x - x}{x \cdot x})) \cdot (\frac{1}{x} + \frac{x - x}{x \cdot x})
\end{autobreak}
\end{align*}

C = \begin{align*}
\begin{autobreak}
(x^{x} \cdot (\ln(x) + \frac{x}{x}) \cdot (\ln(x) + \frac{x}{x}) + x^{x} \cdot (\frac{1}{x} + \frac{x - x}{x \cdot x})) \cdot (\frac{1}{x} + \frac{x - x}{x \cdot x})
\end{autobreak}
\end{align*}

D = \begin{align*}
\begin{autobreak}
(x^{x} \cdot (\ln(x) + \frac{x}{x}) \cdot (\ln(x) + \frac{x}{x}) + x^{x} \cdot (\frac{1}{x} + \frac{x - x}{x \cdot x})) \cdot (\frac{1}{x} + \frac{x - x}{x \cdot x})
\end{autobreak}
\end{align*}

E = \begin{align*}
\begin{autobreak}
(x^{x} \cdot (\ln(x) + \frac{x}{x}) \cdot (\ln(x) + \frac{x}{x}) + x^{x} \cdot (\frac{1}{x} + \frac{x - x}{x \cdot x})) \cdot (\ln(x) + \frac{x}{x})
\end{autobreak}
\end{align*}

F = \begin{align*}
\begin{autobreak}
(0 - (x - x) \cdot (x + x)) \cdot ((x + x) \cdot x \cdot x + x \cdot x \cdot (x + x))
\end{autobreak}
\end{align*}

G = \begin{align*}
\begin{autobreak}
x^{x} \cdot (\ln(x) + \frac{x}{x}) \cdot (\frac{-1}{x \cdot x} + \frac{0 - (x - x) \cdot (x + x)}{x \cdot x \cdot x \cdot x})
\end{autobreak}
\end{align*}

H = \begin{align*}
\begin{autobreak}
x^{x} \cdot (\ln(x) + \frac{x}{x}) \cdot (\frac{-1}{x \cdot x} + \frac{0 - (x - x) \cdot (x + x)}{x \cdot x \cdot x \cdot x})
\end{autobreak}
\end{align*}

I = \begin{align*}
\begin{autobreak}
x^{x} \cdot (\ln(x) + \frac{x}{x}) \cdot (\frac{-1}{x \cdot x} + \frac{0 - (x - x) \cdot (x + x)}{x \cdot x \cdot x \cdot x})
\end{autobreak}
\end{align*}

J = \begin{align*}
\begin{autobreak}
(0 - (x - x) \cdot 2) \cdot x \cdot x \cdot x \cdot x
\end{autobreak}
\end{align*}

K = \begin{align*}
\begin{autobreak}
x \cdot x \cdot x \cdot x \cdot x \cdot x \cdot x \cdot x
\end{autobreak}
\end{align*}

Текущий ход событий: \begin{align*}
\begin{autobreak}
(E + x^{x} \cdot (\ln(x) + \frac{x}{x}) \cdot (\frac{1}{x} + \frac{x - x}{x \cdot x}) + A) \cdot (\ln(x) + \frac{x}{x}) + B + C + G + D + H + I + x^{x} \cdot (\frac{0 - -1 \cdot (x + x)}{x \cdot x \cdot x \cdot x} + \frac{J - F}{K})
\end{autobreak}
\end{align*}

\noindent\hrulefill\begin{center}
Небольшой взмах посохом — и план сражения выглядит куда приличнее.
\end{center}
A = \begin{align*}
\begin{autobreak}
x^{x} \cdot (\ln(x) + \frac{x}{x}) \cdot (\frac{1}{x} + \frac{x - x}{x \cdot x}) + x^{x} \cdot (\frac{-1}{x \cdot x} + \frac{0 - (x - x) \cdot (x + x)}{x \cdot x \cdot x \cdot x})
\end{autobreak}
\end{align*}

B = \begin{align*}
\begin{autobreak}
(x^{x} \cdot (\ln(x) + \frac{x}{x}) \cdot (\ln(x) + \frac{x}{x}) + x^{x} \cdot (\frac{1}{x} + \frac{x - x}{x \cdot x})) \cdot (\frac{1}{x} + \frac{x - x}{x \cdot x})
\end{autobreak}
\end{align*}

C = \begin{align*}
\begin{autobreak}
(x^{x} \cdot (\ln(x) + \frac{x}{x}) \cdot (\ln(x) + \frac{x}{x}) + x^{x} \cdot (\frac{1}{x} + \frac{x - x}{x \cdot x})) \cdot (\frac{1}{x} + \frac{x - x}{x \cdot x})
\end{autobreak}
\end{align*}

D = \begin{align*}
\begin{autobreak}
(x^{x} \cdot (\ln(x) + \frac{x}{x}) \cdot (\ln(x) + \frac{x}{x}) + x^{x} \cdot (\frac{1}{x} + \frac{x - x}{x \cdot x})) \cdot (\frac{1}{x} + \frac{x - x}{x \cdot x})
\end{autobreak}
\end{align*}

E = \begin{align*}
\begin{autobreak}
(x^{x} \cdot (\ln(x) + \frac{x}{x}) \cdot (\ln(x) + \frac{x}{x}) + x^{x} \cdot (\frac{1}{x} + \frac{x - x}{x \cdot x})) \cdot (\ln(x) + \frac{x}{x})
\end{autobreak}
\end{align*}

F = \begin{align*}
\begin{autobreak}
(0 - (x - x) \cdot (x + x)) \cdot ((x + x) \cdot x \cdot x + x \cdot x \cdot (x + x))
\end{autobreak}
\end{align*}

G = \begin{align*}
\begin{autobreak}
x^{x} \cdot (\ln(x) + \frac{x}{x}) \cdot (\frac{-1}{x \cdot x} + \frac{0 - (x - x) \cdot (x + x)}{x \cdot x \cdot x \cdot x})
\end{autobreak}
\end{align*}

H = \begin{align*}
\begin{autobreak}
x^{x} \cdot (\ln(x) + \frac{x}{x}) \cdot (\frac{-1}{x \cdot x} + \frac{0 - (x - x) \cdot (x + x)}{x \cdot x \cdot x \cdot x})
\end{autobreak}
\end{align*}

I = \begin{align*}
\begin{autobreak}
x^{x} \cdot (\ln(x) + \frac{x}{x}) \cdot (\frac{-1}{x \cdot x} + \frac{0 - (x - x) \cdot (x + x)}{x \cdot x \cdot x \cdot x})
\end{autobreak}
\end{align*}

J = \begin{align*}
\begin{autobreak}
(0 - (x - x) \cdot 2) \cdot x \cdot x \cdot x \cdot x
\end{autobreak}
\end{align*}

K = \begin{align*}
\begin{autobreak}
x \cdot x \cdot x \cdot x \cdot x \cdot x \cdot x \cdot x
\end{autobreak}
\end{align*}

L = \begin{align*}
\begin{autobreak}
x^{x} \cdot (\ln(x) + \frac{x}{x}) \cdot (\frac{1}{x} + \frac{x - x}{x \cdot x})
\end{autobreak}
\end{align*}

M = \begin{align*}
\begin{autobreak}
\frac{0 - -1 \cdot (x + x)}{x \cdot x \cdot x \cdot x}
\end{autobreak}
\end{align*}

\begin{align*}
\begin{autobreak}
\frac{d}{dx}((E + L + A) \cdot (\ln(x) + \frac{x}{x}) + B + C + G + D + H + I + x^{x} \cdot (M + \frac{J - F}{K})) = \frac{d}{dx}((E + L + A) \cdot (\ln(x) + \frac{x}{x}) + B + C + G) + \frac{d}{dx}(D + H + I + x^{x} \cdot (M + \frac{J - F}{K}))
\end{autobreak}
\end{align*}

\begin{center}
Один из гоблинов упал
\end{center}
A = \begin{align*}
\begin{autobreak}
(x^{x} \cdot (\ln(x) + \frac{x}{x}) \cdot (\ln(x) + \frac{x}{x}) + x^{x} \cdot (\frac{1}{x} + \frac{x - x}{x \cdot x})) \cdot (\frac{1}{x} + \frac{x - x}{x \cdot x})
\end{autobreak}
\end{align*}

B = \begin{align*}
\begin{autobreak}
(0 - (x - x) \cdot (x + x)) \cdot ((x + x) \cdot x \cdot x + x \cdot x \cdot (x + x))
\end{autobreak}
\end{align*}

C = \begin{align*}
\begin{autobreak}
x^{x} \cdot (\ln(x) + \frac{x}{x}) \cdot (\frac{-1}{x \cdot x} + \frac{0 - (x - x) \cdot (x + x)}{x \cdot x \cdot x \cdot x})
\end{autobreak}
\end{align*}

D = \begin{align*}
\begin{autobreak}
x^{x} \cdot (\ln(x) + \frac{x}{x}) \cdot (\frac{-1}{x \cdot x} + \frac{0 - (x - x) \cdot (x + x)}{x \cdot x \cdot x \cdot x})
\end{autobreak}
\end{align*}

E = \begin{align*}
\begin{autobreak}
(0 - (x - x) \cdot 2) \cdot x \cdot x \cdot x \cdot x
\end{autobreak}
\end{align*}

F = \begin{align*}
\begin{autobreak}
x \cdot x \cdot x \cdot x \cdot x \cdot x \cdot x \cdot x
\end{autobreak}
\end{align*}

\begin{align*}
\begin{autobreak}
\frac{d}{dx}(A + C + D + x^{x} \cdot (\frac{0 - -1 \cdot (x + x)}{x \cdot x \cdot x \cdot x} + \frac{E - B}{F})) = \frac{d}{dx}(A + C) + \frac{d}{dx}(D + x^{x} \cdot (\frac{0 - -1 \cdot (x + x)}{x \cdot x \cdot x \cdot x} + \frac{E - B}{F}))
\end{autobreak}
\end{align*}

\begin{center}
Родственники Боблина продолжают лезть к вам, держите посох крепче!
\end{center}
A = \begin{align*}
\begin{autobreak}
(0 - (x - x) \cdot (x + x)) \cdot ((x + x) \cdot x \cdot x + x \cdot x \cdot (x + x))
\end{autobreak}
\end{align*}

B = \begin{align*}
\begin{autobreak}
x^{x} \cdot (\ln(x) + \frac{x}{x}) \cdot (\frac{-1}{x \cdot x} + \frac{0 - (x - x) \cdot (x + x)}{x \cdot x \cdot x \cdot x})
\end{autobreak}
\end{align*}

C = \begin{align*}
\begin{autobreak}
(0 - (x - x) \cdot 2) \cdot x \cdot x \cdot x \cdot x
\end{autobreak}
\end{align*}

D = \begin{align*}
\begin{autobreak}
x \cdot x \cdot x \cdot x \cdot x \cdot x \cdot x \cdot x
\end{autobreak}
\end{align*}

\begin{align*}
\begin{autobreak}
\frac{d}{dx}(B + x^{x} \cdot (\frac{0 - -1 \cdot (x + x)}{x \cdot x \cdot x \cdot x} + \frac{C - A}{D})) = \frac{d}{dx}(B) + \frac{d}{dx}(x^{x} \cdot (\frac{0 - -1 \cdot (x + x)}{x \cdot x \cdot x \cdot x} + \frac{C - A}{D}))
\end{autobreak}
\end{align*}

\begin{center}
Битва продолжается, не теряйте духу, они когда-то, наверное, закончаться!
\end{center}
A = \begin{align*}
\begin{autobreak}
(0 - (x - x) \cdot (x + x)) \cdot ((x + x) \cdot x \cdot x + x \cdot x \cdot (x + x))
\end{autobreak}
\end{align*}

B = \begin{align*}
\begin{autobreak}
(0 - (x - x) \cdot 2) \cdot x \cdot x \cdot x \cdot x
\end{autobreak}
\end{align*}

\begin{align*}
\begin{autobreak}
\frac{d}{dx}(x^{x} \cdot (\frac{0 - -1 \cdot (x + x)}{x \cdot x \cdot x \cdot x} + \frac{B - A}{x \cdot x \cdot x \cdot x \cdot x \cdot x \cdot x \cdot x})) = \frac{d}{dx}(x^{x}) \cdot \frac{0 - -1 \cdot (x + x)}{x \cdot x \cdot x \cdot x} + \frac{B - A}{x \cdot x \cdot x \cdot x \cdot x \cdot x \cdot x \cdot x} + x^{x} \cdot \frac{d}{dx}(\frac{0 - -1 \cdot (x + x)}{x \cdot x \cdot x \cdot x} + \frac{B - A}{x \cdot x \cdot x \cdot x \cdot x \cdot x \cdot x \cdot x})
\end{autobreak}
\end{align*}

\begin{center}
Их не становиться меньше, откуда они только лезут?!
\end{center}
A = \begin{align*}
\begin{autobreak}
(0 - (x - x) \cdot (x + x)) \cdot ((x + x) \cdot x \cdot x + x \cdot x \cdot (x + x))
\end{autobreak}
\end{align*}

B = \begin{align*}
\begin{autobreak}
(0 - (x - x) \cdot 2) \cdot x \cdot x \cdot x \cdot x
\end{autobreak}
\end{align*}

\begin{align*}
\begin{autobreak}
\frac{d}{dx}(\frac{0 - -1 \cdot (x + x)}{x \cdot x \cdot x \cdot x} + \frac{B - A}{x \cdot x \cdot x \cdot x \cdot x \cdot x \cdot x \cdot x}) = \frac{d}{dx}(\frac{0 - -1 \cdot (x + x)}{x \cdot x \cdot x \cdot x}) + \frac{d}{dx}(\frac{B - A}{x \cdot x \cdot x \cdot x \cdot x \cdot x \cdot x \cdot x})
\end{autobreak}
\end{align*}

\begin{center}
Вот так и рождаются легенды о герое, истребившем половину рода Боблина.
\end{center}
A = \begin{align*}
\begin{autobreak}
(0 - (x - x) \cdot (x + x)) \cdot ((x + x) \cdot x \cdot x + x \cdot x \cdot (x + x))
\end{autobreak}
\end{align*}

B = \begin{align*}
\begin{autobreak}
(0 - (x - x) \cdot 2) \cdot x \cdot x \cdot x \cdot x
\end{autobreak}
\end{align*}

\begin{align*}
\begin{autobreak}
\frac{d}{dx}(\frac{B - A}{x \cdot x \cdot x \cdot x \cdot x \cdot x \cdot x \cdot x}) = \frac{\frac{d}{dx}(B - A) \cdot x \cdot x \cdot x \cdot x \cdot x \cdot x \cdot x \cdot x - B - A \cdot \frac{d}{dx}(x \cdot x \cdot x \cdot x \cdot x \cdot x \cdot x \cdot x)}{{x \cdot x \cdot x \cdot x \cdot x \cdot x \cdot x \cdot x}^2}
\end{autobreak}
\end{align*}

\begin{center}
Небольшой взмах посохом — и план сражения выглядит куда приличнее.
\end{center}
\begin{align*}
\begin{autobreak}
\frac{d}{dx}(x \cdot x \cdot x \cdot x \cdot x \cdot x \cdot x \cdot x) = \frac{d}{dx}(x \cdot x \cdot x \cdot x) \cdot x \cdot x \cdot x \cdot x + x \cdot x \cdot x \cdot x \cdot \frac{d}{dx}(x \cdot x \cdot x \cdot x)
\end{autobreak}
\end{align*}

\begin{center}
Один из гоблинов упал
\end{center}
\begin{align*}
\begin{autobreak}
\frac{d}{dx}(x \cdot x \cdot x \cdot x) = \frac{d}{dx}(x \cdot x) \cdot x \cdot x + x \cdot x \cdot \frac{d}{dx}(x \cdot x)
\end{autobreak}
\end{align*}

\begin{center}
Родственники Боблина продолжают лезть к вам, держите посох крепче!
\end{center}
\begin{align*}
\begin{autobreak}
\frac{d}{dx}(x \cdot x) = \frac{d}{dx}(x) \cdot x + x \cdot \frac{d}{dx}(x)
\end{autobreak}
\end{align*}

\begin{center}
Ваше заклинание свернуло невестку Боблина в шарик
\end{center}
\begin{align*}
\begin{autobreak}
\frac{d}{dx}(x) = 1
\end{autobreak}
\end{align*}

\begin{center}
ААХХАХААХАХ Гоблин-Боблин
\end{center}
\begin{align*}
\begin{autobreak}
\frac{d}{dx}(x) = 1
\end{autobreak}
\end{align*}

\begin{center}
Битва продолжается, не теряйте духу, они когда-то, наверное, закончаться!
\end{center}
\begin{align*}
\begin{autobreak}
\frac{d}{dx}(x \cdot x) = \frac{d}{dx}(x) \cdot x + x \cdot \frac{d}{dx}(x)
\end{autobreak}
\end{align*}

\begin{center}
Ваше вошшебство откатило деверя Боблина до младенчества
\end{center}
\begin{align*}
\begin{autobreak}
\frac{d}{dx}(x) = 1
\end{autobreak}
\end{align*}

\begin{center}
Полиморф сработал отлично: зять Боблина теперь лягушка
\end{center}
\begin{align*}
\begin{autobreak}
\frac{d}{dx}(x) = 1
\end{autobreak}
\end{align*}

\begin{center}
Их не становиться меньше, откуда они только лезут?!
\end{center}
\begin{align*}
\begin{autobreak}
\frac{d}{dx}(x \cdot x \cdot x \cdot x) = \frac{d}{dx}(x \cdot x) \cdot x \cdot x + x \cdot x \cdot \frac{d}{dx}(x \cdot x)
\end{autobreak}
\end{align*}

\begin{center}
Вот так и рождаются легенды о герое, истребившем половину рода Боблина.
\end{center}
\begin{align*}
\begin{autobreak}
\frac{d}{dx}(x \cdot x) = \frac{d}{dx}(x) \cdot x + x \cdot \frac{d}{dx}(x)
\end{autobreak}
\end{align*}

\begin{center}
Поздравляю! От свояка Боблина осталась только полоыина
\end{center}
\begin{align*}
\begin{autobreak}
\frac{d}{dx}(x) = 1
\end{autobreak}
\end{align*}

\begin{center}
Ваше заклинание свернуло невестку Боблина в шарик
\end{center}
\begin{align*}
\begin{autobreak}
\frac{d}{dx}(x) = 1
\end{autobreak}
\end{align*}

\begin{center}
Небольшой взмах посохом — и план сражения выглядит куда приличнее.
\end{center}
\begin{align*}
\begin{autobreak}
\frac{d}{dx}(x \cdot x) = \frac{d}{dx}(x) \cdot x + x \cdot \frac{d}{dx}(x)
\end{autobreak}
\end{align*}

\begin{center}
ААХХАХААХАХ Гоблин-Боблин
\end{center}
\begin{align*}
\begin{autobreak}
\frac{d}{dx}(x) = 1
\end{autobreak}
\end{align*}

\begin{center}
Ваше вошшебство откатило деверя Боблина до младенчества
\end{center}
\begin{align*}
\begin{autobreak}
\frac{d}{dx}(x) = 1
\end{autobreak}
\end{align*}

\begin{center}
Один из гоблинов упал
\end{center}
A = \begin{align*}
\begin{autobreak}
(0 - (x - x) \cdot (x + x)) \cdot ((x + x) \cdot x \cdot x + x \cdot x \cdot (x + x))
\end{autobreak}
\end{align*}

B = \begin{align*}
\begin{autobreak}
(0 - (x - x) \cdot 2) \cdot x \cdot x \cdot x \cdot x
\end{autobreak}
\end{align*}

\begin{align*}
\begin{autobreak}
\frac{d}{dx}(B - A) = \frac{d}{dx}(B) - \frac{d}{dx}(A)
\end{autobreak}
\end{align*}

\begin{center}
Родственники Боблина продолжают лезть к вам, держите посох крепче!
\end{center}
A = \begin{align*}
\begin{autobreak}
(0 - (x - x) \cdot (x + x)) \cdot ((x + x) \cdot x \cdot x + x \cdot x \cdot (x + x))
\end{autobreak}
\end{align*}

\begin{align*}
\begin{autobreak}
\frac{d}{dx}(A) = \frac{d}{dx}(0 - (x - x) \cdot (x + x)) \cdot (x + x) \cdot x \cdot x + x \cdot x \cdot (x + x) + 0 - (x - x) \cdot (x + x) \cdot \frac{d}{dx}((x + x) \cdot x \cdot x + x \cdot x \cdot (x + x))
\end{autobreak}
\end{align*}

\begin{center}
Битва продолжается, не теряйте духу, они когда-то, наверное, закончаться!
\end{center}
\begin{align*}
\begin{autobreak}
\frac{d}{dx}((x + x) \cdot x \cdot x + x \cdot x \cdot (x + x)) = \frac{d}{dx}((x + x) \cdot x \cdot x) + \frac{d}{dx}(x \cdot x \cdot (x + x))
\end{autobreak}
\end{align*}

\begin{center}
Их не становиться меньше, откуда они только лезут?!
\end{center}
\begin{align*}
\begin{autobreak}
\frac{d}{dx}(x \cdot x \cdot (x + x)) = \frac{d}{dx}(x \cdot x) \cdot x + x + x \cdot x \cdot \frac{d}{dx}(x + x)
\end{autobreak}
\end{align*}

\begin{center}
Вот так и рождаются легенды о герое, истребившем половину рода Боблина.
\end{center}
\begin{align*}
\begin{autobreak}
\frac{d}{dx}(x + x) = \frac{d}{dx}(x) + \frac{d}{dx}(x)
\end{autobreak}
\end{align*}

\begin{center}
Полиморф сработал отлично: зять Боблина теперь лягушка
\end{center}
\begin{align*}
\begin{autobreak}
\frac{d}{dx}(x) = 1
\end{autobreak}
\end{align*}

\begin{center}
Поздравляю! От свояка Боблина осталась только полоыина
\end{center}
\begin{align*}
\begin{autobreak}
\frac{d}{dx}(x) = 1
\end{autobreak}
\end{align*}

\begin{center}
Небольшой взмах посохом — и план сражения выглядит куда приличнее.
\end{center}
\begin{align*}
\begin{autobreak}
\frac{d}{dx}(x \cdot x) = \frac{d}{dx}(x) \cdot x + x \cdot \frac{d}{dx}(x)
\end{autobreak}
\end{align*}

\begin{center}
Ваше заклинание свернуло невестку Боблина в шарик
\end{center}
\begin{align*}
\begin{autobreak}
\frac{d}{dx}(x) = 1
\end{autobreak}
\end{align*}

\begin{center}
ААХХАХААХАХ Гоблин-Боблин
\end{center}
\begin{align*}
\begin{autobreak}
\frac{d}{dx}(x) = 1
\end{autobreak}
\end{align*}

\begin{center}
Один из гоблинов упал
\end{center}
\begin{align*}
\begin{autobreak}
\frac{d}{dx}((x + x) \cdot x \cdot x) = \frac{d}{dx}(x + x) \cdot x \cdot x + x + x \cdot \frac{d}{dx}(x \cdot x)
\end{autobreak}
\end{align*}

\begin{center}
Родственники Боблина продолжают лезть к вам, держите посох крепче!
\end{center}
\begin{align*}
\begin{autobreak}
\frac{d}{dx}(x \cdot x) = \frac{d}{dx}(x) \cdot x + x \cdot \frac{d}{dx}(x)
\end{autobreak}
\end{align*}

\begin{center}
Ваше вошшебство откатило деверя Боблина до младенчества
\end{center}
\begin{align*}
\begin{autobreak}
\frac{d}{dx}(x) = 1
\end{autobreak}
\end{align*}

\begin{center}
Полиморф сработал отлично: зять Боблина теперь лягушка
\end{center}
\begin{align*}
\begin{autobreak}
\frac{d}{dx}(x) = 1
\end{autobreak}
\end{align*}

\begin{center}
Битва продолжается, не теряйте духу, они когда-то, наверное, закончаться!
\end{center}
\begin{align*}
\begin{autobreak}
\frac{d}{dx}(x + x) = \frac{d}{dx}(x) + \frac{d}{dx}(x)
\end{autobreak}
\end{align*}

\begin{center}
Поздравляю! От свояка Боблина осталась только полоыина
\end{center}
\begin{align*}
\begin{autobreak}
\frac{d}{dx}(x) = 1
\end{autobreak}
\end{align*}

\begin{center}
Ваше заклинание свернуло невестку Боблина в шарик
\end{center}
\begin{align*}
\begin{autobreak}
\frac{d}{dx}(x) = 1
\end{autobreak}
\end{align*}

\begin{center}
Их не становиться меньше, откуда они только лезут?!
\end{center}
\begin{align*}
\begin{autobreak}
\frac{d}{dx}(0 - (x - x) \cdot (x + x)) = \frac{d}{dx}(0) - \frac{d}{dx}((x - x) \cdot (x + x))
\end{autobreak}
\end{align*}

\begin{center}
Вот так и рождаются легенды о герое, истребившем половину рода Боблина.
\end{center}
\begin{align*}
\begin{autobreak}
\frac{d}{dx}((x - x) \cdot (x + x)) = \frac{d}{dx}(x - x) \cdot x + x + x - x \cdot \frac{d}{dx}(x + x)
\end{autobreak}
\end{align*}

\begin{center}
Небольшой взмах посохом — и план сражения выглядит куда приличнее.
\end{center}
\begin{align*}
\begin{autobreak}
\frac{d}{dx}(x + x) = \frac{d}{dx}(x) + \frac{d}{dx}(x)
\end{autobreak}
\end{align*}

\begin{center}
ААХХАХААХАХ Гоблин-Боблин
\end{center}
\begin{align*}
\begin{autobreak}
\frac{d}{dx}(x) = 1
\end{autobreak}
\end{align*}

\begin{center}
Ваше вошшебство откатило деверя Боблина до младенчества
\end{center}
\begin{align*}
\begin{autobreak}
\frac{d}{dx}(x) = 1
\end{autobreak}
\end{align*}

\begin{center}
Один из гоблинов упал
\end{center}
\begin{align*}
\begin{autobreak}
\frac{d}{dx}(x - x) = \frac{d}{dx}(x) - \frac{d}{dx}(x)
\end{autobreak}
\end{align*}

\begin{center}
Полиморф сработал отлично: зять Боблина теперь лягушка
\end{center}
\begin{align*}
\begin{autobreak}
\frac{d}{dx}(x) = 1
\end{autobreak}
\end{align*}

\begin{center}
Поздравляю! От свояка Боблина осталась только полоыина
\end{center}
\begin{align*}
\begin{autobreak}
\frac{d}{dx}(x) = 1
\end{autobreak}
\end{align*}

\begin{center}
Ваше колдовство низвело сестру Боблина до атомов
\end{center}
\begin{align*}
\begin{autobreak}
\frac{d}{dx}(0) = 0
\end{autobreak}
\end{align*}

\begin{center}
Родственники Боблина продолжают лезть к вам, держите посох крепче!
\end{center}
A = \begin{align*}
\begin{autobreak}
(0 - (x - x) \cdot 2) \cdot x \cdot x \cdot x \cdot x
\end{autobreak}
\end{align*}

\begin{align*}
\begin{autobreak}
\frac{d}{dx}(A) = \frac{d}{dx}(0 - (x - x) \cdot 2) \cdot x \cdot x \cdot x \cdot x + 0 - (x - x) \cdot 2 \cdot \frac{d}{dx}(x \cdot x \cdot x \cdot x)
\end{autobreak}
\end{align*}

\begin{center}
Битва продолжается, не теряйте духу, они когда-то, наверное, закончаться!
\end{center}
\begin{align*}
\begin{autobreak}
\frac{d}{dx}(x \cdot x \cdot x \cdot x) = \frac{d}{dx}(x \cdot x) \cdot x \cdot x + x \cdot x \cdot \frac{d}{dx}(x \cdot x)
\end{autobreak}
\end{align*}

\begin{center}
Их не становиться меньше, откуда они только лезут?!
\end{center}
\begin{align*}
\begin{autobreak}
\frac{d}{dx}(x \cdot x) = \frac{d}{dx}(x) \cdot x + x \cdot \frac{d}{dx}(x)
\end{autobreak}
\end{align*}

\begin{center}
Ваше заклинание свернуло невестку Боблина в шарик
\end{center}
\begin{align*}
\begin{autobreak}
\frac{d}{dx}(x) = 1
\end{autobreak}
\end{align*}

\begin{center}
ААХХАХААХАХ Гоблин-Боблин
\end{center}
\begin{align*}
\begin{autobreak}
\frac{d}{dx}(x) = 1
\end{autobreak}
\end{align*}

\begin{center}
Вот так и рождаются легенды о герое, истребившем половину рода Боблина.
\end{center}
\begin{align*}
\begin{autobreak}
\frac{d}{dx}(x \cdot x) = \frac{d}{dx}(x) \cdot x + x \cdot \frac{d}{dx}(x)
\end{autobreak}
\end{align*}

\begin{center}
Ваше вошшебство откатило деверя Боблина до младенчества
\end{center}
\begin{align*}
\begin{autobreak}
\frac{d}{dx}(x) = 1
\end{autobreak}
\end{align*}

\begin{center}
Полиморф сработал отлично: зять Боблина теперь лягушка
\end{center}
\begin{align*}
\begin{autobreak}
\frac{d}{dx}(x) = 1
\end{autobreak}
\end{align*}

\begin{center}
Небольшой взмах посохом — и план сражения выглядит куда приличнее.
\end{center}
\begin{align*}
\begin{autobreak}
\frac{d}{dx}(0 - (x - x) \cdot 2) = \frac{d}{dx}(0) - \frac{d}{dx}((x - x) \cdot 2)
\end{autobreak}
\end{align*}

\begin{center}
Один из гоблинов упал
\end{center}
\begin{align*}
\begin{autobreak}
\frac{d}{dx}((x - x) \cdot 2) = \frac{d}{dx}(x - x) \cdot 2 + x - x \cdot \frac{d}{dx}(2)
\end{autobreak}
\end{align*}

\begin{center}
Вы разложили дядю Боблина на молекулы
\end{center}
\begin{align*}
\begin{autobreak}
\frac{d}{dx}(2) = 0
\end{autobreak}
\end{align*}

\begin{center}
Родственники Боблина продолжают лезть к вам, держите посох крепче!
\end{center}
\begin{align*}
\begin{autobreak}
\frac{d}{dx}(x - x) = \frac{d}{dx}(x) - \frac{d}{dx}(x)
\end{autobreak}
\end{align*}

\begin{center}
Поздравляю! От свояка Боблина осталась только полоыина
\end{center}
\begin{align*}
\begin{autobreak}
\frac{d}{dx}(x) = 1
\end{autobreak}
\end{align*}

\begin{center}
Ваше заклинание свернуло невестку Боблина в шарик
\end{center}
\begin{align*}
\begin{autobreak}
\frac{d}{dx}(x) = 1
\end{autobreak}
\end{align*}

\begin{center}
Ваше волшебство оказалось не по зубам тёте Боблина, кстати, куда она делась?
\end{center}
\begin{align*}
\begin{autobreak}
\frac{d}{dx}(0) = 0
\end{autobreak}
\end{align*}

\begin{center}
Битва продолжается, не теряйте духу, они когда-то, наверное, закончаться!
\end{center}
\begin{align*}
\begin{autobreak}
\frac{d}{dx}(\frac{0 - -1 \cdot (x + x)}{x \cdot x \cdot x \cdot x}) = \frac{\frac{d}{dx}(0 - -1 \cdot (x + x)) \cdot x \cdot x \cdot x \cdot x - 0 - -1 \cdot (x + x) \cdot \frac{d}{dx}(x \cdot x \cdot x \cdot x)}{{x \cdot x \cdot x \cdot x}^2}
\end{autobreak}
\end{align*}

\begin{center}
Их не становиться меньше, откуда они только лезут?!
\end{center}
\begin{align*}
\begin{autobreak}
\frac{d}{dx}(x \cdot x \cdot x \cdot x) = \frac{d}{dx}(x \cdot x) \cdot x \cdot x + x \cdot x \cdot \frac{d}{dx}(x \cdot x)
\end{autobreak}
\end{align*}

\begin{center}
Вот так и рождаются легенды о герое, истребившем половину рода Боблина.
\end{center}
\begin{align*}
\begin{autobreak}
\frac{d}{dx}(x \cdot x) = \frac{d}{dx}(x) \cdot x + x \cdot \frac{d}{dx}(x)
\end{autobreak}
\end{align*}

\begin{center}
ААХХАХААХАХ Гоблин-Боблин
\end{center}
\begin{align*}
\begin{autobreak}
\frac{d}{dx}(x) = 1
\end{autobreak}
\end{align*}

\begin{center}
Ваше вошшебство откатило деверя Боблина до младенчества
\end{center}
\begin{align*}
\begin{autobreak}
\frac{d}{dx}(x) = 1
\end{autobreak}
\end{align*}

\begin{center}
Небольшой взмах посохом — и план сражения выглядит куда приличнее.
\end{center}
\begin{align*}
\begin{autobreak}
\frac{d}{dx}(x \cdot x) = \frac{d}{dx}(x) \cdot x + x \cdot \frac{d}{dx}(x)
\end{autobreak}
\end{align*}

\begin{center}
Полиморф сработал отлично: зять Боблина теперь лягушка
\end{center}
\begin{align*}
\begin{autobreak}
\frac{d}{dx}(x) = 1
\end{autobreak}
\end{align*}

\begin{center}
Поздравляю! От свояка Боблина осталась только полоыина
\end{center}
\begin{align*}
\begin{autobreak}
\frac{d}{dx}(x) = 1
\end{autobreak}
\end{align*}

\begin{center}
Один из гоблинов упал
\end{center}
\begin{align*}
\begin{autobreak}
\frac{d}{dx}(0 - -1 \cdot (x + x)) = \frac{d}{dx}(0) - \frac{d}{dx}(-1 \cdot (x + x))
\end{autobreak}
\end{align*}

\begin{center}
Родственники Боблина продолжают лезть к вам, держите посох крепче!
\end{center}
\begin{align*}
\begin{autobreak}
\frac{d}{dx}(-1 \cdot (x + x)) = \frac{d}{dx}(-1) \cdot x + x + -1 \cdot \frac{d}{dx}(x + x)
\end{autobreak}
\end{align*}

\begin{center}
Битва продолжается, не теряйте духу, они когда-то, наверное, закончаться!
\end{center}
\begin{align*}
\begin{autobreak}
\frac{d}{dx}(x + x) = \frac{d}{dx}(x) + \frac{d}{dx}(x)
\end{autobreak}
\end{align*}

\begin{center}
Ваше заклинание свернуло невестку Боблина в шарик
\end{center}
\begin{align*}
\begin{autobreak}
\frac{d}{dx}(x) = 1
\end{autobreak}
\end{align*}

\begin{center}
ААХХАХААХАХ Гоблин-Боблин
\end{center}
\begin{align*}
\begin{autobreak}
\frac{d}{dx}(x) = 1
\end{autobreak}
\end{align*}

\begin{center}
Огненный шар испарил бабушку Боблина
\end{center}
\begin{align*}
\begin{autobreak}
\frac{d}{dx}(-1) = 0
\end{autobreak}
\end{align*}

\begin{center}
Заклинание хаоса раскидало части племянника Боблина по разным планам
\end{center}
\begin{align*}
\begin{autobreak}
\frac{d}{dx}(0) = 0
\end{autobreak}
\end{align*}

\begin{center}
Их не становиться меньше, откуда они только лезут?!
\end{center}
\begin{align*}
\begin{autobreak}
\frac{d}{dx}(x^{x}) = x^{x} \cdot ( \frac{d}{dx}(x) \cdot \ln(x) + x \cdot \frac{d}{dx}(x) / x )
\end{autobreak}
\end{align*}

\begin{center}
Ваше вошшебство откатило деверя Боблина до младенчества
\end{center}
\begin{align*}
\begin{autobreak}
\frac{d}{dx}(x) = 1
\end{autobreak}
\end{align*}

\begin{center}
Полиморф сработал отлично: зять Боблина теперь лягушка
\end{center}
\begin{align*}
\begin{autobreak}
\frac{d}{dx}(x) = 1
\end{autobreak}
\end{align*}

\begin{center}
Вот так и рождаются легенды о герое, истребившем половину рода Боблина.
\end{center}
A = \begin{align*}
\begin{autobreak}
x^{x} \cdot (\ln(x) + \frac{x}{x}) \cdot (\frac{-1}{x \cdot x} + \frac{0 - (x - x) \cdot (x + x)}{x \cdot x \cdot x \cdot x})
\end{autobreak}
\end{align*}

\begin{align*}
\begin{autobreak}
\frac{d}{dx}(A) = \frac{d}{dx}(x^{x} \cdot (\ln(x) + \frac{x}{x})) \cdot \frac{-1}{x \cdot x} + \frac{0 - (x - x) \cdot (x + x)}{x \cdot x \cdot x \cdot x} + x^{x} \cdot (\ln(x) + \frac{x}{x}) \cdot \frac{d}{dx}(\frac{-1}{x \cdot x} + \frac{0 - (x - x) \cdot (x + x)}{x \cdot x \cdot x \cdot x})
\end{autobreak}
\end{align*}

\begin{center}
Небольшой взмах посохом — и план сражения выглядит куда приличнее.
\end{center}
\begin{align*}
\begin{autobreak}
\frac{d}{dx}(\frac{-1}{x \cdot x} + \frac{0 - (x - x) \cdot (x + x)}{x \cdot x \cdot x \cdot x}) = \frac{d}{dx}(\frac{-1}{x \cdot x}) + \frac{d}{dx}(\frac{0 - (x - x) \cdot (x + x)}{x \cdot x \cdot x \cdot x})
\end{autobreak}
\end{align*}

\begin{center}
Один из гоблинов упал
\end{center}
\begin{align*}
\begin{autobreak}
\frac{d}{dx}(\frac{0 - (x - x) \cdot (x + x)}{x \cdot x \cdot x \cdot x}) = \frac{\frac{d}{dx}(0 - (x - x) \cdot (x + x)) \cdot x \cdot x \cdot x \cdot x - 0 - (x - x) \cdot (x + x) \cdot \frac{d}{dx}(x \cdot x \cdot x \cdot x)}{{x \cdot x \cdot x \cdot x}^2}
\end{autobreak}
\end{align*}

\begin{center}
Родственники Боблина продолжают лезть к вам, держите посох крепче!
\end{center}
\begin{align*}
\begin{autobreak}
\frac{d}{dx}(x \cdot x \cdot x \cdot x) = \frac{d}{dx}(x \cdot x) \cdot x \cdot x + x \cdot x \cdot \frac{d}{dx}(x \cdot x)
\end{autobreak}
\end{align*}

\begin{center}
Битва продолжается, не теряйте духу, они когда-то, наверное, закончаться!
\end{center}
\begin{align*}
\begin{autobreak}
\frac{d}{dx}(x \cdot x) = \frac{d}{dx}(x) \cdot x + x \cdot \frac{d}{dx}(x)
\end{autobreak}
\end{align*}

\begin{center}
Поздравляю! От свояка Боблина осталась только полоыина
\end{center}
\begin{align*}
\begin{autobreak}
\frac{d}{dx}(x) = 1
\end{autobreak}
\end{align*}

\begin{center}
Ваше заклинание свернуло невестку Боблина в шарик
\end{center}
\begin{align*}
\begin{autobreak}
\frac{d}{dx}(x) = 1
\end{autobreak}
\end{align*}

\begin{center}
Их не становиться меньше, откуда они только лезут?!
\end{center}
\begin{align*}
\begin{autobreak}
\frac{d}{dx}(x \cdot x) = \frac{d}{dx}(x) \cdot x + x \cdot \frac{d}{dx}(x)
\end{autobreak}
\end{align*}

\begin{center}
ААХХАХААХАХ Гоблин-Боблин
\end{center}
\begin{align*}
\begin{autobreak}
\frac{d}{dx}(x) = 1
\end{autobreak}
\end{align*}

\begin{center}
Ваше вошшебство откатило деверя Боблина до младенчества
\end{center}
\begin{align*}
\begin{autobreak}
\frac{d}{dx}(x) = 1
\end{autobreak}
\end{align*}

\begin{center}
Вот так и рождаются легенды о герое, истребившем половину рода Боблина.
\end{center}
\begin{align*}
\begin{autobreak}
\frac{d}{dx}(0 - (x - x) \cdot (x + x)) = \frac{d}{dx}(0) - \frac{d}{dx}((x - x) \cdot (x + x))
\end{autobreak}
\end{align*}

\begin{center}
Небольшой взмах посохом — и план сражения выглядит куда приличнее.
\end{center}
\begin{align*}
\begin{autobreak}
\frac{d}{dx}((x - x) \cdot (x + x)) = \frac{d}{dx}(x - x) \cdot x + x + x - x \cdot \frac{d}{dx}(x + x)
\end{autobreak}
\end{align*}

\begin{center}
Один из гоблинов упал
\end{center}
\begin{align*}
\begin{autobreak}
\frac{d}{dx}(x + x) = \frac{d}{dx}(x) + \frac{d}{dx}(x)
\end{autobreak}
\end{align*}

\begin{center}
Полиморф сработал отлично: зять Боблина теперь лягушка
\end{center}
\begin{align*}
\begin{autobreak}
\frac{d}{dx}(x) = 1
\end{autobreak}
\end{align*}

\begin{center}
Поздравляю! От свояка Боблина осталась только полоыина
\end{center}
\begin{align*}
\begin{autobreak}
\frac{d}{dx}(x) = 1
\end{autobreak}
\end{align*}

\begin{center}
Родственники Боблина продолжают лезть к вам, держите посох крепче!
\end{center}
\begin{align*}
\begin{autobreak}
\frac{d}{dx}(x - x) = \frac{d}{dx}(x) - \frac{d}{dx}(x)
\end{autobreak}
\end{align*}

\begin{center}
Ваше заклинание свернуло невестку Боблина в шарик
\end{center}
\begin{align*}
\begin{autobreak}
\frac{d}{dx}(x) = 1
\end{autobreak}
\end{align*}

\begin{center}
ААХХАХААХАХ Гоблин-Боблин
\end{center}
\begin{align*}
\begin{autobreak}
\frac{d}{dx}(x) = 1
\end{autobreak}
\end{align*}

\begin{center}
Ваш портал небытия вежливо удалил тещу Боблина из этого измерения
\end{center}
\begin{align*}
\begin{autobreak}
\frac{d}{dx}(0) = 0
\end{autobreak}
\end{align*}

\begin{center}
Битва продолжается, не теряйте духу, они когда-то, наверное, закончаться!
\end{center}
\begin{align*}
\begin{autobreak}
\frac{d}{dx}(\frac{-1}{x \cdot x}) = \frac{\frac{d}{dx}(-1) \cdot x \cdot x - -1 \cdot \frac{d}{dx}(x \cdot x)}{{x \cdot x}^2}
\end{autobreak}
\end{align*}

\begin{center}
Их не становиться меньше, откуда они только лезут?!
\end{center}
\begin{align*}
\begin{autobreak}
\frac{d}{dx}(x \cdot x) = \frac{d}{dx}(x) \cdot x + x \cdot \frac{d}{dx}(x)
\end{autobreak}
\end{align*}

\begin{center}
Ваше вошшебство откатило деверя Боблина до младенчества
\end{center}
\begin{align*}
\begin{autobreak}
\frac{d}{dx}(x) = 1
\end{autobreak}
\end{align*}

\begin{center}
Полиморф сработал отлично: зять Боблина теперь лягушка
\end{center}
\begin{align*}
\begin{autobreak}
\frac{d}{dx}(x) = 1
\end{autobreak}
\end{align*}

\begin{center}
Ваше заклинание дезинтегрировало брата Боблина
\end{center}
\begin{align*}
\begin{autobreak}
\frac{d}{dx}(-1) = 0
\end{autobreak}
\end{align*}

\begin{center}
Вот так и рождаются легенды о герое, истребившем половину рода Боблина.
\end{center}
\begin{align*}
\begin{autobreak}
\frac{d}{dx}(x^{x} \cdot (\ln(x) + \frac{x}{x})) = \frac{d}{dx}(x^{x}) \cdot \ln(x) + \frac{x}{x} + x^{x} \cdot \frac{d}{dx}(\ln(x) + \frac{x}{x})
\end{autobreak}
\end{align*}

\begin{center}
Небольшой взмах посохом — и план сражения выглядит куда приличнее.
\end{center}
\begin{align*}
\begin{autobreak}
\frac{d}{dx}(\ln(x) + \frac{x}{x}) = \frac{d}{dx}(\ln(x)) + \frac{d}{dx}(\frac{x}{x})
\end{autobreak}
\end{align*}

\begin{center}
Один из гоблинов упал
\end{center}
\begin{align*}
\begin{autobreak}
\frac{d}{dx}(\frac{x}{x}) = \frac{\frac{d}{dx}(x) \cdot x - x \cdot \frac{d}{dx}(x)}{{x}^2}
\end{autobreak}
\end{align*}

\begin{center}
Поздравляю! От свояка Боблина осталась только полоыина
\end{center}
\begin{align*}
\begin{autobreak}
\frac{d}{dx}(x) = 1
\end{autobreak}
\end{align*}

\begin{center}
Ваше заклинание свернуло невестку Боблина в шарик
\end{center}
\begin{align*}
\begin{autobreak}
\frac{d}{dx}(x) = 1
\end{autobreak}
\end{align*}

\begin{center}
Родственники Боблина продолжают лезть к вам, держите посох крепче!
\end{center}
\begin{align*}
\begin{autobreak}
\frac{d}{dx}(\ln(x)) = \frac{\frac{d}{dx}(x)}{x}
\end{autobreak}
\end{align*}

\begin{center}
ААХХАХААХАХ Гоблин-Боблин
\end{center}
\begin{align*}
\begin{autobreak}
\frac{d}{dx}(x) = 1
\end{autobreak}
\end{align*}

\begin{center}
Битва продолжается, не теряйте духу, они когда-то, наверное, закончаться!
\end{center}
\begin{align*}
\begin{autobreak}
\frac{d}{dx}(x^{x}) = x^{x} \cdot ( \frac{d}{dx}(x) \cdot \ln(x) + x \cdot \frac{d}{dx}(x) / x )
\end{autobreak}
\end{align*}

\begin{center}
Ваше вошшебство откатило деверя Боблина до младенчества
\end{center}
\begin{align*}
\begin{autobreak}
\frac{d}{dx}(x) = 1
\end{autobreak}
\end{align*}

\begin{center}
Полиморф сработал отлично: зять Боблина теперь лягушка
\end{center}
\begin{align*}
\begin{autobreak}
\frac{d}{dx}(x) = 1
\end{autobreak}
\end{align*}

\begin{center}
Их не становиться меньше, откуда они только лезут?!
\end{center}
A = \begin{align*}
\begin{autobreak}
(x^{x} \cdot (\ln(x) + \frac{x}{x}) \cdot (\ln(x) + \frac{x}{x}) + x^{x} \cdot (\frac{1}{x} + \frac{x - x}{x \cdot x})) \cdot (\frac{1}{x} + \frac{x - x}{x \cdot x})
\end{autobreak}
\end{align*}

B = \begin{align*}
\begin{autobreak}
x^{x} \cdot (\ln(x) + \frac{x}{x}) \cdot (\frac{-1}{x \cdot x} + \frac{0 - (x - x) \cdot (x + x)}{x \cdot x \cdot x \cdot x})
\end{autobreak}
\end{align*}

\begin{align*}
\begin{autobreak}
\frac{d}{dx}(A + B) = \frac{d}{dx}(A) + \frac{d}{dx}(B)
\end{autobreak}
\end{align*}

\begin{center}
Вот так и рождаются легенды о герое, истребившем половину рода Боблина.
\end{center}
A = \begin{align*}
\begin{autobreak}
x^{x} \cdot (\ln(x) + \frac{x}{x}) \cdot (\frac{-1}{x \cdot x} + \frac{0 - (x - x) \cdot (x + x)}{x \cdot x \cdot x \cdot x})
\end{autobreak}
\end{align*}

\begin{align*}
\begin{autobreak}
\frac{d}{dx}(A) = \frac{d}{dx}(x^{x} \cdot (\ln(x) + \frac{x}{x})) \cdot \frac{-1}{x \cdot x} + \frac{0 - (x - x) \cdot (x + x)}{x \cdot x \cdot x \cdot x} + x^{x} \cdot (\ln(x) + \frac{x}{x}) \cdot \frac{d}{dx}(\frac{-1}{x \cdot x} + \frac{0 - (x - x) \cdot (x + x)}{x \cdot x \cdot x \cdot x})
\end{autobreak}
\end{align*}

\begin{center}
Небольшой взмах посохом — и план сражения выглядит куда приличнее.
\end{center}
\begin{align*}
\begin{autobreak}
\frac{d}{dx}(\frac{-1}{x \cdot x} + \frac{0 - (x - x) \cdot (x + x)}{x \cdot x \cdot x \cdot x}) = \frac{d}{dx}(\frac{-1}{x \cdot x}) + \frac{d}{dx}(\frac{0 - (x - x) \cdot (x + x)}{x \cdot x \cdot x \cdot x})
\end{autobreak}
\end{align*}

\begin{center}
Один из гоблинов упал
\end{center}
\begin{align*}
\begin{autobreak}
\frac{d}{dx}(\frac{0 - (x - x) \cdot (x + x)}{x \cdot x \cdot x \cdot x}) = \frac{\frac{d}{dx}(0 - (x - x) \cdot (x + x)) \cdot x \cdot x \cdot x \cdot x - 0 - (x - x) \cdot (x + x) \cdot \frac{d}{dx}(x \cdot x \cdot x \cdot x)}{{x \cdot x \cdot x \cdot x}^2}
\end{autobreak}
\end{align*}

\begin{center}
Родственники Боблина продолжают лезть к вам, держите посох крепче!
\end{center}
\begin{align*}
\begin{autobreak}
\frac{d}{dx}(x \cdot x \cdot x \cdot x) = \frac{d}{dx}(x \cdot x) \cdot x \cdot x + x \cdot x \cdot \frac{d}{dx}(x \cdot x)
\end{autobreak}
\end{align*}

\begin{center}
Битва продолжается, не теряйте духу, они когда-то, наверное, закончаться!
\end{center}
\begin{align*}
\begin{autobreak}
\frac{d}{dx}(x \cdot x) = \frac{d}{dx}(x) \cdot x + x \cdot \frac{d}{dx}(x)
\end{autobreak}
\end{align*}

\begin{center}
Поздравляю! От свояка Боблина осталась только полоыина
\end{center}
\begin{align*}
\begin{autobreak}
\frac{d}{dx}(x) = 1
\end{autobreak}
\end{align*}

\begin{center}
Ваше заклинание свернуло невестку Боблина в шарик
\end{center}
\begin{align*}
\begin{autobreak}
\frac{d}{dx}(x) = 1
\end{autobreak}
\end{align*}

\begin{center}
Их не становиться меньше, откуда они только лезут?!
\end{center}
\begin{align*}
\begin{autobreak}
\frac{d}{dx}(x \cdot x) = \frac{d}{dx}(x) \cdot x + x \cdot \frac{d}{dx}(x)
\end{autobreak}
\end{align*}

\begin{center}
ААХХАХААХАХ Гоблин-Боблин
\end{center}
\begin{align*}
\begin{autobreak}
\frac{d}{dx}(x) = 1
\end{autobreak}
\end{align*}

\begin{center}
Ваше вошшебство откатило деверя Боблина до младенчества
\end{center}
\begin{align*}
\begin{autobreak}
\frac{d}{dx}(x) = 1
\end{autobreak}
\end{align*}

\begin{center}
Вот так и рождаются легенды о герое, истребившем половину рода Боблина.
\end{center}
\begin{align*}
\begin{autobreak}
\frac{d}{dx}(0 - (x - x) \cdot (x + x)) = \frac{d}{dx}(0) - \frac{d}{dx}((x - x) \cdot (x + x))
\end{autobreak}
\end{align*}

\begin{center}
Небольшой взмах посохом — и план сражения выглядит куда приличнее.
\end{center}
\begin{align*}
\begin{autobreak}
\frac{d}{dx}((x - x) \cdot (x + x)) = \frac{d}{dx}(x - x) \cdot x + x + x - x \cdot \frac{d}{dx}(x + x)
\end{autobreak}
\end{align*}

\begin{center}
Один из гоблинов упал
\end{center}
\begin{align*}
\begin{autobreak}
\frac{d}{dx}(x + x) = \frac{d}{dx}(x) + \frac{d}{dx}(x)
\end{autobreak}
\end{align*}

\begin{center}
Полиморф сработал отлично: зять Боблина теперь лягушка
\end{center}
\begin{align*}
\begin{autobreak}
\frac{d}{dx}(x) = 1
\end{autobreak}
\end{align*}

\begin{center}
Поздравляю! От свояка Боблина осталась только полоыина
\end{center}
\begin{align*}
\begin{autobreak}
\frac{d}{dx}(x) = 1
\end{autobreak}
\end{align*}

\begin{center}
Родственники Боблина продолжают лезть к вам, держите посох крепче!
\end{center}
\begin{align*}
\begin{autobreak}
\frac{d}{dx}(x - x) = \frac{d}{dx}(x) - \frac{d}{dx}(x)
\end{autobreak}
\end{align*}

\begin{center}
Ваше заклинание свернуло невестку Боблина в шарик
\end{center}
\begin{align*}
\begin{autobreak}
\frac{d}{dx}(x) = 1
\end{autobreak}
\end{align*}

\begin{center}
ААХХАХААХАХ Гоблин-Боблин
\end{center}
\begin{align*}
\begin{autobreak}
\frac{d}{dx}(x) = 1
\end{autobreak}
\end{align*}

\begin{center}
Ваше колдовство низвело сестру Боблина до атомов
\end{center}
\begin{align*}
\begin{autobreak}
\frac{d}{dx}(0) = 0
\end{autobreak}
\end{align*}

\begin{center}
Битва продолжается, не теряйте духу, они когда-то, наверное, закончаться!
\end{center}
\begin{align*}
\begin{autobreak}
\frac{d}{dx}(\frac{-1}{x \cdot x}) = \frac{\frac{d}{dx}(-1) \cdot x \cdot x - -1 \cdot \frac{d}{dx}(x \cdot x)}{{x \cdot x}^2}
\end{autobreak}
\end{align*}

\begin{center}
Их не становиться меньше, откуда они только лезут?!
\end{center}
\begin{align*}
\begin{autobreak}
\frac{d}{dx}(x \cdot x) = \frac{d}{dx}(x) \cdot x + x \cdot \frac{d}{dx}(x)
\end{autobreak}
\end{align*}

\begin{center}
Ваше вошшебство откатило деверя Боблина до младенчества
\end{center}
\begin{align*}
\begin{autobreak}
\frac{d}{dx}(x) = 1
\end{autobreak}
\end{align*}

\begin{center}
Полиморф сработал отлично: зять Боблина теперь лягушка
\end{center}
\begin{align*}
\begin{autobreak}
\frac{d}{dx}(x) = 1
\end{autobreak}
\end{align*}

\begin{center}
Вы разложили дядю Боблина на молекулы
\end{center}
\begin{align*}
\begin{autobreak}
\frac{d}{dx}(-1) = 0
\end{autobreak}
\end{align*}

\begin{center}
Вот так и рождаются легенды о герое, истребившем половину рода Боблина.
\end{center}
\begin{align*}
\begin{autobreak}
\frac{d}{dx}(x^{x} \cdot (\ln(x) + \frac{x}{x})) = \frac{d}{dx}(x^{x}) \cdot \ln(x) + \frac{x}{x} + x^{x} \cdot \frac{d}{dx}(\ln(x) + \frac{x}{x})
\end{autobreak}
\end{align*}

\begin{center}
Небольшой взмах посохом — и план сражения выглядит куда приличнее.
\end{center}
\begin{align*}
\begin{autobreak}
\frac{d}{dx}(\ln(x) + \frac{x}{x}) = \frac{d}{dx}(\ln(x)) + \frac{d}{dx}(\frac{x}{x})
\end{autobreak}
\end{align*}

\begin{center}
Один из гоблинов упал
\end{center}
\begin{align*}
\begin{autobreak}
\frac{d}{dx}(\frac{x}{x}) = \frac{\frac{d}{dx}(x) \cdot x - x \cdot \frac{d}{dx}(x)}{{x}^2}
\end{autobreak}
\end{align*}

\begin{center}
Поздравляю! От свояка Боблина осталась только полоыина
\end{center}
\begin{align*}
\begin{autobreak}
\frac{d}{dx}(x) = 1
\end{autobreak}
\end{align*}

\begin{center}
Ваше заклинание свернуло невестку Боблина в шарик
\end{center}
\begin{align*}
\begin{autobreak}
\frac{d}{dx}(x) = 1
\end{autobreak}
\end{align*}

\begin{center}
Родственники Боблина продолжают лезть к вам, держите посох крепче!
\end{center}
\begin{align*}
\begin{autobreak}
\frac{d}{dx}(\ln(x)) = \frac{\frac{d}{dx}(x)}{x}
\end{autobreak}
\end{align*}

\begin{center}
ААХХАХААХАХ Гоблин-Боблин
\end{center}
\begin{align*}
\begin{autobreak}
\frac{d}{dx}(x) = 1
\end{autobreak}
\end{align*}

\begin{center}
Битва продолжается, не теряйте духу, они когда-то, наверное, закончаться!
\end{center}
\begin{align*}
\begin{autobreak}
\frac{d}{dx}(x^{x}) = x^{x} \cdot ( \frac{d}{dx}(x) \cdot \ln(x) + x \cdot \frac{d}{dx}(x) / x )
\end{autobreak}
\end{align*}

\begin{center}
Ваше вошшебство откатило деверя Боблина до младенчества
\end{center}
\begin{align*}
\begin{autobreak}
\frac{d}{dx}(x) = 1
\end{autobreak}
\end{align*}

\begin{center}
Полиморф сработал отлично: зять Боблина теперь лягушка
\end{center}
\begin{align*}
\begin{autobreak}
\frac{d}{dx}(x) = 1
\end{autobreak}
\end{align*}

\begin{center}
Их не становиться меньше, откуда они только лезут?!
\end{center}
A = \begin{align*}
\begin{autobreak}
(x^{x} \cdot (\ln(x) + \frac{x}{x}) \cdot (\ln(x) + \frac{x}{x}) + x^{x} \cdot (\frac{1}{x} + \frac{x - x}{x \cdot x})) \cdot (\frac{1}{x} + \frac{x - x}{x \cdot x})
\end{autobreak}
\end{align*}

\begin{align*}
\begin{autobreak}
\frac{d}{dx}(A) = \frac{d}{dx}(x^{x} \cdot (\ln(x) + \frac{x}{x}) \cdot (\ln(x) + \frac{x}{x}) + x^{x} \cdot (\frac{1}{x} + \frac{x - x}{x \cdot x})) \cdot \frac{1}{x} + \frac{x - x}{x \cdot x} + x^{x} \cdot (\ln(x) + \frac{x}{x}) \cdot (\ln(x) + \frac{x}{x}) + x^{x} \cdot (\frac{1}{x} + \frac{x - x}{x \cdot x}) \cdot \frac{d}{dx}(\frac{1}{x} + \frac{x - x}{x \cdot x})
\end{autobreak}
\end{align*}

\begin{center}
Вот так и рождаются легенды о герое, истребившем половину рода Боблина.
\end{center}
\begin{align*}
\begin{autobreak}
\frac{d}{dx}(\frac{1}{x} + \frac{x - x}{x \cdot x}) = \frac{d}{dx}(\frac{1}{x}) + \frac{d}{dx}(\frac{x - x}{x \cdot x})
\end{autobreak}
\end{align*}

\begin{center}
Небольшой взмах посохом — и план сражения выглядит куда приличнее.
\end{center}
\begin{align*}
\begin{autobreak}
\frac{d}{dx}(\frac{x - x}{x \cdot x}) = \frac{\frac{d}{dx}(x - x) \cdot x \cdot x - x - x \cdot \frac{d}{dx}(x \cdot x)}{{x \cdot x}^2}
\end{autobreak}
\end{align*}

\begin{center}
Один из гоблинов упал
\end{center}
\begin{align*}
\begin{autobreak}
\frac{d}{dx}(x \cdot x) = \frac{d}{dx}(x) \cdot x + x \cdot \frac{d}{dx}(x)
\end{autobreak}
\end{align*}

\begin{center}
Поздравляю! От свояка Боблина осталась только полоыина
\end{center}
\begin{align*}
\begin{autobreak}
\frac{d}{dx}(x) = 1
\end{autobreak}
\end{align*}

\begin{center}
Ваше заклинание свернуло невестку Боблина в шарик
\end{center}
\begin{align*}
\begin{autobreak}
\frac{d}{dx}(x) = 1
\end{autobreak}
\end{align*}

\begin{center}
Родственники Боблина продолжают лезть к вам, держите посох крепче!
\end{center}
\begin{align*}
\begin{autobreak}
\frac{d}{dx}(x - x) = \frac{d}{dx}(x) - \frac{d}{dx}(x)
\end{autobreak}
\end{align*}

\begin{center}
ААХХАХААХАХ Гоблин-Боблин
\end{center}
\begin{align*}
\begin{autobreak}
\frac{d}{dx}(x) = 1
\end{autobreak}
\end{align*}

\begin{center}
Ваше вошшебство откатило деверя Боблина до младенчества
\end{center}
\begin{align*}
\begin{autobreak}
\frac{d}{dx}(x) = 1
\end{autobreak}
\end{align*}

\begin{center}
Битва продолжается, не теряйте духу, они когда-то, наверное, закончаться!
\end{center}
\begin{align*}
\begin{autobreak}
\frac{d}{dx}(\frac{1}{x}) = \frac{\frac{d}{dx}(1) \cdot x - 1 \cdot \frac{d}{dx}(x)}{{x}^2}
\end{autobreak}
\end{align*}

\begin{center}
Полиморф сработал отлично: зять Боблина теперь лягушка
\end{center}
\begin{align*}
\begin{autobreak}
\frac{d}{dx}(x) = 1
\end{autobreak}
\end{align*}

\begin{center}
Ваше волшебство оказалось не по зубам тёте Боблина, кстати, куда она делась?
\end{center}
\begin{align*}
\begin{autobreak}
\frac{d}{dx}(1) = 0
\end{autobreak}
\end{align*}

\begin{center}
Их не становиться меньше, откуда они только лезут?!
\end{center}
A = \begin{align*}
\begin{autobreak}
x^{x} \cdot (\ln(x) + \frac{x}{x}) \cdot (\ln(x) + \frac{x}{x}) + x^{x} \cdot (\frac{1}{x} + \frac{x - x}{x \cdot x})
\end{autobreak}
\end{align*}

\begin{align*}
\begin{autobreak}
\frac{d}{dx}(A) = \frac{d}{dx}(x^{x} \cdot (\ln(x) + \frac{x}{x}) \cdot (\ln(x) + \frac{x}{x})) + \frac{d}{dx}(x^{x} \cdot (\frac{1}{x} + \frac{x - x}{x \cdot x}))
\end{autobreak}
\end{align*}

\begin{center}
Вот так и рождаются легенды о герое, истребившем половину рода Боблина.
\end{center}
\begin{align*}
\begin{autobreak}
\frac{d}{dx}(x^{x} \cdot (\frac{1}{x} + \frac{x - x}{x \cdot x})) = \frac{d}{dx}(x^{x}) \cdot \frac{1}{x} + \frac{x - x}{x \cdot x} + x^{x} \cdot \frac{d}{dx}(\frac{1}{x} + \frac{x - x}{x \cdot x})
\end{autobreak}
\end{align*}

\begin{center}
Небольшой взмах посохом — и план сражения выглядит куда приличнее.
\end{center}
\begin{align*}
\begin{autobreak}
\frac{d}{dx}(\frac{1}{x} + \frac{x - x}{x \cdot x}) = \frac{d}{dx}(\frac{1}{x}) + \frac{d}{dx}(\frac{x - x}{x \cdot x})
\end{autobreak}
\end{align*}

\begin{center}
Один из гоблинов упал
\end{center}
\begin{align*}
\begin{autobreak}
\frac{d}{dx}(\frac{x - x}{x \cdot x}) = \frac{\frac{d}{dx}(x - x) \cdot x \cdot x - x - x \cdot \frac{d}{dx}(x \cdot x)}{{x \cdot x}^2}
\end{autobreak}
\end{align*}

\begin{center}
Родственники Боблина продолжают лезть к вам, держите посох крепче!
\end{center}
\begin{align*}
\begin{autobreak}
\frac{d}{dx}(x \cdot x) = \frac{d}{dx}(x) \cdot x + x \cdot \frac{d}{dx}(x)
\end{autobreak}
\end{align*}

\begin{center}
Поздравляю! От свояка Боблина осталась только полоыина
\end{center}
\begin{align*}
\begin{autobreak}
\frac{d}{dx}(x) = 1
\end{autobreak}
\end{align*}

\begin{center}
Ваше заклинание свернуло невестку Боблина в шарик
\end{center}
\begin{align*}
\begin{autobreak}
\frac{d}{dx}(x) = 1
\end{autobreak}
\end{align*}

\begin{center}
Битва продолжается, не теряйте духу, они когда-то, наверное, закончаться!
\end{center}
\begin{align*}
\begin{autobreak}
\frac{d}{dx}(x - x) = \frac{d}{dx}(x) - \frac{d}{dx}(x)
\end{autobreak}
\end{align*}

\begin{center}
ААХХАХААХАХ Гоблин-Боблин
\end{center}
\begin{align*}
\begin{autobreak}
\frac{d}{dx}(x) = 1
\end{autobreak}
\end{align*}

\begin{center}
Ваше вошшебство откатило деверя Боблина до младенчества
\end{center}
\begin{align*}
\begin{autobreak}
\frac{d}{dx}(x) = 1
\end{autobreak}
\end{align*}

\begin{center}
Их не становиться меньше, откуда они только лезут?!
\end{center}
\begin{align*}
\begin{autobreak}
\frac{d}{dx}(\frac{1}{x}) = \frac{\frac{d}{dx}(1) \cdot x - 1 \cdot \frac{d}{dx}(x)}{{x}^2}
\end{autobreak}
\end{align*}

\begin{center}
Полиморф сработал отлично: зять Боблина теперь лягушка
\end{center}
\begin{align*}
\begin{autobreak}
\frac{d}{dx}(x) = 1
\end{autobreak}
\end{align*}

\begin{center}
Огненный шар испарил бабушку Боблина
\end{center}
\begin{align*}
\begin{autobreak}
\frac{d}{dx}(1) = 0
\end{autobreak}
\end{align*}

\begin{center}
Вот так и рождаются легенды о герое, истребившем половину рода Боблина.
\end{center}
\begin{align*}
\begin{autobreak}
\frac{d}{dx}(x^{x}) = x^{x} \cdot ( \frac{d}{dx}(x) \cdot \ln(x) + x \cdot \frac{d}{dx}(x) / x )
\end{autobreak}
\end{align*}

\begin{center}
Поздравляю! От свояка Боблина осталась только полоыина
\end{center}
\begin{align*}
\begin{autobreak}
\frac{d}{dx}(x) = 1
\end{autobreak}
\end{align*}

\begin{center}
Ваше заклинание свернуло невестку Боблина в шарик
\end{center}
\begin{align*}
\begin{autobreak}
\frac{d}{dx}(x) = 1
\end{autobreak}
\end{align*}

\begin{center}
Небольшой взмах посохом — и план сражения выглядит куда приличнее.
\end{center}
\begin{align*}
\begin{autobreak}
\frac{d}{dx}(x^{x} \cdot (\ln(x) + \frac{x}{x}) \cdot (\ln(x) + \frac{x}{x})) = \frac{d}{dx}(x^{x} \cdot (\ln(x) + \frac{x}{x})) \cdot \ln(x) + \frac{x}{x} + x^{x} \cdot (\ln(x) + \frac{x}{x}) \cdot \frac{d}{dx}(\ln(x) + \frac{x}{x})
\end{autobreak}
\end{align*}

\begin{center}
Один из гоблинов упал
\end{center}
\begin{align*}
\begin{autobreak}
\frac{d}{dx}(\ln(x) + \frac{x}{x}) = \frac{d}{dx}(\ln(x)) + \frac{d}{dx}(\frac{x}{x})
\end{autobreak}
\end{align*}

\begin{center}
Родственники Боблина продолжают лезть к вам, держите посох крепче!
\end{center}
\begin{align*}
\begin{autobreak}
\frac{d}{dx}(\frac{x}{x}) = \frac{\frac{d}{dx}(x) \cdot x - x \cdot \frac{d}{dx}(x)}{{x}^2}
\end{autobreak}
\end{align*}

\begin{center}
ААХХАХААХАХ Гоблин-Боблин
\end{center}
\begin{align*}
\begin{autobreak}
\frac{d}{dx}(x) = 1
\end{autobreak}
\end{align*}

\begin{center}
Ваше вошшебство откатило деверя Боблина до младенчества
\end{center}
\begin{align*}
\begin{autobreak}
\frac{d}{dx}(x) = 1
\end{autobreak}
\end{align*}

\begin{center}
Битва продолжается, не теряйте духу, они когда-то, наверное, закончаться!
\end{center}
\begin{align*}
\begin{autobreak}
\frac{d}{dx}(\ln(x)) = \frac{\frac{d}{dx}(x)}{x}
\end{autobreak}
\end{align*}

\begin{center}
Полиморф сработал отлично: зять Боблина теперь лягушка
\end{center}
\begin{align*}
\begin{autobreak}
\frac{d}{dx}(x) = 1
\end{autobreak}
\end{align*}

\begin{center}
Их не становиться меньше, откуда они только лезут?!
\end{center}
\begin{align*}
\begin{autobreak}
\frac{d}{dx}(x^{x} \cdot (\ln(x) + \frac{x}{x})) = \frac{d}{dx}(x^{x}) \cdot \ln(x) + \frac{x}{x} + x^{x} \cdot \frac{d}{dx}(\ln(x) + \frac{x}{x})
\end{autobreak}
\end{align*}

\begin{center}
Вот так и рождаются легенды о герое, истребившем половину рода Боблина.
\end{center}
\begin{align*}
\begin{autobreak}
\frac{d}{dx}(\ln(x) + \frac{x}{x}) = \frac{d}{dx}(\ln(x)) + \frac{d}{dx}(\frac{x}{x})
\end{autobreak}
\end{align*}

\begin{center}
Небольшой взмах посохом — и план сражения выглядит куда приличнее.
\end{center}
\begin{align*}
\begin{autobreak}
\frac{d}{dx}(\frac{x}{x}) = \frac{\frac{d}{dx}(x) \cdot x - x \cdot \frac{d}{dx}(x)}{{x}^2}
\end{autobreak}
\end{align*}

\begin{center}
Поздравляю! От свояка Боблина осталась только полоыина
\end{center}
\begin{align*}
\begin{autobreak}
\frac{d}{dx}(x) = 1
\end{autobreak}
\end{align*}

\begin{center}
Ваше заклинание свернуло невестку Боблина в шарик
\end{center}
\begin{align*}
\begin{autobreak}
\frac{d}{dx}(x) = 1
\end{autobreak}
\end{align*}

\begin{center}
Один из гоблинов упал
\end{center}
\begin{align*}
\begin{autobreak}
\frac{d}{dx}(\ln(x)) = \frac{\frac{d}{dx}(x)}{x}
\end{autobreak}
\end{align*}

\begin{center}
ААХХАХААХАХ Гоблин-Боблин
\end{center}
\begin{align*}
\begin{autobreak}
\frac{d}{dx}(x) = 1
\end{autobreak}
\end{align*}

\begin{center}
Родственники Боблина продолжают лезть к вам, держите посох крепче!
\end{center}
\begin{align*}
\begin{autobreak}
\frac{d}{dx}(x^{x}) = x^{x} \cdot ( \frac{d}{dx}(x) \cdot \ln(x) + x \cdot \frac{d}{dx}(x) / x )
\end{autobreak}
\end{align*}

\begin{center}
Ваше вошшебство откатило деверя Боблина до младенчества
\end{center}
\begin{align*}
\begin{autobreak}
\frac{d}{dx}(x) = 1
\end{autobreak}
\end{align*}

\begin{center}
Полиморф сработал отлично: зять Боблина теперь лягушка
\end{center}
\begin{align*}
\begin{autobreak}
\frac{d}{dx}(x) = 1
\end{autobreak}
\end{align*}

\begin{center}
Битва продолжается, не теряйте духу, они когда-то, наверное, закончаться!
\end{center}
A = \begin{align*}
\begin{autobreak}
x^{x} \cdot (\ln(x) + \frac{x}{x}) \cdot (\frac{1}{x} + \frac{x - x}{x \cdot x}) + x^{x} \cdot (\frac{-1}{x \cdot x} + \frac{0 - (x - x) \cdot (x + x)}{x \cdot x \cdot x \cdot x})
\end{autobreak}
\end{align*}

B = \begin{align*}
\begin{autobreak}
(x^{x} \cdot (\ln(x) + \frac{x}{x}) \cdot (\ln(x) + \frac{x}{x}) + x^{x} \cdot (\frac{1}{x} + \frac{x - x}{x \cdot x})) \cdot (\frac{1}{x} + \frac{x - x}{x \cdot x})
\end{autobreak}
\end{align*}

C = \begin{align*}
\begin{autobreak}
(x^{x} \cdot (\ln(x) + \frac{x}{x}) \cdot (\ln(x) + \frac{x}{x}) + x^{x} \cdot (\frac{1}{x} + \frac{x - x}{x \cdot x})) \cdot (\frac{1}{x} + \frac{x - x}{x \cdot x})
\end{autobreak}
\end{align*}

D = \begin{align*}
\begin{autobreak}
(x^{x} \cdot (\ln(x) + \frac{x}{x}) \cdot (\ln(x) + \frac{x}{x}) + x^{x} \cdot (\frac{1}{x} + \frac{x - x}{x \cdot x})) \cdot (\ln(x) + \frac{x}{x})
\end{autobreak}
\end{align*}

E = \begin{align*}
\begin{autobreak}
x^{x} \cdot (\ln(x) + \frac{x}{x}) \cdot (\frac{-1}{x \cdot x} + \frac{0 - (x - x) \cdot (x + x)}{x \cdot x \cdot x \cdot x})
\end{autobreak}
\end{align*}

F = \begin{align*}
\begin{autobreak}
x^{x} \cdot (\ln(x) + \frac{x}{x}) \cdot (\frac{1}{x} + \frac{x - x}{x \cdot x})
\end{autobreak}
\end{align*}

\begin{align*}
\begin{autobreak}
\frac{d}{dx}((D + F + A) \cdot (\ln(x) + \frac{x}{x}) + B + C + E) = \frac{d}{dx}((D + F + A) \cdot (\ln(x) + \frac{x}{x}) + B) + \frac{d}{dx}(C + E)
\end{autobreak}
\end{align*}

\begin{center}
Их не становиться меньше, откуда они только лезут?!
\end{center}
A = \begin{align*}
\begin{autobreak}
(x^{x} \cdot (\ln(x) + \frac{x}{x}) \cdot (\ln(x) + \frac{x}{x}) + x^{x} \cdot (\frac{1}{x} + \frac{x - x}{x \cdot x})) \cdot (\frac{1}{x} + \frac{x - x}{x \cdot x})
\end{autobreak}
\end{align*}

B = \begin{align*}
\begin{autobreak}
x^{x} \cdot (\ln(x) + \frac{x}{x}) \cdot (\frac{-1}{x \cdot x} + \frac{0 - (x - x) \cdot (x + x)}{x \cdot x \cdot x \cdot x})
\end{autobreak}
\end{align*}

\begin{align*}
\begin{autobreak}
\frac{d}{dx}(A + B) = \frac{d}{dx}(A) + \frac{d}{dx}(B)
\end{autobreak}
\end{align*}

\begin{center}
Вот так и рождаются легенды о герое, истребившем половину рода Боблина.
\end{center}
A = \begin{align*}
\begin{autobreak}
x^{x} \cdot (\ln(x) + \frac{x}{x}) \cdot (\frac{-1}{x \cdot x} + \frac{0 - (x - x) \cdot (x + x)}{x \cdot x \cdot x \cdot x})
\end{autobreak}
\end{align*}

\begin{align*}
\begin{autobreak}
\frac{d}{dx}(A) = \frac{d}{dx}(x^{x} \cdot (\ln(x) + \frac{x}{x})) \cdot \frac{-1}{x \cdot x} + \frac{0 - (x - x) \cdot (x + x)}{x \cdot x \cdot x \cdot x} + x^{x} \cdot (\ln(x) + \frac{x}{x}) \cdot \frac{d}{dx}(\frac{-1}{x \cdot x} + \frac{0 - (x - x) \cdot (x + x)}{x \cdot x \cdot x \cdot x})
\end{autobreak}
\end{align*}

\begin{center}
Небольшой взмах посохом — и план сражения выглядит куда приличнее.
\end{center}
\begin{align*}
\begin{autobreak}
\frac{d}{dx}(\frac{-1}{x \cdot x} + \frac{0 - (x - x) \cdot (x + x)}{x \cdot x \cdot x \cdot x}) = \frac{d}{dx}(\frac{-1}{x \cdot x}) + \frac{d}{dx}(\frac{0 - (x - x) \cdot (x + x)}{x \cdot x \cdot x \cdot x})
\end{autobreak}
\end{align*}

\begin{center}
Один из гоблинов упал
\end{center}
\begin{align*}
\begin{autobreak}
\frac{d}{dx}(\frac{0 - (x - x) \cdot (x + x)}{x \cdot x \cdot x \cdot x}) = \frac{\frac{d}{dx}(0 - (x - x) \cdot (x + x)) \cdot x \cdot x \cdot x \cdot x - 0 - (x - x) \cdot (x + x) \cdot \frac{d}{dx}(x \cdot x \cdot x \cdot x)}{{x \cdot x \cdot x \cdot x}^2}
\end{autobreak}
\end{align*}

\begin{center}
Родственники Боблина продолжают лезть к вам, держите посох крепче!
\end{center}
\begin{align*}
\begin{autobreak}
\frac{d}{dx}(x \cdot x \cdot x \cdot x) = \frac{d}{dx}(x \cdot x) \cdot x \cdot x + x \cdot x \cdot \frac{d}{dx}(x \cdot x)
\end{autobreak}
\end{align*}

\begin{center}
Битва продолжается, не теряйте духу, они когда-то, наверное, закончаться!
\end{center}
\begin{align*}
\begin{autobreak}
\frac{d}{dx}(x \cdot x) = \frac{d}{dx}(x) \cdot x + x \cdot \frac{d}{dx}(x)
\end{autobreak}
\end{align*}

\begin{center}
Поздравляю! От свояка Боблина осталась только полоыина
\end{center}
\begin{align*}
\begin{autobreak}
\frac{d}{dx}(x) = 1
\end{autobreak}
\end{align*}

\begin{center}
Ваше заклинание свернуло невестку Боблина в шарик
\end{center}
\begin{align*}
\begin{autobreak}
\frac{d}{dx}(x) = 1
\end{autobreak}
\end{align*}

\begin{center}
Их не становиться меньше, откуда они только лезут?!
\end{center}
\begin{align*}
\begin{autobreak}
\frac{d}{dx}(x \cdot x) = \frac{d}{dx}(x) \cdot x + x \cdot \frac{d}{dx}(x)
\end{autobreak}
\end{align*}

\begin{center}
ААХХАХААХАХ Гоблин-Боблин
\end{center}
\begin{align*}
\begin{autobreak}
\frac{d}{dx}(x) = 1
\end{autobreak}
\end{align*}

\begin{center}
Ваше вошшебство откатило деверя Боблина до младенчества
\end{center}
\begin{align*}
\begin{autobreak}
\frac{d}{dx}(x) = 1
\end{autobreak}
\end{align*}

\begin{center}
Вот так и рождаются легенды о герое, истребившем половину рода Боблина.
\end{center}
\begin{align*}
\begin{autobreak}
\frac{d}{dx}(0 - (x - x) \cdot (x + x)) = \frac{d}{dx}(0) - \frac{d}{dx}((x - x) \cdot (x + x))
\end{autobreak}
\end{align*}

\begin{center}
Небольшой взмах посохом — и план сражения выглядит куда приличнее.
\end{center}
\begin{align*}
\begin{autobreak}
\frac{d}{dx}((x - x) \cdot (x + x)) = \frac{d}{dx}(x - x) \cdot x + x + x - x \cdot \frac{d}{dx}(x + x)
\end{autobreak}
\end{align*}

\begin{center}
Один из гоблинов упал
\end{center}
\begin{align*}
\begin{autobreak}
\frac{d}{dx}(x + x) = \frac{d}{dx}(x) + \frac{d}{dx}(x)
\end{autobreak}
\end{align*}

\begin{center}
Полиморф сработал отлично: зять Боблина теперь лягушка
\end{center}
\begin{align*}
\begin{autobreak}
\frac{d}{dx}(x) = 1
\end{autobreak}
\end{align*}

\begin{center}
Поздравляю! От свояка Боблина осталась только полоыина
\end{center}
\begin{align*}
\begin{autobreak}
\frac{d}{dx}(x) = 1
\end{autobreak}
\end{align*}

\begin{center}
Родственники Боблина продолжают лезть к вам, держите посох крепче!
\end{center}
\begin{align*}
\begin{autobreak}
\frac{d}{dx}(x - x) = \frac{d}{dx}(x) - \frac{d}{dx}(x)
\end{autobreak}
\end{align*}

\begin{center}
Ваше заклинание свернуло невестку Боблина в шарик
\end{center}
\begin{align*}
\begin{autobreak}
\frac{d}{dx}(x) = 1
\end{autobreak}
\end{align*}

\begin{center}
ААХХАХААХАХ Гоблин-Боблин
\end{center}
\begin{align*}
\begin{autobreak}
\frac{d}{dx}(x) = 1
\end{autobreak}
\end{align*}

\begin{center}
Заклинание хаоса раскидало части племянника Боблина по разным планам
\end{center}
\begin{align*}
\begin{autobreak}
\frac{d}{dx}(0) = 0
\end{autobreak}
\end{align*}

\begin{center}
Битва продолжается, не теряйте духу, они когда-то, наверное, закончаться!
\end{center}
\begin{align*}
\begin{autobreak}
\frac{d}{dx}(\frac{-1}{x \cdot x}) = \frac{\frac{d}{dx}(-1) \cdot x \cdot x - -1 \cdot \frac{d}{dx}(x \cdot x)}{{x \cdot x}^2}
\end{autobreak}
\end{align*}

\begin{center}
Их не становиться меньше, откуда они только лезут?!
\end{center}
\begin{align*}
\begin{autobreak}
\frac{d}{dx}(x \cdot x) = \frac{d}{dx}(x) \cdot x + x \cdot \frac{d}{dx}(x)
\end{autobreak}
\end{align*}

\begin{center}
Ваше вошшебство откатило деверя Боблина до младенчества
\end{center}
\begin{align*}
\begin{autobreak}
\frac{d}{dx}(x) = 1
\end{autobreak}
\end{align*}

\begin{center}
Полиморф сработал отлично: зять Боблина теперь лягушка
\end{center}
\begin{align*}
\begin{autobreak}
\frac{d}{dx}(x) = 1
\end{autobreak}
\end{align*}

\begin{center}
Ваш портал небытия вежливо удалил тещу Боблина из этого измерения
\end{center}
\begin{align*}
\begin{autobreak}
\frac{d}{dx}(-1) = 0
\end{autobreak}
\end{align*}

\begin{center}
Вот так и рождаются легенды о герое, истребившем половину рода Боблина.
\end{center}
\begin{align*}
\begin{autobreak}
\frac{d}{dx}(x^{x} \cdot (\ln(x) + \frac{x}{x})) = \frac{d}{dx}(x^{x}) \cdot \ln(x) + \frac{x}{x} + x^{x} \cdot \frac{d}{dx}(\ln(x) + \frac{x}{x})
\end{autobreak}
\end{align*}

\begin{center}
Небольшой взмах посохом — и план сражения выглядит куда приличнее.
\end{center}
\begin{align*}
\begin{autobreak}
\frac{d}{dx}(\ln(x) + \frac{x}{x}) = \frac{d}{dx}(\ln(x)) + \frac{d}{dx}(\frac{x}{x})
\end{autobreak}
\end{align*}

\begin{center}
Один из гоблинов упал
\end{center}
\begin{align*}
\begin{autobreak}
\frac{d}{dx}(\frac{x}{x}) = \frac{\frac{d}{dx}(x) \cdot x - x \cdot \frac{d}{dx}(x)}{{x}^2}
\end{autobreak}
\end{align*}

\begin{center}
Поздравляю! От свояка Боблина осталась только полоыина
\end{center}
\begin{align*}
\begin{autobreak}
\frac{d}{dx}(x) = 1
\end{autobreak}
\end{align*}

\begin{center}
Ваше заклинание свернуло невестку Боблина в шарик
\end{center}
\begin{align*}
\begin{autobreak}
\frac{d}{dx}(x) = 1
\end{autobreak}
\end{align*}

\begin{center}
Родственники Боблина продолжают лезть к вам, держите посох крепче!
\end{center}
\begin{align*}
\begin{autobreak}
\frac{d}{dx}(\ln(x)) = \frac{\frac{d}{dx}(x)}{x}
\end{autobreak}
\end{align*}

\begin{center}
ААХХАХААХАХ Гоблин-Боблин
\end{center}
\begin{align*}
\begin{autobreak}
\frac{d}{dx}(x) = 1
\end{autobreak}
\end{align*}

\begin{center}
Битва продолжается, не теряйте духу, они когда-то, наверное, закончаться!
\end{center}
\begin{align*}
\begin{autobreak}
\frac{d}{dx}(x^{x}) = x^{x} \cdot ( \frac{d}{dx}(x) \cdot \ln(x) + x \cdot \frac{d}{dx}(x) / x )
\end{autobreak}
\end{align*}

\begin{center}
Ваше вошшебство откатило деверя Боблина до младенчества
\end{center}
\begin{align*}
\begin{autobreak}
\frac{d}{dx}(x) = 1
\end{autobreak}
\end{align*}

\begin{center}
Полиморф сработал отлично: зять Боблина теперь лягушка
\end{center}
\begin{align*}
\begin{autobreak}
\frac{d}{dx}(x) = 1
\end{autobreak}
\end{align*}

\begin{center}
Их не становиться меньше, откуда они только лезут?!
\end{center}
A = \begin{align*}
\begin{autobreak}
(x^{x} \cdot (\ln(x) + \frac{x}{x}) \cdot (\ln(x) + \frac{x}{x}) + x^{x} \cdot (\frac{1}{x} + \frac{x - x}{x \cdot x})) \cdot (\frac{1}{x} + \frac{x - x}{x \cdot x})
\end{autobreak}
\end{align*}

\begin{align*}
\begin{autobreak}
\frac{d}{dx}(A) = \frac{d}{dx}(x^{x} \cdot (\ln(x) + \frac{x}{x}) \cdot (\ln(x) + \frac{x}{x}) + x^{x} \cdot (\frac{1}{x} + \frac{x - x}{x \cdot x})) \cdot \frac{1}{x} + \frac{x - x}{x \cdot x} + x^{x} \cdot (\ln(x) + \frac{x}{x}) \cdot (\ln(x) + \frac{x}{x}) + x^{x} \cdot (\frac{1}{x} + \frac{x - x}{x \cdot x}) \cdot \frac{d}{dx}(\frac{1}{x} + \frac{x - x}{x \cdot x})
\end{autobreak}
\end{align*}

\begin{center}
Вот так и рождаются легенды о герое, истребившем половину рода Боблина.
\end{center}
\begin{align*}
\begin{autobreak}
\frac{d}{dx}(\frac{1}{x} + \frac{x - x}{x \cdot x}) = \frac{d}{dx}(\frac{1}{x}) + \frac{d}{dx}(\frac{x - x}{x \cdot x})
\end{autobreak}
\end{align*}

\begin{center}
Небольшой взмах посохом — и план сражения выглядит куда приличнее.
\end{center}
\begin{align*}
\begin{autobreak}
\frac{d}{dx}(\frac{x - x}{x \cdot x}) = \frac{\frac{d}{dx}(x - x) \cdot x \cdot x - x - x \cdot \frac{d}{dx}(x \cdot x)}{{x \cdot x}^2}
\end{autobreak}
\end{align*}

\begin{center}
Один из гоблинов упал
\end{center}
\begin{align*}
\begin{autobreak}
\frac{d}{dx}(x \cdot x) = \frac{d}{dx}(x) \cdot x + x \cdot \frac{d}{dx}(x)
\end{autobreak}
\end{align*}

\begin{center}
Поздравляю! От свояка Боблина осталась только полоыина
\end{center}
\begin{align*}
\begin{autobreak}
\frac{d}{dx}(x) = 1
\end{autobreak}
\end{align*}

\begin{center}
Ваше заклинание свернуло невестку Боблина в шарик
\end{center}
\begin{align*}
\begin{autobreak}
\frac{d}{dx}(x) = 1
\end{autobreak}
\end{align*}

\begin{center}
Родственники Боблина продолжают лезть к вам, держите посох крепче!
\end{center}
\begin{align*}
\begin{autobreak}
\frac{d}{dx}(x - x) = \frac{d}{dx}(x) - \frac{d}{dx}(x)
\end{autobreak}
\end{align*}

\begin{center}
ААХХАХААХАХ Гоблин-Боблин
\end{center}
\begin{align*}
\begin{autobreak}
\frac{d}{dx}(x) = 1
\end{autobreak}
\end{align*}

\begin{center}
Ваше вошшебство откатило деверя Боблина до младенчества
\end{center}
\begin{align*}
\begin{autobreak}
\frac{d}{dx}(x) = 1
\end{autobreak}
\end{align*}

\begin{center}
Битва продолжается, не теряйте духу, они когда-то, наверное, закончаться!
\end{center}
\begin{align*}
\begin{autobreak}
\frac{d}{dx}(\frac{1}{x}) = \frac{\frac{d}{dx}(1) \cdot x - 1 \cdot \frac{d}{dx}(x)}{{x}^2}
\end{autobreak}
\end{align*}

\begin{center}
Полиморф сработал отлично: зять Боблина теперь лягушка
\end{center}
\begin{align*}
\begin{autobreak}
\frac{d}{dx}(x) = 1
\end{autobreak}
\end{align*}

\begin{center}
Ваше заклинание дезинтегрировало брата Боблина
\end{center}
\begin{align*}
\begin{autobreak}
\frac{d}{dx}(1) = 0
\end{autobreak}
\end{align*}

\begin{center}
Их не становиться меньше, откуда они только лезут?!
\end{center}
A = \begin{align*}
\begin{autobreak}
x^{x} \cdot (\ln(x) + \frac{x}{x}) \cdot (\ln(x) + \frac{x}{x}) + x^{x} \cdot (\frac{1}{x} + \frac{x - x}{x \cdot x})
\end{autobreak}
\end{align*}

\begin{align*}
\begin{autobreak}
\frac{d}{dx}(A) = \frac{d}{dx}(x^{x} \cdot (\ln(x) + \frac{x}{x}) \cdot (\ln(x) + \frac{x}{x})) + \frac{d}{dx}(x^{x} \cdot (\frac{1}{x} + \frac{x - x}{x \cdot x}))
\end{autobreak}
\end{align*}

\begin{center}
Вот так и рождаются легенды о герое, истребившем половину рода Боблина.
\end{center}
\begin{align*}
\begin{autobreak}
\frac{d}{dx}(x^{x} \cdot (\frac{1}{x} + \frac{x - x}{x \cdot x})) = \frac{d}{dx}(x^{x}) \cdot \frac{1}{x} + \frac{x - x}{x \cdot x} + x^{x} \cdot \frac{d}{dx}(\frac{1}{x} + \frac{x - x}{x \cdot x})
\end{autobreak}
\end{align*}

\begin{center}
Небольшой взмах посохом — и план сражения выглядит куда приличнее.
\end{center}
\begin{align*}
\begin{autobreak}
\frac{d}{dx}(\frac{1}{x} + \frac{x - x}{x \cdot x}) = \frac{d}{dx}(\frac{1}{x}) + \frac{d}{dx}(\frac{x - x}{x \cdot x})
\end{autobreak}
\end{align*}

\begin{center}
Один из гоблинов упал
\end{center}
\begin{align*}
\begin{autobreak}
\frac{d}{dx}(\frac{x - x}{x \cdot x}) = \frac{\frac{d}{dx}(x - x) \cdot x \cdot x - x - x \cdot \frac{d}{dx}(x \cdot x)}{{x \cdot x}^2}
\end{autobreak}
\end{align*}

\begin{center}
Родственники Боблина продолжают лезть к вам, держите посох крепче!
\end{center}
\begin{align*}
\begin{autobreak}
\frac{d}{dx}(x \cdot x) = \frac{d}{dx}(x) \cdot x + x \cdot \frac{d}{dx}(x)
\end{autobreak}
\end{align*}

\begin{center}
Поздравляю! От свояка Боблина осталась только полоыина
\end{center}
\begin{align*}
\begin{autobreak}
\frac{d}{dx}(x) = 1
\end{autobreak}
\end{align*}

\begin{center}
Ваше заклинание свернуло невестку Боблина в шарик
\end{center}
\begin{align*}
\begin{autobreak}
\frac{d}{dx}(x) = 1
\end{autobreak}
\end{align*}

\begin{center}
Битва продолжается, не теряйте духу, они когда-то, наверное, закончаться!
\end{center}
\begin{align*}
\begin{autobreak}
\frac{d}{dx}(x - x) = \frac{d}{dx}(x) - \frac{d}{dx}(x)
\end{autobreak}
\end{align*}

\begin{center}
ААХХАХААХАХ Гоблин-Боблин
\end{center}
\begin{align*}
\begin{autobreak}
\frac{d}{dx}(x) = 1
\end{autobreak}
\end{align*}

\begin{center}
Ваше вошшебство откатило деверя Боблина до младенчества
\end{center}
\begin{align*}
\begin{autobreak}
\frac{d}{dx}(x) = 1
\end{autobreak}
\end{align*}

\begin{center}
Их не становиться меньше, откуда они только лезут?!
\end{center}
\begin{align*}
\begin{autobreak}
\frac{d}{dx}(\frac{1}{x}) = \frac{\frac{d}{dx}(1) \cdot x - 1 \cdot \frac{d}{dx}(x)}{{x}^2}
\end{autobreak}
\end{align*}

\begin{center}
Полиморф сработал отлично: зять Боблина теперь лягушка
\end{center}
\begin{align*}
\begin{autobreak}
\frac{d}{dx}(x) = 1
\end{autobreak}
\end{align*}

\begin{center}
Ваше колдовство низвело сестру Боблина до атомов
\end{center}
\begin{align*}
\begin{autobreak}
\frac{d}{dx}(1) = 0
\end{autobreak}
\end{align*}

\begin{center}
Вот так и рождаются легенды о герое, истребившем половину рода Боблина.
\end{center}
\begin{align*}
\begin{autobreak}
\frac{d}{dx}(x^{x}) = x^{x} \cdot ( \frac{d}{dx}(x) \cdot \ln(x) + x \cdot \frac{d}{dx}(x) / x )
\end{autobreak}
\end{align*}

\begin{center}
Поздравляю! От свояка Боблина осталась только полоыина
\end{center}
\begin{align*}
\begin{autobreak}
\frac{d}{dx}(x) = 1
\end{autobreak}
\end{align*}

\begin{center}
Ваше заклинание свернуло невестку Боблина в шарик
\end{center}
\begin{align*}
\begin{autobreak}
\frac{d}{dx}(x) = 1
\end{autobreak}
\end{align*}

\begin{center}
Небольшой взмах посохом — и план сражения выглядит куда приличнее.
\end{center}
\begin{align*}
\begin{autobreak}
\frac{d}{dx}(x^{x} \cdot (\ln(x) + \frac{x}{x}) \cdot (\ln(x) + \frac{x}{x})) = \frac{d}{dx}(x^{x} \cdot (\ln(x) + \frac{x}{x})) \cdot \ln(x) + \frac{x}{x} + x^{x} \cdot (\ln(x) + \frac{x}{x}) \cdot \frac{d}{dx}(\ln(x) + \frac{x}{x})
\end{autobreak}
\end{align*}

\begin{center}
Один из гоблинов упал
\end{center}
\begin{align*}
\begin{autobreak}
\frac{d}{dx}(\ln(x) + \frac{x}{x}) = \frac{d}{dx}(\ln(x)) + \frac{d}{dx}(\frac{x}{x})
\end{autobreak}
\end{align*}

\begin{center}
Родственники Боблина продолжают лезть к вам, держите посох крепче!
\end{center}
\begin{align*}
\begin{autobreak}
\frac{d}{dx}(\frac{x}{x}) = \frac{\frac{d}{dx}(x) \cdot x - x \cdot \frac{d}{dx}(x)}{{x}^2}
\end{autobreak}
\end{align*}

\begin{center}
ААХХАХААХАХ Гоблин-Боблин
\end{center}
\begin{align*}
\begin{autobreak}
\frac{d}{dx}(x) = 1
\end{autobreak}
\end{align*}

\begin{center}
Ваше вошшебство откатило деверя Боблина до младенчества
\end{center}
\begin{align*}
\begin{autobreak}
\frac{d}{dx}(x) = 1
\end{autobreak}
\end{align*}

\begin{center}
Битва продолжается, не теряйте духу, они когда-то, наверное, закончаться!
\end{center}
\begin{align*}
\begin{autobreak}
\frac{d}{dx}(\ln(x)) = \frac{\frac{d}{dx}(x)}{x}
\end{autobreak}
\end{align*}

\begin{center}
Полиморф сработал отлично: зять Боблина теперь лягушка
\end{center}
\begin{align*}
\begin{autobreak}
\frac{d}{dx}(x) = 1
\end{autobreak}
\end{align*}

\begin{center}
Их не становиться меньше, откуда они только лезут?!
\end{center}
\begin{align*}
\begin{autobreak}
\frac{d}{dx}(x^{x} \cdot (\ln(x) + \frac{x}{x})) = \frac{d}{dx}(x^{x}) \cdot \ln(x) + \frac{x}{x} + x^{x} \cdot \frac{d}{dx}(\ln(x) + \frac{x}{x})
\end{autobreak}
\end{align*}

\begin{center}
Вот так и рождаются легенды о герое, истребившем половину рода Боблина.
\end{center}
\begin{align*}
\begin{autobreak}
\frac{d}{dx}(\ln(x) + \frac{x}{x}) = \frac{d}{dx}(\ln(x)) + \frac{d}{dx}(\frac{x}{x})
\end{autobreak}
\end{align*}

\begin{center}
Небольшой взмах посохом — и план сражения выглядит куда приличнее.
\end{center}
\begin{align*}
\begin{autobreak}
\frac{d}{dx}(\frac{x}{x}) = \frac{\frac{d}{dx}(x) \cdot x - x \cdot \frac{d}{dx}(x)}{{x}^2}
\end{autobreak}
\end{align*}

\begin{center}
Поздравляю! От свояка Боблина осталась только полоыина
\end{center}
\begin{align*}
\begin{autobreak}
\frac{d}{dx}(x) = 1
\end{autobreak}
\end{align*}

\begin{center}
Ваше заклинание свернуло невестку Боблина в шарик
\end{center}
\begin{align*}
\begin{autobreak}
\frac{d}{dx}(x) = 1
\end{autobreak}
\end{align*}

\begin{center}
Один из гоблинов упал
\end{center}
\begin{align*}
\begin{autobreak}
\frac{d}{dx}(\ln(x)) = \frac{\frac{d}{dx}(x)}{x}
\end{autobreak}
\end{align*}

\begin{center}
ААХХАХААХАХ Гоблин-Боблин
\end{center}
\begin{align*}
\begin{autobreak}
\frac{d}{dx}(x) = 1
\end{autobreak}
\end{align*}

\begin{center}
Родственники Боблина продолжают лезть к вам, держите посох крепче!
\end{center}
\begin{align*}
\begin{autobreak}
\frac{d}{dx}(x^{x}) = x^{x} \cdot ( \frac{d}{dx}(x) \cdot \ln(x) + x \cdot \frac{d}{dx}(x) / x )
\end{autobreak}
\end{align*}

\begin{center}
Ваше вошшебство откатило деверя Боблина до младенчества
\end{center}
\begin{align*}
\begin{autobreak}
\frac{d}{dx}(x) = 1
\end{autobreak}
\end{align*}

\begin{center}
Полиморф сработал отлично: зять Боблина теперь лягушка
\end{center}
\begin{align*}
\begin{autobreak}
\frac{d}{dx}(x) = 1
\end{autobreak}
\end{align*}

\begin{center}
Битва продолжается, не теряйте духу, они когда-то, наверное, закончаться!
\end{center}
A = \begin{align*}
\begin{autobreak}
x^{x} \cdot (\ln(x) + \frac{x}{x}) \cdot (\frac{1}{x} + \frac{x - x}{x \cdot x}) + x^{x} \cdot (\frac{-1}{x \cdot x} + \frac{0 - (x - x) \cdot (x + x)}{x \cdot x \cdot x \cdot x})
\end{autobreak}
\end{align*}

B = \begin{align*}
\begin{autobreak}
(x^{x} \cdot (\ln(x) + \frac{x}{x}) \cdot (\ln(x) + \frac{x}{x}) + x^{x} \cdot (\frac{1}{x} + \frac{x - x}{x \cdot x})) \cdot (\frac{1}{x} + \frac{x - x}{x \cdot x})
\end{autobreak}
\end{align*}

C = \begin{align*}
\begin{autobreak}
(x^{x} \cdot (\ln(x) + \frac{x}{x}) \cdot (\ln(x) + \frac{x}{x}) + x^{x} \cdot (\frac{1}{x} + \frac{x - x}{x \cdot x})) \cdot (\ln(x) + \frac{x}{x})
\end{autobreak}
\end{align*}

D = \begin{align*}
\begin{autobreak}
x^{x} \cdot (\ln(x) + \frac{x}{x}) \cdot (\frac{1}{x} + \frac{x - x}{x \cdot x})
\end{autobreak}
\end{align*}

\begin{align*}
\begin{autobreak}
\frac{d}{dx}((C + D + A) \cdot (\ln(x) + \frac{x}{x}) + B) = \frac{d}{dx}((C + D + A) \cdot (\ln(x) + \frac{x}{x})) + \frac{d}{dx}(B)
\end{autobreak}
\end{align*}

\begin{center}
Их не становиться меньше, откуда они только лезут?!
\end{center}
A = \begin{align*}
\begin{autobreak}
(x^{x} \cdot (\ln(x) + \frac{x}{x}) \cdot (\ln(x) + \frac{x}{x}) + x^{x} \cdot (\frac{1}{x} + \frac{x - x}{x \cdot x})) \cdot (\frac{1}{x} + \frac{x - x}{x \cdot x})
\end{autobreak}
\end{align*}

\begin{align*}
\begin{autobreak}
\frac{d}{dx}(A) = \frac{d}{dx}(x^{x} \cdot (\ln(x) + \frac{x}{x}) \cdot (\ln(x) + \frac{x}{x}) + x^{x} \cdot (\frac{1}{x} + \frac{x - x}{x \cdot x})) \cdot \frac{1}{x} + \frac{x - x}{x \cdot x} + x^{x} \cdot (\ln(x) + \frac{x}{x}) \cdot (\ln(x) + \frac{x}{x}) + x^{x} \cdot (\frac{1}{x} + \frac{x - x}{x \cdot x}) \cdot \frac{d}{dx}(\frac{1}{x} + \frac{x - x}{x \cdot x})
\end{autobreak}
\end{align*}

\begin{center}
Вот так и рождаются легенды о герое, истребившем половину рода Боблина.
\end{center}
\begin{align*}
\begin{autobreak}
\frac{d}{dx}(\frac{1}{x} + \frac{x - x}{x \cdot x}) = \frac{d}{dx}(\frac{1}{x}) + \frac{d}{dx}(\frac{x - x}{x \cdot x})
\end{autobreak}
\end{align*}

\begin{center}
Небольшой взмах посохом — и план сражения выглядит куда приличнее.
\end{center}
\begin{align*}
\begin{autobreak}
\frac{d}{dx}(\frac{x - x}{x \cdot x}) = \frac{\frac{d}{dx}(x - x) \cdot x \cdot x - x - x \cdot \frac{d}{dx}(x \cdot x)}{{x \cdot x}^2}
\end{autobreak}
\end{align*}

\begin{center}
Один из гоблинов упал
\end{center}
\begin{align*}
\begin{autobreak}
\frac{d}{dx}(x \cdot x) = \frac{d}{dx}(x) \cdot x + x \cdot \frac{d}{dx}(x)
\end{autobreak}
\end{align*}

\begin{center}
Поздравляю! От свояка Боблина осталась только полоыина
\end{center}
\begin{align*}
\begin{autobreak}
\frac{d}{dx}(x) = 1
\end{autobreak}
\end{align*}

\begin{center}
Ваше заклинание свернуло невестку Боблина в шарик
\end{center}
\begin{align*}
\begin{autobreak}
\frac{d}{dx}(x) = 1
\end{autobreak}
\end{align*}

\begin{center}
Родственники Боблина продолжают лезть к вам, держите посох крепче!
\end{center}
\begin{align*}
\begin{autobreak}
\frac{d}{dx}(x - x) = \frac{d}{dx}(x) - \frac{d}{dx}(x)
\end{autobreak}
\end{align*}

\begin{center}
ААХХАХААХАХ Гоблин-Боблин
\end{center}
\begin{align*}
\begin{autobreak}
\frac{d}{dx}(x) = 1
\end{autobreak}
\end{align*}

\begin{center}
Ваше вошшебство откатило деверя Боблина до младенчества
\end{center}
\begin{align*}
\begin{autobreak}
\frac{d}{dx}(x) = 1
\end{autobreak}
\end{align*}

\begin{center}
Битва продолжается, не теряйте духу, они когда-то, наверное, закончаться!
\end{center}
\begin{align*}
\begin{autobreak}
\frac{d}{dx}(\frac{1}{x}) = \frac{\frac{d}{dx}(1) \cdot x - 1 \cdot \frac{d}{dx}(x)}{{x}^2}
\end{autobreak}
\end{align*}

\begin{center}
Полиморф сработал отлично: зять Боблина теперь лягушка
\end{center}
\begin{align*}
\begin{autobreak}
\frac{d}{dx}(x) = 1
\end{autobreak}
\end{align*}

\begin{center}
Вы разложили дядю Боблина на молекулы
\end{center}
\begin{align*}
\begin{autobreak}
\frac{d}{dx}(1) = 0
\end{autobreak}
\end{align*}

\begin{center}
Их не становиться меньше, откуда они только лезут?!
\end{center}
A = \begin{align*}
\begin{autobreak}
x^{x} \cdot (\ln(x) + \frac{x}{x}) \cdot (\ln(x) + \frac{x}{x}) + x^{x} \cdot (\frac{1}{x} + \frac{x - x}{x \cdot x})
\end{autobreak}
\end{align*}

\begin{align*}
\begin{autobreak}
\frac{d}{dx}(A) = \frac{d}{dx}(x^{x} \cdot (\ln(x) + \frac{x}{x}) \cdot (\ln(x) + \frac{x}{x})) + \frac{d}{dx}(x^{x} \cdot (\frac{1}{x} + \frac{x - x}{x \cdot x}))
\end{autobreak}
\end{align*}

\begin{center}
Вот так и рождаются легенды о герое, истребившем половину рода Боблина.
\end{center}
\begin{align*}
\begin{autobreak}
\frac{d}{dx}(x^{x} \cdot (\frac{1}{x} + \frac{x - x}{x \cdot x})) = \frac{d}{dx}(x^{x}) \cdot \frac{1}{x} + \frac{x - x}{x \cdot x} + x^{x} \cdot \frac{d}{dx}(\frac{1}{x} + \frac{x - x}{x \cdot x})
\end{autobreak}
\end{align*}

\begin{center}
Небольшой взмах посохом — и план сражения выглядит куда приличнее.
\end{center}
\begin{align*}
\begin{autobreak}
\frac{d}{dx}(\frac{1}{x} + \frac{x - x}{x \cdot x}) = \frac{d}{dx}(\frac{1}{x}) + \frac{d}{dx}(\frac{x - x}{x \cdot x})
\end{autobreak}
\end{align*}

\begin{center}
Один из гоблинов упал
\end{center}
\begin{align*}
\begin{autobreak}
\frac{d}{dx}(\frac{x - x}{x \cdot x}) = \frac{\frac{d}{dx}(x - x) \cdot x \cdot x - x - x \cdot \frac{d}{dx}(x \cdot x)}{{x \cdot x}^2}
\end{autobreak}
\end{align*}

\begin{center}
Родственники Боблина продолжают лезть к вам, держите посох крепче!
\end{center}
\begin{align*}
\begin{autobreak}
\frac{d}{dx}(x \cdot x) = \frac{d}{dx}(x) \cdot x + x \cdot \frac{d}{dx}(x)
\end{autobreak}
\end{align*}

\begin{center}
Поздравляю! От свояка Боблина осталась только полоыина
\end{center}
\begin{align*}
\begin{autobreak}
\frac{d}{dx}(x) = 1
\end{autobreak}
\end{align*}

\begin{center}
Ваше заклинание свернуло невестку Боблина в шарик
\end{center}
\begin{align*}
\begin{autobreak}
\frac{d}{dx}(x) = 1
\end{autobreak}
\end{align*}

\begin{center}
Битва продолжается, не теряйте духу, они когда-то, наверное, закончаться!
\end{center}
\begin{align*}
\begin{autobreak}
\frac{d}{dx}(x - x) = \frac{d}{dx}(x) - \frac{d}{dx}(x)
\end{autobreak}
\end{align*}

\begin{center}
ААХХАХААХАХ Гоблин-Боблин
\end{center}
\begin{align*}
\begin{autobreak}
\frac{d}{dx}(x) = 1
\end{autobreak}
\end{align*}

\begin{center}
Ваше вошшебство откатило деверя Боблина до младенчества
\end{center}
\begin{align*}
\begin{autobreak}
\frac{d}{dx}(x) = 1
\end{autobreak}
\end{align*}

\begin{center}
Их не становиться меньше, откуда они только лезут?!
\end{center}
\begin{align*}
\begin{autobreak}
\frac{d}{dx}(\frac{1}{x}) = \frac{\frac{d}{dx}(1) \cdot x - 1 \cdot \frac{d}{dx}(x)}{{x}^2}
\end{autobreak}
\end{align*}

\begin{center}
Полиморф сработал отлично: зять Боблина теперь лягушка
\end{center}
\begin{align*}
\begin{autobreak}
\frac{d}{dx}(x) = 1
\end{autobreak}
\end{align*}

\begin{center}
Ваше волшебство оказалось не по зубам тёте Боблина, кстати, куда она делась?
\end{center}
\begin{align*}
\begin{autobreak}
\frac{d}{dx}(1) = 0
\end{autobreak}
\end{align*}

\begin{center}
Вот так и рождаются легенды о герое, истребившем половину рода Боблина.
\end{center}
\begin{align*}
\begin{autobreak}
\frac{d}{dx}(x^{x}) = x^{x} \cdot ( \frac{d}{dx}(x) \cdot \ln(x) + x \cdot \frac{d}{dx}(x) / x )
\end{autobreak}
\end{align*}

\begin{center}
Поздравляю! От свояка Боблина осталась только полоыина
\end{center}
\begin{align*}
\begin{autobreak}
\frac{d}{dx}(x) = 1
\end{autobreak}
\end{align*}

\begin{center}
Ваше заклинание свернуло невестку Боблина в шарик
\end{center}
\begin{align*}
\begin{autobreak}
\frac{d}{dx}(x) = 1
\end{autobreak}
\end{align*}

\begin{center}
Небольшой взмах посохом — и план сражения выглядит куда приличнее.
\end{center}
\begin{align*}
\begin{autobreak}
\frac{d}{dx}(x^{x} \cdot (\ln(x) + \frac{x}{x}) \cdot (\ln(x) + \frac{x}{x})) = \frac{d}{dx}(x^{x} \cdot (\ln(x) + \frac{x}{x})) \cdot \ln(x) + \frac{x}{x} + x^{x} \cdot (\ln(x) + \frac{x}{x}) \cdot \frac{d}{dx}(\ln(x) + \frac{x}{x})
\end{autobreak}
\end{align*}

\begin{center}
Один из гоблинов упал
\end{center}
\begin{align*}
\begin{autobreak}
\frac{d}{dx}(\ln(x) + \frac{x}{x}) = \frac{d}{dx}(\ln(x)) + \frac{d}{dx}(\frac{x}{x})
\end{autobreak}
\end{align*}

\begin{center}
Родственники Боблина продолжают лезть к вам, держите посох крепче!
\end{center}
\begin{align*}
\begin{autobreak}
\frac{d}{dx}(\frac{x}{x}) = \frac{\frac{d}{dx}(x) \cdot x - x \cdot \frac{d}{dx}(x)}{{x}^2}
\end{autobreak}
\end{align*}

\begin{center}
ААХХАХААХАХ Гоблин-Боблин
\end{center}
\begin{align*}
\begin{autobreak}
\frac{d}{dx}(x) = 1
\end{autobreak}
\end{align*}

\begin{center}
Ваше вошшебство откатило деверя Боблина до младенчества
\end{center}
\begin{align*}
\begin{autobreak}
\frac{d}{dx}(x) = 1
\end{autobreak}
\end{align*}

\begin{center}
Битва продолжается, не теряйте духу, они когда-то, наверное, закончаться!
\end{center}
\begin{align*}
\begin{autobreak}
\frac{d}{dx}(\ln(x)) = \frac{\frac{d}{dx}(x)}{x}
\end{autobreak}
\end{align*}

\begin{center}
Полиморф сработал отлично: зять Боблина теперь лягушка
\end{center}
\begin{align*}
\begin{autobreak}
\frac{d}{dx}(x) = 1
\end{autobreak}
\end{align*}

\begin{center}
Их не становиться меньше, откуда они только лезут?!
\end{center}
\begin{align*}
\begin{autobreak}
\frac{d}{dx}(x^{x} \cdot (\ln(x) + \frac{x}{x})) = \frac{d}{dx}(x^{x}) \cdot \ln(x) + \frac{x}{x} + x^{x} \cdot \frac{d}{dx}(\ln(x) + \frac{x}{x})
\end{autobreak}
\end{align*}

\begin{center}
Вот так и рождаются легенды о герое, истребившем половину рода Боблина.
\end{center}
\begin{align*}
\begin{autobreak}
\frac{d}{dx}(\ln(x) + \frac{x}{x}) = \frac{d}{dx}(\ln(x)) + \frac{d}{dx}(\frac{x}{x})
\end{autobreak}
\end{align*}

\begin{center}
Небольшой взмах посохом — и план сражения выглядит куда приличнее.
\end{center}
\begin{align*}
\begin{autobreak}
\frac{d}{dx}(\frac{x}{x}) = \frac{\frac{d}{dx}(x) \cdot x - x \cdot \frac{d}{dx}(x)}{{x}^2}
\end{autobreak}
\end{align*}

\begin{center}
Поздравляю! От свояка Боблина осталась только полоыина
\end{center}
\begin{align*}
\begin{autobreak}
\frac{d}{dx}(x) = 1
\end{autobreak}
\end{align*}

\begin{center}
Ваше заклинание свернуло невестку Боблина в шарик
\end{center}
\begin{align*}
\begin{autobreak}
\frac{d}{dx}(x) = 1
\end{autobreak}
\end{align*}

\begin{center}
Один из гоблинов упал
\end{center}
\begin{align*}
\begin{autobreak}
\frac{d}{dx}(\ln(x)) = \frac{\frac{d}{dx}(x)}{x}
\end{autobreak}
\end{align*}

\begin{center}
ААХХАХААХАХ Гоблин-Боблин
\end{center}
\begin{align*}
\begin{autobreak}
\frac{d}{dx}(x) = 1
\end{autobreak}
\end{align*}

\begin{center}
Родственники Боблина продолжают лезть к вам, держите посох крепче!
\end{center}
\begin{align*}
\begin{autobreak}
\frac{d}{dx}(x^{x}) = x^{x} \cdot ( \frac{d}{dx}(x) \cdot \ln(x) + x \cdot \frac{d}{dx}(x) / x )
\end{autobreak}
\end{align*}

\begin{center}
Ваше вошшебство откатило деверя Боблина до младенчества
\end{center}
\begin{align*}
\begin{autobreak}
\frac{d}{dx}(x) = 1
\end{autobreak}
\end{align*}

\begin{center}
Полиморф сработал отлично: зять Боблина теперь лягушка
\end{center}
\begin{align*}
\begin{autobreak}
\frac{d}{dx}(x) = 1
\end{autobreak}
\end{align*}

\begin{center}
Битва продолжается, не теряйте духу, они когда-то, наверное, закончаться!
\end{center}
A = \begin{align*}
\begin{autobreak}
x^{x} \cdot (\ln(x) + \frac{x}{x}) \cdot (\frac{1}{x} + \frac{x - x}{x \cdot x}) + x^{x} \cdot (\frac{-1}{x \cdot x} + \frac{0 - (x - x) \cdot (x + x)}{x \cdot x \cdot x \cdot x})
\end{autobreak}
\end{align*}

B = \begin{align*}
\begin{autobreak}
(x^{x} \cdot (\ln(x) + \frac{x}{x}) \cdot (\ln(x) + \frac{x}{x}) + x^{x} \cdot (\frac{1}{x} + \frac{x - x}{x \cdot x})) \cdot (\ln(x) + \frac{x}{x})
\end{autobreak}
\end{align*}

C = \begin{align*}
\begin{autobreak}
x^{x} \cdot (\ln(x) + \frac{x}{x}) \cdot (\frac{1}{x} + \frac{x - x}{x \cdot x})
\end{autobreak}
\end{align*}

\begin{align*}
\begin{autobreak}
\frac{d}{dx}((B + C + A) \cdot (\ln(x) + \frac{x}{x})) = \frac{d}{dx}(B + C + A) \cdot \ln(x) + \frac{x}{x} + B + C + A \cdot \frac{d}{dx}(\ln(x) + \frac{x}{x})
\end{autobreak}
\end{align*}

\begin{center}
Их не становиться меньше, откуда они только лезут?!
\end{center}
\begin{align*}
\begin{autobreak}
\frac{d}{dx}(\ln(x) + \frac{x}{x}) = \frac{d}{dx}(\ln(x)) + \frac{d}{dx}(\frac{x}{x})
\end{autobreak}
\end{align*}

\begin{center}
Вот так и рождаются легенды о герое, истребившем половину рода Боблина.
\end{center}
\begin{align*}
\begin{autobreak}
\frac{d}{dx}(\frac{x}{x}) = \frac{\frac{d}{dx}(x) \cdot x - x \cdot \frac{d}{dx}(x)}{{x}^2}
\end{autobreak}
\end{align*}

\begin{center}
Поздравляю! От свояка Боблина осталась только полоыина
\end{center}
\begin{align*}
\begin{autobreak}
\frac{d}{dx}(x) = 1
\end{autobreak}
\end{align*}

\begin{center}
Ваше заклинание свернуло невестку Боблина в шарик
\end{center}
\begin{align*}
\begin{autobreak}
\frac{d}{dx}(x) = 1
\end{autobreak}
\end{align*}

\begin{center}
Небольшой взмах посохом — и план сражения выглядит куда приличнее.
\end{center}
\begin{align*}
\begin{autobreak}
\frac{d}{dx}(\ln(x)) = \frac{\frac{d}{dx}(x)}{x}
\end{autobreak}
\end{align*}

\begin{center}
ААХХАХААХАХ Гоблин-Боблин
\end{center}
\begin{align*}
\begin{autobreak}
\frac{d}{dx}(x) = 1
\end{autobreak}
\end{align*}

\begin{center}
Один из гоблинов упал
\end{center}
A = \begin{align*}
\begin{autobreak}
x^{x} \cdot (\ln(x) + \frac{x}{x}) \cdot (\frac{1}{x} + \frac{x - x}{x \cdot x}) + x^{x} \cdot (\frac{-1}{x \cdot x} + \frac{0 - (x - x) \cdot (x + x)}{x \cdot x \cdot x \cdot x})
\end{autobreak}
\end{align*}

B = \begin{align*}
\begin{autobreak}
(x^{x} \cdot (\ln(x) + \frac{x}{x}) \cdot (\ln(x) + \frac{x}{x}) + x^{x} \cdot (\frac{1}{x} + \frac{x - x}{x \cdot x})) \cdot (\ln(x) + \frac{x}{x})
\end{autobreak}
\end{align*}

C = \begin{align*}
\begin{autobreak}
x^{x} \cdot (\ln(x) + \frac{x}{x}) \cdot (\frac{1}{x} + \frac{x - x}{x \cdot x})
\end{autobreak}
\end{align*}

\begin{align*}
\begin{autobreak}
\frac{d}{dx}(B + C + A) = \frac{d}{dx}(B + C) + \frac{d}{dx}(A)
\end{autobreak}
\end{align*}

\begin{center}
Родственники Боблина продолжают лезть к вам, держите посох крепче!
\end{center}
A = \begin{align*}
\begin{autobreak}
x^{x} \cdot (\ln(x) + \frac{x}{x}) \cdot (\frac{1}{x} + \frac{x - x}{x \cdot x}) + x^{x} \cdot (\frac{-1}{x \cdot x} + \frac{0 - (x - x) \cdot (x + x)}{x \cdot x \cdot x \cdot x})
\end{autobreak}
\end{align*}

\begin{align*}
\begin{autobreak}
\frac{d}{dx}(A) = \frac{d}{dx}(x^{x} \cdot (\ln(x) + \frac{x}{x}) \cdot (\frac{1}{x} + \frac{x - x}{x \cdot x})) + \frac{d}{dx}(x^{x} \cdot (\frac{-1}{x \cdot x} + \frac{0 - (x - x) \cdot (x + x)}{x \cdot x \cdot x \cdot x}))
\end{autobreak}
\end{align*}

\begin{center}
Битва продолжается, не теряйте духу, они когда-то, наверное, закончаться!
\end{center}
\begin{align*}
\begin{autobreak}
\frac{d}{dx}(x^{x} \cdot (\frac{-1}{x \cdot x} + \frac{0 - (x - x) \cdot (x + x)}{x \cdot x \cdot x \cdot x})) = \frac{d}{dx}(x^{x}) \cdot \frac{-1}{x \cdot x} + \frac{0 - (x - x) \cdot (x + x)}{x \cdot x \cdot x \cdot x} + x^{x} \cdot \frac{d}{dx}(\frac{-1}{x \cdot x} + \frac{0 - (x - x) \cdot (x + x)}{x \cdot x \cdot x \cdot x})
\end{autobreak}
\end{align*}

\begin{center}
Их не становиться меньше, откуда они только лезут?!
\end{center}
\begin{align*}
\begin{autobreak}
\frac{d}{dx}(\frac{-1}{x \cdot x} + \frac{0 - (x - x) \cdot (x + x)}{x \cdot x \cdot x \cdot x}) = \frac{d}{dx}(\frac{-1}{x \cdot x}) + \frac{d}{dx}(\frac{0 - (x - x) \cdot (x + x)}{x \cdot x \cdot x \cdot x})
\end{autobreak}
\end{align*}

\begin{center}
Вот так и рождаются легенды о герое, истребившем половину рода Боблина.
\end{center}
\begin{align*}
\begin{autobreak}
\frac{d}{dx}(\frac{0 - (x - x) \cdot (x + x)}{x \cdot x \cdot x \cdot x}) = \frac{\frac{d}{dx}(0 - (x - x) \cdot (x + x)) \cdot x \cdot x \cdot x \cdot x - 0 - (x - x) \cdot (x + x) \cdot \frac{d}{dx}(x \cdot x \cdot x \cdot x)}{{x \cdot x \cdot x \cdot x}^2}
\end{autobreak}
\end{align*}

\begin{center}
Небольшой взмах посохом — и план сражения выглядит куда приличнее.
\end{center}
\begin{align*}
\begin{autobreak}
\frac{d}{dx}(x \cdot x \cdot x \cdot x) = \frac{d}{dx}(x \cdot x) \cdot x \cdot x + x \cdot x \cdot \frac{d}{dx}(x \cdot x)
\end{autobreak}
\end{align*}

\begin{center}
Один из гоблинов упал
\end{center}
\begin{align*}
\begin{autobreak}
\frac{d}{dx}(x \cdot x) = \frac{d}{dx}(x) \cdot x + x \cdot \frac{d}{dx}(x)
\end{autobreak}
\end{align*}

\begin{center}
Ваше вошшебство откатило деверя Боблина до младенчества
\end{center}
\begin{align*}
\begin{autobreak}
\frac{d}{dx}(x) = 1
\end{autobreak}
\end{align*}

\begin{center}
Полиморф сработал отлично: зять Боблина теперь лягушка
\end{center}
\begin{align*}
\begin{autobreak}
\frac{d}{dx}(x) = 1
\end{autobreak}
\end{align*}

\begin{center}
Родственники Боблина продолжают лезть к вам, держите посох крепче!
\end{center}
\begin{align*}
\begin{autobreak}
\frac{d}{dx}(x \cdot x) = \frac{d}{dx}(x) \cdot x + x \cdot \frac{d}{dx}(x)
\end{autobreak}
\end{align*}

\begin{center}
Поздравляю! От свояка Боблина осталась только полоыина
\end{center}
\begin{align*}
\begin{autobreak}
\frac{d}{dx}(x) = 1
\end{autobreak}
\end{align*}

\begin{center}
Ваше заклинание свернуло невестку Боблина в шарик
\end{center}
\begin{align*}
\begin{autobreak}
\frac{d}{dx}(x) = 1
\end{autobreak}
\end{align*}

\begin{center}
Битва продолжается, не теряйте духу, они когда-то, наверное, закончаться!
\end{center}
\begin{align*}
\begin{autobreak}
\frac{d}{dx}(0 - (x - x) \cdot (x + x)) = \frac{d}{dx}(0) - \frac{d}{dx}((x - x) \cdot (x + x))
\end{autobreak}
\end{align*}

\begin{center}
Их не становиться меньше, откуда они только лезут?!
\end{center}
\begin{align*}
\begin{autobreak}
\frac{d}{dx}((x - x) \cdot (x + x)) = \frac{d}{dx}(x - x) \cdot x + x + x - x \cdot \frac{d}{dx}(x + x)
\end{autobreak}
\end{align*}

\begin{center}
Вот так и рождаются легенды о герое, истребившем половину рода Боблина.
\end{center}
\begin{align*}
\begin{autobreak}
\frac{d}{dx}(x + x) = \frac{d}{dx}(x) + \frac{d}{dx}(x)
\end{autobreak}
\end{align*}

\begin{center}
ААХХАХААХАХ Гоблин-Боблин
\end{center}
\begin{align*}
\begin{autobreak}
\frac{d}{dx}(x) = 1
\end{autobreak}
\end{align*}

\begin{center}
Ваше вошшебство откатило деверя Боблина до младенчества
\end{center}
\begin{align*}
\begin{autobreak}
\frac{d}{dx}(x) = 1
\end{autobreak}
\end{align*}

\begin{center}
Небольшой взмах посохом — и план сражения выглядит куда приличнее.
\end{center}
\begin{align*}
\begin{autobreak}
\frac{d}{dx}(x - x) = \frac{d}{dx}(x) - \frac{d}{dx}(x)
\end{autobreak}
\end{align*}

\begin{center}
Полиморф сработал отлично: зять Боблина теперь лягушка
\end{center}
\begin{align*}
\begin{autobreak}
\frac{d}{dx}(x) = 1
\end{autobreak}
\end{align*}

\begin{center}
Поздравляю! От свояка Боблина осталась только полоыина
\end{center}
\begin{align*}
\begin{autobreak}
\frac{d}{dx}(x) = 1
\end{autobreak}
\end{align*}

\begin{center}
Огненный шар испарил бабушку Боблина
\end{center}
\begin{align*}
\begin{autobreak}
\frac{d}{dx}(0) = 0
\end{autobreak}
\end{align*}

\begin{center}
Один из гоблинов упал
\end{center}
\begin{align*}
\begin{autobreak}
\frac{d}{dx}(\frac{-1}{x \cdot x}) = \frac{\frac{d}{dx}(-1) \cdot x \cdot x - -1 \cdot \frac{d}{dx}(x \cdot x)}{{x \cdot x}^2}
\end{autobreak}
\end{align*}

\begin{center}
Родственники Боблина продолжают лезть к вам, держите посох крепче!
\end{center}
\begin{align*}
\begin{autobreak}
\frac{d}{dx}(x \cdot x) = \frac{d}{dx}(x) \cdot x + x \cdot \frac{d}{dx}(x)
\end{autobreak}
\end{align*}

\begin{center}
Ваше заклинание свернуло невестку Боблина в шарик
\end{center}
\begin{align*}
\begin{autobreak}
\frac{d}{dx}(x) = 1
\end{autobreak}
\end{align*}

\begin{center}
ААХХАХААХАХ Гоблин-Боблин
\end{center}
\begin{align*}
\begin{autobreak}
\frac{d}{dx}(x) = 1
\end{autobreak}
\end{align*}

\begin{center}
Заклинание хаоса раскидало части племянника Боблина по разным планам
\end{center}
\begin{align*}
\begin{autobreak}
\frac{d}{dx}(-1) = 0
\end{autobreak}
\end{align*}

\begin{center}
Битва продолжается, не теряйте духу, они когда-то, наверное, закончаться!
\end{center}
\begin{align*}
\begin{autobreak}
\frac{d}{dx}(x^{x}) = x^{x} \cdot ( \frac{d}{dx}(x) \cdot \ln(x) + x \cdot \frac{d}{dx}(x) / x )
\end{autobreak}
\end{align*}

\begin{center}
Ваше вошшебство откатило деверя Боблина до младенчества
\end{center}
\begin{align*}
\begin{autobreak}
\frac{d}{dx}(x) = 1
\end{autobreak}
\end{align*}

\begin{center}
Полиморф сработал отлично: зять Боблина теперь лягушка
\end{center}
\begin{align*}
\begin{autobreak}
\frac{d}{dx}(x) = 1
\end{autobreak}
\end{align*}

\begin{center}
Их не становиться меньше, откуда они только лезут?!
\end{center}
\begin{align*}
\begin{autobreak}
\frac{d}{dx}(x^{x} \cdot (\ln(x) + \frac{x}{x}) \cdot (\frac{1}{x} + \frac{x - x}{x \cdot x})) = \frac{d}{dx}(x^{x} \cdot (\ln(x) + \frac{x}{x})) \cdot \frac{1}{x} + \frac{x - x}{x \cdot x} + x^{x} \cdot (\ln(x) + \frac{x}{x}) \cdot \frac{d}{dx}(\frac{1}{x} + \frac{x - x}{x \cdot x})
\end{autobreak}
\end{align*}

\begin{center}
Вот так и рождаются легенды о герое, истребившем половину рода Боблина.
\end{center}
\begin{align*}
\begin{autobreak}
\frac{d}{dx}(\frac{1}{x} + \frac{x - x}{x \cdot x}) = \frac{d}{dx}(\frac{1}{x}) + \frac{d}{dx}(\frac{x - x}{x \cdot x})
\end{autobreak}
\end{align*}

\begin{center}
Небольшой взмах посохом — и план сражения выглядит куда приличнее.
\end{center}
\begin{align*}
\begin{autobreak}
\frac{d}{dx}(\frac{x - x}{x \cdot x}) = \frac{\frac{d}{dx}(x - x) \cdot x \cdot x - x - x \cdot \frac{d}{dx}(x \cdot x)}{{x \cdot x}^2}
\end{autobreak}
\end{align*}

\begin{center}
Один из гоблинов упал
\end{center}
\begin{align*}
\begin{autobreak}
\frac{d}{dx}(x \cdot x) = \frac{d}{dx}(x) \cdot x + x \cdot \frac{d}{dx}(x)
\end{autobreak}
\end{align*}

\begin{center}
Поздравляю! От свояка Боблина осталась только полоыина
\end{center}
\begin{align*}
\begin{autobreak}
\frac{d}{dx}(x) = 1
\end{autobreak}
\end{align*}

\begin{center}
Ваше заклинание свернуло невестку Боблина в шарик
\end{center}
\begin{align*}
\begin{autobreak}
\frac{d}{dx}(x) = 1
\end{autobreak}
\end{align*}

\begin{center}
Родственники Боблина продолжают лезть к вам, держите посох крепче!
\end{center}
\begin{align*}
\begin{autobreak}
\frac{d}{dx}(x - x) = \frac{d}{dx}(x) - \frac{d}{dx}(x)
\end{autobreak}
\end{align*}

\begin{center}
ААХХАХААХАХ Гоблин-Боблин
\end{center}
\begin{align*}
\begin{autobreak}
\frac{d}{dx}(x) = 1
\end{autobreak}
\end{align*}

\begin{center}
Ваше вошшебство откатило деверя Боблина до младенчества
\end{center}
\begin{align*}
\begin{autobreak}
\frac{d}{dx}(x) = 1
\end{autobreak}
\end{align*}

\begin{center}
Битва продолжается, не теряйте духу, они когда-то, наверное, закончаться!
\end{center}
\begin{align*}
\begin{autobreak}
\frac{d}{dx}(\frac{1}{x}) = \frac{\frac{d}{dx}(1) \cdot x - 1 \cdot \frac{d}{dx}(x)}{{x}^2}
\end{autobreak}
\end{align*}

\begin{center}
Полиморф сработал отлично: зять Боблина теперь лягушка
\end{center}
\begin{align*}
\begin{autobreak}
\frac{d}{dx}(x) = 1
\end{autobreak}
\end{align*}

\begin{center}
Ваш портал небытия вежливо удалил тещу Боблина из этого измерения
\end{center}
\begin{align*}
\begin{autobreak}
\frac{d}{dx}(1) = 0
\end{autobreak}
\end{align*}

\begin{center}
Их не становиться меньше, откуда они только лезут?!
\end{center}
\begin{align*}
\begin{autobreak}
\frac{d}{dx}(x^{x} \cdot (\ln(x) + \frac{x}{x})) = \frac{d}{dx}(x^{x}) \cdot \ln(x) + \frac{x}{x} + x^{x} \cdot \frac{d}{dx}(\ln(x) + \frac{x}{x})
\end{autobreak}
\end{align*}

\begin{center}
Вот так и рождаются легенды о герое, истребившем половину рода Боблина.
\end{center}
\begin{align*}
\begin{autobreak}
\frac{d}{dx}(\ln(x) + \frac{x}{x}) = \frac{d}{dx}(\ln(x)) + \frac{d}{dx}(\frac{x}{x})
\end{autobreak}
\end{align*}

\begin{center}
Небольшой взмах посохом — и план сражения выглядит куда приличнее.
\end{center}
\begin{align*}
\begin{autobreak}
\frac{d}{dx}(\frac{x}{x}) = \frac{\frac{d}{dx}(x) \cdot x - x \cdot \frac{d}{dx}(x)}{{x}^2}
\end{autobreak}
\end{align*}

\begin{center}
Поздравляю! От свояка Боблина осталась только полоыина
\end{center}
\begin{align*}
\begin{autobreak}
\frac{d}{dx}(x) = 1
\end{autobreak}
\end{align*}

\begin{center}
Ваше заклинание свернуло невестку Боблина в шарик
\end{center}
\begin{align*}
\begin{autobreak}
\frac{d}{dx}(x) = 1
\end{autobreak}
\end{align*}

\begin{center}
Один из гоблинов упал
\end{center}
\begin{align*}
\begin{autobreak}
\frac{d}{dx}(\ln(x)) = \frac{\frac{d}{dx}(x)}{x}
\end{autobreak}
\end{align*}

\begin{center}
ААХХАХААХАХ Гоблин-Боблин
\end{center}
\begin{align*}
\begin{autobreak}
\frac{d}{dx}(x) = 1
\end{autobreak}
\end{align*}

\begin{center}
Родственники Боблина продолжают лезть к вам, держите посох крепче!
\end{center}
\begin{align*}
\begin{autobreak}
\frac{d}{dx}(x^{x}) = x^{x} \cdot ( \frac{d}{dx}(x) \cdot \ln(x) + x \cdot \frac{d}{dx}(x) / x )
\end{autobreak}
\end{align*}

\begin{center}
Ваше вошшебство откатило деверя Боблина до младенчества
\end{center}
\begin{align*}
\begin{autobreak}
\frac{d}{dx}(x) = 1
\end{autobreak}
\end{align*}

\begin{center}
Полиморф сработал отлично: зять Боблина теперь лягушка
\end{center}
\begin{align*}
\begin{autobreak}
\frac{d}{dx}(x) = 1
\end{autobreak}
\end{align*}

\begin{center}
Битва продолжается, не теряйте духу, они когда-то, наверное, закончаться!
\end{center}
A = \begin{align*}
\begin{autobreak}
(x^{x} \cdot (\ln(x) + \frac{x}{x}) \cdot (\ln(x) + \frac{x}{x}) + x^{x} \cdot (\frac{1}{x} + \frac{x - x}{x \cdot x})) \cdot (\ln(x) + \frac{x}{x})
\end{autobreak}
\end{align*}

\begin{align*}
\begin{autobreak}
\frac{d}{dx}(A + x^{x} \cdot (\ln(x) + \frac{x}{x}) \cdot (\frac{1}{x} + \frac{x - x}{x \cdot x})) = \frac{d}{dx}(A) + \frac{d}{dx}(x^{x} \cdot (\ln(x) + \frac{x}{x}) \cdot (\frac{1}{x} + \frac{x - x}{x \cdot x}))
\end{autobreak}
\end{align*}

\begin{center}
Их не становиться меньше, откуда они только лезут?!
\end{center}
\begin{align*}
\begin{autobreak}
\frac{d}{dx}(x^{x} \cdot (\ln(x) + \frac{x}{x}) \cdot (\frac{1}{x} + \frac{x - x}{x \cdot x})) = \frac{d}{dx}(x^{x} \cdot (\ln(x) + \frac{x}{x})) \cdot \frac{1}{x} + \frac{x - x}{x \cdot x} + x^{x} \cdot (\ln(x) + \frac{x}{x}) \cdot \frac{d}{dx}(\frac{1}{x} + \frac{x - x}{x \cdot x})
\end{autobreak}
\end{align*}

\begin{center}
Вот так и рождаются легенды о герое, истребившем половину рода Боблина.
\end{center}
\begin{align*}
\begin{autobreak}
\frac{d}{dx}(\frac{1}{x} + \frac{x - x}{x \cdot x}) = \frac{d}{dx}(\frac{1}{x}) + \frac{d}{dx}(\frac{x - x}{x \cdot x})
\end{autobreak}
\end{align*}

\begin{center}
Небольшой взмах посохом — и план сражения выглядит куда приличнее.
\end{center}
\begin{align*}
\begin{autobreak}
\frac{d}{dx}(\frac{x - x}{x \cdot x}) = \frac{\frac{d}{dx}(x - x) \cdot x \cdot x - x - x \cdot \frac{d}{dx}(x \cdot x)}{{x \cdot x}^2}
\end{autobreak}
\end{align*}

\begin{center}
Один из гоблинов упал
\end{center}
\begin{align*}
\begin{autobreak}
\frac{d}{dx}(x \cdot x) = \frac{d}{dx}(x) \cdot x + x \cdot \frac{d}{dx}(x)
\end{autobreak}
\end{align*}

\begin{center}
Поздравляю! От свояка Боблина осталась только полоыина
\end{center}
\begin{align*}
\begin{autobreak}
\frac{d}{dx}(x) = 1
\end{autobreak}
\end{align*}

\begin{center}
Ваше заклинание свернуло невестку Боблина в шарик
\end{center}
\begin{align*}
\begin{autobreak}
\frac{d}{dx}(x) = 1
\end{autobreak}
\end{align*}

\begin{center}
Родственники Боблина продолжают лезть к вам, держите посох крепче!
\end{center}
\begin{align*}
\begin{autobreak}
\frac{d}{dx}(x - x) = \frac{d}{dx}(x) - \frac{d}{dx}(x)
\end{autobreak}
\end{align*}

\begin{center}
ААХХАХААХАХ Гоблин-Боблин
\end{center}
\begin{align*}
\begin{autobreak}
\frac{d}{dx}(x) = 1
\end{autobreak}
\end{align*}

\begin{center}
Ваше вошшебство откатило деверя Боблина до младенчества
\end{center}
\begin{align*}
\begin{autobreak}
\frac{d}{dx}(x) = 1
\end{autobreak}
\end{align*}

\begin{center}
Битва продолжается, не теряйте духу, они когда-то, наверное, закончаться!
\end{center}
\begin{align*}
\begin{autobreak}
\frac{d}{dx}(\frac{1}{x}) = \frac{\frac{d}{dx}(1) \cdot x - 1 \cdot \frac{d}{dx}(x)}{{x}^2}
\end{autobreak}
\end{align*}

\begin{center}
Полиморф сработал отлично: зять Боблина теперь лягушка
\end{center}
\begin{align*}
\begin{autobreak}
\frac{d}{dx}(x) = 1
\end{autobreak}
\end{align*}

\begin{center}
Ваше заклинание дезинтегрировало брата Боблина
\end{center}
\begin{align*}
\begin{autobreak}
\frac{d}{dx}(1) = 0
\end{autobreak}
\end{align*}

\begin{center}
Их не становиться меньше, откуда они только лезут?!
\end{center}
\begin{align*}
\begin{autobreak}
\frac{d}{dx}(x^{x} \cdot (\ln(x) + \frac{x}{x})) = \frac{d}{dx}(x^{x}) \cdot \ln(x) + \frac{x}{x} + x^{x} \cdot \frac{d}{dx}(\ln(x) + \frac{x}{x})
\end{autobreak}
\end{align*}

\begin{center}
Вот так и рождаются легенды о герое, истребившем половину рода Боблина.
\end{center}
\begin{align*}
\begin{autobreak}
\frac{d}{dx}(\ln(x) + \frac{x}{x}) = \frac{d}{dx}(\ln(x)) + \frac{d}{dx}(\frac{x}{x})
\end{autobreak}
\end{align*}

\begin{center}
Небольшой взмах посохом — и план сражения выглядит куда приличнее.
\end{center}
\begin{align*}
\begin{autobreak}
\frac{d}{dx}(\frac{x}{x}) = \frac{\frac{d}{dx}(x) \cdot x - x \cdot \frac{d}{dx}(x)}{{x}^2}
\end{autobreak}
\end{align*}

\begin{center}
Поздравляю! От свояка Боблина осталась только полоыина
\end{center}
\begin{align*}
\begin{autobreak}
\frac{d}{dx}(x) = 1
\end{autobreak}
\end{align*}

\begin{center}
Ваше заклинание свернуло невестку Боблина в шарик
\end{center}
\begin{align*}
\begin{autobreak}
\frac{d}{dx}(x) = 1
\end{autobreak}
\end{align*}

\begin{center}
Один из гоблинов упал
\end{center}
\begin{align*}
\begin{autobreak}
\frac{d}{dx}(\ln(x)) = \frac{\frac{d}{dx}(x)}{x}
\end{autobreak}
\end{align*}

\begin{center}
ААХХАХААХАХ Гоблин-Боблин
\end{center}
\begin{align*}
\begin{autobreak}
\frac{d}{dx}(x) = 1
\end{autobreak}
\end{align*}

\begin{center}
Родственники Боблина продолжают лезть к вам, держите посох крепче!
\end{center}
\begin{align*}
\begin{autobreak}
\frac{d}{dx}(x^{x}) = x^{x} \cdot ( \frac{d}{dx}(x) \cdot \ln(x) + x \cdot \frac{d}{dx}(x) / x )
\end{autobreak}
\end{align*}

\begin{center}
Ваше вошшебство откатило деверя Боблина до младенчества
\end{center}
\begin{align*}
\begin{autobreak}
\frac{d}{dx}(x) = 1
\end{autobreak}
\end{align*}

\begin{center}
Полиморф сработал отлично: зять Боблина теперь лягушка
\end{center}
\begin{align*}
\begin{autobreak}
\frac{d}{dx}(x) = 1
\end{autobreak}
\end{align*}

\begin{center}
Битва продолжается, не теряйте духу, они когда-то, наверное, закончаться!
\end{center}
A = \begin{align*}
\begin{autobreak}
(x^{x} \cdot (\ln(x) + \frac{x}{x}) \cdot (\ln(x) + \frac{x}{x}) + x^{x} \cdot (\frac{1}{x} + \frac{x - x}{x \cdot x})) \cdot (\ln(x) + \frac{x}{x})
\end{autobreak}
\end{align*}

\begin{align*}
\begin{autobreak}
\frac{d}{dx}(A) = \frac{d}{dx}(x^{x} \cdot (\ln(x) + \frac{x}{x}) \cdot (\ln(x) + \frac{x}{x}) + x^{x} \cdot (\frac{1}{x} + \frac{x - x}{x \cdot x})) \cdot \ln(x) + \frac{x}{x} + x^{x} \cdot (\ln(x) + \frac{x}{x}) \cdot (\ln(x) + \frac{x}{x}) + x^{x} \cdot (\frac{1}{x} + \frac{x - x}{x \cdot x}) \cdot \frac{d}{dx}(\ln(x) + \frac{x}{x})
\end{autobreak}
\end{align*}

\begin{center}
Их не становиться меньше, откуда они только лезут?!
\end{center}
\begin{align*}
\begin{autobreak}
\frac{d}{dx}(\ln(x) + \frac{x}{x}) = \frac{d}{dx}(\ln(x)) + \frac{d}{dx}(\frac{x}{x})
\end{autobreak}
\end{align*}

\begin{center}
Вот так и рождаются легенды о герое, истребившем половину рода Боблина.
\end{center}
\begin{align*}
\begin{autobreak}
\frac{d}{dx}(\frac{x}{x}) = \frac{\frac{d}{dx}(x) \cdot x - x \cdot \frac{d}{dx}(x)}{{x}^2}
\end{autobreak}
\end{align*}

\begin{center}
Поздравляю! От свояка Боблина осталась только полоыина
\end{center}
\begin{align*}
\begin{autobreak}
\frac{d}{dx}(x) = 1
\end{autobreak}
\end{align*}

\begin{center}
Ваше заклинание свернуло невестку Боблина в шарик
\end{center}
\begin{align*}
\begin{autobreak}
\frac{d}{dx}(x) = 1
\end{autobreak}
\end{align*}

\begin{center}
Небольшой взмах посохом — и план сражения выглядит куда приличнее.
\end{center}
\begin{align*}
\begin{autobreak}
\frac{d}{dx}(\ln(x)) = \frac{\frac{d}{dx}(x)}{x}
\end{autobreak}
\end{align*}

\begin{center}
ААХХАХААХАХ Гоблин-Боблин
\end{center}
\begin{align*}
\begin{autobreak}
\frac{d}{dx}(x) = 1
\end{autobreak}
\end{align*}

\begin{center}
Один из гоблинов упал
\end{center}
A = \begin{align*}
\begin{autobreak}
x^{x} \cdot (\ln(x) + \frac{x}{x}) \cdot (\ln(x) + \frac{x}{x}) + x^{x} \cdot (\frac{1}{x} + \frac{x - x}{x \cdot x})
\end{autobreak}
\end{align*}

\begin{align*}
\begin{autobreak}
\frac{d}{dx}(A) = \frac{d}{dx}(x^{x} \cdot (\ln(x) + \frac{x}{x}) \cdot (\ln(x) + \frac{x}{x})) + \frac{d}{dx}(x^{x} \cdot (\frac{1}{x} + \frac{x - x}{x \cdot x}))
\end{autobreak}
\end{align*}

\begin{center}
Родственники Боблина продолжают лезть к вам, держите посох крепче!
\end{center}
\begin{align*}
\begin{autobreak}
\frac{d}{dx}(x^{x} \cdot (\frac{1}{x} + \frac{x - x}{x \cdot x})) = \frac{d}{dx}(x^{x}) \cdot \frac{1}{x} + \frac{x - x}{x \cdot x} + x^{x} \cdot \frac{d}{dx}(\frac{1}{x} + \frac{x - x}{x \cdot x})
\end{autobreak}
\end{align*}

\begin{center}
Битва продолжается, не теряйте духу, они когда-то, наверное, закончаться!
\end{center}
\begin{align*}
\begin{autobreak}
\frac{d}{dx}(\frac{1}{x} + \frac{x - x}{x \cdot x}) = \frac{d}{dx}(\frac{1}{x}) + \frac{d}{dx}(\frac{x - x}{x \cdot x})
\end{autobreak}
\end{align*}

\begin{center}
Их не становиться меньше, откуда они только лезут?!
\end{center}
\begin{align*}
\begin{autobreak}
\frac{d}{dx}(\frac{x - x}{x \cdot x}) = \frac{\frac{d}{dx}(x - x) \cdot x \cdot x - x - x \cdot \frac{d}{dx}(x \cdot x)}{{x \cdot x}^2}
\end{autobreak}
\end{align*}

\begin{center}
Вот так и рождаются легенды о герое, истребившем половину рода Боблина.
\end{center}
\begin{align*}
\begin{autobreak}
\frac{d}{dx}(x \cdot x) = \frac{d}{dx}(x) \cdot x + x \cdot \frac{d}{dx}(x)
\end{autobreak}
\end{align*}

\begin{center}
Ваше вошшебство откатило деверя Боблина до младенчества
\end{center}
\begin{align*}
\begin{autobreak}
\frac{d}{dx}(x) = 1
\end{autobreak}
\end{align*}

\begin{center}
Полиморф сработал отлично: зять Боблина теперь лягушка
\end{center}
\begin{align*}
\begin{autobreak}
\frac{d}{dx}(x) = 1
\end{autobreak}
\end{align*}

\begin{center}
Небольшой взмах посохом — и план сражения выглядит куда приличнее.
\end{center}
\begin{align*}
\begin{autobreak}
\frac{d}{dx}(x - x) = \frac{d}{dx}(x) - \frac{d}{dx}(x)
\end{autobreak}
\end{align*}

\begin{center}
Поздравляю! От свояка Боблина осталась только полоыина
\end{center}
\begin{align*}
\begin{autobreak}
\frac{d}{dx}(x) = 1
\end{autobreak}
\end{align*}

\begin{center}
Ваше заклинание свернуло невестку Боблина в шарик
\end{center}
\begin{align*}
\begin{autobreak}
\frac{d}{dx}(x) = 1
\end{autobreak}
\end{align*}

\begin{center}
Один из гоблинов упал
\end{center}
\begin{align*}
\begin{autobreak}
\frac{d}{dx}(\frac{1}{x}) = \frac{\frac{d}{dx}(1) \cdot x - 1 \cdot \frac{d}{dx}(x)}{{x}^2}
\end{autobreak}
\end{align*}

\begin{center}
ААХХАХААХАХ Гоблин-Боблин
\end{center}
\begin{align*}
\begin{autobreak}
\frac{d}{dx}(x) = 1
\end{autobreak}
\end{align*}

\begin{center}
Ваше колдовство низвело сестру Боблина до атомов
\end{center}
\begin{align*}
\begin{autobreak}
\frac{d}{dx}(1) = 0
\end{autobreak}
\end{align*}

\begin{center}
Родственники Боблина продолжают лезть к вам, держите посох крепче!
\end{center}
\begin{align*}
\begin{autobreak}
\frac{d}{dx}(x^{x}) = x^{x} \cdot ( \frac{d}{dx}(x) \cdot \ln(x) + x \cdot \frac{d}{dx}(x) / x )
\end{autobreak}
\end{align*}

\begin{center}
Ваше вошшебство откатило деверя Боблина до младенчества
\end{center}
\begin{align*}
\begin{autobreak}
\frac{d}{dx}(x) = 1
\end{autobreak}
\end{align*}

\begin{center}
Полиморф сработал отлично: зять Боблина теперь лягушка
\end{center}
\begin{align*}
\begin{autobreak}
\frac{d}{dx}(x) = 1
\end{autobreak}
\end{align*}

\begin{center}
Битва продолжается, не теряйте духу, они когда-то, наверное, закончаться!
\end{center}
\begin{align*}
\begin{autobreak}
\frac{d}{dx}(x^{x} \cdot (\ln(x) + \frac{x}{x}) \cdot (\ln(x) + \frac{x}{x})) = \frac{d}{dx}(x^{x} \cdot (\ln(x) + \frac{x}{x})) \cdot \ln(x) + \frac{x}{x} + x^{x} \cdot (\ln(x) + \frac{x}{x}) \cdot \frac{d}{dx}(\ln(x) + \frac{x}{x})
\end{autobreak}
\end{align*}

\begin{center}
Их не становиться меньше, откуда они только лезут?!
\end{center}
\begin{align*}
\begin{autobreak}
\frac{d}{dx}(\ln(x) + \frac{x}{x}) = \frac{d}{dx}(\ln(x)) + \frac{d}{dx}(\frac{x}{x})
\end{autobreak}
\end{align*}

\begin{center}
Вот так и рождаются легенды о герое, истребившем половину рода Боблина.
\end{center}
\begin{align*}
\begin{autobreak}
\frac{d}{dx}(\frac{x}{x}) = \frac{\frac{d}{dx}(x) \cdot x - x \cdot \frac{d}{dx}(x)}{{x}^2}
\end{autobreak}
\end{align*}

\begin{center}
Поздравляю! От свояка Боблина осталась только полоыина
\end{center}
\begin{align*}
\begin{autobreak}
\frac{d}{dx}(x) = 1
\end{autobreak}
\end{align*}

\begin{center}
Ваше заклинание свернуло невестку Боблина в шарик
\end{center}
\begin{align*}
\begin{autobreak}
\frac{d}{dx}(x) = 1
\end{autobreak}
\end{align*}

\begin{center}
Небольшой взмах посохом — и план сражения выглядит куда приличнее.
\end{center}
\begin{align*}
\begin{autobreak}
\frac{d}{dx}(\ln(x)) = \frac{\frac{d}{dx}(x)}{x}
\end{autobreak}
\end{align*}

\begin{center}
ААХХАХААХАХ Гоблин-Боблин
\end{center}
\begin{align*}
\begin{autobreak}
\frac{d}{dx}(x) = 1
\end{autobreak}
\end{align*}

\begin{center}
Один из гоблинов упал
\end{center}
\begin{align*}
\begin{autobreak}
\frac{d}{dx}(x^{x} \cdot (\ln(x) + \frac{x}{x})) = \frac{d}{dx}(x^{x}) \cdot \ln(x) + \frac{x}{x} + x^{x} \cdot \frac{d}{dx}(\ln(x) + \frac{x}{x})
\end{autobreak}
\end{align*}

\begin{center}
Родственники Боблина продолжают лезть к вам, держите посох крепче!
\end{center}
\begin{align*}
\begin{autobreak}
\frac{d}{dx}(\ln(x) + \frac{x}{x}) = \frac{d}{dx}(\ln(x)) + \frac{d}{dx}(\frac{x}{x})
\end{autobreak}
\end{align*}

\begin{center}
Битва продолжается, не теряйте духу, они когда-то, наверное, закончаться!
\end{center}
\begin{align*}
\begin{autobreak}
\frac{d}{dx}(\frac{x}{x}) = \frac{\frac{d}{dx}(x) \cdot x - x \cdot \frac{d}{dx}(x)}{{x}^2}
\end{autobreak}
\end{align*}

\begin{center}
Ваше вошшебство откатило деверя Боблина до младенчества
\end{center}
\begin{align*}
\begin{autobreak}
\frac{d}{dx}(x) = 1
\end{autobreak}
\end{align*}

\begin{center}
Полиморф сработал отлично: зять Боблина теперь лягушка
\end{center}
\begin{align*}
\begin{autobreak}
\frac{d}{dx}(x) = 1
\end{autobreak}
\end{align*}

\begin{center}
Их не становиться меньше, откуда они только лезут?!
\end{center}
\begin{align*}
\begin{autobreak}
\frac{d}{dx}(\ln(x)) = \frac{\frac{d}{dx}(x)}{x}
\end{autobreak}
\end{align*}

\begin{center}
Поздравляю! От свояка Боблина осталась только полоыина
\end{center}
\begin{align*}
\begin{autobreak}
\frac{d}{dx}(x) = 1
\end{autobreak}
\end{align*}

\begin{center}
Вот так и рождаются легенды о герое, истребившем половину рода Боблина.
\end{center}
\begin{align*}
\begin{autobreak}
\frac{d}{dx}(x^{x}) = x^{x} \cdot ( \frac{d}{dx}(x) \cdot \ln(x) + x \cdot \frac{d}{dx}(x) / x )
\end{autobreak}
\end{align*}

\begin{center}
Ваше заклинание свернуло невестку Боблина в шарик
\end{center}
\begin{align*}
\begin{autobreak}
\frac{d}{dx}(x) = 1
\end{autobreak}
\end{align*}

\begin{center}
ААХХАХААХАХ Гоблин-Боблин
\end{center}
\begin{align*}
\begin{autobreak}
\frac{d}{dx}(x) = 1
\end{autobreak}
\end{align*}

\begin{align*}
\begin{autobreak}
1 + \frac{1}{1} \cdot (x - 1)^{1} + \frac{3.99509e+45}{2} \cdot (x - 1)^{2} + \frac{1.76433e+46}{6} \cdot (x - 1)^{3} + \frac{7.80483e+46}{24} \cdot (x - 1)^{4} + \frac{3.45832e+47}{120} \cdot (x - 1)^{5}
\end{autobreak}
\end{align*}

\end{document}
