\documentclass[a4paper,12pt]{article}
\usepackage[T2A]{fontenc}
\usepackage[utf8]{inputenc}
\usepackage[russian]{babel}
\usepackage{amsmath}
\usepackage{amssymb}
\usepackage{autobreak}
\usepackage{hyperref}
\setcounter{secnumdepth}{0}
\begin{document}
\title{Сокращение рода боблина}
\maketitle
\newpage
\begin{titlepage}
    \centering

    {\Large Уставший волшебник}\\[1cm]

    {\huge\bfseries «Завершение рода Боблина»}\\[0.5cm]

    \raggedright

    \textbf{Предыстория} \\[0.3cm]

    Жил-был самый обычный гоблин по имени Боблин и его очень большая семья. \\ 
    Как-то раз, одним жарким летом они все вместе решили отправиться на пикник.\\ 
    Они нашли великолепную полянку посреди болота: солнышко, зеленая трава, тенеко от непонятно башни, одним словом - благодать. \\ 
    Шел 5-ый час гоблинской пъянки, тут уже нервы волшебника живущего в башне не выдержали. \\ 
    и он решил обрушиить свой праведный гнев не семейство Боблина, истребив некоторую его часть. \\ 
    \vspace{0.8cm}
    \textbf{Боевой журнал}\\[0.3cm]

    В башне стоял особенный артефакт, который записывал ход сражения в виде странного набора символов.\\ 
    Которые лишь сам маг был способен понять, здесь и будет приведет этот боевой журнал. \\     \vfill

    \raggedleft
    \textit{«Если на странице стало больше знаков — значит, кто-то из клана Боблина опять что-то натворил.»}\\[0.3cm]

\end{titlepage}
\tableofcontents
\newpage
f(x) = \begin{align*}
\begin{autobreak}
\sin(15 \cdot x^{3}) + \cos(6 \cdot x + 2)^{3}
\end{autobreak}
\end{align*}

\begin{center}
Небольшой взмах посохом — и план сражения выглядит куда приличнее.
\end{center}
\begin{align*}
\begin{autobreak}
\frac{d}{dx}(\sin(15 \cdot x^{3}) + \cos(6 \cdot x + 2)^{3}) = \frac{d}{dx}(\sin(15 \cdot x^{3})) + \frac{d}{dx}(\cos(6 \cdot x + 2)^{3})
\end{autobreak}
\end{align*}

\begin{center}
Один из гоблинов упал
\end{center}
\begin{align*}
\begin{autobreak}
\frac{d}{dx}(\cos(6 \cdot x + 2)^{3}) = \cos(6 \cdot x + 2)^3 \cdot (\frac{d}{dx}(3) \cdot \ln(\cos(6 \cdot x + 2)) + 3 \cdot \frac{\frac{d}{dx}(\cos(6 \cdot x + 2))}{\cos(6 \cdot x + 2)})
\end{autobreak}
\end{align*}

\begin{center}
Родственники Боблина продолжают лезть к вам, держите посох крепче!
\end{center}
\begin{align*}
\begin{autobreak}
\frac{d}{dx}(\cos(6 \cdot x + 2)) = -\sin(6 \cdot x + 2) \cdot \frac{d}{dx}(6 \cdot x + 2)
\end{autobreak}
\end{align*}

\begin{center}
Битва продолжается, не теряйте духу, они когда-то, наверное, закончаться!
\end{center}
\begin{align*}
\begin{autobreak}
\frac{d}{dx}(6 \cdot x + 2) = \frac{d}{dx}(6 \cdot x) + \frac{d}{dx}(2)
\end{autobreak}
\end{align*}

\begin{center}
Огненный шар испарил бабушку Боблина
\end{center}
\begin{align*}
\begin{autobreak}
\frac{d}{dx}(2) = 0
\end{autobreak}
\end{align*}

\begin{center}
Их не становиться меньше, откуда они только лезут?!
\end{center}
\begin{align*}
\begin{autobreak}
\frac{d}{dx}(6 \cdot x) = \frac{d}{dx}(6) \cdot x + 6 \cdot \frac{d}{dx}(x)
\end{autobreak}
\end{align*}

\begin{center}
Ваше заклинание свернуло невестку Боблина в шарик
\end{center}
\begin{align*}
\begin{autobreak}
\frac{d}{dx}(x) = 1
\end{autobreak}
\end{align*}

\begin{center}
Заклинание хаоса раскидало части племянника Боблина по разным планам
\end{center}
\begin{align*}
\begin{autobreak}
\frac{d}{dx}(6) = 0
\end{autobreak}
\end{align*}

\begin{center}
Ваш портал небытия вежливо удалил тещу Боблина из этого измерения
\end{center}
\begin{align*}
\begin{autobreak}
\frac{d}{dx}(3) = 0
\end{autobreak}
\end{align*}

\begin{center}
Вот так и рождаются легенды о герое, истребившем половину рода Боблина.
\end{center}
\begin{align*}
\begin{autobreak}
\frac{d}{dx}(\sin(15 \cdot x^{3})) = \cos(15 \cdot x^{3}) \cdot \frac{d}{dx}(15 \cdot x^{3})
\end{autobreak}
\end{align*}

\begin{center}
Небольшой взмах посохом — и план сражения выглядит куда приличнее.
\end{center}
\begin{align*}
\begin{autobreak}
\frac{d}{dx}(15 \cdot x^{3}) = \frac{d}{dx}(15) \cdot x^{3} + 15 \cdot \frac{d}{dx}(x^{3})
\end{autobreak}
\end{align*}

\begin{center}
Один из гоблинов упал
\end{center}
\begin{align*}
\begin{autobreak}
\frac{d}{dx}(x^{3}) = x^3 \cdot (\frac{d}{dx}(3) \cdot \ln(x) + 3 \cdot \frac{\frac{d}{dx}(x)}{x})
\end{autobreak}
\end{align*}

\begin{center}
ААХХАХААХАХ Гоблин-Боблин
\end{center}
\begin{align*}
\begin{autobreak}
\frac{d}{dx}(x) = 1
\end{autobreak}
\end{align*}

\begin{center}
Ваше заклинание дезинтегрировало брата Боблина
\end{center}
\begin{align*}
\begin{autobreak}
\frac{d}{dx}(3) = 0
\end{autobreak}
\end{align*}

\begin{center}
Ваше колдовство низвело сестру Боблина до атомов
\end{center}
\begin{align*}
\begin{autobreak}
\frac{d}{dx}(15) = 0
\end{autobreak}
\end{align*}

f(x)|dx = \begin{align*}
\begin{autobreak}
\cos(15 \cdot x^{3}) \cdot (0 \cdot x^{3} + 15 \cdot x^{3} \cdot (0 \cdot \ln(x) + 3 \cdot \frac{1}{x})) + \cos(6 \cdot x + 2)^{3} \cdot (0 \cdot \ln(\cos(6 \cdot x + 2)) + 3 \cdot \frac{(0 - \sin(6 \cdot x + 2)) \cdot (0 \cdot x + 6 \cdot 1 + 0)}{\cos(6 \cdot x + 2)})
\end{autobreak}
\end{align*}

optimized f(x)|dx = \begin{align*}
\begin{autobreak}
\cos(15 \cdot x^{3}) \cdot 15 \cdot x^{3} \cdot 3 \cdot \frac{1}{x} + \cos(6 \cdot x + 2)^{3} \cdot 3 \cdot \frac{(0 - \sin(6 \cdot x + 2)) \cdot 6}{\cos(6 \cdot x + 2)}
\end{autobreak}
\end{align*}

f(x)` = \begin{align*}
\begin{autobreak}
\cos(15 \cdot x^{3}) \cdot 15 \cdot x^{3} \cdot 3 \cdot \frac{1}{x} + \cos(6 \cdot x + 2)^{3} \cdot 3 \cdot \frac{(0 - \sin(6 \cdot x + 2)) \cdot 6}{\cos(6 \cdot x + 2)}
\end{autobreak}
\end{align*}

\end{document}
