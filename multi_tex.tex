\documentclass[a4paper,12pt]{article}
\usepackage[T2A]{fontenc}
\usepackage[utf8]{inputenc}
\usepackage[russian]{babel}
\usepackage{amsmath}
\usepackage{amssymb}
\usepackage{autobreak}
\usepackage{hyperref}
\setcounter{secnumdepth}{0}
\begin{document}
\title{Сокращение рода боблина}
\maketitle
\newpage
\begin{titlepage}
    \centering

    {\Large Уставший волшебник}\\[1cm]

    {\huge\bfseries «Завершение рода Боблина»}\\[0.5cm]

    \raggedright

    \textbf{Предыстория} \\[0.3cm]

    Жил-был самый обычный гоблин по имени Боблин и его очень большая семья. \\ 
    Как-то раз, одним жарким летом они все вместе решили отправиться на пикник.\\ 
    Они нашли великолепную полянку посреди болота: солнышко, зеленая трава, тенеко от непонятно башни, одним словом - благодать. \\ 
    Шел 5-ый час гоблинской пъянки, тут уже нервы волшебника живущего в башне не выдержали. \\ 
    и он решил обрушиить свой праведный гнев не семейство Боблина, истребив некоторую его часть. \\ 
    \vspace{0.8cm}
    \textbf{Боевой журнал}\\[0.3cm]

    В башне стоял особенный артефакт, который записывал ход сражения в виде странного набора символов.\\ 
    Которые лишь сам маг был способен понять, здесь и будет приведет этот боевой журнал. \\     \vfill

    \raggedleft
    \textit{«Если на странице стало больше знаков — значит, кто-то из клана Боблина опять что-то натворил.»}\\[0.3cm]

\end{titlepage}
\tableofcontents
\newpage
f(x) = \begin{align*}
\begin{autobreak}
\frac{2}{x^{3}}
\end{autobreak}
\end{align*}

\begin{center}
Небольшой взмах посохом — и план сражения выглядит куда приличнее.
\end{center}
\begin{align*}
\begin{autobreak}
\frac{d}{dx}(\frac{2}{x^{3}}) = \frac{\frac{d}{dx}(2) \cdot x^{3} - 2 \cdot \frac{d}{dx}(x^{3})}{x^{3}^2}
\end{autobreak}
\end{align*}

\begin{center}
Один из гоблинов упал
\end{center}
\begin{align*}
\begin{autobreak}
\frac{d}{dx}(x^{3}) = 3 \cdot x^{3 - 1} \cdot \frac{d}{dx}(x)
\end{autobreak}
\end{align*}

\begin{center}
Вы разложили дядю Боблина на молекулы
\end{center}
\begin{align*}
\begin{autobreak}
\frac{d}{dx}(2) = 0
\end{autobreak}
\end{align*}

f(x)|dx = \begin{align*}
\begin{autobreak}
\frac{0 \cdot x^{3} - 2 \cdot x^{(3 - 1)} \cdot 3}{x^{3} \cdot x^{3}}
\end{autobreak}
\end{align*}

optimized f(x)|dx = \begin{align*}
\begin{autobreak}
\frac{0 - 2 \cdot x^{2} \cdot 3}{x^{3} \cdot x^{3}}
\end{autobreak}
\end{align*}

f(x)` = \begin{align*}
\begin{autobreak}
\frac{0 - 2 \cdot x^{2} \cdot 3}{x^{3} \cdot x^{3}}
\end{autobreak}
\end{align*}

\end{document}
