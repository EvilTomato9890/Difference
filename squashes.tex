\documentclass[a4paper,12pt]{article}
\usepackage[T2A]{fontenc}
\usepackage[utf8]{inputenc}
\usepackage[russian]{babel}
\usepackage{amsmath}
\usepackage{amssymb}
\usepackage{autobreak}
\usepackage{hyperref}
\setcounter{secnumdepth}{0}
\begin{document}
\title{Сокращение рода боблина}
\maketitle
\newpage
\begin{titlepage}
    \centering

    {\Large Уставший волшебник}\\[1cm]

    {\huge\bfseries «Завершение рода Боблина»}\\[0.5cm]

    \raggedright

    \textbf{Предыстория} \\[0.3cm]

    Жил-был самый обычный гоблин по имени Боблин и его очень большая семья. \\ 
    Как-то раз, одним жарким летом они все вместе решили отправиться на пикник.\\ 
    Они нашли великолепную полянку посреди болота: солнышко, зеленая трава, тенеко от непонятно башни, одним словом - благодать. \\ 
    Шел 5-ый час гоблинской пъянки, тут уже нервы волшебника живущего в башне не выдержали. \\ 
    и он решил обрушиить свой праведный гнев не семейство Боблина, истребив некоторую его часть. \\ 
    \vspace{0.8cm}
    \textbf{Боевой журнал}\\[0.3cm]

    В башне стоял особенный артефакт, который записывал ход сражения в виде странного набора символов.\\ 
    Которые лишь сам маг был способен понять, здесь и будет приведет этот боевой журнал. \\     \vfill

    \raggedleft
    \textit{«Если на странице стало больше знаков — значит, кто-то из клана Боблина опять что-то натворил.»}\\[0.3cm]

\end{titlepage}
\tableofcontents
\newpage
f(x) = \begin{align*}
\begin{autobreak}
(\cos(x) \cdot \cos(x) \cdot \cos(x) \cdot \cos(x))^{(\cos(x) \cdot \cos(x))}
\end{autobreak}
\end{align*}

\begin{center}
Небольшой взмах посохом — и план сражения выглядит куда приличнее.
\end{center}
A = \begin{align*}
\begin{autobreak}
(\cos(x) \cdot \cos(x) \cdot \cos(x) \cdot \cos(x))^{(\cos(x) \cdot \cos(x))}
\end{autobreak}
\end{align*}

\begin{align*}
\begin{autobreak}
\frac{d}{dx}(A) = A \cdot ( \frac{d}{dx}(\cos(x) \cdot \cos(x)) \cdot \ln(\cos(x) \cdot \cos(x) \cdot \cos(x) \cdot \cos(x)) + \cos(x) \cdot \cos(x) \cdot \frac{d}{dx}(\cos(x) \cdot \cos(x) \cdot \cos(x) \cdot \cos(x)) / \cos(x) \cdot \cos(x) \cdot \cos(x) \cdot \cos(x) )
\end{autobreak}
\end{align*}

\begin{center}
Один из гоблинов упал
\end{center}
\begin{align*}
\begin{autobreak}
\frac{d}{dx}(\cos(x) \cdot \cos(x) \cdot \cos(x) \cdot \cos(x)) = \frac{d}{dx}(\cos(x) \cdot \cos(x) \cdot \cos(x)) \cdot \cos(x) + \cos(x) \cdot \cos(x) \cdot \cos(x) \cdot \frac{d}{dx}(\cos(x))
\end{autobreak}
\end{align*}

\begin{center}
Родственники Боблина продолжают лезть к вам, держите посох крепче!
\end{center}
\begin{align*}
\begin{autobreak}
\frac{d}{dx}(\cos(x)) = -\sin(x) \cdot \frac{d}{dx}(x)
\end{autobreak}
\end{align*}

\begin{center}
Ваше вошшебство откатило деверя Боблина до младенчества
\end{center}
\begin{align*}
\begin{autobreak}
\frac{d}{dx}(x) = 1
\end{autobreak}
\end{align*}

\begin{center}
Битва продолжается, не теряйте духу, они когда-то, наверное, закончаться!
\end{center}
\begin{align*}
\begin{autobreak}
\frac{d}{dx}(\cos(x) \cdot \cos(x) \cdot \cos(x)) = \frac{d}{dx}(\cos(x) \cdot \cos(x)) \cdot \cos(x) + \cos(x) \cdot \cos(x) \cdot \frac{d}{dx}(\cos(x))
\end{autobreak}
\end{align*}

\begin{center}
Их не становиться меньше, откуда они только лезут?!
\end{center}
\begin{align*}
\begin{autobreak}
\frac{d}{dx}(\cos(x)) = -\sin(x) \cdot \frac{d}{dx}(x)
\end{autobreak}
\end{align*}

\begin{center}
Полиморф сработал отлично: зять Боблина теперь лягушка
\end{center}
\begin{align*}
\begin{autobreak}
\frac{d}{dx}(x) = 1
\end{autobreak}
\end{align*}

\begin{center}
Вот так и рождаются легенды о герое, истребившем половину рода Боблина.
\end{center}
\begin{align*}
\begin{autobreak}
\frac{d}{dx}(\cos(x) \cdot \cos(x)) = \frac{d}{dx}(\cos(x)) \cdot \cos(x) + \cos(x) \cdot \frac{d}{dx}(\cos(x))
\end{autobreak}
\end{align*}

\begin{center}
Небольшой взмах посохом — и план сражения выглядит куда приличнее.
\end{center}
\begin{align*}
\begin{autobreak}
\frac{d}{dx}(\cos(x)) = -\sin(x) \cdot \frac{d}{dx}(x)
\end{autobreak}
\end{align*}

\begin{center}
Поздравляю! От свояка Боблина осталась только полоыина
\end{center}
\begin{align*}
\begin{autobreak}
\frac{d}{dx}(x) = 1
\end{autobreak}
\end{align*}

\begin{center}
Один из гоблинов упал
\end{center}
\begin{align*}
\begin{autobreak}
\frac{d}{dx}(\cos(x)) = -\sin(x) \cdot \frac{d}{dx}(x)
\end{autobreak}
\end{align*}

\begin{center}
Ваше заклинание свернуло невестку Боблина в шарик
\end{center}
\begin{align*}
\begin{autobreak}
\frac{d}{dx}(x) = 1
\end{autobreak}
\end{align*}

\begin{center}
Родственники Боблина продолжают лезть к вам, держите посох крепче!
\end{center}
\begin{align*}
\begin{autobreak}
\frac{d}{dx}(\cos(x) \cdot \cos(x)) = \frac{d}{dx}(\cos(x)) \cdot \cos(x) + \cos(x) \cdot \frac{d}{dx}(\cos(x))
\end{autobreak}
\end{align*}

\begin{center}
Битва продолжается, не теряйте духу, они когда-то, наверное, закончаться!
\end{center}
\begin{align*}
\begin{autobreak}
\frac{d}{dx}(\cos(x)) = -\sin(x) \cdot \frac{d}{dx}(x)
\end{autobreak}
\end{align*}

\begin{center}
ААХХАХААХАХ Гоблин-Боблин
\end{center}
\begin{align*}
\begin{autobreak}
\frac{d}{dx}(x) = 1
\end{autobreak}
\end{align*}

\begin{center}
Их не становиться меньше, откуда они только лезут?!
\end{center}
\begin{align*}
\begin{autobreak}
\frac{d}{dx}(\cos(x)) = -\sin(x) \cdot \frac{d}{dx}(x)
\end{autobreak}
\end{align*}

\begin{center}
Ваше вошшебство откатило деверя Боблина до младенчества
\end{center}
\begin{align*}
\begin{autobreak}
\frac{d}{dx}(x) = 1
\end{autobreak}
\end{align*}

f(x)|dx = \begin{align*}
\begin{autobreak}
(\cos(x) \cdot \cos(x) \cdot \cos(x) \cdot \cos(x))^{(\cos(x) \cdot \cos(x))} \cdot (((0 - \sin(x)) \cdot 1 \cdot \cos(x) + \cos(x) \cdot (0 - \sin(x)) \cdot 1) \cdot \ln(\cos(x) \cdot \cos(x) \cdot \cos(x) \cdot \cos(x)) + ((((0 - \sin(x)) \cdot 1 \cdot \cos(x) + \cos(x) \cdot (0 - \sin(x)) \cdot 1) \cdot \cos(x) + \cos(x) \cdot \cos(x) \cdot (0 - \sin(x)) \cdot 1) \cdot \cos(x) + \cos(x) \cdot \cos(x) \cdot \cos(x) \cdot (0 - \sin(x)) \cdot 1) \cdot \frac{\cos(x) \cdot \cos(x)}{\cos(x) \cdot \cos(x) \cdot \cos(x) \cdot \cos(x)})
\end{autobreak}
\end{align*}

A = \begin{align*}
\begin{autobreak}
((0 - \sin(x)) \cdot 1 \cdot \cos(x) + \cos(x) \cdot (0 - \sin(x)) \cdot 1) \cdot \cos(x)
\end{autobreak}
\end{align*}

B = \begin{align*}
\begin{autobreak}
(0 - \sin(x)) \cdot 1 \cdot \cos(x) + \cos(x) \cdot (0 - \sin(x)) \cdot 1
\end{autobreak}
\end{align*}

C = \begin{align*}
\begin{autobreak}
\cos(x) \cdot \cos(x) \cdot \cos(x) \cdot (0 - \sin(x)) \cdot 1
\end{autobreak}
\end{align*}

D = \begin{align*}
\begin{autobreak}
(\cos(x) \cdot \cos(x) \cdot \cos(x) \cdot \cos(x))^{(\cos(x) \cdot \cos(x))}
\end{autobreak}
\end{align*}

E = \begin{align*}
\begin{autobreak}
\cos(x) \cdot \cos(x) \cdot (0 - \sin(x)) \cdot 1
\end{autobreak}
\end{align*}

F = \begin{align*}
\begin{autobreak}
\ln(\cos(x) \cdot \cos(x) \cdot \cos(x) \cdot \cos(x))
\end{autobreak}
\end{align*}

f(x)|dx with squashes = \begin{align*}
\begin{autobreak}
D \cdot (B \cdot F + ((A + E) \cdot \cos(x) + C) \cdot \frac{\cos(x) \cdot \cos(x)}{\cos(x) \cdot \cos(x) \cdot \cos(x) \cdot \cos(x)})
\end{autobreak}
\end{align*}

\end{document}
