\documentclass[a4paper,12pt]{article}
\usepackage[T2A]{fontenc}
\usepackage[utf8]{inputenc}
\usepackage[russian]{babel}
\usepackage{amsmath}
\usepackage{amssymb}
\usepackage{autobreak}
\usepackage{hyperref}
\usepackage{graphicx}
\setcounter{secnumdepth}{0}
\begin{document}
\title{Сокращение рода боблина}
\maketitle
\newpage
\begin{titlepage}
    \centering

    {\Large Уставший волшебник}\\[1cm]

    {\huge\bfseries «Завершение рода Боблина»}\\[0.5cm]

    \raggedright

    \textbf{Предыстория} \\[0.3cm]

    Жил-был самый обычный гоблин по имени Боблин и его очень большая семья. \\ 
    Как-то раз, одним жарким летом они все вместе решили отправиться на пикник.\\ 
    Они нашли великолепную полянку посреди болота: солнышко, зеленая трава, тенеко от непонятно башни, одним словом - благодать. \\ 
    Шел 5-ый час гоблинской пъянки, тут уже нервы волшебника живущего в башне не выдержали. \\ 
    и он решил обрушиить свой праведный гнев не семейство Боблина, истребив некоторую его часть. \\ 
    \vspace{0.8cm}
    \textbf{Боевой журнал}\\[0.3cm]

    В башне стоял особенный артефакт, который записывал ход сражения в виде странного набора символов.\\ 
    Которые лишь сам маг был способен понять, здесь и будет приведет этот боевой журнал. \\     \vfill

    \raggedleft
    \textit{«Если на странице стало больше знаков — значит, кто-то из клана Боблина опять что-то натворил.»}\\[0.3cm]

\end{titlepage}
\tableofcontents
\newpage
\begin{center}
f(x) = \begin{math}
\cos(x^{2}) + 3^{x}
\end{math}\par\vspace{2em}
\end{center}
\section{Прибывает Тейлор и куча дальних родственников}
\begin{center}
Текущий ход событий: \begin{math}
\cos(x^{2}) + 3^{x}
\end{math}\par\vspace{2em}
\end{center}
\subsection{Прибывает 1-ая волна родственников Тейлора-Боблина}
\begin{center}
Текущий ход событий: \begin{math}
\cos(x^{2}) + 3^{x}
\end{math}\par\vspace{2em}
\end{center}
\noindent\hrulefill\begin{center}
Вот так и рождаются легенды о герое, истребившем половину рода Боблина.
\end{center}
\begin{center}
\begin{math}
\frac{d}{dx}(\cos(x^{2}) + 3^{x}) = \frac{d}{dx}(\cos(x^{2})) + \frac{d}{dx}(3^{x})
\end{math}\par\vspace{2em}
\end{center}
\begin{center}
Небольшой взмах посохом — и план сражения выглядит куда приличнее.
\end{center}
\begin{center}
\begin{math}
\frac{d}{dx}(3^{x}) = 3^{x} \cdot \ln(3) \cdot \frac{d}{dx}(x)
\end{math}\par\vspace{2em}
\end{center}
\begin{center}
Полиморф сработал отлично: зять Боблина теперь лягушка
\end{center}
\begin{center}
\begin{math}
\frac{d}{dx}(x) = 1
\end{math}\par\vspace{2em}
\end{center}
\begin{center}
Один из гоблинов упал
\end{center}
\begin{center}
\begin{math}
\frac{d}{dx}(\cos(x^{2})) = -\sin(x^{2}) \cdot \frac{d}{dx}(x^{2})
\end{math}\par\vspace{2em}
\end{center}
\begin{center}
Родственники Боблина продолжают лезть к вам, держите посох крепче!
\end{center}
\begin{center}
\begin{math}
\frac{d}{dx}(x^{2}) = 2 \cdot {x}^{2 - 1} \cdot \frac{d}{dx}(x)
\end{math}\par\vspace{2em}
\end{center}
\subsection{Прибывает 2-ая волна родственников Тейлора-Боблина}
\begin{center}
Текущий ход событий: \begin{math}
(0 - \sin(x^{2})) \cdot x \cdot 2 + 3^{x} \cdot 1.09861
\end{math}\par\vspace{2em}
\end{center}
\noindent\hrulefill\begin{center}
Битва продолжается, не теряйте духу, они когда-то, наверное, закончаться!
\end{center}
\begin{center}
A = \begin{math}
(0 - \sin(x^{2})) \cdot x \cdot 2 + 3^{x} \cdot 1.09861
\end{math}\par\vspace{2em}
\end{center}
\begin{center}
\begin{math}
\frac{d}{dx}(A) = \frac{d}{dx}((0 - \sin(x^{2})) \cdot x \cdot 2) + \frac{d}{dx}(3^{x} \cdot 1.09861)
\end{math}\par\vspace{2em}
\end{center}
\begin{center}
Их не становиться меньше, откуда они только лезут?!
\end{center}
\begin{center}
\begin{math}
\frac{d}{dx}(3^{x} \cdot 1.09861) = \frac{d}{dx}(3^{x}) \cdot 1.09861 + 3^{x} \cdot \frac{d}{dx}(1.09861)
\end{math}\par\vspace{2em}
\end{center}
\begin{center}
Вы разложили дядю Боблина на молекулы
\end{center}
\begin{center}
\begin{math}
\frac{d}{dx}(1.09861) = 0
\end{math}\par\vspace{2em}
\end{center}
\begin{center}
Вот так и рождаются легенды о герое, истребившем половину рода Боблина.
\end{center}
\begin{center}
\begin{math}
\frac{d}{dx}(3^{x}) = 3^{x} \cdot \ln(3) \cdot \frac{d}{dx}(x)
\end{math}\par\vspace{2em}
\end{center}
\begin{center}
Поздравляю! От свояка Боблина осталась только полоыина
\end{center}
\begin{center}
\begin{math}
\frac{d}{dx}(x) = 1
\end{math}\par\vspace{2em}
\end{center}
\begin{center}
Небольшой взмах посохом — и план сражения выглядит куда приличнее.
\end{center}
\begin{center}
\begin{math}
\frac{d}{dx}((0 - \sin(x^{2})) \cdot x \cdot 2) = \frac{d}{dx}(0 - \sin(x^{2})) \cdot x \cdot 2 + 0 - \sin(x^{2}) \cdot \frac{d}{dx}(x \cdot 2)
\end{math}\par\vspace{2em}
\end{center}
\begin{center}
Один из гоблинов упал
\end{center}
\begin{center}
\begin{math}
\frac{d}{dx}(x \cdot 2) = \frac{d}{dx}(x) \cdot 2 + x \cdot \frac{d}{dx}(2)
\end{math}\par\vspace{2em}
\end{center}
\begin{center}
Ваше волшебство оказалось не по зубам тёте Боблина, кстати, куда она делась?
\end{center}
\begin{center}
\begin{math}
\frac{d}{dx}(2) = 0
\end{math}\par\vspace{2em}
\end{center}
\begin{center}
Ваше заклинание свернуло невестку Боблина в шарик
\end{center}
\begin{center}
\begin{math}
\frac{d}{dx}(x) = 1
\end{math}\par\vspace{2em}
\end{center}
\begin{center}
Родственники Боблина продолжают лезть к вам, держите посох крепче!
\end{center}
\begin{center}
\begin{math}
\frac{d}{dx}(0 - \sin(x^{2})) = \frac{d}{dx}(0) - \frac{d}{dx}(\sin(x^{2}))
\end{math}\par\vspace{2em}
\end{center}
\begin{center}
Битва продолжается, не теряйте духу, они когда-то, наверное, закончаться!
\end{center}
\begin{center}
\begin{math}
\frac{d}{dx}(\sin(x^{2})) = \cos(x^{2}) \cdot \frac{d}{dx}(x^{2})
\end{math}\par\vspace{2em}
\end{center}
\begin{center}
Их не становиться меньше, откуда они только лезут?!
\end{center}
\begin{center}
\begin{math}
\frac{d}{dx}(x^{2}) = 2 \cdot {x}^{2 - 1} \cdot \frac{d}{dx}(x)
\end{math}\par\vspace{2em}
\end{center}
\begin{center}
Огненный шар испарил бабушку Боблина
\end{center}
\begin{center}
\begin{math}
\frac{d}{dx}(0) = 0
\end{math}\par\vspace{2em}
\end{center}
\subsection{Прибывает 3-ая волна родственников Тейлора-Боблина}
\begin{center}
A = \begin{math}
(0 - \cos(x^{2}) \cdot x \cdot 2) \cdot x \cdot 2 + (0 - \sin(x^{2})) \cdot 2
\end{math}\par\vspace{2em}
\end{center}
\begin{center}
Текущий ход событий: \begin{math}
A + 3^{x} \cdot 1.09861 \cdot 1.09861
\end{math}\par\vspace{2em}
\end{center}
\noindent\hrulefill\begin{center}
Вот так и рождаются легенды о герое, истребившем половину рода Боблина.
\end{center}
\begin{center}
A = \begin{math}
(0 - \cos(x^{2}) \cdot x \cdot 2) \cdot x \cdot 2 + (0 - \sin(x^{2})) \cdot 2
\end{math}\par\vspace{2em}
\end{center}
\begin{center}
\begin{math}
\frac{d}{dx}(A + 3^{x} \cdot 1.09861 \cdot 1.09861) = \frac{d}{dx}((0 - \cos(x^{2}) \cdot x \cdot 2) \cdot x \cdot 2 + (0 - \sin(x^{2})) \cdot 2) + \frac{d}{dx}(3^{x} \cdot 1.09861 \cdot 1.09861)
\end{math}\par\vspace{2em}
\end{center}
\begin{center}
Небольшой взмах посохом — и план сражения выглядит куда приличнее.
\end{center}
\begin{center}
\begin{math}
\frac{d}{dx}(3^{x} \cdot 1.09861 \cdot 1.09861) = \frac{d}{dx}(3^{x} \cdot 1.09861) \cdot 1.09861 + 3^{x} \cdot 1.09861 \cdot \frac{d}{dx}(1.09861)
\end{math}\par\vspace{2em}
\end{center}
\begin{center}
Заклинание хаоса раскидало части племянника Боблина по разным планам
\end{center}
\begin{center}
\begin{math}
\frac{d}{dx}(1.09861) = 0
\end{math}\par\vspace{2em}
\end{center}
\begin{center}
Один из гоблинов упал
\end{center}
\begin{center}
\begin{math}
\frac{d}{dx}(3^{x} \cdot 1.09861) = \frac{d}{dx}(3^{x}) \cdot 1.09861 + 3^{x} \cdot \frac{d}{dx}(1.09861)
\end{math}\par\vspace{2em}
\end{center}
\begin{center}
Ваш портал небытия вежливо удалил тещу Боблина из этого измерения
\end{center}
\begin{center}
\begin{math}
\frac{d}{dx}(1.09861) = 0
\end{math}\par\vspace{2em}
\end{center}
\begin{center}
Родственники Боблина продолжают лезть к вам, держите посох крепче!
\end{center}
\begin{center}
\begin{math}
\frac{d}{dx}(3^{x}) = 3^{x} \cdot \ln(3) \cdot \frac{d}{dx}(x)
\end{math}\par\vspace{2em}
\end{center}
\begin{center}
ААХХАХААХАХ Гоблин-Боблин
\end{center}
\begin{center}
\begin{math}
\frac{d}{dx}(x) = 1
\end{math}\par\vspace{2em}
\end{center}
\begin{center}
Битва продолжается, не теряйте духу, они когда-то, наверное, закончаться!
\end{center}
\begin{center}
A = \begin{math}
(0 - \cos(x^{2}) \cdot x \cdot 2) \cdot x \cdot 2 + (0 - \sin(x^{2})) \cdot 2
\end{math}\par\vspace{2em}
\end{center}
\begin{center}
\begin{math}
\frac{d}{dx}(A) = \frac{d}{dx}((0 - \cos(x^{2}) \cdot x \cdot 2) \cdot x \cdot 2) + \frac{d}{dx}((0 - \sin(x^{2})) \cdot 2)
\end{math}\par\vspace{2em}
\end{center}
\begin{center}
Их не становиться меньше, откуда они только лезут?!
\end{center}
\begin{center}
\begin{math}
\frac{d}{dx}((0 - \sin(x^{2})) \cdot 2) = \frac{d}{dx}(0 - \sin(x^{2})) \cdot 2 + 0 - \sin(x^{2}) \cdot \frac{d}{dx}(2)
\end{math}\par\vspace{2em}
\end{center}
\begin{center}
Ваше заклинание дезинтегрировало брата Боблина
\end{center}
\begin{center}
\begin{math}
\frac{d}{dx}(2) = 0
\end{math}\par\vspace{2em}
\end{center}
\begin{center}
Вот так и рождаются легенды о герое, истребившем половину рода Боблина.
\end{center}
\begin{center}
\begin{math}
\frac{d}{dx}(0 - \sin(x^{2})) = \frac{d}{dx}(0) - \frac{d}{dx}(\sin(x^{2}))
\end{math}\par\vspace{2em}
\end{center}
\begin{center}
Небольшой взмах посохом — и план сражения выглядит куда приличнее.
\end{center}
\begin{center}
\begin{math}
\frac{d}{dx}(\sin(x^{2})) = \cos(x^{2}) \cdot \frac{d}{dx}(x^{2})
\end{math}\par\vspace{2em}
\end{center}
\begin{center}
Один из гоблинов упал
\end{center}
\begin{center}
\begin{math}
\frac{d}{dx}(x^{2}) = 2 \cdot {x}^{2 - 1} \cdot \frac{d}{dx}(x)
\end{math}\par\vspace{2em}
\end{center}
\begin{center}
Ваше колдовство низвело сестру Боблина до атомов
\end{center}
\begin{center}
\begin{math}
\frac{d}{dx}(0) = 0
\end{math}\par\vspace{2em}
\end{center}
\begin{center}
Родственники Боблина продолжают лезть к вам, держите посох крепче!
\end{center}
\begin{center}
A = \begin{math}
(0 - \cos(x^{2}) \cdot x \cdot 2) \cdot x \cdot 2
\end{math}\par\vspace{2em}
\end{center}
\begin{center}
\begin{math}
\frac{d}{dx}(A) = \frac{d}{dx}(0 - \cos(x^{2}) \cdot x \cdot 2) \cdot x \cdot 2 + 0 - \cos(x^{2}) \cdot x \cdot 2 \cdot \frac{d}{dx}(x \cdot 2)
\end{math}\par\vspace{2em}
\end{center}
\begin{center}
Битва продолжается, не теряйте духу, они когда-то, наверное, закончаться!
\end{center}
\begin{center}
\begin{math}
\frac{d}{dx}(x \cdot 2) = \frac{d}{dx}(x) \cdot 2 + x \cdot \frac{d}{dx}(2)
\end{math}\par\vspace{2em}
\end{center}
\begin{center}
Вы разложили дядю Боблина на молекулы
\end{center}
\begin{center}
\begin{math}
\frac{d}{dx}(2) = 0
\end{math}\par\vspace{2em}
\end{center}
\begin{center}
Ваше вошшебство откатило деверя Боблина до младенчества
\end{center}
\begin{center}
\begin{math}
\frac{d}{dx}(x) = 1
\end{math}\par\vspace{2em}
\end{center}
\begin{center}
Их не становиться меньше, откуда они только лезут?!
\end{center}
\begin{center}
\begin{math}
\frac{d}{dx}(0 - \cos(x^{2}) \cdot x \cdot 2) = \frac{d}{dx}(0) - \frac{d}{dx}(\cos(x^{2}) \cdot x \cdot 2)
\end{math}\par\vspace{2em}
\end{center}
\begin{center}
Вот так и рождаются легенды о герое, истребившем половину рода Боблина.
\end{center}
\begin{center}
\begin{math}
\frac{d}{dx}(\cos(x^{2}) \cdot x \cdot 2) = \frac{d}{dx}(\cos(x^{2})) \cdot x \cdot 2 + \cos(x^{2}) \cdot \frac{d}{dx}(x \cdot 2)
\end{math}\par\vspace{2em}
\end{center}
\begin{center}
Небольшой взмах посохом — и план сражения выглядит куда приличнее.
\end{center}
\begin{center}
\begin{math}
\frac{d}{dx}(x \cdot 2) = \frac{d}{dx}(x) \cdot 2 + x \cdot \frac{d}{dx}(2)
\end{math}\par\vspace{2em}
\end{center}
\begin{center}
Ваше волшебство оказалось не по зубам тёте Боблина, кстати, куда она делась?
\end{center}
\begin{center}
\begin{math}
\frac{d}{dx}(2) = 0
\end{math}\par\vspace{2em}
\end{center}
\begin{center}
Полиморф сработал отлично: зять Боблина теперь лягушка
\end{center}
\begin{center}
\begin{math}
\frac{d}{dx}(x) = 1
\end{math}\par\vspace{2em}
\end{center}
\begin{center}
Один из гоблинов упал
\end{center}
\begin{center}
\begin{math}
\frac{d}{dx}(\cos(x^{2})) = -\sin(x^{2}) \cdot \frac{d}{dx}(x^{2})
\end{math}\par\vspace{2em}
\end{center}
\begin{center}
Родственники Боблина продолжают лезть к вам, держите посох крепче!
\end{center}
\begin{center}
\begin{math}
\frac{d}{dx}(x^{2}) = 2 \cdot {x}^{2 - 1} \cdot \frac{d}{dx}(x)
\end{math}\par\vspace{2em}
\end{center}
\begin{center}
Огненный шар испарил бабушку Боблина
\end{center}
\begin{center}
\begin{math}
\frac{d}{dx}(0) = 0
\end{math}\par\vspace{2em}
\end{center}
\subsection{Прибывает 4-ая волна родственников Тейлора-Боблина}
\begin{center}
A = \begin{math}
0 - ((0 - \sin(x^{2})) \cdot x \cdot 2 \cdot x \cdot 2 + \cos(x^{2}) \cdot 2)
\end{math}\par\vspace{2em}
\end{center}
\begin{center}
B = \begin{math}
(0 - \cos(x^{2}) \cdot x \cdot 2) \cdot 2
\end{math}\par\vspace{2em}
\end{center}
\begin{center}
C = \begin{math}
(0 - \cos(x^{2}) \cdot x \cdot 2) \cdot 2
\end{math}\par\vspace{2em}
\end{center}
\begin{center}
Текущий ход событий: \begin{math}
A \cdot x \cdot 2 + B + C + 3^{x} \cdot 1.09861 \cdot 1.09861 \cdot 1.09861
\end{math}\par\vspace{2em}
\end{center}
\noindent\hrulefill\begin{center}
Битва продолжается, не теряйте духу, они когда-то, наверное, закончаться!
\end{center}
\begin{center}
A = \begin{math}
0 - ((0 - \sin(x^{2})) \cdot x \cdot 2 \cdot x \cdot 2 + \cos(x^{2}) \cdot 2)
\end{math}\par\vspace{2em}
\end{center}
\begin{center}
B = \begin{math}
(0 - \cos(x^{2}) \cdot x \cdot 2) \cdot 2
\end{math}\par\vspace{2em}
\end{center}
\begin{center}
C = \begin{math}
(0 - \cos(x^{2}) \cdot x \cdot 2) \cdot 2
\end{math}\par\vspace{2em}
\end{center}
\begin{center}
D = \begin{math}
3^{x} \cdot 1.09861 \cdot 1.09861 \cdot 1.09861
\end{math}\par\vspace{2em}
\end{center}
\begin{center}
\begin{math}
\frac{d}{dx}(A \cdot x \cdot 2 + B + C + D) = \frac{d}{dx}((0 - ((0 - \sin(x^{2})) \cdot x \cdot 2 \cdot x \cdot 2 + \cos(x^{2}) \cdot 2)) \cdot x \cdot 2 + (0 - \cos(x^{2}) \cdot x \cdot 2) \cdot 2 + (0 - \cos(x^{2}) \cdot x \cdot 2) \cdot 2) + \frac{d}{dx}(3^{x} \cdot 1.09861 \cdot 1.09861 \cdot 1.09861)
\end{math}\par\vspace{2em}
\end{center}
\begin{center}
Их не становиться меньше, откуда они только лезут?!
\end{center}
\begin{center}
\begin{math}
\frac{d}{dx}(3^{x} \cdot 1.09861 \cdot 1.09861 \cdot 1.09861) = \frac{d}{dx}(3^{x} \cdot 1.09861 \cdot 1.09861) \cdot 1.09861 + 3^{x} \cdot 1.09861 \cdot 1.09861 \cdot \frac{d}{dx}(1.09861)
\end{math}\par\vspace{2em}
\end{center}
\begin{center}
Заклинание хаоса раскидало части племянника Боблина по разным планам
\end{center}
\begin{center}
\begin{math}
\frac{d}{dx}(1.09861) = 0
\end{math}\par\vspace{2em}
\end{center}
\begin{center}
Вот так и рождаются легенды о герое, истребившем половину рода Боблина.
\end{center}
\begin{center}
\begin{math}
\frac{d}{dx}(3^{x} \cdot 1.09861 \cdot 1.09861) = \frac{d}{dx}(3^{x} \cdot 1.09861) \cdot 1.09861 + 3^{x} \cdot 1.09861 \cdot \frac{d}{dx}(1.09861)
\end{math}\par\vspace{2em}
\end{center}
\begin{center}
Ваш портал небытия вежливо удалил тещу Боблина из этого измерения
\end{center}
\begin{center}
\begin{math}
\frac{d}{dx}(1.09861) = 0
\end{math}\par\vspace{2em}
\end{center}
\begin{center}
Небольшой взмах посохом — и план сражения выглядит куда приличнее.
\end{center}
\begin{center}
\begin{math}
\frac{d}{dx}(3^{x} \cdot 1.09861) = \frac{d}{dx}(3^{x}) \cdot 1.09861 + 3^{x} \cdot \frac{d}{dx}(1.09861)
\end{math}\par\vspace{2em}
\end{center}
\begin{center}
Ваше заклинание дезинтегрировало брата Боблина
\end{center}
\begin{center}
\begin{math}
\frac{d}{dx}(1.09861) = 0
\end{math}\par\vspace{2em}
\end{center}
\begin{center}
Один из гоблинов упал
\end{center}
\begin{center}
\begin{math}
\frac{d}{dx}(3^{x}) = 3^{x} \cdot \ln(3) \cdot \frac{d}{dx}(x)
\end{math}\par\vspace{2em}
\end{center}
\begin{center}
Поздравляю! От свояка Боблина осталась только полоыина
\end{center}
\begin{center}
\begin{math}
\frac{d}{dx}(x) = 1
\end{math}\par\vspace{2em}
\end{center}
\begin{center}
Родственники Боблина продолжают лезть к вам, держите посох крепче!
\end{center}
\begin{center}
A = \begin{math}
0 - ((0 - \sin(x^{2})) \cdot x \cdot 2 \cdot x \cdot 2 + \cos(x^{2}) \cdot 2)
\end{math}\par\vspace{2em}
\end{center}
\begin{center}
B = \begin{math}
(0 - \cos(x^{2}) \cdot x \cdot 2) \cdot 2
\end{math}\par\vspace{2em}
\end{center}
\begin{center}
C = \begin{math}
(0 - \cos(x^{2}) \cdot x \cdot 2) \cdot 2
\end{math}\par\vspace{2em}
\end{center}
\begin{center}
\begin{math}
\frac{d}{dx}(A \cdot x \cdot 2 + B + C) = \frac{d}{dx}((0 - ((0 - \sin(x^{2})) \cdot x \cdot 2 \cdot x \cdot 2 + \cos(x^{2}) \cdot 2)) \cdot x \cdot 2 + (0 - \cos(x^{2}) \cdot x \cdot 2) \cdot 2) + \frac{d}{dx}((0 - \cos(x^{2}) \cdot x \cdot 2) \cdot 2)
\end{math}\par\vspace{2em}
\end{center}
\begin{center}
Битва продолжается, не теряйте духу, они когда-то, наверное, закончаться!
\end{center}
\begin{center}
A = \begin{math}
(0 - \cos(x^{2}) \cdot x \cdot 2) \cdot 2
\end{math}\par\vspace{2em}
\end{center}
\begin{center}
\begin{math}
\frac{d}{dx}(A) = \frac{d}{dx}(0 - \cos(x^{2}) \cdot x \cdot 2) \cdot 2 + 0 - \cos(x^{2}) \cdot x \cdot 2 \cdot \frac{d}{dx}(2)
\end{math}\par\vspace{2em}
\end{center}
\begin{center}
Ваше колдовство низвело сестру Боблина до атомов
\end{center}
\begin{center}
\begin{math}
\frac{d}{dx}(2) = 0
\end{math}\par\vspace{2em}
\end{center}
\begin{center}
Их не становиться меньше, откуда они только лезут?!
\end{center}
\begin{center}
\begin{math}
\frac{d}{dx}(0 - \cos(x^{2}) \cdot x \cdot 2) = \frac{d}{dx}(0) - \frac{d}{dx}(\cos(x^{2}) \cdot x \cdot 2)
\end{math}\par\vspace{2em}
\end{center}
\begin{center}
Вот так и рождаются легенды о герое, истребившем половину рода Боблина.
\end{center}
\begin{center}
\begin{math}
\frac{d}{dx}(\cos(x^{2}) \cdot x \cdot 2) = \frac{d}{dx}(\cos(x^{2})) \cdot x \cdot 2 + \cos(x^{2}) \cdot \frac{d}{dx}(x \cdot 2)
\end{math}\par\vspace{2em}
\end{center}
\begin{center}
Небольшой взмах посохом — и план сражения выглядит куда приличнее.
\end{center}
\begin{center}
\begin{math}
\frac{d}{dx}(x \cdot 2) = \frac{d}{dx}(x) \cdot 2 + x \cdot \frac{d}{dx}(2)
\end{math}\par\vspace{2em}
\end{center}
\begin{center}
Вы разложили дядю Боблина на молекулы
\end{center}
\begin{center}
\begin{math}
\frac{d}{dx}(2) = 0
\end{math}\par\vspace{2em}
\end{center}
\begin{center}
Ваше заклинание свернуло невестку Боблина в шарик
\end{center}
\begin{center}
\begin{math}
\frac{d}{dx}(x) = 1
\end{math}\par\vspace{2em}
\end{center}
\begin{center}
Один из гоблинов упал
\end{center}
\begin{center}
\begin{math}
\frac{d}{dx}(\cos(x^{2})) = -\sin(x^{2}) \cdot \frac{d}{dx}(x^{2})
\end{math}\par\vspace{2em}
\end{center}
\begin{center}
Родственники Боблина продолжают лезть к вам, держите посох крепче!
\end{center}
\begin{center}
\begin{math}
\frac{d}{dx}(x^{2}) = 2 \cdot {x}^{2 - 1} \cdot \frac{d}{dx}(x)
\end{math}\par\vspace{2em}
\end{center}
\begin{center}
Ваше волшебство оказалось не по зубам тёте Боблина, кстати, куда она делась?
\end{center}
\begin{center}
\begin{math}
\frac{d}{dx}(0) = 0
\end{math}\par\vspace{2em}
\end{center}
\begin{center}
Битва продолжается, не теряйте духу, они когда-то, наверное, закончаться!
\end{center}
\begin{center}
A = \begin{math}
0 - ((0 - \sin(x^{2})) \cdot x \cdot 2 \cdot x \cdot 2 + \cos(x^{2}) \cdot 2)
\end{math}\par\vspace{2em}
\end{center}
\begin{center}
B = \begin{math}
(0 - \cos(x^{2}) \cdot x \cdot 2) \cdot 2
\end{math}\par\vspace{2em}
\end{center}
\begin{center}
\begin{math}
\frac{d}{dx}(A \cdot x \cdot 2 + B) = \frac{d}{dx}((0 - ((0 - \sin(x^{2})) \cdot x \cdot 2 \cdot x \cdot 2 + \cos(x^{2}) \cdot 2)) \cdot x \cdot 2) + \frac{d}{dx}((0 - \cos(x^{2}) \cdot x \cdot 2) \cdot 2)
\end{math}\par\vspace{2em}
\end{center}
\begin{center}
Их не становиться меньше, откуда они только лезут?!
\end{center}
\begin{center}
A = \begin{math}
(0 - \cos(x^{2}) \cdot x \cdot 2) \cdot 2
\end{math}\par\vspace{2em}
\end{center}
\begin{center}
\begin{math}
\frac{d}{dx}(A) = \frac{d}{dx}(0 - \cos(x^{2}) \cdot x \cdot 2) \cdot 2 + 0 - \cos(x^{2}) \cdot x \cdot 2 \cdot \frac{d}{dx}(2)
\end{math}\par\vspace{2em}
\end{center}
\begin{center}
Огненный шар испарил бабушку Боблина
\end{center}
\begin{center}
\begin{math}
\frac{d}{dx}(2) = 0
\end{math}\par\vspace{2em}
\end{center}
\begin{center}
Вот так и рождаются легенды о герое, истребившем половину рода Боблина.
\end{center}
\begin{center}
\begin{math}
\frac{d}{dx}(0 - \cos(x^{2}) \cdot x \cdot 2) = \frac{d}{dx}(0) - \frac{d}{dx}(\cos(x^{2}) \cdot x \cdot 2)
\end{math}\par\vspace{2em}
\end{center}
\begin{center}
Небольшой взмах посохом — и план сражения выглядит куда приличнее.
\end{center}
\begin{center}
\begin{math}
\frac{d}{dx}(\cos(x^{2}) \cdot x \cdot 2) = \frac{d}{dx}(\cos(x^{2})) \cdot x \cdot 2 + \cos(x^{2}) \cdot \frac{d}{dx}(x \cdot 2)
\end{math}\par\vspace{2em}
\end{center}
\begin{center}
Один из гоблинов упал
\end{center}
\begin{center}
\begin{math}
\frac{d}{dx}(x \cdot 2) = \frac{d}{dx}(x) \cdot 2 + x \cdot \frac{d}{dx}(2)
\end{math}\par\vspace{2em}
\end{center}
\begin{center}
Заклинание хаоса раскидало части племянника Боблина по разным планам
\end{center}
\begin{center}
\begin{math}
\frac{d}{dx}(2) = 0
\end{math}\par\vspace{2em}
\end{center}
\begin{center}
ААХХАХААХАХ Гоблин-Боблин
\end{center}
\begin{center}
\begin{math}
\frac{d}{dx}(x) = 1
\end{math}\par\vspace{2em}
\end{center}
\begin{center}
Родственники Боблина продолжают лезть к вам, держите посох крепче!
\end{center}
\begin{center}
\begin{math}
\frac{d}{dx}(\cos(x^{2})) = -\sin(x^{2}) \cdot \frac{d}{dx}(x^{2})
\end{math}\par\vspace{2em}
\end{center}
\begin{center}
Битва продолжается, не теряйте духу, они когда-то, наверное, закончаться!
\end{center}
\begin{center}
\begin{math}
\frac{d}{dx}(x^{2}) = 2 \cdot {x}^{2 - 1} \cdot \frac{d}{dx}(x)
\end{math}\par\vspace{2em}
\end{center}
\begin{center}
Ваш портал небытия вежливо удалил тещу Боблина из этого измерения
\end{center}
\begin{center}
\begin{math}
\frac{d}{dx}(0) = 0
\end{math}\par\vspace{2em}
\end{center}
\begin{center}
Их не становиться меньше, откуда они только лезут?!
\end{center}
\begin{center}
A = \begin{math}
0 - ((0 - \sin(x^{2})) \cdot x \cdot 2 \cdot x \cdot 2 + \cos(x^{2}) \cdot 2)
\end{math}\par\vspace{2em}
\end{center}
\begin{center}
\begin{math}
\frac{d}{dx}(A \cdot x \cdot 2) = \frac{d}{dx}(0 - ((0 - \sin(x^{2})) \cdot x \cdot 2 \cdot x \cdot 2 + \cos(x^{2}) \cdot 2)) \cdot x \cdot 2 + 0 - ((0 - \sin(x^{2})) \cdot x \cdot 2 \cdot x \cdot 2 + \cos(x^{2}) \cdot 2) \cdot \frac{d}{dx}(x \cdot 2)
\end{math}\par\vspace{2em}
\end{center}
\begin{center}
Вот так и рождаются легенды о герое, истребившем половину рода Боблина.
\end{center}
\begin{center}
\begin{math}
\frac{d}{dx}(x \cdot 2) = \frac{d}{dx}(x) \cdot 2 + x \cdot \frac{d}{dx}(2)
\end{math}\par\vspace{2em}
\end{center}
\begin{center}
Ваше заклинание дезинтегрировало брата Боблина
\end{center}
\begin{center}
\begin{math}
\frac{d}{dx}(2) = 0
\end{math}\par\vspace{2em}
\end{center}
\begin{center}
Ваше вошшебство откатило деверя Боблина до младенчества
\end{center}
\begin{center}
\begin{math}
\frac{d}{dx}(x) = 1
\end{math}\par\vspace{2em}
\end{center}
\begin{center}
Небольшой взмах посохом — и план сражения выглядит куда приличнее.
\end{center}
\begin{center}
A = \begin{math}
0 - ((0 - \sin(x^{2})) \cdot x \cdot 2 \cdot x \cdot 2 + \cos(x^{2}) \cdot 2)
\end{math}\par\vspace{2em}
\end{center}
\begin{center}
\begin{math}
\frac{d}{dx}(A) = \frac{d}{dx}(0) - \frac{d}{dx}((0 - \sin(x^{2})) \cdot x \cdot 2 \cdot x \cdot 2 + \cos(x^{2}) \cdot 2)
\end{math}\par\vspace{2em}
\end{center}
\begin{center}
Один из гоблинов упал
\end{center}
\begin{center}
A = \begin{math}
(0 - \sin(x^{2})) \cdot x \cdot 2 \cdot x \cdot 2 + \cos(x^{2}) \cdot 2
\end{math}\par\vspace{2em}
\end{center}
\begin{center}
\begin{math}
\frac{d}{dx}(A) = \frac{d}{dx}((0 - \sin(x^{2})) \cdot x \cdot 2 \cdot x \cdot 2) + \frac{d}{dx}(\cos(x^{2}) \cdot 2)
\end{math}\par\vspace{2em}
\end{center}
\begin{center}
Родственники Боблина продолжают лезть к вам, держите посох крепче!
\end{center}
\begin{center}
\begin{math}
\frac{d}{dx}(\cos(x^{2}) \cdot 2) = \frac{d}{dx}(\cos(x^{2})) \cdot 2 + \cos(x^{2}) \cdot \frac{d}{dx}(2)
\end{math}\par\vspace{2em}
\end{center}
\begin{center}
Ваше колдовство низвело сестру Боблина до атомов
\end{center}
\begin{center}
\begin{math}
\frac{d}{dx}(2) = 0
\end{math}\par\vspace{2em}
\end{center}
\begin{center}
Битва продолжается, не теряйте духу, они когда-то, наверное, закончаться!
\end{center}
\begin{center}
\begin{math}
\frac{d}{dx}(\cos(x^{2})) = -\sin(x^{2}) \cdot \frac{d}{dx}(x^{2})
\end{math}\par\vspace{2em}
\end{center}
\begin{center}
Их не становиться меньше, откуда они только лезут?!
\end{center}
\begin{center}
\begin{math}
\frac{d}{dx}(x^{2}) = 2 \cdot {x}^{2 - 1} \cdot \frac{d}{dx}(x)
\end{math}\par\vspace{2em}
\end{center}
\begin{center}
Вот так и рождаются легенды о герое, истребившем половину рода Боблина.
\end{center}
\begin{center}
A = \begin{math}
(0 - \sin(x^{2})) \cdot x \cdot 2 \cdot x \cdot 2
\end{math}\par\vspace{2em}
\end{center}
\begin{center}
\begin{math}
\frac{d}{dx}(A) = \frac{d}{dx}((0 - \sin(x^{2})) \cdot x \cdot 2) \cdot x \cdot 2 + (0 - \sin(x^{2})) \cdot x \cdot 2 \cdot \frac{d}{dx}(x \cdot 2)
\end{math}\par\vspace{2em}
\end{center}
\begin{center}
Небольшой взмах посохом — и план сражения выглядит куда приличнее.
\end{center}
\begin{center}
\begin{math}
\frac{d}{dx}(x \cdot 2) = \frac{d}{dx}(x) \cdot 2 + x \cdot \frac{d}{dx}(2)
\end{math}\par\vspace{2em}
\end{center}
\begin{center}
Вы разложили дядю Боблина на молекулы
\end{center}
\begin{center}
\begin{math}
\frac{d}{dx}(2) = 0
\end{math}\par\vspace{2em}
\end{center}
\begin{center}
Полиморф сработал отлично: зять Боблина теперь лягушка
\end{center}
\begin{center}
\begin{math}
\frac{d}{dx}(x) = 1
\end{math}\par\vspace{2em}
\end{center}
\begin{center}
Один из гоблинов упал
\end{center}
\begin{center}
\begin{math}
\frac{d}{dx}((0 - \sin(x^{2})) \cdot x \cdot 2) = \frac{d}{dx}(0 - \sin(x^{2})) \cdot x \cdot 2 + 0 - \sin(x^{2}) \cdot \frac{d}{dx}(x \cdot 2)
\end{math}\par\vspace{2em}
\end{center}
\begin{center}
Родственники Боблина продолжают лезть к вам, держите посох крепче!
\end{center}
\begin{center}
\begin{math}
\frac{d}{dx}(x \cdot 2) = \frac{d}{dx}(x) \cdot 2 + x \cdot \frac{d}{dx}(2)
\end{math}\par\vspace{2em}
\end{center}
\begin{center}
Ваше волшебство оказалось не по зубам тёте Боблина, кстати, куда она делась?
\end{center}
\begin{center}
\begin{math}
\frac{d}{dx}(2) = 0
\end{math}\par\vspace{2em}
\end{center}
\begin{center}
Поздравляю! От свояка Боблина осталась только полоыина
\end{center}
\begin{center}
\begin{math}
\frac{d}{dx}(x) = 1
\end{math}\par\vspace{2em}
\end{center}
\begin{center}
Битва продолжается, не теряйте духу, они когда-то, наверное, закончаться!
\end{center}
\begin{center}
\begin{math}
\frac{d}{dx}(0 - \sin(x^{2})) = \frac{d}{dx}(0) - \frac{d}{dx}(\sin(x^{2}))
\end{math}\par\vspace{2em}
\end{center}
\begin{center}
Их не становиться меньше, откуда они только лезут?!
\end{center}
\begin{center}
\begin{math}
\frac{d}{dx}(\sin(x^{2})) = \cos(x^{2}) \cdot \frac{d}{dx}(x^{2})
\end{math}\par\vspace{2em}
\end{center}
\begin{center}
Вот так и рождаются легенды о герое, истребившем половину рода Боблина.
\end{center}
\begin{center}
\begin{math}
\frac{d}{dx}(x^{2}) = 2 \cdot {x}^{2 - 1} \cdot \frac{d}{dx}(x)
\end{math}\par\vspace{2em}
\end{center}
\begin{center}
Огненный шар испарил бабушку Боблина
\end{center}
\begin{center}
\begin{math}
\frac{d}{dx}(0) = 0
\end{math}\par\vspace{2em}
\end{center}
\begin{center}
Заклинание хаоса раскидало части племянника Боблина по разным планам
\end{center}
\begin{center}
\begin{math}
\frac{d}{dx}(0) = 0
\end{math}\par\vspace{2em}
\end{center}
\subsection{Прибывает 5-ая волна родственников Тейлора-Боблина}
\begin{center}
A = \begin{math}
(0 - ((0 - \sin(x^{2})) \cdot x \cdot 2 \cdot x \cdot 2 + \cos(x^{2}) \cdot 2)) \cdot 2
\end{math}\par\vspace{2em}
\end{center}
\begin{center}
B = \begin{math}
(0 - ((0 - \sin(x^{2})) \cdot x \cdot 2 \cdot x \cdot 2 + \cos(x^{2}) \cdot 2)) \cdot 2
\end{math}\par\vspace{2em}
\end{center}
\begin{center}
C = \begin{math}
(0 - ((0 - \sin(x^{2})) \cdot x \cdot 2 \cdot x \cdot 2 + \cos(x^{2}) \cdot 2)) \cdot 2
\end{math}\par\vspace{2em}
\end{center}
\begin{center}
D = \begin{math}
(0 - \cos(x^{2}) \cdot x \cdot 2) \cdot x \cdot 2 + (0 - \sin(x^{2})) \cdot 2
\end{math}\par\vspace{2em}
\end{center}
\begin{center}
E = \begin{math}
(0 - \sin(x^{2})) \cdot x \cdot 2 \cdot 2
\end{math}\par\vspace{2em}
\end{center}
\begin{center}
F = \begin{math}
(0 - \sin(x^{2})) \cdot x \cdot 2 \cdot 2
\end{math}\par\vspace{2em}
\end{center}
\begin{center}
G = \begin{math}
3^{x} \cdot 1.09861 \cdot 1.09861 \cdot 1.09861 \cdot 1.09861
\end{math}\par\vspace{2em}
\end{center}
\begin{center}
Текущий ход событий: \begin{math}
(0 - (D \cdot x \cdot 2 + E + F)) \cdot x \cdot 2 + A + B + C + G
\end{math}\par\vspace{2em}
\end{center}
\noindent\hrulefill\begin{center}
Небольшой взмах посохом — и план сражения выглядит куда приличнее.
\end{center}
\begin{center}
A = \begin{math}
(0 - ((0 - \sin(x^{2})) \cdot x \cdot 2 \cdot x \cdot 2 + \cos(x^{2}) \cdot 2)) \cdot 2
\end{math}\par\vspace{2em}
\end{center}
\begin{center}
B = \begin{math}
(0 - ((0 - \sin(x^{2})) \cdot x \cdot 2 \cdot x \cdot 2 + \cos(x^{2}) \cdot 2)) \cdot 2
\end{math}\par\vspace{2em}
\end{center}
\begin{center}
C = \begin{math}
(0 - ((0 - \sin(x^{2})) \cdot x \cdot 2 \cdot x \cdot 2 + \cos(x^{2}) \cdot 2)) \cdot 2
\end{math}\par\vspace{2em}
\end{center}
\begin{center}
D = \begin{math}
(0 - \cos(x^{2}) \cdot x \cdot 2) \cdot x \cdot 2 + (0 - \sin(x^{2})) \cdot 2
\end{math}\par\vspace{2em}
\end{center}
\begin{center}
E = \begin{math}
(0 - \sin(x^{2})) \cdot x \cdot 2 \cdot 2
\end{math}\par\vspace{2em}
\end{center}
\begin{center}
F = \begin{math}
(0 - \sin(x^{2})) \cdot x \cdot 2 \cdot 2
\end{math}\par\vspace{2em}
\end{center}
\begin{center}
G = \begin{math}
3^{x} \cdot 1.09861 \cdot 1.09861 \cdot 1.09861 \cdot 1.09861
\end{math}\par\vspace{2em}
\end{center}
\begin{center}
H = \begin{math}
x \cdot 2
\end{math}\par\vspace{2em}
\end{center}
\begin{center}
I = \begin{math}
x \cdot 2
\end{math}\par\vspace{2em}
\end{center}
\begin{center}
J = \begin{math}
0
\end{math}\par\vspace{2em}
\end{center}
\begin{center}
\begin{math}
\frac{d}{dx}((J - (D \cdot H + E + F)) \cdot I + A + B + C + G) = \frac{d}{dx}((0 - (((0 - \cos(x^{2}) \cdot x \cdot 2) \cdot x \cdot 2 + (0 - \sin(x^{2})) \cdot 2) \cdot x \cdot 2 + (0 - \sin(x^{2})) \cdot x \cdot 2 \cdot 2 + (0 - \sin(x^{2})) \cdot x \cdot 2 \cdot 2)) \cdot x \cdot 2 + (0 - ((0 - \sin(x^{2})) \cdot x \cdot 2 \cdot x \cdot 2 + \cos(x^{2}) \cdot 2)) \cdot 2 + (0 - ((0 - \sin(x^{2})) \cdot x \cdot 2 \cdot x \cdot 2 + \cos(x^{2}) \cdot 2)) \cdot 2 + (0 - ((0 - \sin(x^{2})) \cdot x \cdot 2 \cdot x \cdot 2 + \cos(x^{2}) \cdot 2)) \cdot 2) + \frac{d}{dx}(3^{x} \cdot 1.09861 \cdot 1.09861 \cdot 1.09861 \cdot 1.09861)
\end{math}\par\vspace{2em}
\end{center}
\begin{center}
Один из гоблинов упал
\end{center}
\begin{center}
A = \begin{math}
3^{x} \cdot 1.09861 \cdot 1.09861 \cdot 1.09861 \cdot 1.09861
\end{math}\par\vspace{2em}
\end{center}
\begin{center}
\begin{math}
\frac{d}{dx}(A) = \frac{d}{dx}(3^{x} \cdot 1.09861 \cdot 1.09861 \cdot 1.09861) \cdot 1.09861 + 3^{x} \cdot 1.09861 \cdot 1.09861 \cdot 1.09861 \cdot \frac{d}{dx}(1.09861)
\end{math}\par\vspace{2em}
\end{center}
\begin{center}
Ваш портал небытия вежливо удалил тещу Боблина из этого измерения
\end{center}
\begin{center}
\begin{math}
\frac{d}{dx}(1.09861) = 0
\end{math}\par\vspace{2em}
\end{center}
\begin{center}
Родственники Боблина продолжают лезть к вам, держите посох крепче!
\end{center}
\begin{center}
\begin{math}
\frac{d}{dx}(3^{x} \cdot 1.09861 \cdot 1.09861 \cdot 1.09861) = \frac{d}{dx}(3^{x} \cdot 1.09861 \cdot 1.09861) \cdot 1.09861 + 3^{x} \cdot 1.09861 \cdot 1.09861 \cdot \frac{d}{dx}(1.09861)
\end{math}\par\vspace{2em}
\end{center}
\begin{center}
Ваше заклинание дезинтегрировало брата Боблина
\end{center}
\begin{center}
\begin{math}
\frac{d}{dx}(1.09861) = 0
\end{math}\par\vspace{2em}
\end{center}
\begin{center}
Битва продолжается, не теряйте духу, они когда-то, наверное, закончаться!
\end{center}
\begin{center}
\begin{math}
\frac{d}{dx}(3^{x} \cdot 1.09861 \cdot 1.09861) = \frac{d}{dx}(3^{x} \cdot 1.09861) \cdot 1.09861 + 3^{x} \cdot 1.09861 \cdot \frac{d}{dx}(1.09861)
\end{math}\par\vspace{2em}
\end{center}
\begin{center}
Ваше колдовство низвело сестру Боблина до атомов
\end{center}
\begin{center}
\begin{math}
\frac{d}{dx}(1.09861) = 0
\end{math}\par\vspace{2em}
\end{center}
\begin{center}
Их не становиться меньше, откуда они только лезут?!
\end{center}
\begin{center}
\begin{math}
\frac{d}{dx}(3^{x} \cdot 1.09861) = \frac{d}{dx}(3^{x}) \cdot 1.09861 + 3^{x} \cdot \frac{d}{dx}(1.09861)
\end{math}\par\vspace{2em}
\end{center}
\begin{center}
Вы разложили дядю Боблина на молекулы
\end{center}
\begin{center}
\begin{math}
\frac{d}{dx}(1.09861) = 0
\end{math}\par\vspace{2em}
\end{center}
\begin{center}
Вот так и рождаются легенды о герое, истребившем половину рода Боблина.
\end{center}
\begin{center}
\begin{math}
\frac{d}{dx}(3^{x}) = 3^{x} \cdot \ln(3) \cdot \frac{d}{dx}(x)
\end{math}\par\vspace{2em}
\end{center}
\begin{center}
Ваше заклинание свернуло невестку Боблина в шарик
\end{center}
\begin{center}
\begin{math}
\frac{d}{dx}(x) = 1
\end{math}\par\vspace{2em}
\end{center}
\begin{center}
Небольшой взмах посохом — и план сражения выглядит куда приличнее.
\end{center}
\begin{center}
A = \begin{math}
(0 - ((0 - \sin(x^{2})) \cdot x \cdot 2 \cdot x \cdot 2 + \cos(x^{2}) \cdot 2)) \cdot 2
\end{math}\par\vspace{2em}
\end{center}
\begin{center}
B = \begin{math}
(0 - ((0 - \sin(x^{2})) \cdot x \cdot 2 \cdot x \cdot 2 + \cos(x^{2}) \cdot 2)) \cdot 2
\end{math}\par\vspace{2em}
\end{center}
\begin{center}
C = \begin{math}
(0 - ((0 - \sin(x^{2})) \cdot x \cdot 2 \cdot x \cdot 2 + \cos(x^{2}) \cdot 2)) \cdot 2
\end{math}\par\vspace{2em}
\end{center}
\begin{center}
D = \begin{math}
(0 - \cos(x^{2}) \cdot x \cdot 2) \cdot x \cdot 2 + (0 - \sin(x^{2})) \cdot 2
\end{math}\par\vspace{2em}
\end{center}
\begin{center}
E = \begin{math}
(0 - \sin(x^{2})) \cdot x \cdot 2 \cdot 2
\end{math}\par\vspace{2em}
\end{center}
\begin{center}
F = \begin{math}
(0 - \sin(x^{2})) \cdot x \cdot 2 \cdot 2
\end{math}\par\vspace{2em}
\end{center}
\begin{center}
G = \begin{math}
x \cdot 2
\end{math}\par\vspace{2em}
\end{center}
\begin{center}
H = \begin{math}
x \cdot 2
\end{math}\par\vspace{2em}
\end{center}
\begin{center}
I = \begin{math}
0
\end{math}\par\vspace{2em}
\end{center}
\begin{center}
\begin{math}
\frac{d}{dx}((I - (D \cdot G + E + F)) \cdot H + A + B + C) = \frac{d}{dx}((0 - (((0 - \cos(x^{2}) \cdot x \cdot 2) \cdot x \cdot 2 + (0 - \sin(x^{2})) \cdot 2) \cdot x \cdot 2 + (0 - \sin(x^{2})) \cdot x \cdot 2 \cdot 2 + (0 - \sin(x^{2})) \cdot x \cdot 2 \cdot 2)) \cdot x \cdot 2 + (0 - ((0 - \sin(x^{2})) \cdot x \cdot 2 \cdot x \cdot 2 + \cos(x^{2}) \cdot 2)) \cdot 2 + (0 - ((0 - \sin(x^{2})) \cdot x \cdot 2 \cdot x \cdot 2 + \cos(x^{2}) \cdot 2)) \cdot 2) + \frac{d}{dx}((0 - ((0 - \sin(x^{2})) \cdot x \cdot 2 \cdot x \cdot 2 + \cos(x^{2}) \cdot 2)) \cdot 2)
\end{math}\par\vspace{2em}
\end{center}
\begin{center}
Один из гоблинов упал
\end{center}
\begin{center}
A = \begin{math}
(0 - ((0 - \sin(x^{2})) \cdot x \cdot 2 \cdot x \cdot 2 + \cos(x^{2}) \cdot 2)) \cdot 2
\end{math}\par\vspace{2em}
\end{center}
\begin{center}
\begin{math}
\frac{d}{dx}(A) = \frac{d}{dx}(0 - ((0 - \sin(x^{2})) \cdot x \cdot 2 \cdot x \cdot 2 + \cos(x^{2}) \cdot 2)) \cdot 2 + 0 - ((0 - \sin(x^{2})) \cdot x \cdot 2 \cdot x \cdot 2 + \cos(x^{2}) \cdot 2) \cdot \frac{d}{dx}(2)
\end{math}\par\vspace{2em}
\end{center}
\begin{center}
Ваше волшебство оказалось не по зубам тёте Боблина, кстати, куда она делась?
\end{center}
\begin{center}
\begin{math}
\frac{d}{dx}(2) = 0
\end{math}\par\vspace{2em}
\end{center}
\begin{center}
Родственники Боблина продолжают лезть к вам, держите посох крепче!
\end{center}
\begin{center}
A = \begin{math}
0 - ((0 - \sin(x^{2})) \cdot x \cdot 2 \cdot x \cdot 2 + \cos(x^{2}) \cdot 2)
\end{math}\par\vspace{2em}
\end{center}
\begin{center}
\begin{math}
\frac{d}{dx}(A) = \frac{d}{dx}(0) - \frac{d}{dx}((0 - \sin(x^{2})) \cdot x \cdot 2 \cdot x \cdot 2 + \cos(x^{2}) \cdot 2)
\end{math}\par\vspace{2em}
\end{center}
\begin{center}
Битва продолжается, не теряйте духу, они когда-то, наверное, закончаться!
\end{center}
\begin{center}
A = \begin{math}
(0 - \sin(x^{2})) \cdot x \cdot 2 \cdot x \cdot 2 + \cos(x^{2}) \cdot 2
\end{math}\par\vspace{2em}
\end{center}
\begin{center}
\begin{math}
\frac{d}{dx}(A) = \frac{d}{dx}((0 - \sin(x^{2})) \cdot x \cdot 2 \cdot x \cdot 2) + \frac{d}{dx}(\cos(x^{2}) \cdot 2)
\end{math}\par\vspace{2em}
\end{center}
\begin{center}
Их не становиться меньше, откуда они только лезут?!
\end{center}
\begin{center}
\begin{math}
\frac{d}{dx}(\cos(x^{2}) \cdot 2) = \frac{d}{dx}(\cos(x^{2})) \cdot 2 + \cos(x^{2}) \cdot \frac{d}{dx}(2)
\end{math}\par\vspace{2em}
\end{center}
\begin{center}
Огненный шар испарил бабушку Боблина
\end{center}
\begin{center}
\begin{math}
\frac{d}{dx}(2) = 0
\end{math}\par\vspace{2em}
\end{center}
\begin{center}
Вот так и рождаются легенды о герое, истребившем половину рода Боблина.
\end{center}
\begin{center}
\begin{math}
\frac{d}{dx}(\cos(x^{2})) = -\sin(x^{2}) \cdot \frac{d}{dx}(x^{2})
\end{math}\par\vspace{2em}
\end{center}
\begin{center}
Небольшой взмах посохом — и план сражения выглядит куда приличнее.
\end{center}
\begin{center}
\begin{math}
\frac{d}{dx}(x^{2}) = 2 \cdot {x}^{2 - 1} \cdot \frac{d}{dx}(x)
\end{math}\par\vspace{2em}
\end{center}
\begin{center}
Один из гоблинов упал
\end{center}
\begin{center}
A = \begin{math}
(0 - \sin(x^{2})) \cdot x \cdot 2 \cdot x \cdot 2
\end{math}\par\vspace{2em}
\end{center}
\begin{center}
\begin{math}
\frac{d}{dx}(A) = \frac{d}{dx}((0 - \sin(x^{2})) \cdot x \cdot 2) \cdot x \cdot 2 + (0 - \sin(x^{2})) \cdot x \cdot 2 \cdot \frac{d}{dx}(x \cdot 2)
\end{math}\par\vspace{2em}
\end{center}
\begin{center}
Родственники Боблина продолжают лезть к вам, держите посох крепче!
\end{center}
\begin{center}
\begin{math}
\frac{d}{dx}(x \cdot 2) = \frac{d}{dx}(x) \cdot 2 + x \cdot \frac{d}{dx}(2)
\end{math}\par\vspace{2em}
\end{center}
\begin{center}
Заклинание хаоса раскидало части племянника Боблина по разным планам
\end{center}
\begin{center}
\begin{math}
\frac{d}{dx}(2) = 0
\end{math}\par\vspace{2em}
\end{center}
\begin{center}
ААХХАХААХАХ Гоблин-Боблин
\end{center}
\begin{center}
\begin{math}
\frac{d}{dx}(x) = 1
\end{math}\par\vspace{2em}
\end{center}
\begin{center}
Битва продолжается, не теряйте духу, они когда-то, наверное, закончаться!
\end{center}
\begin{center}
\begin{math}
\frac{d}{dx}((0 - \sin(x^{2})) \cdot x \cdot 2) = \frac{d}{dx}(0 - \sin(x^{2})) \cdot x \cdot 2 + 0 - \sin(x^{2}) \cdot \frac{d}{dx}(x \cdot 2)
\end{math}\par\vspace{2em}
\end{center}
\begin{center}
Их не становиться меньше, откуда они только лезут?!
\end{center}
\begin{center}
\begin{math}
\frac{d}{dx}(x \cdot 2) = \frac{d}{dx}(x) \cdot 2 + x \cdot \frac{d}{dx}(2)
\end{math}\par\vspace{2em}
\end{center}
\begin{center}
Ваш портал небытия вежливо удалил тещу Боблина из этого измерения
\end{center}
\begin{center}
\begin{math}
\frac{d}{dx}(2) = 0
\end{math}\par\vspace{2em}
\end{center}
\begin{center}
Ваше вошшебство откатило деверя Боблина до младенчества
\end{center}
\begin{center}
\begin{math}
\frac{d}{dx}(x) = 1
\end{math}\par\vspace{2em}
\end{center}
\begin{center}
Вот так и рождаются легенды о герое, истребившем половину рода Боблина.
\end{center}
\begin{center}
\begin{math}
\frac{d}{dx}(0 - \sin(x^{2})) = \frac{d}{dx}(0) - \frac{d}{dx}(\sin(x^{2}))
\end{math}\par\vspace{2em}
\end{center}
\begin{center}
Небольшой взмах посохом — и план сражения выглядит куда приличнее.
\end{center}
\begin{center}
\begin{math}
\frac{d}{dx}(\sin(x^{2})) = \cos(x^{2}) \cdot \frac{d}{dx}(x^{2})
\end{math}\par\vspace{2em}
\end{center}
\begin{center}
Один из гоблинов упал
\end{center}
\begin{center}
\begin{math}
\frac{d}{dx}(x^{2}) = 2 \cdot {x}^{2 - 1} \cdot \frac{d}{dx}(x)
\end{math}\par\vspace{2em}
\end{center}
\begin{center}
Ваше заклинание дезинтегрировало брата Боблина
\end{center}
\begin{center}
\begin{math}
\frac{d}{dx}(0) = 0
\end{math}\par\vspace{2em}
\end{center}
\begin{center}
Ваше колдовство низвело сестру Боблина до атомов
\end{center}
\begin{center}
\begin{math}
\frac{d}{dx}(0) = 0
\end{math}\par\vspace{2em}
\end{center}
\begin{center}
Родственники Боблина продолжают лезть к вам, держите посох крепче!
\end{center}
\begin{center}
A = \begin{math}
(0 - ((0 - \sin(x^{2})) \cdot x \cdot 2 \cdot x \cdot 2 + \cos(x^{2}) \cdot 2)) \cdot 2
\end{math}\par\vspace{2em}
\end{center}
\begin{center}
B = \begin{math}
(0 - ((0 - \sin(x^{2})) \cdot x \cdot 2 \cdot x \cdot 2 + \cos(x^{2}) \cdot 2)) \cdot 2
\end{math}\par\vspace{2em}
\end{center}
\begin{center}
C = \begin{math}
(0 - \cos(x^{2}) \cdot x \cdot 2) \cdot x \cdot 2 + (0 - \sin(x^{2})) \cdot 2
\end{math}\par\vspace{2em}
\end{center}
\begin{center}
D = \begin{math}
(0 - \sin(x^{2})) \cdot x \cdot 2 \cdot 2
\end{math}\par\vspace{2em}
\end{center}
\begin{center}
E = \begin{math}
(0 - \sin(x^{2})) \cdot x \cdot 2 \cdot 2
\end{math}\par\vspace{2em}
\end{center}
\begin{center}
F = \begin{math}
x \cdot 2
\end{math}\par\vspace{2em}
\end{center}
\begin{center}
G = \begin{math}
x \cdot 2
\end{math}\par\vspace{2em}
\end{center}
\begin{center}
H = \begin{math}
0
\end{math}\par\vspace{2em}
\end{center}
\begin{center}
\begin{math}
\frac{d}{dx}((H - (C \cdot F + D + E)) \cdot G + A + B) = \frac{d}{dx}((0 - (((0 - \cos(x^{2}) \cdot x \cdot 2) \cdot x \cdot 2 + (0 - \sin(x^{2})) \cdot 2) \cdot x \cdot 2 + (0 - \sin(x^{2})) \cdot x \cdot 2 \cdot 2 + (0 - \sin(x^{2})) \cdot x \cdot 2 \cdot 2)) \cdot x \cdot 2 + (0 - ((0 - \sin(x^{2})) \cdot x \cdot 2 \cdot x \cdot 2 + \cos(x^{2}) \cdot 2)) \cdot 2) + \frac{d}{dx}((0 - ((0 - \sin(x^{2})) \cdot x \cdot 2 \cdot x \cdot 2 + \cos(x^{2}) \cdot 2)) \cdot 2)
\end{math}\par\vspace{2em}
\end{center}
\begin{center}
Битва продолжается, не теряйте духу, они когда-то, наверное, закончаться!
\end{center}
\begin{center}
A = \begin{math}
(0 - ((0 - \sin(x^{2})) \cdot x \cdot 2 \cdot x \cdot 2 + \cos(x^{2}) \cdot 2)) \cdot 2
\end{math}\par\vspace{2em}
\end{center}
\begin{center}
\begin{math}
\frac{d}{dx}(A) = \frac{d}{dx}(0 - ((0 - \sin(x^{2})) \cdot x \cdot 2 \cdot x \cdot 2 + \cos(x^{2}) \cdot 2)) \cdot 2 + 0 - ((0 - \sin(x^{2})) \cdot x \cdot 2 \cdot x \cdot 2 + \cos(x^{2}) \cdot 2) \cdot \frac{d}{dx}(2)
\end{math}\par\vspace{2em}
\end{center}
\begin{center}
Вы разложили дядю Боблина на молекулы
\end{center}
\begin{center}
\begin{math}
\frac{d}{dx}(2) = 0
\end{math}\par\vspace{2em}
\end{center}
\begin{center}
Их не становиться меньше, откуда они только лезут?!
\end{center}
\begin{center}
A = \begin{math}
0 - ((0 - \sin(x^{2})) \cdot x \cdot 2 \cdot x \cdot 2 + \cos(x^{2}) \cdot 2)
\end{math}\par\vspace{2em}
\end{center}
\begin{center}
\begin{math}
\frac{d}{dx}(A) = \frac{d}{dx}(0) - \frac{d}{dx}((0 - \sin(x^{2})) \cdot x \cdot 2 \cdot x \cdot 2 + \cos(x^{2}) \cdot 2)
\end{math}\par\vspace{2em}
\end{center}
\begin{center}
Вот так и рождаются легенды о герое, истребившем половину рода Боблина.
\end{center}
\begin{center}
A = \begin{math}
(0 - \sin(x^{2})) \cdot x \cdot 2 \cdot x \cdot 2 + \cos(x^{2}) \cdot 2
\end{math}\par\vspace{2em}
\end{center}
\begin{center}
\begin{math}
\frac{d}{dx}(A) = \frac{d}{dx}((0 - \sin(x^{2})) \cdot x \cdot 2 \cdot x \cdot 2) + \frac{d}{dx}(\cos(x^{2}) \cdot 2)
\end{math}\par\vspace{2em}
\end{center}
\begin{center}
Небольшой взмах посохом — и план сражения выглядит куда приличнее.
\end{center}
\begin{center}
\begin{math}
\frac{d}{dx}(\cos(x^{2}) \cdot 2) = \frac{d}{dx}(\cos(x^{2})) \cdot 2 + \cos(x^{2}) \cdot \frac{d}{dx}(2)
\end{math}\par\vspace{2em}
\end{center}
\begin{center}
Ваше волшебство оказалось не по зубам тёте Боблина, кстати, куда она делась?
\end{center}
\begin{center}
\begin{math}
\frac{d}{dx}(2) = 0
\end{math}\par\vspace{2em}
\end{center}
\begin{center}
Один из гоблинов упал
\end{center}
\begin{center}
\begin{math}
\frac{d}{dx}(\cos(x^{2})) = -\sin(x^{2}) \cdot \frac{d}{dx}(x^{2})
\end{math}\par\vspace{2em}
\end{center}
\begin{center}
Родственники Боблина продолжают лезть к вам, держите посох крепче!
\end{center}
\begin{center}
\begin{math}
\frac{d}{dx}(x^{2}) = 2 \cdot {x}^{2 - 1} \cdot \frac{d}{dx}(x)
\end{math}\par\vspace{2em}
\end{center}
\begin{center}
Битва продолжается, не теряйте духу, они когда-то, наверное, закончаться!
\end{center}
\begin{center}
A = \begin{math}
(0 - \sin(x^{2})) \cdot x \cdot 2 \cdot x \cdot 2
\end{math}\par\vspace{2em}
\end{center}
\begin{center}
\begin{math}
\frac{d}{dx}(A) = \frac{d}{dx}((0 - \sin(x^{2})) \cdot x \cdot 2) \cdot x \cdot 2 + (0 - \sin(x^{2})) \cdot x \cdot 2 \cdot \frac{d}{dx}(x \cdot 2)
\end{math}\par\vspace{2em}
\end{center}
\begin{center}
Их не становиться меньше, откуда они только лезут?!
\end{center}
\begin{center}
\begin{math}
\frac{d}{dx}(x \cdot 2) = \frac{d}{dx}(x) \cdot 2 + x \cdot \frac{d}{dx}(2)
\end{math}\par\vspace{2em}
\end{center}
\begin{center}
Огненный шар испарил бабушку Боблина
\end{center}
\begin{center}
\begin{math}
\frac{d}{dx}(2) = 0
\end{math}\par\vspace{2em}
\end{center}
\begin{center}
Полиморф сработал отлично: зять Боблина теперь лягушка
\end{center}
\begin{center}
\begin{math}
\frac{d}{dx}(x) = 1
\end{math}\par\vspace{2em}
\end{center}
\begin{center}
Вот так и рождаются легенды о герое, истребившем половину рода Боблина.
\end{center}
\begin{center}
\begin{math}
\frac{d}{dx}((0 - \sin(x^{2})) \cdot x \cdot 2) = \frac{d}{dx}(0 - \sin(x^{2})) \cdot x \cdot 2 + 0 - \sin(x^{2}) \cdot \frac{d}{dx}(x \cdot 2)
\end{math}\par\vspace{2em}
\end{center}
\begin{center}
Небольшой взмах посохом — и план сражения выглядит куда приличнее.
\end{center}
\begin{center}
\begin{math}
\frac{d}{dx}(x \cdot 2) = \frac{d}{dx}(x) \cdot 2 + x \cdot \frac{d}{dx}(2)
\end{math}\par\vspace{2em}
\end{center}
\begin{center}
Заклинание хаоса раскидало части племянника Боблина по разным планам
\end{center}
\begin{center}
\begin{math}
\frac{d}{dx}(2) = 0
\end{math}\par\vspace{2em}
\end{center}
\begin{center}
Поздравляю! От свояка Боблина осталась только полоыина
\end{center}
\begin{center}
\begin{math}
\frac{d}{dx}(x) = 1
\end{math}\par\vspace{2em}
\end{center}
\begin{center}
Один из гоблинов упал
\end{center}
\begin{center}
\begin{math}
\frac{d}{dx}(0 - \sin(x^{2})) = \frac{d}{dx}(0) - \frac{d}{dx}(\sin(x^{2}))
\end{math}\par\vspace{2em}
\end{center}
\begin{center}
Родственники Боблина продолжают лезть к вам, держите посох крепче!
\end{center}
\begin{center}
\begin{math}
\frac{d}{dx}(\sin(x^{2})) = \cos(x^{2}) \cdot \frac{d}{dx}(x^{2})
\end{math}\par\vspace{2em}
\end{center}
\begin{center}
Битва продолжается, не теряйте духу, они когда-то, наверное, закончаться!
\end{center}
\begin{center}
\begin{math}
\frac{d}{dx}(x^{2}) = 2 \cdot {x}^{2 - 1} \cdot \frac{d}{dx}(x)
\end{math}\par\vspace{2em}
\end{center}
\begin{center}
Ваш портал небытия вежливо удалил тещу Боблина из этого измерения
\end{center}
\begin{center}
\begin{math}
\frac{d}{dx}(0) = 0
\end{math}\par\vspace{2em}
\end{center}
\begin{center}
Ваше заклинание дезинтегрировало брата Боблина
\end{center}
\begin{center}
\begin{math}
\frac{d}{dx}(0) = 0
\end{math}\par\vspace{2em}
\end{center}
\begin{center}
Их не становиться меньше, откуда они только лезут?!
\end{center}
\begin{center}
A = \begin{math}
(0 - ((0 - \sin(x^{2})) \cdot x \cdot 2 \cdot x \cdot 2 + \cos(x^{2}) \cdot 2)) \cdot 2
\end{math}\par\vspace{2em}
\end{center}
\begin{center}
B = \begin{math}
(0 - \cos(x^{2}) \cdot x \cdot 2) \cdot x \cdot 2 + (0 - \sin(x^{2})) \cdot 2
\end{math}\par\vspace{2em}
\end{center}
\begin{center}
C = \begin{math}
(0 - \sin(x^{2})) \cdot x \cdot 2 \cdot 2
\end{math}\par\vspace{2em}
\end{center}
\begin{center}
D = \begin{math}
(0 - \sin(x^{2})) \cdot x \cdot 2 \cdot 2
\end{math}\par\vspace{2em}
\end{center}
\begin{center}
E = \begin{math}
x \cdot 2
\end{math}\par\vspace{2em}
\end{center}
\begin{center}
F = \begin{math}
x \cdot 2
\end{math}\par\vspace{2em}
\end{center}
\begin{center}
\begin{math}
\frac{d}{dx}((0 - (B \cdot E + C + D)) \cdot F + A) = \frac{d}{dx}((0 - (((0 - \cos(x^{2}) \cdot x \cdot 2) \cdot x \cdot 2 + (0 - \sin(x^{2})) \cdot 2) \cdot x \cdot 2 + (0 - \sin(x^{2})) \cdot x \cdot 2 \cdot 2 + (0 - \sin(x^{2})) \cdot x \cdot 2 \cdot 2)) \cdot x \cdot 2) + \frac{d}{dx}((0 - ((0 - \sin(x^{2})) \cdot x \cdot 2 \cdot x \cdot 2 + \cos(x^{2}) \cdot 2)) \cdot 2)
\end{math}\par\vspace{2em}
\end{center}
\begin{center}
Вот так и рождаются легенды о герое, истребившем половину рода Боблина.
\end{center}
\begin{center}
A = \begin{math}
(0 - ((0 - \sin(x^{2})) \cdot x \cdot 2 \cdot x \cdot 2 + \cos(x^{2}) \cdot 2)) \cdot 2
\end{math}\par\vspace{2em}
\end{center}
\begin{center}
\begin{math}
\frac{d}{dx}(A) = \frac{d}{dx}(0 - ((0 - \sin(x^{2})) \cdot x \cdot 2 \cdot x \cdot 2 + \cos(x^{2}) \cdot 2)) \cdot 2 + 0 - ((0 - \sin(x^{2})) \cdot x \cdot 2 \cdot x \cdot 2 + \cos(x^{2}) \cdot 2) \cdot \frac{d}{dx}(2)
\end{math}\par\vspace{2em}
\end{center}
\begin{center}
Ваше колдовство низвело сестру Боблина до атомов
\end{center}
\begin{center}
\begin{math}
\frac{d}{dx}(2) = 0
\end{math}\par\vspace{2em}
\end{center}
\begin{center}
Небольшой взмах посохом — и план сражения выглядит куда приличнее.
\end{center}
\begin{center}
A = \begin{math}
0 - ((0 - \sin(x^{2})) \cdot x \cdot 2 \cdot x \cdot 2 + \cos(x^{2}) \cdot 2)
\end{math}\par\vspace{2em}
\end{center}
\begin{center}
\begin{math}
\frac{d}{dx}(A) = \frac{d}{dx}(0) - \frac{d}{dx}((0 - \sin(x^{2})) \cdot x \cdot 2 \cdot x \cdot 2 + \cos(x^{2}) \cdot 2)
\end{math}\par\vspace{2em}
\end{center}
\begin{center}
Один из гоблинов упал
\end{center}
\begin{center}
A = \begin{math}
(0 - \sin(x^{2})) \cdot x \cdot 2 \cdot x \cdot 2 + \cos(x^{2}) \cdot 2
\end{math}\par\vspace{2em}
\end{center}
\begin{center}
\begin{math}
\frac{d}{dx}(A) = \frac{d}{dx}((0 - \sin(x^{2})) \cdot x \cdot 2 \cdot x \cdot 2) + \frac{d}{dx}(\cos(x^{2}) \cdot 2)
\end{math}\par\vspace{2em}
\end{center}
\begin{center}
Родственники Боблина продолжают лезть к вам, держите посох крепче!
\end{center}
\begin{center}
\begin{math}
\frac{d}{dx}(\cos(x^{2}) \cdot 2) = \frac{d}{dx}(\cos(x^{2})) \cdot 2 + \cos(x^{2}) \cdot \frac{d}{dx}(2)
\end{math}\par\vspace{2em}
\end{center}
\begin{center}
Вы разложили дядю Боблина на молекулы
\end{center}
\begin{center}
\begin{math}
\frac{d}{dx}(2) = 0
\end{math}\par\vspace{2em}
\end{center}
\begin{center}
Битва продолжается, не теряйте духу, они когда-то, наверное, закончаться!
\end{center}
\begin{center}
\begin{math}
\frac{d}{dx}(\cos(x^{2})) = -\sin(x^{2}) \cdot \frac{d}{dx}(x^{2})
\end{math}\par\vspace{2em}
\end{center}
\begin{center}
Их не становиться меньше, откуда они только лезут?!
\end{center}
\begin{center}
\begin{math}
\frac{d}{dx}(x^{2}) = 2 \cdot {x}^{2 - 1} \cdot \frac{d}{dx}(x)
\end{math}\par\vspace{2em}
\end{center}
\begin{center}
Вот так и рождаются легенды о герое, истребившем половину рода Боблина.
\end{center}
\begin{center}
A = \begin{math}
(0 - \sin(x^{2})) \cdot x \cdot 2 \cdot x \cdot 2
\end{math}\par\vspace{2em}
\end{center}
\begin{center}
\begin{math}
\frac{d}{dx}(A) = \frac{d}{dx}((0 - \sin(x^{2})) \cdot x \cdot 2) \cdot x \cdot 2 + (0 - \sin(x^{2})) \cdot x \cdot 2 \cdot \frac{d}{dx}(x \cdot 2)
\end{math}\par\vspace{2em}
\end{center}
\begin{center}
Небольшой взмах посохом — и план сражения выглядит куда приличнее.
\end{center}
\begin{center}
\begin{math}
\frac{d}{dx}(x \cdot 2) = \frac{d}{dx}(x) \cdot 2 + x \cdot \frac{d}{dx}(2)
\end{math}\par\vspace{2em}
\end{center}
\begin{center}
Ваше волшебство оказалось не по зубам тёте Боблина, кстати, куда она делась?
\end{center}
\begin{center}
\begin{math}
\frac{d}{dx}(2) = 0
\end{math}\par\vspace{2em}
\end{center}
\begin{center}
Ваше заклинание свернуло невестку Боблина в шарик
\end{center}
\begin{center}
\begin{math}
\frac{d}{dx}(x) = 1
\end{math}\par\vspace{2em}
\end{center}
\begin{center}
Один из гоблинов упал
\end{center}
\begin{center}
\begin{math}
\frac{d}{dx}((0 - \sin(x^{2})) \cdot x \cdot 2) = \frac{d}{dx}(0 - \sin(x^{2})) \cdot x \cdot 2 + 0 - \sin(x^{2}) \cdot \frac{d}{dx}(x \cdot 2)
\end{math}\par\vspace{2em}
\end{center}
\begin{center}
Родственники Боблина продолжают лезть к вам, держите посох крепче!
\end{center}
\begin{center}
\begin{math}
\frac{d}{dx}(x \cdot 2) = \frac{d}{dx}(x) \cdot 2 + x \cdot \frac{d}{dx}(2)
\end{math}\par\vspace{2em}
\end{center}
\begin{center}
Огненный шар испарил бабушку Боблина
\end{center}
\begin{center}
\begin{math}
\frac{d}{dx}(2) = 0
\end{math}\par\vspace{2em}
\end{center}
\begin{center}
ААХХАХААХАХ Гоблин-Боблин
\end{center}
\begin{center}
\begin{math}
\frac{d}{dx}(x) = 1
\end{math}\par\vspace{2em}
\end{center}
\begin{center}
Битва продолжается, не теряйте духу, они когда-то, наверное, закончаться!
\end{center}
\begin{center}
\begin{math}
\frac{d}{dx}(0 - \sin(x^{2})) = \frac{d}{dx}(0) - \frac{d}{dx}(\sin(x^{2}))
\end{math}\par\vspace{2em}
\end{center}
\begin{center}
Их не становиться меньше, откуда они только лезут?!
\end{center}
\begin{center}
\begin{math}
\frac{d}{dx}(\sin(x^{2})) = \cos(x^{2}) \cdot \frac{d}{dx}(x^{2})
\end{math}\par\vspace{2em}
\end{center}
\begin{center}
Вот так и рождаются легенды о герое, истребившем половину рода Боблина.
\end{center}
\begin{center}
\begin{math}
\frac{d}{dx}(x^{2}) = 2 \cdot {x}^{2 - 1} \cdot \frac{d}{dx}(x)
\end{math}\par\vspace{2em}
\end{center}
\begin{center}
Заклинание хаоса раскидало части племянника Боблина по разным планам
\end{center}
\begin{center}
\begin{math}
\frac{d}{dx}(0) = 0
\end{math}\par\vspace{2em}
\end{center}
\begin{center}
Ваш портал небытия вежливо удалил тещу Боблина из этого измерения
\end{center}
\begin{center}
\begin{math}
\frac{d}{dx}(0) = 0
\end{math}\par\vspace{2em}
\end{center}
\begin{center}
Небольшой взмах посохом — и план сражения выглядит куда приличнее.
\end{center}
\begin{center}
A = \begin{math}
(0 - \cos(x^{2}) \cdot x \cdot 2) \cdot x \cdot 2 + (0 - \sin(x^{2})) \cdot 2
\end{math}\par\vspace{2em}
\end{center}
\begin{center}
B = \begin{math}
(0 - \sin(x^{2})) \cdot x \cdot 2 \cdot 2
\end{math}\par\vspace{2em}
\end{center}
\begin{center}
C = \begin{math}
(0 - \sin(x^{2})) \cdot x \cdot 2 \cdot 2
\end{math}\par\vspace{2em}
\end{center}
\begin{center}
D = \begin{math}
x \cdot 2
\end{math}\par\vspace{2em}
\end{center}
\begin{center}
E = \begin{math}
x \cdot 2
\end{math}\par\vspace{2em}
\end{center}
\begin{center}
\begin{math}
\frac{d}{dx}((0 - (A \cdot D + B + C)) \cdot E) = \frac{d}{dx}(0 - (((0 - \cos(x^{2}) \cdot x \cdot 2) \cdot x \cdot 2 + (0 - \sin(x^{2})) \cdot 2) \cdot x \cdot 2 + (0 - \sin(x^{2})) \cdot x \cdot 2 \cdot 2 + (0 - \sin(x^{2})) \cdot x \cdot 2 \cdot 2)) \cdot x \cdot 2 + 0 - (((0 - \cos(x^{2}) \cdot x \cdot 2) \cdot x \cdot 2 + (0 - \sin(x^{2})) \cdot 2) \cdot x \cdot 2 + (0 - \sin(x^{2})) \cdot x \cdot 2 \cdot 2 + (0 - \sin(x^{2})) \cdot x \cdot 2 \cdot 2) \cdot \frac{d}{dx}(x \cdot 2)
\end{math}\par\vspace{2em}
\end{center}
\begin{center}
Один из гоблинов упал
\end{center}
\begin{center}
\begin{math}
\frac{d}{dx}(x \cdot 2) = \frac{d}{dx}(x) \cdot 2 + x \cdot \frac{d}{dx}(2)
\end{math}\par\vspace{2em}
\end{center}
\begin{center}
Ваше заклинание дезинтегрировало брата Боблина
\end{center}
\begin{center}
\begin{math}
\frac{d}{dx}(2) = 0
\end{math}\par\vspace{2em}
\end{center}
\begin{center}
Ваше вошшебство откатило деверя Боблина до младенчества
\end{center}
\begin{center}
\begin{math}
\frac{d}{dx}(x) = 1
\end{math}\par\vspace{2em}
\end{center}
\begin{center}
Родственники Боблина продолжают лезть к вам, держите посох крепче!
\end{center}
\begin{center}
A = \begin{math}
(0 - \cos(x^{2}) \cdot x \cdot 2) \cdot x \cdot 2 + (0 - \sin(x^{2})) \cdot 2
\end{math}\par\vspace{2em}
\end{center}
\begin{center}
B = \begin{math}
(0 - \sin(x^{2})) \cdot x \cdot 2 \cdot 2
\end{math}\par\vspace{2em}
\end{center}
\begin{center}
C = \begin{math}
(0 - \sin(x^{2})) \cdot x \cdot 2 \cdot 2
\end{math}\par\vspace{2em}
\end{center}
\begin{center}
\begin{math}
\frac{d}{dx}(0 - (A \cdot x \cdot 2 + B + C)) = \frac{d}{dx}(0) - \frac{d}{dx}(((0 - \cos(x^{2}) \cdot x \cdot 2) \cdot x \cdot 2 + (0 - \sin(x^{2})) \cdot 2) \cdot x \cdot 2 + (0 - \sin(x^{2})) \cdot x \cdot 2 \cdot 2 + (0 - \sin(x^{2})) \cdot x \cdot 2 \cdot 2)
\end{math}\par\vspace{2em}
\end{center}
\begin{center}
Битва продолжается, не теряйте духу, они когда-то, наверное, закончаться!
\end{center}
\begin{center}
A = \begin{math}
(0 - \cos(x^{2}) \cdot x \cdot 2) \cdot x \cdot 2 + (0 - \sin(x^{2})) \cdot 2
\end{math}\par\vspace{2em}
\end{center}
\begin{center}
B = \begin{math}
(0 - \sin(x^{2})) \cdot x \cdot 2 \cdot 2
\end{math}\par\vspace{2em}
\end{center}
\begin{center}
C = \begin{math}
(0 - \sin(x^{2})) \cdot x \cdot 2 \cdot 2
\end{math}\par\vspace{2em}
\end{center}
\begin{center}
\begin{math}
\frac{d}{dx}(A \cdot x \cdot 2 + B + C) = \frac{d}{dx}(((0 - \cos(x^{2}) \cdot x \cdot 2) \cdot x \cdot 2 + (0 - \sin(x^{2})) \cdot 2) \cdot x \cdot 2 + (0 - \sin(x^{2})) \cdot x \cdot 2 \cdot 2) + \frac{d}{dx}((0 - \sin(x^{2})) \cdot x \cdot 2 \cdot 2)
\end{math}\par\vspace{2em}
\end{center}
\begin{center}
Их не становиться меньше, откуда они только лезут?!
\end{center}
\begin{center}
A = \begin{math}
(0 - \sin(x^{2})) \cdot x \cdot 2 \cdot 2
\end{math}\par\vspace{2em}
\end{center}
\begin{center}
\begin{math}
\frac{d}{dx}(A) = \frac{d}{dx}((0 - \sin(x^{2})) \cdot x \cdot 2) \cdot 2 + (0 - \sin(x^{2})) \cdot x \cdot 2 \cdot \frac{d}{dx}(2)
\end{math}\par\vspace{2em}
\end{center}
\begin{center}
Ваше колдовство низвело сестру Боблина до атомов
\end{center}
\begin{center}
\begin{math}
\frac{d}{dx}(2) = 0
\end{math}\par\vspace{2em}
\end{center}
\begin{center}
Вот так и рождаются легенды о герое, истребившем половину рода Боблина.
\end{center}
\begin{center}
\begin{math}
\frac{d}{dx}((0 - \sin(x^{2})) \cdot x \cdot 2) = \frac{d}{dx}(0 - \sin(x^{2})) \cdot x \cdot 2 + 0 - \sin(x^{2}) \cdot \frac{d}{dx}(x \cdot 2)
\end{math}\par\vspace{2em}
\end{center}
\begin{center}
Небольшой взмах посохом — и план сражения выглядит куда приличнее.
\end{center}
\begin{center}
\begin{math}
\frac{d}{dx}(x \cdot 2) = \frac{d}{dx}(x) \cdot 2 + x \cdot \frac{d}{dx}(2)
\end{math}\par\vspace{2em}
\end{center}
\begin{center}
Вы разложили дядю Боблина на молекулы
\end{center}
\begin{center}
\begin{math}
\frac{d}{dx}(2) = 0
\end{math}\par\vspace{2em}
\end{center}
\begin{center}
Полиморф сработал отлично: зять Боблина теперь лягушка
\end{center}
\begin{center}
\begin{math}
\frac{d}{dx}(x) = 1
\end{math}\par\vspace{2em}
\end{center}
\begin{center}
Один из гоблинов упал
\end{center}
\begin{center}
\begin{math}
\frac{d}{dx}(0 - \sin(x^{2})) = \frac{d}{dx}(0) - \frac{d}{dx}(\sin(x^{2}))
\end{math}\par\vspace{2em}
\end{center}
\begin{center}
Родственники Боблина продолжают лезть к вам, держите посох крепче!
\end{center}
\begin{center}
\begin{math}
\frac{d}{dx}(\sin(x^{2})) = \cos(x^{2}) \cdot \frac{d}{dx}(x^{2})
\end{math}\par\vspace{2em}
\end{center}
\begin{center}
Битва продолжается, не теряйте духу, они когда-то, наверное, закончаться!
\end{center}
\begin{center}
\begin{math}
\frac{d}{dx}(x^{2}) = 2 \cdot {x}^{2 - 1} \cdot \frac{d}{dx}(x)
\end{math}\par\vspace{2em}
\end{center}
\begin{center}
Ваше волшебство оказалось не по зубам тёте Боблина, кстати, куда она делась?
\end{center}
\begin{center}
\begin{math}
\frac{d}{dx}(0) = 0
\end{math}\par\vspace{2em}
\end{center}
\begin{center}
Их не становиться меньше, откуда они только лезут?!
\end{center}
\begin{center}
A = \begin{math}
(0 - \cos(x^{2}) \cdot x \cdot 2) \cdot x \cdot 2 + (0 - \sin(x^{2})) \cdot 2
\end{math}\par\vspace{2em}
\end{center}
\begin{center}
B = \begin{math}
(0 - \sin(x^{2})) \cdot x \cdot 2 \cdot 2
\end{math}\par\vspace{2em}
\end{center}
\begin{center}
\begin{math}
\frac{d}{dx}(A \cdot x \cdot 2 + B) = \frac{d}{dx}(((0 - \cos(x^{2}) \cdot x \cdot 2) \cdot x \cdot 2 + (0 - \sin(x^{2})) \cdot 2) \cdot x \cdot 2) + \frac{d}{dx}((0 - \sin(x^{2})) \cdot x \cdot 2 \cdot 2)
\end{math}\par\vspace{2em}
\end{center}
\begin{center}
Вот так и рождаются легенды о герое, истребившем половину рода Боблина.
\end{center}
\begin{center}
A = \begin{math}
(0 - \sin(x^{2})) \cdot x \cdot 2 \cdot 2
\end{math}\par\vspace{2em}
\end{center}
\begin{center}
\begin{math}
\frac{d}{dx}(A) = \frac{d}{dx}((0 - \sin(x^{2})) \cdot x \cdot 2) \cdot 2 + (0 - \sin(x^{2})) \cdot x \cdot 2 \cdot \frac{d}{dx}(2)
\end{math}\par\vspace{2em}
\end{center}
\begin{center}
Огненный шар испарил бабушку Боблина
\end{center}
\begin{center}
\begin{math}
\frac{d}{dx}(2) = 0
\end{math}\par\vspace{2em}
\end{center}
\begin{center}
Небольшой взмах посохом — и план сражения выглядит куда приличнее.
\end{center}
\begin{center}
\begin{math}
\frac{d}{dx}((0 - \sin(x^{2})) \cdot x \cdot 2) = \frac{d}{dx}(0 - \sin(x^{2})) \cdot x \cdot 2 + 0 - \sin(x^{2}) \cdot \frac{d}{dx}(x \cdot 2)
\end{math}\par\vspace{2em}
\end{center}
\begin{center}
Один из гоблинов упал
\end{center}
\begin{center}
\begin{math}
\frac{d}{dx}(x \cdot 2) = \frac{d}{dx}(x) \cdot 2 + x \cdot \frac{d}{dx}(2)
\end{math}\par\vspace{2em}
\end{center}
\begin{center}
Заклинание хаоса раскидало части племянника Боблина по разным планам
\end{center}
\begin{center}
\begin{math}
\frac{d}{dx}(2) = 0
\end{math}\par\vspace{2em}
\end{center}
\begin{center}
Поздравляю! От свояка Боблина осталась только полоыина
\end{center}
\begin{center}
\begin{math}
\frac{d}{dx}(x) = 1
\end{math}\par\vspace{2em}
\end{center}
\begin{center}
Родственники Боблина продолжают лезть к вам, держите посох крепче!
\end{center}
\begin{center}
\begin{math}
\frac{d}{dx}(0 - \sin(x^{2})) = \frac{d}{dx}(0) - \frac{d}{dx}(\sin(x^{2}))
\end{math}\par\vspace{2em}
\end{center}
\begin{center}
Битва продолжается, не теряйте духу, они когда-то, наверное, закончаться!
\end{center}
\begin{center}
\begin{math}
\frac{d}{dx}(\sin(x^{2})) = \cos(x^{2}) \cdot \frac{d}{dx}(x^{2})
\end{math}\par\vspace{2em}
\end{center}
\begin{center}
Их не становиться меньше, откуда они только лезут?!
\end{center}
\begin{center}
\begin{math}
\frac{d}{dx}(x^{2}) = 2 \cdot {x}^{2 - 1} \cdot \frac{d}{dx}(x)
\end{math}\par\vspace{2em}
\end{center}
\begin{center}
Ваш портал небытия вежливо удалил тещу Боблина из этого измерения
\end{center}
\begin{center}
\begin{math}
\frac{d}{dx}(0) = 0
\end{math}\par\vspace{2em}
\end{center}
\begin{center}
Вот так и рождаются легенды о герое, истребившем половину рода Боблина.
\end{center}
\begin{center}
A = \begin{math}
(0 - \cos(x^{2}) \cdot x \cdot 2) \cdot x \cdot 2 + (0 - \sin(x^{2})) \cdot 2
\end{math}\par\vspace{2em}
\end{center}
\begin{center}
\begin{math}
\frac{d}{dx}(A \cdot x \cdot 2) = \frac{d}{dx}((0 - \cos(x^{2}) \cdot x \cdot 2) \cdot x \cdot 2 + (0 - \sin(x^{2})) \cdot 2) \cdot x \cdot 2 + (0 - \cos(x^{2}) \cdot x \cdot 2) \cdot x \cdot 2 + (0 - \sin(x^{2})) \cdot 2 \cdot \frac{d}{dx}(x \cdot 2)
\end{math}\par\vspace{2em}
\end{center}
\begin{center}
Небольшой взмах посохом — и план сражения выглядит куда приличнее.
\end{center}
\begin{center}
\begin{math}
\frac{d}{dx}(x \cdot 2) = \frac{d}{dx}(x) \cdot 2 + x \cdot \frac{d}{dx}(2)
\end{math}\par\vspace{2em}
\end{center}
\begin{center}
Ваше заклинание дезинтегрировало брата Боблина
\end{center}
\begin{center}
\begin{math}
\frac{d}{dx}(2) = 0
\end{math}\par\vspace{2em}
\end{center}
\begin{center}
Ваше заклинание свернуло невестку Боблина в шарик
\end{center}
\begin{center}
\begin{math}
\frac{d}{dx}(x) = 1
\end{math}\par\vspace{2em}
\end{center}
\begin{center}
Один из гоблинов упал
\end{center}
\begin{center}
A = \begin{math}
(0 - \cos(x^{2}) \cdot x \cdot 2) \cdot x \cdot 2 + (0 - \sin(x^{2})) \cdot 2
\end{math}\par\vspace{2em}
\end{center}
\begin{center}
\begin{math}
\frac{d}{dx}(A) = \frac{d}{dx}((0 - \cos(x^{2}) \cdot x \cdot 2) \cdot x \cdot 2) + \frac{d}{dx}((0 - \sin(x^{2})) \cdot 2)
\end{math}\par\vspace{2em}
\end{center}
\begin{center}
Родственники Боблина продолжают лезть к вам, держите посох крепче!
\end{center}
\begin{center}
\begin{math}
\frac{d}{dx}((0 - \sin(x^{2})) \cdot 2) = \frac{d}{dx}(0 - \sin(x^{2})) \cdot 2 + 0 - \sin(x^{2}) \cdot \frac{d}{dx}(2)
\end{math}\par\vspace{2em}
\end{center}
\begin{center}
Ваше колдовство низвело сестру Боблина до атомов
\end{center}
\begin{center}
\begin{math}
\frac{d}{dx}(2) = 0
\end{math}\par\vspace{2em}
\end{center}
\begin{center}
Битва продолжается, не теряйте духу, они когда-то, наверное, закончаться!
\end{center}
\begin{center}
\begin{math}
\frac{d}{dx}(0 - \sin(x^{2})) = \frac{d}{dx}(0) - \frac{d}{dx}(\sin(x^{2}))
\end{math}\par\vspace{2em}
\end{center}
\begin{center}
Их не становиться меньше, откуда они только лезут?!
\end{center}
\begin{center}
\begin{math}
\frac{d}{dx}(\sin(x^{2})) = \cos(x^{2}) \cdot \frac{d}{dx}(x^{2})
\end{math}\par\vspace{2em}
\end{center}
\begin{center}
Вот так и рождаются легенды о герое, истребившем половину рода Боблина.
\end{center}
\begin{center}
\begin{math}
\frac{d}{dx}(x^{2}) = 2 \cdot {x}^{2 - 1} \cdot \frac{d}{dx}(x)
\end{math}\par\vspace{2em}
\end{center}
\begin{center}
Вы разложили дядю Боблина на молекулы
\end{center}
\begin{center}
\begin{math}
\frac{d}{dx}(0) = 0
\end{math}\par\vspace{2em}
\end{center}
\begin{center}
Небольшой взмах посохом — и план сражения выглядит куда приличнее.
\end{center}
\begin{center}
A = \begin{math}
(0 - \cos(x^{2}) \cdot x \cdot 2) \cdot x \cdot 2
\end{math}\par\vspace{2em}
\end{center}
\begin{center}
\begin{math}
\frac{d}{dx}(A) = \frac{d}{dx}(0 - \cos(x^{2}) \cdot x \cdot 2) \cdot x \cdot 2 + 0 - \cos(x^{2}) \cdot x \cdot 2 \cdot \frac{d}{dx}(x \cdot 2)
\end{math}\par\vspace{2em}
\end{center}
\begin{center}
Один из гоблинов упал
\end{center}
\begin{center}
\begin{math}
\frac{d}{dx}(x \cdot 2) = \frac{d}{dx}(x) \cdot 2 + x \cdot \frac{d}{dx}(2)
\end{math}\par\vspace{2em}
\end{center}
\begin{center}
Ваше волшебство оказалось не по зубам тёте Боблина, кстати, куда она делась?
\end{center}
\begin{center}
\begin{math}
\frac{d}{dx}(2) = 0
\end{math}\par\vspace{2em}
\end{center}
\begin{center}
ААХХАХААХАХ Гоблин-Боблин
\end{center}
\begin{center}
\begin{math}
\frac{d}{dx}(x) = 1
\end{math}\par\vspace{2em}
\end{center}
\begin{center}
Родственники Боблина продолжают лезть к вам, держите посох крепче!
\end{center}
\begin{center}
\begin{math}
\frac{d}{dx}(0 - \cos(x^{2}) \cdot x \cdot 2) = \frac{d}{dx}(0) - \frac{d}{dx}(\cos(x^{2}) \cdot x \cdot 2)
\end{math}\par\vspace{2em}
\end{center}
\begin{center}
Битва продолжается, не теряйте духу, они когда-то, наверное, закончаться!
\end{center}
\begin{center}
\begin{math}
\frac{d}{dx}(\cos(x^{2}) \cdot x \cdot 2) = \frac{d}{dx}(\cos(x^{2})) \cdot x \cdot 2 + \cos(x^{2}) \cdot \frac{d}{dx}(x \cdot 2)
\end{math}\par\vspace{2em}
\end{center}
\begin{center}
Их не становиться меньше, откуда они только лезут?!
\end{center}
\begin{center}
\begin{math}
\frac{d}{dx}(x \cdot 2) = \frac{d}{dx}(x) \cdot 2 + x \cdot \frac{d}{dx}(2)
\end{math}\par\vspace{2em}
\end{center}
\begin{center}
Огненный шар испарил бабушку Боблина
\end{center}
\begin{center}
\begin{math}
\frac{d}{dx}(2) = 0
\end{math}\par\vspace{2em}
\end{center}
\begin{center}
Ваше вошшебство откатило деверя Боблина до младенчества
\end{center}
\begin{center}
\begin{math}
\frac{d}{dx}(x) = 1
\end{math}\par\vspace{2em}
\end{center}
\begin{center}
Вот так и рождаются легенды о герое, истребившем половину рода Боблина.
\end{center}
\begin{center}
\begin{math}
\frac{d}{dx}(\cos(x^{2})) = -\sin(x^{2}) \cdot \frac{d}{dx}(x^{2})
\end{math}\par\vspace{2em}
\end{center}
\begin{center}
Небольшой взмах посохом — и план сражения выглядит куда приличнее.
\end{center}
\begin{center}
\begin{math}
\frac{d}{dx}(x^{2}) = 2 \cdot {x}^{2 - 1} \cdot \frac{d}{dx}(x)
\end{math}\par\vspace{2em}
\end{center}
\begin{center}
Заклинание хаоса раскидало части племянника Боблина по разным планам
\end{center}
\begin{center}
\begin{math}
\frac{d}{dx}(0) = 0
\end{math}\par\vspace{2em}
\end{center}
\begin{center}
Ваш портал небытия вежливо удалил тещу Боблина из этого измерения
\end{center}
\begin{center}
\begin{math}
\frac{d}{dx}(0) = 0
\end{math}\par\vspace{2em}
\end{center}
\section{Полный план сражения с Тейлором-Боблином}
\begin{center}
A = \begin{math}
2 + \frac{268.286}{1} \cdot (x - 0)^{1} + \frac{194.433}{2} \cdot (x - 0)^{2} + \frac{130.387}{6} \cdot (x - 0)^{3}
\end{math}\par\vspace{2em}
\end{center}
\begin{center}
f(x) = \begin{math}
A + \frac{10095.3}{24} \cdot (x - 0)^{4} + \frac{33368.7}{120} \cdot (x - 0)^{5}
\end{math}\par\vspace{2em}
\end{center}
\includegraphics[height=7.5cm]{./graphs/teylor_plot.png}\end{document}
