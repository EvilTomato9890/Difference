\documentclass[a4paper,12pt]{article}
\usepackage[T2A]{fontenc}
\usepackage[utf8]{inputenc}
\usepackage[russian]{babel}
\usepackage{amsmath}
\usepackage{amssymb}
\usepackage{autobreak}
\usepackage{hyperref}
\setcounter{secnumdepth}{0}
\begin{document}
\title{Сокращение рода боблина}
\maketitle
\newpage
\begin{titlepage}
    \centering

    {\Large Уставший волшебник}\\[1cm]

    {\huge\bfseries «Завершение рода Боблина»}\\[0.5cm]

    \raggedright

    \textbf{Предыстория} \\[0.3cm]

    Жил-был самый обычный гоблин по имени Боблин и его очень большая семья. \\ 
    Как-то раз, одним жарким летом они все вместе решили отправиться на пикник.\\ 
    Они нашли великолепную полянку посреди болота: солнышко, зеленая трава, тенеко от непонятно башни, одним словом - благодать. \\ 
    Шел 5-ый час гоблинской пъянки, тут уже нервы волшебника живущего в башне не выдержали. \\ 
    и он решил обрушиить свой праведный гнев не семейство Боблина, истребив некоторую его часть. \\ 
    \vspace{0.8cm}
    \textbf{Боевой журнал}\\[0.3cm]

    В башне стоял особенный артефакт, который записывал ход сражения в виде странного набора символов.\\ 
    Которые лишь сам маг был способен понять, здесь и будет приведет этот боевой журнал. \\     \vfill

    \raggedleft
    \textit{«Если на странице стало больше знаков — значит, кто-то из клана Боблина опять что-то натворил.»}\\[0.3cm]

\end{titlepage}
\tableofcontents
\newpage
f(x) = \begin{align*}
\begin{autobreak}
\cos(\sin(x)) + x^{3}
\end{autobreak}
\end{align*}

\section{Прибывает Тейлор и куча дальних родственников}
Текущий ход событий: \begin{align*}
\begin{autobreak}
\cos(\sin(x)) + x^{3}
\end{autobreak}
\end{align*}

\subsection{Прибывает 1-ая волна родственников Тейлора-Боблина}
Текущий ход событий: \begin{align*}
\begin{autobreak}
\cos(\sin(x)) + x^{3}
\end{autobreak}
\end{align*}

\noindent\hrulefill\begin{center}
Вот так и рождаются легенды о герое, истребившем половину рода Боблина.
\end{center}
\begin{align*}
\begin{autobreak}
\frac{d}{dx}(\cos(\sin(x)) + x^{3}) = \frac{d}{dx}(\cos(\sin(x))) + \frac{d}{dx}(x^{3})
\end{autobreak}
\end{align*}

\begin{center}
Небольшой взмах посохом — и план сражения выглядит куда приличнее.
\end{center}
\begin{align*}
\begin{autobreak}
\frac{d}{dx}(x^{3}) = 3 \cdot {x}^{3 - 1} \cdot \frac{d}{dx}(x)
\end{autobreak}
\end{align*}

\begin{center}
Один из гоблинов упал
\end{center}
\begin{align*}
\begin{autobreak}
\frac{d}{dx}(\cos(\sin(x))) = -\sin(\sin(x)) \cdot \frac{d}{dx}(\sin(x))
\end{autobreak}
\end{align*}

\begin{center}
Родственники Боблина продолжают лезть к вам, держите посох крепче!
\end{center}
\begin{align*}
\begin{autobreak}
\frac{d}{dx}(\sin(x)) = \cos(x) \cdot \frac{d}{dx}(x)
\end{autobreak}
\end{align*}

\begin{center}
Полиморф сработал отлично: зять Боблина теперь лягушка
\end{center}
\begin{align*}
\begin{autobreak}
\frac{d}{dx}(x) = 1
\end{autobreak}
\end{align*}

\subsection{Прибывает 2-ая волна родственников Тейлора-Боблина}
Текущий ход событий: \begin{align*}
\begin{autobreak}
(0 - \sin(\sin(x))) \cdot \cos(x) + x^{2} \cdot 3
\end{autobreak}
\end{align*}

\noindent\hrulefill\begin{center}
Битва продолжается, не теряйте духу, они когда-то, наверное, закончаться!
\end{center}
A = \begin{align*}
\begin{autobreak}
(0 - \sin(\sin(x))) \cdot \cos(x) + x^{2} \cdot 3
\end{autobreak}
\end{align*}

\begin{align*}
\begin{autobreak}
\frac{d}{dx}(A) = \frac{d}{dx}((0 - \sin(\sin(x))) \cdot \cos(x)) + \frac{d}{dx}(x^{2} \cdot 3)
\end{autobreak}
\end{align*}

\begin{center}
Их не становиться меньше, откуда они только лезут?!
\end{center}
\begin{align*}
\begin{autobreak}
\frac{d}{dx}(x^{2} \cdot 3) = \frac{d}{dx}(x^{2}) \cdot 3 + x^{2} \cdot \frac{d}{dx}(3)
\end{autobreak}
\end{align*}

\begin{center}
Вы разложили дядю Боблина на молекулы
\end{center}
\begin{align*}
\begin{autobreak}
\frac{d}{dx}(3) = 0
\end{autobreak}
\end{align*}

\begin{center}
Вот так и рождаются легенды о герое, истребившем половину рода Боблина.
\end{center}
\begin{align*}
\begin{autobreak}
\frac{d}{dx}(x^{2}) = 2 \cdot {x}^{2 - 1} \cdot \frac{d}{dx}(x)
\end{autobreak}
\end{align*}

\begin{center}
Небольшой взмах посохом — и план сражения выглядит куда приличнее.
\end{center}
\begin{align*}
\begin{autobreak}
\frac{d}{dx}((0 - \sin(\sin(x))) \cdot \cos(x)) = \frac{d}{dx}(0 - \sin(\sin(x))) \cdot \cos(x) + 0 - \sin(\sin(x)) \cdot \frac{d}{dx}(\cos(x))
\end{autobreak}
\end{align*}

\begin{center}
Один из гоблинов упал
\end{center}
\begin{align*}
\begin{autobreak}
\frac{d}{dx}(\cos(x)) = -\sin(x) \cdot \frac{d}{dx}(x)
\end{autobreak}
\end{align*}

\begin{center}
Поздравляю! От свояка Боблина осталась только полоыина
\end{center}
\begin{align*}
\begin{autobreak}
\frac{d}{dx}(x) = 1
\end{autobreak}
\end{align*}

\begin{center}
Родственники Боблина продолжают лезть к вам, держите посох крепче!
\end{center}
\begin{align*}
\begin{autobreak}
\frac{d}{dx}(0 - \sin(\sin(x))) = \frac{d}{dx}(0) - \frac{d}{dx}(\sin(\sin(x)))
\end{autobreak}
\end{align*}

\begin{center}
Битва продолжается, не теряйте духу, они когда-то, наверное, закончаться!
\end{center}
\begin{align*}
\begin{autobreak}
\frac{d}{dx}(\sin(\sin(x))) = \cos(\sin(x)) \cdot \frac{d}{dx}(\sin(x))
\end{autobreak}
\end{align*}

\begin{center}
Их не становиться меньше, откуда они только лезут?!
\end{center}
\begin{align*}
\begin{autobreak}
\frac{d}{dx}(\sin(x)) = \cos(x) \cdot \frac{d}{dx}(x)
\end{autobreak}
\end{align*}

\begin{center}
Ваше заклинание свернуло невестку Боблина в шарик
\end{center}
\begin{align*}
\begin{autobreak}
\frac{d}{dx}(x) = 1
\end{autobreak}
\end{align*}

\begin{center}
Ваше волшебство оказалось не по зубам тёте Боблина, кстати, куда она делась?
\end{center}
\begin{align*}
\begin{autobreak}
\frac{d}{dx}(0) = 0
\end{autobreak}
\end{align*}

\subsection{Прибывает 3-ая волна родственников Тейлора-Боблина}
A = \begin{align*}
\begin{autobreak}
(0 - \cos(\sin(x)) \cdot \cos(x)) \cdot \cos(x) + (0 - \sin(\sin(x))) \cdot (0 - \sin(x))
\end{autobreak}
\end{align*}

Текущий ход событий: \begin{align*}
\begin{autobreak}
A + x \cdot 2 \cdot 3
\end{autobreak}
\end{align*}

\noindent\hrulefill\begin{center}
Вот так и рождаются легенды о герое, истребившем половину рода Боблина.
\end{center}
A = \begin{align*}
\begin{autobreak}
(0 - \cos(\sin(x)) \cdot \cos(x)) \cdot \cos(x) + (0 - \sin(\sin(x))) \cdot (0 - \sin(x))
\end{autobreak}
\end{align*}

\begin{align*}
\begin{autobreak}
\frac{d}{dx}(A + x \cdot 2 \cdot 3) = \frac{d}{dx}(A) + \frac{d}{dx}(x \cdot 2 \cdot 3)
\end{autobreak}
\end{align*}

\begin{center}
Небольшой взмах посохом — и план сражения выглядит куда приличнее.
\end{center}
\begin{align*}
\begin{autobreak}
\frac{d}{dx}(x \cdot 2 \cdot 3) = \frac{d}{dx}(x \cdot 2) \cdot 3 + x \cdot 2 \cdot \frac{d}{dx}(3)
\end{autobreak}
\end{align*}

\begin{center}
Огненный шар испарил бабушку Боблина
\end{center}
\begin{align*}
\begin{autobreak}
\frac{d}{dx}(3) = 0
\end{autobreak}
\end{align*}

\begin{center}
Один из гоблинов упал
\end{center}
\begin{align*}
\begin{autobreak}
\frac{d}{dx}(x \cdot 2) = \frac{d}{dx}(x) \cdot 2 + x \cdot \frac{d}{dx}(2)
\end{autobreak}
\end{align*}

\begin{center}
Заклинание хаоса раскидало части племянника Боблина по разным планам
\end{center}
\begin{align*}
\begin{autobreak}
\frac{d}{dx}(2) = 0
\end{autobreak}
\end{align*}

\begin{center}
ААХХАХААХАХ Гоблин-Боблин
\end{center}
\begin{align*}
\begin{autobreak}
\frac{d}{dx}(x) = 1
\end{autobreak}
\end{align*}

\begin{center}
Родственники Боблина продолжают лезть к вам, держите посох крепче!
\end{center}
A = \begin{align*}
\begin{autobreak}
(0 - \cos(\sin(x)) \cdot \cos(x)) \cdot \cos(x) + (0 - \sin(\sin(x))) \cdot (0 - \sin(x))
\end{autobreak}
\end{align*}

\begin{align*}
\begin{autobreak}
\frac{d}{dx}(A) = \frac{d}{dx}((0 - \cos(\sin(x)) \cdot \cos(x)) \cdot \cos(x)) + \frac{d}{dx}((0 - \sin(\sin(x))) \cdot (0 - \sin(x)))
\end{autobreak}
\end{align*}

\begin{center}
Битва продолжается, не теряйте духу, они когда-то, наверное, закончаться!
\end{center}
\begin{align*}
\begin{autobreak}
\frac{d}{dx}((0 - \sin(\sin(x))) \cdot (0 - \sin(x))) = \frac{d}{dx}(0 - \sin(\sin(x))) \cdot 0 - \sin(x) + 0 - \sin(\sin(x)) \cdot \frac{d}{dx}(0 - \sin(x))
\end{autobreak}
\end{align*}

\begin{center}
Их не становиться меньше, откуда они только лезут?!
\end{center}
\begin{align*}
\begin{autobreak}
\frac{d}{dx}(0 - \sin(x)) = \frac{d}{dx}(0) - \frac{d}{dx}(\sin(x))
\end{autobreak}
\end{align*}

\begin{center}
Вот так и рождаются легенды о герое, истребившем половину рода Боблина.
\end{center}
\begin{align*}
\begin{autobreak}
\frac{d}{dx}(\sin(x)) = \cos(x) \cdot \frac{d}{dx}(x)
\end{autobreak}
\end{align*}

\begin{center}
Ваше вошшебство откатило деверя Боблина до младенчества
\end{center}
\begin{align*}
\begin{autobreak}
\frac{d}{dx}(x) = 1
\end{autobreak}
\end{align*}

\begin{center}
Ваш портал небытия вежливо удалил тещу Боблина из этого измерения
\end{center}
\begin{align*}
\begin{autobreak}
\frac{d}{dx}(0) = 0
\end{autobreak}
\end{align*}

\begin{center}
Небольшой взмах посохом — и план сражения выглядит куда приличнее.
\end{center}
\begin{align*}
\begin{autobreak}
\frac{d}{dx}(0 - \sin(\sin(x))) = \frac{d}{dx}(0) - \frac{d}{dx}(\sin(\sin(x)))
\end{autobreak}
\end{align*}

\begin{center}
Один из гоблинов упал
\end{center}
\begin{align*}
\begin{autobreak}
\frac{d}{dx}(\sin(\sin(x))) = \cos(\sin(x)) \cdot \frac{d}{dx}(\sin(x))
\end{autobreak}
\end{align*}

\begin{center}
Родственники Боблина продолжают лезть к вам, держите посох крепче!
\end{center}
\begin{align*}
\begin{autobreak}
\frac{d}{dx}(\sin(x)) = \cos(x) \cdot \frac{d}{dx}(x)
\end{autobreak}
\end{align*}

\begin{center}
Полиморф сработал отлично: зять Боблина теперь лягушка
\end{center}
\begin{align*}
\begin{autobreak}
\frac{d}{dx}(x) = 1
\end{autobreak}
\end{align*}

\begin{center}
Ваше заклинание дезинтегрировало брата Боблина
\end{center}
\begin{align*}
\begin{autobreak}
\frac{d}{dx}(0) = 0
\end{autobreak}
\end{align*}

\begin{center}
Битва продолжается, не теряйте духу, они когда-то, наверное, закончаться!
\end{center}
\begin{align*}
\begin{autobreak}
\frac{d}{dx}((0 - \cos(\sin(x)) \cdot \cos(x)) \cdot \cos(x)) = \frac{d}{dx}(0 - \cos(\sin(x)) \cdot \cos(x)) \cdot \cos(x) + 0 - \cos(\sin(x)) \cdot \cos(x) \cdot \frac{d}{dx}(\cos(x))
\end{autobreak}
\end{align*}

\begin{center}
Их не становиться меньше, откуда они только лезут?!
\end{center}
\begin{align*}
\begin{autobreak}
\frac{d}{dx}(\cos(x)) = -\sin(x) \cdot \frac{d}{dx}(x)
\end{autobreak}
\end{align*}

\begin{center}
Поздравляю! От свояка Боблина осталась только полоыина
\end{center}
\begin{align*}
\begin{autobreak}
\frac{d}{dx}(x) = 1
\end{autobreak}
\end{align*}

\begin{center}
Вот так и рождаются легенды о герое, истребившем половину рода Боблина.
\end{center}
\begin{align*}
\begin{autobreak}
\frac{d}{dx}(0 - \cos(\sin(x)) \cdot \cos(x)) = \frac{d}{dx}(0) - \frac{d}{dx}(\cos(\sin(x)) \cdot \cos(x))
\end{autobreak}
\end{align*}

\begin{center}
Небольшой взмах посохом — и план сражения выглядит куда приличнее.
\end{center}
\begin{align*}
\begin{autobreak}
\frac{d}{dx}(\cos(\sin(x)) \cdot \cos(x)) = \frac{d}{dx}(\cos(\sin(x))) \cdot \cos(x) + \cos(\sin(x)) \cdot \frac{d}{dx}(\cos(x))
\end{autobreak}
\end{align*}

\begin{center}
Один из гоблинов упал
\end{center}
\begin{align*}
\begin{autobreak}
\frac{d}{dx}(\cos(x)) = -\sin(x) \cdot \frac{d}{dx}(x)
\end{autobreak}
\end{align*}

\begin{center}
Ваше заклинание свернуло невестку Боблина в шарик
\end{center}
\begin{align*}
\begin{autobreak}
\frac{d}{dx}(x) = 1
\end{autobreak}
\end{align*}

\begin{center}
Родственники Боблина продолжают лезть к вам, держите посох крепче!
\end{center}
\begin{align*}
\begin{autobreak}
\frac{d}{dx}(\cos(\sin(x))) = -\sin(\sin(x)) \cdot \frac{d}{dx}(\sin(x))
\end{autobreak}
\end{align*}

\begin{center}
Битва продолжается, не теряйте духу, они когда-то, наверное, закончаться!
\end{center}
\begin{align*}
\begin{autobreak}
\frac{d}{dx}(\sin(x)) = \cos(x) \cdot \frac{d}{dx}(x)
\end{autobreak}
\end{align*}

\begin{center}
ААХХАХААХАХ Гоблин-Боблин
\end{center}
\begin{align*}
\begin{autobreak}
\frac{d}{dx}(x) = 1
\end{autobreak}
\end{align*}

\begin{center}
Ваше колдовство низвело сестру Боблина до атомов
\end{center}
\begin{align*}
\begin{autobreak}
\frac{d}{dx}(0) = 0
\end{autobreak}
\end{align*}

\subsection{Прибывает 4-ая волна родственников Тейлора-Боблина}
A = \begin{align*}
\begin{autobreak}
(0 - ((0 - \sin(\sin(x))) \cdot \cos(x) \cdot \cos(x) + \cos(\sin(x)) \cdot (0 - \sin(x)))) \cdot \cos(x)
\end{autobreak}
\end{align*}

B = \begin{align*}
\begin{autobreak}
(0 - \cos(\sin(x)) \cdot \cos(x)) \cdot (0 - \sin(x)) + (0 - \sin(\sin(x))) \cdot (0 - \cos(x))
\end{autobreak}
\end{align*}

Текущий ход событий: \begin{align*}
\begin{autobreak}
A + (0 - \cos(\sin(x)) \cdot \cos(x)) \cdot (0 - \sin(x)) + B + 6
\end{autobreak}
\end{align*}

\noindent\hrulefill\begin{center}
Их не становиться меньше, откуда они только лезут?!
\end{center}
A = \begin{align*}
\begin{autobreak}
(0 - ((0 - \sin(\sin(x))) \cdot \cos(x) \cdot \cos(x) + \cos(\sin(x)) \cdot (0 - \sin(x)))) \cdot \cos(x)
\end{autobreak}
\end{align*}

B = \begin{align*}
\begin{autobreak}
(0 - \cos(\sin(x)) \cdot \cos(x)) \cdot (0 - \sin(x)) + (0 - \sin(\sin(x))) \cdot (0 - \cos(x))
\end{autobreak}
\end{align*}

C = \begin{align*}
\begin{autobreak}
(0 - \cos(\sin(x)) \cdot \cos(x)) \cdot (0 - \sin(x))
\end{autobreak}
\end{align*}

\begin{align*}
\begin{autobreak}
\frac{d}{dx}(A + C + B + 6) = \frac{d}{dx}(A + C + B) + \frac{d}{dx}(6)
\end{autobreak}
\end{align*}

\begin{center}
Вы разложили дядю Боблина на молекулы
\end{center}
\begin{align*}
\begin{autobreak}
\frac{d}{dx}(6) = 0
\end{autobreak}
\end{align*}

\begin{center}
Вот так и рождаются легенды о герое, истребившем половину рода Боблина.
\end{center}
A = \begin{align*}
\begin{autobreak}
(0 - ((0 - \sin(\sin(x))) \cdot \cos(x) \cdot \cos(x) + \cos(\sin(x)) \cdot (0 - \sin(x)))) \cdot \cos(x)
\end{autobreak}
\end{align*}

B = \begin{align*}
\begin{autobreak}
(0 - \cos(\sin(x)) \cdot \cos(x)) \cdot (0 - \sin(x)) + (0 - \sin(\sin(x))) \cdot (0 - \cos(x))
\end{autobreak}
\end{align*}

C = \begin{align*}
\begin{autobreak}
(0 - \cos(\sin(x)) \cdot \cos(x)) \cdot (0 - \sin(x))
\end{autobreak}
\end{align*}

\begin{align*}
\begin{autobreak}
\frac{d}{dx}(A + C + B) = \frac{d}{dx}(A + C) + \frac{d}{dx}(B)
\end{autobreak}
\end{align*}

\begin{center}
Небольшой взмах посохом — и план сражения выглядит куда приличнее.
\end{center}
A = \begin{align*}
\begin{autobreak}
(0 - \cos(\sin(x)) \cdot \cos(x)) \cdot (0 - \sin(x)) + (0 - \sin(\sin(x))) \cdot (0 - \cos(x))
\end{autobreak}
\end{align*}

\begin{align*}
\begin{autobreak}
\frac{d}{dx}(A) = \frac{d}{dx}((0 - \cos(\sin(x)) \cdot \cos(x)) \cdot (0 - \sin(x))) + \frac{d}{dx}((0 - \sin(\sin(x))) \cdot (0 - \cos(x)))
\end{autobreak}
\end{align*}

\begin{center}
Один из гоблинов упал
\end{center}
\begin{align*}
\begin{autobreak}
\frac{d}{dx}((0 - \sin(\sin(x))) \cdot (0 - \cos(x))) = \frac{d}{dx}(0 - \sin(\sin(x))) \cdot 0 - \cos(x) + 0 - \sin(\sin(x)) \cdot \frac{d}{dx}(0 - \cos(x))
\end{autobreak}
\end{align*}

\begin{center}
Родственники Боблина продолжают лезть к вам, держите посох крепче!
\end{center}
\begin{align*}
\begin{autobreak}
\frac{d}{dx}(0 - \cos(x)) = \frac{d}{dx}(0) - \frac{d}{dx}(\cos(x))
\end{autobreak}
\end{align*}

\begin{center}
Битва продолжается, не теряйте духу, они когда-то, наверное, закончаться!
\end{center}
\begin{align*}
\begin{autobreak}
\frac{d}{dx}(\cos(x)) = -\sin(x) \cdot \frac{d}{dx}(x)
\end{autobreak}
\end{align*}

\begin{center}
Ваше вошшебство откатило деверя Боблина до младенчества
\end{center}
\begin{align*}
\begin{autobreak}
\frac{d}{dx}(x) = 1
\end{autobreak}
\end{align*}

\begin{center}
Ваше волшебство оказалось не по зубам тёте Боблина, кстати, куда она делась?
\end{center}
\begin{align*}
\begin{autobreak}
\frac{d}{dx}(0) = 0
\end{autobreak}
\end{align*}

\begin{center}
Их не становиться меньше, откуда они только лезут?!
\end{center}
\begin{align*}
\begin{autobreak}
\frac{d}{dx}(0 - \sin(\sin(x))) = \frac{d}{dx}(0) - \frac{d}{dx}(\sin(\sin(x)))
\end{autobreak}
\end{align*}

\begin{center}
Вот так и рождаются легенды о герое, истребившем половину рода Боблина.
\end{center}
\begin{align*}
\begin{autobreak}
\frac{d}{dx}(\sin(\sin(x))) = \cos(\sin(x)) \cdot \frac{d}{dx}(\sin(x))
\end{autobreak}
\end{align*}

\begin{center}
Небольшой взмах посохом — и план сражения выглядит куда приличнее.
\end{center}
\begin{align*}
\begin{autobreak}
\frac{d}{dx}(\sin(x)) = \cos(x) \cdot \frac{d}{dx}(x)
\end{autobreak}
\end{align*}

\begin{center}
Полиморф сработал отлично: зять Боблина теперь лягушка
\end{center}
\begin{align*}
\begin{autobreak}
\frac{d}{dx}(x) = 1
\end{autobreak}
\end{align*}

\begin{center}
Огненный шар испарил бабушку Боблина
\end{center}
\begin{align*}
\begin{autobreak}
\frac{d}{dx}(0) = 0
\end{autobreak}
\end{align*}

\begin{center}
Один из гоблинов упал
\end{center}
A = \begin{align*}
\begin{autobreak}
(0 - \cos(\sin(x)) \cdot \cos(x)) \cdot (0 - \sin(x))
\end{autobreak}
\end{align*}

\begin{align*}
\begin{autobreak}
\frac{d}{dx}(A) = \frac{d}{dx}(0 - \cos(\sin(x)) \cdot \cos(x)) \cdot 0 - \sin(x) + 0 - \cos(\sin(x)) \cdot \cos(x) \cdot \frac{d}{dx}(0 - \sin(x))
\end{autobreak}
\end{align*}

\begin{center}
Родственники Боблина продолжают лезть к вам, держите посох крепче!
\end{center}
\begin{align*}
\begin{autobreak}
\frac{d}{dx}(0 - \sin(x)) = \frac{d}{dx}(0) - \frac{d}{dx}(\sin(x))
\end{autobreak}
\end{align*}

\begin{center}
Битва продолжается, не теряйте духу, они когда-то, наверное, закончаться!
\end{center}
\begin{align*}
\begin{autobreak}
\frac{d}{dx}(\sin(x)) = \cos(x) \cdot \frac{d}{dx}(x)
\end{autobreak}
\end{align*}

\begin{center}
Поздравляю! От свояка Боблина осталась только полоыина
\end{center}
\begin{align*}
\begin{autobreak}
\frac{d}{dx}(x) = 1
\end{autobreak}
\end{align*}

\begin{center}
Заклинание хаоса раскидало части племянника Боблина по разным планам
\end{center}
\begin{align*}
\begin{autobreak}
\frac{d}{dx}(0) = 0
\end{autobreak}
\end{align*}

\begin{center}
Их не становиться меньше, откуда они только лезут?!
\end{center}
\begin{align*}
\begin{autobreak}
\frac{d}{dx}(0 - \cos(\sin(x)) \cdot \cos(x)) = \frac{d}{dx}(0) - \frac{d}{dx}(\cos(\sin(x)) \cdot \cos(x))
\end{autobreak}
\end{align*}

\begin{center}
Вот так и рождаются легенды о герое, истребившем половину рода Боблина.
\end{center}
\begin{align*}
\begin{autobreak}
\frac{d}{dx}(\cos(\sin(x)) \cdot \cos(x)) = \frac{d}{dx}(\cos(\sin(x))) \cdot \cos(x) + \cos(\sin(x)) \cdot \frac{d}{dx}(\cos(x))
\end{autobreak}
\end{align*}

\begin{center}
Небольшой взмах посохом — и план сражения выглядит куда приличнее.
\end{center}
\begin{align*}
\begin{autobreak}
\frac{d}{dx}(\cos(x)) = -\sin(x) \cdot \frac{d}{dx}(x)
\end{autobreak}
\end{align*}

\begin{center}
Ваше заклинание свернуло невестку Боблина в шарик
\end{center}
\begin{align*}
\begin{autobreak}
\frac{d}{dx}(x) = 1
\end{autobreak}
\end{align*}

\begin{center}
Один из гоблинов упал
\end{center}
\begin{align*}
\begin{autobreak}
\frac{d}{dx}(\cos(\sin(x))) = -\sin(\sin(x)) \cdot \frac{d}{dx}(\sin(x))
\end{autobreak}
\end{align*}

\begin{center}
Родственники Боблина продолжают лезть к вам, держите посох крепче!
\end{center}
\begin{align*}
\begin{autobreak}
\frac{d}{dx}(\sin(x)) = \cos(x) \cdot \frac{d}{dx}(x)
\end{autobreak}
\end{align*}

\begin{center}
ААХХАХААХАХ Гоблин-Боблин
\end{center}
\begin{align*}
\begin{autobreak}
\frac{d}{dx}(x) = 1
\end{autobreak}
\end{align*}

\begin{center}
Ваш портал небытия вежливо удалил тещу Боблина из этого измерения
\end{center}
\begin{align*}
\begin{autobreak}
\frac{d}{dx}(0) = 0
\end{autobreak}
\end{align*}

\begin{center}
Битва продолжается, не теряйте духу, они когда-то, наверное, закончаться!
\end{center}
A = \begin{align*}
\begin{autobreak}
(0 - ((0 - \sin(\sin(x))) \cdot \cos(x) \cdot \cos(x) + \cos(\sin(x)) \cdot (0 - \sin(x)))) \cdot \cos(x)
\end{autobreak}
\end{align*}

B = \begin{align*}
\begin{autobreak}
(0 - \cos(\sin(x)) \cdot \cos(x)) \cdot (0 - \sin(x))
\end{autobreak}
\end{align*}

\begin{align*}
\begin{autobreak}
\frac{d}{dx}(A + B) = \frac{d}{dx}(A) + \frac{d}{dx}(B)
\end{autobreak}
\end{align*}

\begin{center}
Их не становиться меньше, откуда они только лезут?!
\end{center}
A = \begin{align*}
\begin{autobreak}
(0 - \cos(\sin(x)) \cdot \cos(x)) \cdot (0 - \sin(x))
\end{autobreak}
\end{align*}

\begin{align*}
\begin{autobreak}
\frac{d}{dx}(A) = \frac{d}{dx}(0 - \cos(\sin(x)) \cdot \cos(x)) \cdot 0 - \sin(x) + 0 - \cos(\sin(x)) \cdot \cos(x) \cdot \frac{d}{dx}(0 - \sin(x))
\end{autobreak}
\end{align*}

\begin{center}
Вот так и рождаются легенды о герое, истребившем половину рода Боблина.
\end{center}
\begin{align*}
\begin{autobreak}
\frac{d}{dx}(0 - \sin(x)) = \frac{d}{dx}(0) - \frac{d}{dx}(\sin(x))
\end{autobreak}
\end{align*}

\begin{center}
Небольшой взмах посохом — и план сражения выглядит куда приличнее.
\end{center}
\begin{align*}
\begin{autobreak}
\frac{d}{dx}(\sin(x)) = \cos(x) \cdot \frac{d}{dx}(x)
\end{autobreak}
\end{align*}

\begin{center}
Ваше вошшебство откатило деверя Боблина до младенчества
\end{center}
\begin{align*}
\begin{autobreak}
\frac{d}{dx}(x) = 1
\end{autobreak}
\end{align*}

\begin{center}
Ваше заклинание дезинтегрировало брата Боблина
\end{center}
\begin{align*}
\begin{autobreak}
\frac{d}{dx}(0) = 0
\end{autobreak}
\end{align*}

\begin{center}
Один из гоблинов упал
\end{center}
\begin{align*}
\begin{autobreak}
\frac{d}{dx}(0 - \cos(\sin(x)) \cdot \cos(x)) = \frac{d}{dx}(0) - \frac{d}{dx}(\cos(\sin(x)) \cdot \cos(x))
\end{autobreak}
\end{align*}

\begin{center}
Родственники Боблина продолжают лезть к вам, держите посох крепче!
\end{center}
\begin{align*}
\begin{autobreak}
\frac{d}{dx}(\cos(\sin(x)) \cdot \cos(x)) = \frac{d}{dx}(\cos(\sin(x))) \cdot \cos(x) + \cos(\sin(x)) \cdot \frac{d}{dx}(\cos(x))
\end{autobreak}
\end{align*}

\begin{center}
Битва продолжается, не теряйте духу, они когда-то, наверное, закончаться!
\end{center}
\begin{align*}
\begin{autobreak}
\frac{d}{dx}(\cos(x)) = -\sin(x) \cdot \frac{d}{dx}(x)
\end{autobreak}
\end{align*}

\begin{center}
Полиморф сработал отлично: зять Боблина теперь лягушка
\end{center}
\begin{align*}
\begin{autobreak}
\frac{d}{dx}(x) = 1
\end{autobreak}
\end{align*}

\begin{center}
Их не становиться меньше, откуда они только лезут?!
\end{center}
\begin{align*}
\begin{autobreak}
\frac{d}{dx}(\cos(\sin(x))) = -\sin(\sin(x)) \cdot \frac{d}{dx}(\sin(x))
\end{autobreak}
\end{align*}

\begin{center}
Вот так и рождаются легенды о герое, истребившем половину рода Боблина.
\end{center}
\begin{align*}
\begin{autobreak}
\frac{d}{dx}(\sin(x)) = \cos(x) \cdot \frac{d}{dx}(x)
\end{autobreak}
\end{align*}

\begin{center}
Поздравляю! От свояка Боблина осталась только полоыина
\end{center}
\begin{align*}
\begin{autobreak}
\frac{d}{dx}(x) = 1
\end{autobreak}
\end{align*}

\begin{center}
Ваше колдовство низвело сестру Боблина до атомов
\end{center}
\begin{align*}
\begin{autobreak}
\frac{d}{dx}(0) = 0
\end{autobreak}
\end{align*}

\begin{center}
Небольшой взмах посохом — и план сражения выглядит куда приличнее.
\end{center}
A = \begin{align*}
\begin{autobreak}
(0 - ((0 - \sin(\sin(x))) \cdot \cos(x) \cdot \cos(x) + \cos(\sin(x)) \cdot (0 - \sin(x)))) \cdot \cos(x)
\end{autobreak}
\end{align*}

\begin{align*}
\begin{autobreak}
\frac{d}{dx}(A) = \frac{d}{dx}(0 - ((0 - \sin(\sin(x))) \cdot \cos(x) \cdot \cos(x) + \cos(\sin(x)) \cdot (0 - \sin(x)))) \cdot \cos(x) + 0 - ((0 - \sin(\sin(x))) \cdot \cos(x) \cdot \cos(x) + \cos(\sin(x)) \cdot (0 - \sin(x))) \cdot \frac{d}{dx}(\cos(x))
\end{autobreak}
\end{align*}

\begin{center}
Один из гоблинов упал
\end{center}
\begin{align*}
\begin{autobreak}
\frac{d}{dx}(\cos(x)) = -\sin(x) \cdot \frac{d}{dx}(x)
\end{autobreak}
\end{align*}

\begin{center}
Ваше заклинание свернуло невестку Боблина в шарик
\end{center}
\begin{align*}
\begin{autobreak}
\frac{d}{dx}(x) = 1
\end{autobreak}
\end{align*}

\begin{center}
Родственники Боблина продолжают лезть к вам, держите посох крепче!
\end{center}
A = \begin{align*}
\begin{autobreak}
0 - ((0 - \sin(\sin(x))) \cdot \cos(x) \cdot \cos(x) + \cos(\sin(x)) \cdot (0 - \sin(x)))
\end{autobreak}
\end{align*}

\begin{align*}
\begin{autobreak}
\frac{d}{dx}(A) = \frac{d}{dx}(0) - \frac{d}{dx}((0 - \sin(\sin(x))) \cdot \cos(x) \cdot \cos(x) + \cos(\sin(x)) \cdot (0 - \sin(x)))
\end{autobreak}
\end{align*}

\begin{center}
Битва продолжается, не теряйте духу, они когда-то, наверное, закончаться!
\end{center}
A = \begin{align*}
\begin{autobreak}
(0 - \sin(\sin(x))) \cdot \cos(x) \cdot \cos(x) + \cos(\sin(x)) \cdot (0 - \sin(x))
\end{autobreak}
\end{align*}

\begin{align*}
\begin{autobreak}
\frac{d}{dx}(A) = \frac{d}{dx}((0 - \sin(\sin(x))) \cdot \cos(x) \cdot \cos(x)) + \frac{d}{dx}(\cos(\sin(x)) \cdot (0 - \sin(x)))
\end{autobreak}
\end{align*}

\begin{center}
Их не становиться меньше, откуда они только лезут?!
\end{center}
\begin{align*}
\begin{autobreak}
\frac{d}{dx}(\cos(\sin(x)) \cdot (0 - \sin(x))) = \frac{d}{dx}(\cos(\sin(x))) \cdot 0 - \sin(x) + \cos(\sin(x)) \cdot \frac{d}{dx}(0 - \sin(x))
\end{autobreak}
\end{align*}

\begin{center}
Вот так и рождаются легенды о герое, истребившем половину рода Боблина.
\end{center}
\begin{align*}
\begin{autobreak}
\frac{d}{dx}(0 - \sin(x)) = \frac{d}{dx}(0) - \frac{d}{dx}(\sin(x))
\end{autobreak}
\end{align*}

\begin{center}
Небольшой взмах посохом — и план сражения выглядит куда приличнее.
\end{center}
\begin{align*}
\begin{autobreak}
\frac{d}{dx}(\sin(x)) = \cos(x) \cdot \frac{d}{dx}(x)
\end{autobreak}
\end{align*}

\begin{center}
ААХХАХААХАХ Гоблин-Боблин
\end{center}
\begin{align*}
\begin{autobreak}
\frac{d}{dx}(x) = 1
\end{autobreak}
\end{align*}

\begin{center}
Вы разложили дядю Боблина на молекулы
\end{center}
\begin{align*}
\begin{autobreak}
\frac{d}{dx}(0) = 0
\end{autobreak}
\end{align*}

\begin{center}
Один из гоблинов упал
\end{center}
\begin{align*}
\begin{autobreak}
\frac{d}{dx}(\cos(\sin(x))) = -\sin(\sin(x)) \cdot \frac{d}{dx}(\sin(x))
\end{autobreak}
\end{align*}

\begin{center}
Родственники Боблина продолжают лезть к вам, держите посох крепче!
\end{center}
\begin{align*}
\begin{autobreak}
\frac{d}{dx}(\sin(x)) = \cos(x) \cdot \frac{d}{dx}(x)
\end{autobreak}
\end{align*}

\begin{center}
Ваше вошшебство откатило деверя Боблина до младенчества
\end{center}
\begin{align*}
\begin{autobreak}
\frac{d}{dx}(x) = 1
\end{autobreak}
\end{align*}

\begin{center}
Битва продолжается, не теряйте духу, они когда-то, наверное, закончаться!
\end{center}
\begin{align*}
\begin{autobreak}
\frac{d}{dx}((0 - \sin(\sin(x))) \cdot \cos(x) \cdot \cos(x)) = \frac{d}{dx}((0 - \sin(\sin(x))) \cdot \cos(x)) \cdot \cos(x) + (0 - \sin(\sin(x))) \cdot \cos(x) \cdot \frac{d}{dx}(\cos(x))
\end{autobreak}
\end{align*}

\begin{center}
Их не становиться меньше, откуда они только лезут?!
\end{center}
\begin{align*}
\begin{autobreak}
\frac{d}{dx}(\cos(x)) = -\sin(x) \cdot \frac{d}{dx}(x)
\end{autobreak}
\end{align*}

\begin{center}
Полиморф сработал отлично: зять Боблина теперь лягушка
\end{center}
\begin{align*}
\begin{autobreak}
\frac{d}{dx}(x) = 1
\end{autobreak}
\end{align*}

\begin{center}
Вот так и рождаются легенды о герое, истребившем половину рода Боблина.
\end{center}
\begin{align*}
\begin{autobreak}
\frac{d}{dx}((0 - \sin(\sin(x))) \cdot \cos(x)) = \frac{d}{dx}(0 - \sin(\sin(x))) \cdot \cos(x) + 0 - \sin(\sin(x)) \cdot \frac{d}{dx}(\cos(x))
\end{autobreak}
\end{align*}

\begin{center}
Небольшой взмах посохом — и план сражения выглядит куда приличнее.
\end{center}
\begin{align*}
\begin{autobreak}
\frac{d}{dx}(\cos(x)) = -\sin(x) \cdot \frac{d}{dx}(x)
\end{autobreak}
\end{align*}

\begin{center}
Поздравляю! От свояка Боблина осталась только полоыина
\end{center}
\begin{align*}
\begin{autobreak}
\frac{d}{dx}(x) = 1
\end{autobreak}
\end{align*}

\begin{center}
Один из гоблинов упал
\end{center}
\begin{align*}
\begin{autobreak}
\frac{d}{dx}(0 - \sin(\sin(x))) = \frac{d}{dx}(0) - \frac{d}{dx}(\sin(\sin(x)))
\end{autobreak}
\end{align*}

\begin{center}
Родственники Боблина продолжают лезть к вам, держите посох крепче!
\end{center}
\begin{align*}
\begin{autobreak}
\frac{d}{dx}(\sin(\sin(x))) = \cos(\sin(x)) \cdot \frac{d}{dx}(\sin(x))
\end{autobreak}
\end{align*}

\begin{center}
Битва продолжается, не теряйте духу, они когда-то, наверное, закончаться!
\end{center}
\begin{align*}
\begin{autobreak}
\frac{d}{dx}(\sin(x)) = \cos(x) \cdot \frac{d}{dx}(x)
\end{autobreak}
\end{align*}

\begin{center}
Ваше заклинание свернуло невестку Боблина в шарик
\end{center}
\begin{align*}
\begin{autobreak}
\frac{d}{dx}(x) = 1
\end{autobreak}
\end{align*}

\begin{center}
Ваше волшебство оказалось не по зубам тёте Боблина, кстати, куда она делась?
\end{center}
\begin{align*}
\begin{autobreak}
\frac{d}{dx}(0) = 0
\end{autobreak}
\end{align*}

\begin{center}
Огненный шар испарил бабушку Боблина
\end{center}
\begin{align*}
\begin{autobreak}
\frac{d}{dx}(0) = 0
\end{autobreak}
\end{align*}

\subsection{Прибывает 5-ая волна родственников Тейлора-Боблина}
A = \begin{align*}
\begin{autobreak}
(0 - \cos(\sin(x)) \cdot \cos(x)) \cdot (0 - \cos(x)) + (0 - \sin(\sin(x))) \cdot (0 - (0 - \sin(x)))
\end{autobreak}
\end{align*}

B = \begin{align*}
\begin{autobreak}
((0 - \cos(\sin(x)) \cdot \cos(x)) \cdot \cos(x) + (0 - \sin(\sin(x))) \cdot (0 - \sin(x))) \cdot \cos(x)
\end{autobreak}
\end{align*}

C = \begin{align*}
\begin{autobreak}
(0 - \sin(\sin(x))) \cdot \cos(x) \cdot (0 - \sin(x)) + \cos(\sin(x)) \cdot (0 - \cos(x))
\end{autobreak}
\end{align*}

D = \begin{align*}
\begin{autobreak}
0 - ((0 - \sin(\sin(x))) \cdot \cos(x) \cdot \cos(x) + \cos(\sin(x)) \cdot (0 - \sin(x)))
\end{autobreak}
\end{align*}

E = \begin{align*}
\begin{autobreak}
0 - ((0 - \sin(\sin(x))) \cdot \cos(x) \cdot \cos(x) + \cos(\sin(x)) \cdot (0 - \sin(x)))
\end{autobreak}
\end{align*}

F = \begin{align*}
\begin{autobreak}
0 - ((0 - \sin(\sin(x))) \cdot \cos(x) \cdot \cos(x) + \cos(\sin(x)) \cdot (0 - \sin(x)))
\end{autobreak}
\end{align*}

G = \begin{align*}
\begin{autobreak}
(0 - \sin(\sin(x))) \cdot \cos(x) \cdot (0 - \sin(x))
\end{autobreak}
\end{align*}

H = \begin{align*}
\begin{autobreak}
(0 - \cos(\sin(x)) \cdot \cos(x)) \cdot (0 - \cos(x))
\end{autobreak}
\end{align*}

I = \begin{align*}
\begin{autobreak}
(0 - \cos(\sin(x)) \cdot \cos(x)) \cdot (0 - \cos(x))
\end{autobreak}
\end{align*}

J = \begin{align*}
\begin{autobreak}
0 - \sin(x)
\end{autobreak}
\end{align*}

K = \begin{align*}
\begin{autobreak}
0 - \sin(x)
\end{autobreak}
\end{align*}

Текущий ход событий: \begin{align*}
\begin{autobreak}
(0 - (B + G + C)) \cdot \cos(x) + D \cdot J + E \cdot K + H + F \cdot (0 - \sin(x)) + I + A
\end{autobreak}
\end{align*}

\noindent\hrulefill\begin{center}
Их не становиться меньше, откуда они только лезут?!
\end{center}
A = \begin{align*}
\begin{autobreak}
(0 - \cos(\sin(x)) \cdot \cos(x)) \cdot (0 - \cos(x)) + (0 - \sin(\sin(x))) \cdot (0 - (0 - \sin(x)))
\end{autobreak}
\end{align*}

B = \begin{align*}
\begin{autobreak}
((0 - \cos(\sin(x)) \cdot \cos(x)) \cdot \cos(x) + (0 - \sin(\sin(x))) \cdot (0 - \sin(x))) \cdot \cos(x)
\end{autobreak}
\end{align*}

C = \begin{align*}
\begin{autobreak}
(0 - \sin(\sin(x))) \cdot \cos(x) \cdot (0 - \sin(x)) + \cos(\sin(x)) \cdot (0 - \cos(x))
\end{autobreak}
\end{align*}

D = \begin{align*}
\begin{autobreak}
0 - ((0 - \sin(\sin(x))) \cdot \cos(x) \cdot \cos(x) + \cos(\sin(x)) \cdot (0 - \sin(x)))
\end{autobreak}
\end{align*}

E = \begin{align*}
\begin{autobreak}
0 - ((0 - \sin(\sin(x))) \cdot \cos(x) \cdot \cos(x) + \cos(\sin(x)) \cdot (0 - \sin(x)))
\end{autobreak}
\end{align*}

F = \begin{align*}
\begin{autobreak}
0 - ((0 - \sin(\sin(x))) \cdot \cos(x) \cdot \cos(x) + \cos(\sin(x)) \cdot (0 - \sin(x)))
\end{autobreak}
\end{align*}

G = \begin{align*}
\begin{autobreak}
(0 - \sin(\sin(x))) \cdot \cos(x) \cdot (0 - \sin(x))
\end{autobreak}
\end{align*}

H = \begin{align*}
\begin{autobreak}
(0 - \cos(\sin(x)) \cdot \cos(x)) \cdot (0 - \cos(x))
\end{autobreak}
\end{align*}

I = \begin{align*}
\begin{autobreak}
(0 - \cos(\sin(x)) \cdot \cos(x)) \cdot (0 - \cos(x))
\end{autobreak}
\end{align*}

J = \begin{align*}
\begin{autobreak}
0 - \sin(x)
\end{autobreak}
\end{align*}

K = \begin{align*}
\begin{autobreak}
0 - \sin(x)
\end{autobreak}
\end{align*}

L = \begin{align*}
\begin{autobreak}
0 - \sin(x)
\end{autobreak}
\end{align*}

M = \begin{align*}
\begin{autobreak}
\cos(x)
\end{autobreak}
\end{align*}

N = \begin{align*}
\begin{autobreak}
0
\end{autobreak}
\end{align*}

\begin{align*}
\begin{autobreak}
\frac{d}{dx}((N - (B + G + C)) \cdot M + D \cdot J + E \cdot K + H + F \cdot L + I + A) = \frac{d}{dx}((N - (B + G + C)) \cdot M + D \cdot J + E \cdot K + H) + \frac{d}{dx}(F \cdot L + I + A)
\end{autobreak}
\end{align*}

\begin{center}
Вот так и рождаются легенды о герое, истребившем половину рода Боблина.
\end{center}
A = \begin{align*}
\begin{autobreak}
(0 - \cos(\sin(x)) \cdot \cos(x)) \cdot (0 - \cos(x)) + (0 - \sin(\sin(x))) \cdot (0 - (0 - \sin(x)))
\end{autobreak}
\end{align*}

B = \begin{align*}
\begin{autobreak}
0 - ((0 - \sin(\sin(x))) \cdot \cos(x) \cdot \cos(x) + \cos(\sin(x)) \cdot (0 - \sin(x)))
\end{autobreak}
\end{align*}

C = \begin{align*}
\begin{autobreak}
(0 - \cos(\sin(x)) \cdot \cos(x)) \cdot (0 - \cos(x))
\end{autobreak}
\end{align*}

\begin{align*}
\begin{autobreak}
\frac{d}{dx}(B \cdot (0 - \sin(x)) + C + A) = \frac{d}{dx}(B \cdot (0 - \sin(x)) + C) + \frac{d}{dx}(A)
\end{autobreak}
\end{align*}

\begin{center}
Небольшой взмах посохом — и план сражения выглядит куда приличнее.
\end{center}
A = \begin{align*}
\begin{autobreak}
(0 - \cos(\sin(x)) \cdot \cos(x)) \cdot (0 - \cos(x)) + (0 - \sin(\sin(x))) \cdot (0 - (0 - \sin(x)))
\end{autobreak}
\end{align*}

\begin{align*}
\begin{autobreak}
\frac{d}{dx}(A) = \frac{d}{dx}((0 - \cos(\sin(x)) \cdot \cos(x)) \cdot (0 - \cos(x))) + \frac{d}{dx}((0 - \sin(\sin(x))) \cdot (0 - (0 - \sin(x))))
\end{autobreak}
\end{align*}

\begin{center}
Один из гоблинов упал
\end{center}
A = \begin{align*}
\begin{autobreak}
(0 - \sin(\sin(x))) \cdot (0 - (0 - \sin(x)))
\end{autobreak}
\end{align*}

\begin{align*}
\begin{autobreak}
\frac{d}{dx}(A) = \frac{d}{dx}(0 - \sin(\sin(x))) \cdot 0 - (0 - \sin(x)) + 0 - \sin(\sin(x)) \cdot \frac{d}{dx}(0 - (0 - \sin(x)))
\end{autobreak}
\end{align*}

\begin{center}
Родственники Боблина продолжают лезть к вам, держите посох крепче!
\end{center}
\begin{align*}
\begin{autobreak}
\frac{d}{dx}(0 - (0 - \sin(x))) = \frac{d}{dx}(0) - \frac{d}{dx}(0 - \sin(x))
\end{autobreak}
\end{align*}

\begin{center}
Битва продолжается, не теряйте духу, они когда-то, наверное, закончаться!
\end{center}
\begin{align*}
\begin{autobreak}
\frac{d}{dx}(0 - \sin(x)) = \frac{d}{dx}(0) - \frac{d}{dx}(\sin(x))
\end{autobreak}
\end{align*}

\begin{center}
Их не становиться меньше, откуда они только лезут?!
\end{center}
\begin{align*}
\begin{autobreak}
\frac{d}{dx}(\sin(x)) = \cos(x) \cdot \frac{d}{dx}(x)
\end{autobreak}
\end{align*}

\begin{center}
ААХХАХААХАХ Гоблин-Боблин
\end{center}
\begin{align*}
\begin{autobreak}
\frac{d}{dx}(x) = 1
\end{autobreak}
\end{align*}

\begin{center}
Заклинание хаоса раскидало части племянника Боблина по разным планам
\end{center}
\begin{align*}
\begin{autobreak}
\frac{d}{dx}(0) = 0
\end{autobreak}
\end{align*}

\begin{center}
Ваш портал небытия вежливо удалил тещу Боблина из этого измерения
\end{center}
\begin{align*}
\begin{autobreak}
\frac{d}{dx}(0) = 0
\end{autobreak}
\end{align*}

\begin{center}
Вот так и рождаются легенды о герое, истребившем половину рода Боблина.
\end{center}
\begin{align*}
\begin{autobreak}
\frac{d}{dx}(0 - \sin(\sin(x))) = \frac{d}{dx}(0) - \frac{d}{dx}(\sin(\sin(x)))
\end{autobreak}
\end{align*}

\begin{center}
Небольшой взмах посохом — и план сражения выглядит куда приличнее.
\end{center}
\begin{align*}
\begin{autobreak}
\frac{d}{dx}(\sin(\sin(x))) = \cos(\sin(x)) \cdot \frac{d}{dx}(\sin(x))
\end{autobreak}
\end{align*}

\begin{center}
Один из гоблинов упал
\end{center}
\begin{align*}
\begin{autobreak}
\frac{d}{dx}(\sin(x)) = \cos(x) \cdot \frac{d}{dx}(x)
\end{autobreak}
\end{align*}

\begin{center}
Ваше вошшебство откатило деверя Боблина до младенчества
\end{center}
\begin{align*}
\begin{autobreak}
\frac{d}{dx}(x) = 1
\end{autobreak}
\end{align*}

\begin{center}
Ваше заклинание дезинтегрировало брата Боблина
\end{center}
\begin{align*}
\begin{autobreak}
\frac{d}{dx}(0) = 0
\end{autobreak}
\end{align*}

\begin{center}
Родственники Боблина продолжают лезть к вам, держите посох крепче!
\end{center}
A = \begin{align*}
\begin{autobreak}
(0 - \cos(\sin(x)) \cdot \cos(x)) \cdot (0 - \cos(x))
\end{autobreak}
\end{align*}

\begin{align*}
\begin{autobreak}
\frac{d}{dx}(A) = \frac{d}{dx}(0 - \cos(\sin(x)) \cdot \cos(x)) \cdot 0 - \cos(x) + 0 - \cos(\sin(x)) \cdot \cos(x) \cdot \frac{d}{dx}(0 - \cos(x))
\end{autobreak}
\end{align*}

\begin{center}
Битва продолжается, не теряйте духу, они когда-то, наверное, закончаться!
\end{center}
\begin{align*}
\begin{autobreak}
\frac{d}{dx}(0 - \cos(x)) = \frac{d}{dx}(0) - \frac{d}{dx}(\cos(x))
\end{autobreak}
\end{align*}

\begin{center}
Их не становиться меньше, откуда они только лезут?!
\end{center}
\begin{align*}
\begin{autobreak}
\frac{d}{dx}(\cos(x)) = -\sin(x) \cdot \frac{d}{dx}(x)
\end{autobreak}
\end{align*}

\begin{center}
Полиморф сработал отлично: зять Боблина теперь лягушка
\end{center}
\begin{align*}
\begin{autobreak}
\frac{d}{dx}(x) = 1
\end{autobreak}
\end{align*}

\begin{center}
Ваше колдовство низвело сестру Боблина до атомов
\end{center}
\begin{align*}
\begin{autobreak}
\frac{d}{dx}(0) = 0
\end{autobreak}
\end{align*}

\begin{center}
Вот так и рождаются легенды о герое, истребившем половину рода Боблина.
\end{center}
\begin{align*}
\begin{autobreak}
\frac{d}{dx}(0 - \cos(\sin(x)) \cdot \cos(x)) = \frac{d}{dx}(0) - \frac{d}{dx}(\cos(\sin(x)) \cdot \cos(x))
\end{autobreak}
\end{align*}

\begin{center}
Небольшой взмах посохом — и план сражения выглядит куда приличнее.
\end{center}
\begin{align*}
\begin{autobreak}
\frac{d}{dx}(\cos(\sin(x)) \cdot \cos(x)) = \frac{d}{dx}(\cos(\sin(x))) \cdot \cos(x) + \cos(\sin(x)) \cdot \frac{d}{dx}(\cos(x))
\end{autobreak}
\end{align*}

\begin{center}
Один из гоблинов упал
\end{center}
\begin{align*}
\begin{autobreak}
\frac{d}{dx}(\cos(x)) = -\sin(x) \cdot \frac{d}{dx}(x)
\end{autobreak}
\end{align*}

\begin{center}
Поздравляю! От свояка Боблина осталась только полоыина
\end{center}
\begin{align*}
\begin{autobreak}
\frac{d}{dx}(x) = 1
\end{autobreak}
\end{align*}

\begin{center}
Родственники Боблина продолжают лезть к вам, держите посох крепче!
\end{center}
\begin{align*}
\begin{autobreak}
\frac{d}{dx}(\cos(\sin(x))) = -\sin(\sin(x)) \cdot \frac{d}{dx}(\sin(x))
\end{autobreak}
\end{align*}

\begin{center}
Битва продолжается, не теряйте духу, они когда-то, наверное, закончаться!
\end{center}
\begin{align*}
\begin{autobreak}
\frac{d}{dx}(\sin(x)) = \cos(x) \cdot \frac{d}{dx}(x)
\end{autobreak}
\end{align*}

\begin{center}
Ваше заклинание свернуло невестку Боблина в шарик
\end{center}
\begin{align*}
\begin{autobreak}
\frac{d}{dx}(x) = 1
\end{autobreak}
\end{align*}

\begin{center}
Вы разложили дядю Боблина на молекулы
\end{center}
\begin{align*}
\begin{autobreak}
\frac{d}{dx}(0) = 0
\end{autobreak}
\end{align*}

\begin{center}
Их не становиться меньше, откуда они только лезут?!
\end{center}
A = \begin{align*}
\begin{autobreak}
0 - ((0 - \sin(\sin(x))) \cdot \cos(x) \cdot \cos(x) + \cos(\sin(x)) \cdot (0 - \sin(x)))
\end{autobreak}
\end{align*}

B = \begin{align*}
\begin{autobreak}
(0 - \cos(\sin(x)) \cdot \cos(x)) \cdot (0 - \cos(x))
\end{autobreak}
\end{align*}

\begin{align*}
\begin{autobreak}
\frac{d}{dx}(A \cdot (0 - \sin(x)) + B) = \frac{d}{dx}(A \cdot (0 - \sin(x))) + \frac{d}{dx}(B)
\end{autobreak}
\end{align*}

\begin{center}
Вот так и рождаются легенды о герое, истребившем половину рода Боблина.
\end{center}
A = \begin{align*}
\begin{autobreak}
(0 - \cos(\sin(x)) \cdot \cos(x)) \cdot (0 - \cos(x))
\end{autobreak}
\end{align*}

\begin{align*}
\begin{autobreak}
\frac{d}{dx}(A) = \frac{d}{dx}(0 - \cos(\sin(x)) \cdot \cos(x)) \cdot 0 - \cos(x) + 0 - \cos(\sin(x)) \cdot \cos(x) \cdot \frac{d}{dx}(0 - \cos(x))
\end{autobreak}
\end{align*}

\begin{center}
Небольшой взмах посохом — и план сражения выглядит куда приличнее.
\end{center}
\begin{align*}
\begin{autobreak}
\frac{d}{dx}(0 - \cos(x)) = \frac{d}{dx}(0) - \frac{d}{dx}(\cos(x))
\end{autobreak}
\end{align*}

\begin{center}
Один из гоблинов упал
\end{center}
\begin{align*}
\begin{autobreak}
\frac{d}{dx}(\cos(x)) = -\sin(x) \cdot \frac{d}{dx}(x)
\end{autobreak}
\end{align*}

\begin{center}
ААХХАХААХАХ Гоблин-Боблин
\end{center}
\begin{align*}
\begin{autobreak}
\frac{d}{dx}(x) = 1
\end{autobreak}
\end{align*}

\begin{center}
Ваше волшебство оказалось не по зубам тёте Боблина, кстати, куда она делась?
\end{center}
\begin{align*}
\begin{autobreak}
\frac{d}{dx}(0) = 0
\end{autobreak}
\end{align*}

\begin{center}
Родственники Боблина продолжают лезть к вам, держите посох крепче!
\end{center}
\begin{align*}
\begin{autobreak}
\frac{d}{dx}(0 - \cos(\sin(x)) \cdot \cos(x)) = \frac{d}{dx}(0) - \frac{d}{dx}(\cos(\sin(x)) \cdot \cos(x))
\end{autobreak}
\end{align*}

\begin{center}
Битва продолжается, не теряйте духу, они когда-то, наверное, закончаться!
\end{center}
\begin{align*}
\begin{autobreak}
\frac{d}{dx}(\cos(\sin(x)) \cdot \cos(x)) = \frac{d}{dx}(\cos(\sin(x))) \cdot \cos(x) + \cos(\sin(x)) \cdot \frac{d}{dx}(\cos(x))
\end{autobreak}
\end{align*}

\begin{center}
Их не становиться меньше, откуда они только лезут?!
\end{center}
\begin{align*}
\begin{autobreak}
\frac{d}{dx}(\cos(x)) = -\sin(x) \cdot \frac{d}{dx}(x)
\end{autobreak}
\end{align*}

\begin{center}
Ваше вошшебство откатило деверя Боблина до младенчества
\end{center}
\begin{align*}
\begin{autobreak}
\frac{d}{dx}(x) = 1
\end{autobreak}
\end{align*}

\begin{center}
Вот так и рождаются легенды о герое, истребившем половину рода Боблина.
\end{center}
\begin{align*}
\begin{autobreak}
\frac{d}{dx}(\cos(\sin(x))) = -\sin(\sin(x)) \cdot \frac{d}{dx}(\sin(x))
\end{autobreak}
\end{align*}

\begin{center}
Небольшой взмах посохом — и план сражения выглядит куда приличнее.
\end{center}
\begin{align*}
\begin{autobreak}
\frac{d}{dx}(\sin(x)) = \cos(x) \cdot \frac{d}{dx}(x)
\end{autobreak}
\end{align*}

\begin{center}
Полиморф сработал отлично: зять Боблина теперь лягушка
\end{center}
\begin{align*}
\begin{autobreak}
\frac{d}{dx}(x) = 1
\end{autobreak}
\end{align*}

\begin{center}
Огненный шар испарил бабушку Боблина
\end{center}
\begin{align*}
\begin{autobreak}
\frac{d}{dx}(0) = 0
\end{autobreak}
\end{align*}

\begin{center}
Один из гоблинов упал
\end{center}
A = \begin{align*}
\begin{autobreak}
0 - ((0 - \sin(\sin(x))) \cdot \cos(x) \cdot \cos(x) + \cos(\sin(x)) \cdot (0 - \sin(x)))
\end{autobreak}
\end{align*}

\begin{align*}
\begin{autobreak}
\frac{d}{dx}(A \cdot (0 - \sin(x))) = \frac{d}{dx}(A) \cdot 0 - \sin(x) + A \cdot \frac{d}{dx}(0 - \sin(x))
\end{autobreak}
\end{align*}

\begin{center}
Родственники Боблина продолжают лезть к вам, держите посох крепче!
\end{center}
\begin{align*}
\begin{autobreak}
\frac{d}{dx}(0 - \sin(x)) = \frac{d}{dx}(0) - \frac{d}{dx}(\sin(x))
\end{autobreak}
\end{align*}

\begin{center}
Битва продолжается, не теряйте духу, они когда-то, наверное, закончаться!
\end{center}
\begin{align*}
\begin{autobreak}
\frac{d}{dx}(\sin(x)) = \cos(x) \cdot \frac{d}{dx}(x)
\end{autobreak}
\end{align*}

\begin{center}
Поздравляю! От свояка Боблина осталась только полоыина
\end{center}
\begin{align*}
\begin{autobreak}
\frac{d}{dx}(x) = 1
\end{autobreak}
\end{align*}

\begin{center}
Заклинание хаоса раскидало части племянника Боблина по разным планам
\end{center}
\begin{align*}
\begin{autobreak}
\frac{d}{dx}(0) = 0
\end{autobreak}
\end{align*}

\begin{center}
Их не становиться меньше, откуда они только лезут?!
\end{center}
A = \begin{align*}
\begin{autobreak}
0 - ((0 - \sin(\sin(x))) \cdot \cos(x) \cdot \cos(x) + \cos(\sin(x)) \cdot (0 - \sin(x)))
\end{autobreak}
\end{align*}

\begin{align*}
\begin{autobreak}
\frac{d}{dx}(A) = \frac{d}{dx}(0) - \frac{d}{dx}((0 - \sin(\sin(x))) \cdot \cos(x) \cdot \cos(x) + \cos(\sin(x)) \cdot (0 - \sin(x)))
\end{autobreak}
\end{align*}

\begin{center}
Вот так и рождаются легенды о герое, истребившем половину рода Боблина.
\end{center}
A = \begin{align*}
\begin{autobreak}
(0 - \sin(\sin(x))) \cdot \cos(x) \cdot \cos(x) + \cos(\sin(x)) \cdot (0 - \sin(x))
\end{autobreak}
\end{align*}

\begin{align*}
\begin{autobreak}
\frac{d}{dx}(A) = \frac{d}{dx}((0 - \sin(\sin(x))) \cdot \cos(x) \cdot \cos(x)) + \frac{d}{dx}(\cos(\sin(x)) \cdot (0 - \sin(x)))
\end{autobreak}
\end{align*}

\begin{center}
Небольшой взмах посохом — и план сражения выглядит куда приличнее.
\end{center}
\begin{align*}
\begin{autobreak}
\frac{d}{dx}(\cos(\sin(x)) \cdot (0 - \sin(x))) = \frac{d}{dx}(\cos(\sin(x))) \cdot 0 - \sin(x) + \cos(\sin(x)) \cdot \frac{d}{dx}(0 - \sin(x))
\end{autobreak}
\end{align*}

\begin{center}
Один из гоблинов упал
\end{center}
\begin{align*}
\begin{autobreak}
\frac{d}{dx}(0 - \sin(x)) = \frac{d}{dx}(0) - \frac{d}{dx}(\sin(x))
\end{autobreak}
\end{align*}

\begin{center}
Родственники Боблина продолжают лезть к вам, держите посох крепче!
\end{center}
\begin{align*}
\begin{autobreak}
\frac{d}{dx}(\sin(x)) = \cos(x) \cdot \frac{d}{dx}(x)
\end{autobreak}
\end{align*}

\begin{center}
Ваше заклинание свернуло невестку Боблина в шарик
\end{center}
\begin{align*}
\begin{autobreak}
\frac{d}{dx}(x) = 1
\end{autobreak}
\end{align*}

\begin{center}
Ваш портал небытия вежливо удалил тещу Боблина из этого измерения
\end{center}
\begin{align*}
\begin{autobreak}
\frac{d}{dx}(0) = 0
\end{autobreak}
\end{align*}

\begin{center}
Битва продолжается, не теряйте духу, они когда-то, наверное, закончаться!
\end{center}
\begin{align*}
\begin{autobreak}
\frac{d}{dx}(\cos(\sin(x))) = -\sin(\sin(x)) \cdot \frac{d}{dx}(\sin(x))
\end{autobreak}
\end{align*}

\begin{center}
Их не становиться меньше, откуда они только лезут?!
\end{center}
\begin{align*}
\begin{autobreak}
\frac{d}{dx}(\sin(x)) = \cos(x) \cdot \frac{d}{dx}(x)
\end{autobreak}
\end{align*}

\begin{center}
ААХХАХААХАХ Гоблин-Боблин
\end{center}
\begin{align*}
\begin{autobreak}
\frac{d}{dx}(x) = 1
\end{autobreak}
\end{align*}

\begin{center}
Вот так и рождаются легенды о герое, истребившем половину рода Боблина.
\end{center}
\begin{align*}
\begin{autobreak}
\frac{d}{dx}((0 - \sin(\sin(x))) \cdot \cos(x) \cdot \cos(x)) = \frac{d}{dx}((0 - \sin(\sin(x))) \cdot \cos(x)) \cdot \cos(x) + (0 - \sin(\sin(x))) \cdot \cos(x) \cdot \frac{d}{dx}(\cos(x))
\end{autobreak}
\end{align*}

\begin{center}
Небольшой взмах посохом — и план сражения выглядит куда приличнее.
\end{center}
\begin{align*}
\begin{autobreak}
\frac{d}{dx}(\cos(x)) = -\sin(x) \cdot \frac{d}{dx}(x)
\end{autobreak}
\end{align*}

\begin{center}
Ваше вошшебство откатило деверя Боблина до младенчества
\end{center}
\begin{align*}
\begin{autobreak}
\frac{d}{dx}(x) = 1
\end{autobreak}
\end{align*}

\begin{center}
Один из гоблинов упал
\end{center}
\begin{align*}
\begin{autobreak}
\frac{d}{dx}((0 - \sin(\sin(x))) \cdot \cos(x)) = \frac{d}{dx}(0 - \sin(\sin(x))) \cdot \cos(x) + 0 - \sin(\sin(x)) \cdot \frac{d}{dx}(\cos(x))
\end{autobreak}
\end{align*}

\begin{center}
Родственники Боблина продолжают лезть к вам, держите посох крепче!
\end{center}
\begin{align*}
\begin{autobreak}
\frac{d}{dx}(\cos(x)) = -\sin(x) \cdot \frac{d}{dx}(x)
\end{autobreak}
\end{align*}

\begin{center}
Полиморф сработал отлично: зять Боблина теперь лягушка
\end{center}
\begin{align*}
\begin{autobreak}
\frac{d}{dx}(x) = 1
\end{autobreak}
\end{align*}

\begin{center}
Битва продолжается, не теряйте духу, они когда-то, наверное, закончаться!
\end{center}
\begin{align*}
\begin{autobreak}
\frac{d}{dx}(0 - \sin(\sin(x))) = \frac{d}{dx}(0) - \frac{d}{dx}(\sin(\sin(x)))
\end{autobreak}
\end{align*}

\begin{center}
Их не становиться меньше, откуда они только лезут?!
\end{center}
\begin{align*}
\begin{autobreak}
\frac{d}{dx}(\sin(\sin(x))) = \cos(\sin(x)) \cdot \frac{d}{dx}(\sin(x))
\end{autobreak}
\end{align*}

\begin{center}
Вот так и рождаются легенды о герое, истребившем половину рода Боблина.
\end{center}
\begin{align*}
\begin{autobreak}
\frac{d}{dx}(\sin(x)) = \cos(x) \cdot \frac{d}{dx}(x)
\end{autobreak}
\end{align*}

\begin{center}
Поздравляю! От свояка Боблина осталась только полоыина
\end{center}
\begin{align*}
\begin{autobreak}
\frac{d}{dx}(x) = 1
\end{autobreak}
\end{align*}

\begin{center}
Ваше заклинание дезинтегрировало брата Боблина
\end{center}
\begin{align*}
\begin{autobreak}
\frac{d}{dx}(0) = 0
\end{autobreak}
\end{align*}

\begin{center}
Ваше колдовство низвело сестру Боблина до атомов
\end{center}
\begin{align*}
\begin{autobreak}
\frac{d}{dx}(0) = 0
\end{autobreak}
\end{align*}

\begin{center}
Небольшой взмах посохом — и план сражения выглядит куда приличнее.
\end{center}
A = \begin{align*}
\begin{autobreak}
((0 - \cos(\sin(x)) \cdot \cos(x)) \cdot \cos(x) + (0 - \sin(\sin(x))) \cdot (0 - \sin(x))) \cdot \cos(x)
\end{autobreak}
\end{align*}

B = \begin{align*}
\begin{autobreak}
(0 - \sin(\sin(x))) \cdot \cos(x) \cdot (0 - \sin(x)) + \cos(\sin(x)) \cdot (0 - \cos(x))
\end{autobreak}
\end{align*}

C = \begin{align*}
\begin{autobreak}
0 - ((0 - \sin(\sin(x))) \cdot \cos(x) \cdot \cos(x) + \cos(\sin(x)) \cdot (0 - \sin(x)))
\end{autobreak}
\end{align*}

D = \begin{align*}
\begin{autobreak}
0 - ((0 - \sin(\sin(x))) \cdot \cos(x) \cdot \cos(x) + \cos(\sin(x)) \cdot (0 - \sin(x)))
\end{autobreak}
\end{align*}

E = \begin{align*}
\begin{autobreak}
(0 - \sin(\sin(x))) \cdot \cos(x) \cdot (0 - \sin(x))
\end{autobreak}
\end{align*}

F = \begin{align*}
\begin{autobreak}
(0 - \cos(\sin(x)) \cdot \cos(x)) \cdot (0 - \cos(x))
\end{autobreak}
\end{align*}

G = \begin{align*}
\begin{autobreak}
0 - \sin(x)
\end{autobreak}
\end{align*}

H = \begin{align*}
\begin{autobreak}
0 - \sin(x)
\end{autobreak}
\end{align*}

I = \begin{align*}
\begin{autobreak}
\cos(x)
\end{autobreak}
\end{align*}

J = \begin{align*}
\begin{autobreak}
0
\end{autobreak}
\end{align*}

\begin{align*}
\begin{autobreak}
\frac{d}{dx}((J - (A + E + B)) \cdot I + C \cdot G + D \cdot H + F) = \frac{d}{dx}((J - (A + E + B)) \cdot I + C \cdot G) + \frac{d}{dx}(D \cdot H + F)
\end{autobreak}
\end{align*}

\begin{center}
Один из гоблинов упал
\end{center}
A = \begin{align*}
\begin{autobreak}
0 - ((0 - \sin(\sin(x))) \cdot \cos(x) \cdot \cos(x) + \cos(\sin(x)) \cdot (0 - \sin(x)))
\end{autobreak}
\end{align*}

B = \begin{align*}
\begin{autobreak}
(0 - \cos(\sin(x)) \cdot \cos(x)) \cdot (0 - \cos(x))
\end{autobreak}
\end{align*}

\begin{align*}
\begin{autobreak}
\frac{d}{dx}(A \cdot (0 - \sin(x)) + B) = \frac{d}{dx}(A \cdot (0 - \sin(x))) + \frac{d}{dx}(B)
\end{autobreak}
\end{align*}

\begin{center}
Родственники Боблина продолжают лезть к вам, держите посох крепче!
\end{center}
A = \begin{align*}
\begin{autobreak}
(0 - \cos(\sin(x)) \cdot \cos(x)) \cdot (0 - \cos(x))
\end{autobreak}
\end{align*}

\begin{align*}
\begin{autobreak}
\frac{d}{dx}(A) = \frac{d}{dx}(0 - \cos(\sin(x)) \cdot \cos(x)) \cdot 0 - \cos(x) + 0 - \cos(\sin(x)) \cdot \cos(x) \cdot \frac{d}{dx}(0 - \cos(x))
\end{autobreak}
\end{align*}

\begin{center}
Битва продолжается, не теряйте духу, они когда-то, наверное, закончаться!
\end{center}
\begin{align*}
\begin{autobreak}
\frac{d}{dx}(0 - \cos(x)) = \frac{d}{dx}(0) - \frac{d}{dx}(\cos(x))
\end{autobreak}
\end{align*}

\begin{center}
Их не становиться меньше, откуда они только лезут?!
\end{center}
\begin{align*}
\begin{autobreak}
\frac{d}{dx}(\cos(x)) = -\sin(x) \cdot \frac{d}{dx}(x)
\end{autobreak}
\end{align*}

\begin{center}
Ваше заклинание свернуло невестку Боблина в шарик
\end{center}
\begin{align*}
\begin{autobreak}
\frac{d}{dx}(x) = 1
\end{autobreak}
\end{align*}

\begin{center}
Вы разложили дядю Боблина на молекулы
\end{center}
\begin{align*}
\begin{autobreak}
\frac{d}{dx}(0) = 0
\end{autobreak}
\end{align*}

\begin{center}
Вот так и рождаются легенды о герое, истребившем половину рода Боблина.
\end{center}
\begin{align*}
\begin{autobreak}
\frac{d}{dx}(0 - \cos(\sin(x)) \cdot \cos(x)) = \frac{d}{dx}(0) - \frac{d}{dx}(\cos(\sin(x)) \cdot \cos(x))
\end{autobreak}
\end{align*}

\begin{center}
Небольшой взмах посохом — и план сражения выглядит куда приличнее.
\end{center}
\begin{align*}
\begin{autobreak}
\frac{d}{dx}(\cos(\sin(x)) \cdot \cos(x)) = \frac{d}{dx}(\cos(\sin(x))) \cdot \cos(x) + \cos(\sin(x)) \cdot \frac{d}{dx}(\cos(x))
\end{autobreak}
\end{align*}

\begin{center}
Один из гоблинов упал
\end{center}
\begin{align*}
\begin{autobreak}
\frac{d}{dx}(\cos(x)) = -\sin(x) \cdot \frac{d}{dx}(x)
\end{autobreak}
\end{align*}

\begin{center}
ААХХАХААХАХ Гоблин-Боблин
\end{center}
\begin{align*}
\begin{autobreak}
\frac{d}{dx}(x) = 1
\end{autobreak}
\end{align*}

\begin{center}
Родственники Боблина продолжают лезть к вам, держите посох крепче!
\end{center}
\begin{align*}
\begin{autobreak}
\frac{d}{dx}(\cos(\sin(x))) = -\sin(\sin(x)) \cdot \frac{d}{dx}(\sin(x))
\end{autobreak}
\end{align*}

\begin{center}
Битва продолжается, не теряйте духу, они когда-то, наверное, закончаться!
\end{center}
\begin{align*}
\begin{autobreak}
\frac{d}{dx}(\sin(x)) = \cos(x) \cdot \frac{d}{dx}(x)
\end{autobreak}
\end{align*}

\begin{center}
Ваше вошшебство откатило деверя Боблина до младенчества
\end{center}
\begin{align*}
\begin{autobreak}
\frac{d}{dx}(x) = 1
\end{autobreak}
\end{align*}

\begin{center}
Ваше волшебство оказалось не по зубам тёте Боблина, кстати, куда она делась?
\end{center}
\begin{align*}
\begin{autobreak}
\frac{d}{dx}(0) = 0
\end{autobreak}
\end{align*}

\begin{center}
Их не становиться меньше, откуда они только лезут?!
\end{center}
A = \begin{align*}
\begin{autobreak}
0 - ((0 - \sin(\sin(x))) \cdot \cos(x) \cdot \cos(x) + \cos(\sin(x)) \cdot (0 - \sin(x)))
\end{autobreak}
\end{align*}

\begin{align*}
\begin{autobreak}
\frac{d}{dx}(A \cdot (0 - \sin(x))) = \frac{d}{dx}(A) \cdot 0 - \sin(x) + A \cdot \frac{d}{dx}(0 - \sin(x))
\end{autobreak}
\end{align*}

\begin{center}
Вот так и рождаются легенды о герое, истребившем половину рода Боблина.
\end{center}
\begin{align*}
\begin{autobreak}
\frac{d}{dx}(0 - \sin(x)) = \frac{d}{dx}(0) - \frac{d}{dx}(\sin(x))
\end{autobreak}
\end{align*}

\begin{center}
Небольшой взмах посохом — и план сражения выглядит куда приличнее.
\end{center}
\begin{align*}
\begin{autobreak}
\frac{d}{dx}(\sin(x)) = \cos(x) \cdot \frac{d}{dx}(x)
\end{autobreak}
\end{align*}

\begin{center}
Полиморф сработал отлично: зять Боблина теперь лягушка
\end{center}
\begin{align*}
\begin{autobreak}
\frac{d}{dx}(x) = 1
\end{autobreak}
\end{align*}

\begin{center}
Огненный шар испарил бабушку Боблина
\end{center}
\begin{align*}
\begin{autobreak}
\frac{d}{dx}(0) = 0
\end{autobreak}
\end{align*}

\begin{center}
Один из гоблинов упал
\end{center}
A = \begin{align*}
\begin{autobreak}
0 - ((0 - \sin(\sin(x))) \cdot \cos(x) \cdot \cos(x) + \cos(\sin(x)) \cdot (0 - \sin(x)))
\end{autobreak}
\end{align*}

\begin{align*}
\begin{autobreak}
\frac{d}{dx}(A) = \frac{d}{dx}(0) - \frac{d}{dx}((0 - \sin(\sin(x))) \cdot \cos(x) \cdot \cos(x) + \cos(\sin(x)) \cdot (0 - \sin(x)))
\end{autobreak}
\end{align*}

\begin{center}
Родственники Боблина продолжают лезть к вам, держите посох крепче!
\end{center}
A = \begin{align*}
\begin{autobreak}
(0 - \sin(\sin(x))) \cdot \cos(x) \cdot \cos(x) + \cos(\sin(x)) \cdot (0 - \sin(x))
\end{autobreak}
\end{align*}

\begin{align*}
\begin{autobreak}
\frac{d}{dx}(A) = \frac{d}{dx}((0 - \sin(\sin(x))) \cdot \cos(x) \cdot \cos(x)) + \frac{d}{dx}(\cos(\sin(x)) \cdot (0 - \sin(x)))
\end{autobreak}
\end{align*}

\begin{center}
Битва продолжается, не теряйте духу, они когда-то, наверное, закончаться!
\end{center}
\begin{align*}
\begin{autobreak}
\frac{d}{dx}(\cos(\sin(x)) \cdot (0 - \sin(x))) = \frac{d}{dx}(\cos(\sin(x))) \cdot 0 - \sin(x) + \cos(\sin(x)) \cdot \frac{d}{dx}(0 - \sin(x))
\end{autobreak}
\end{align*}

\begin{center}
Их не становиться меньше, откуда они только лезут?!
\end{center}
\begin{align*}
\begin{autobreak}
\frac{d}{dx}(0 - \sin(x)) = \frac{d}{dx}(0) - \frac{d}{dx}(\sin(x))
\end{autobreak}
\end{align*}

\begin{center}
Вот так и рождаются легенды о герое, истребившем половину рода Боблина.
\end{center}
\begin{align*}
\begin{autobreak}
\frac{d}{dx}(\sin(x)) = \cos(x) \cdot \frac{d}{dx}(x)
\end{autobreak}
\end{align*}

\begin{center}
Поздравляю! От свояка Боблина осталась только полоыина
\end{center}
\begin{align*}
\begin{autobreak}
\frac{d}{dx}(x) = 1
\end{autobreak}
\end{align*}

\begin{center}
Заклинание хаоса раскидало части племянника Боблина по разным планам
\end{center}
\begin{align*}
\begin{autobreak}
\frac{d}{dx}(0) = 0
\end{autobreak}
\end{align*}

\begin{center}
Небольшой взмах посохом — и план сражения выглядит куда приличнее.
\end{center}
\begin{align*}
\begin{autobreak}
\frac{d}{dx}(\cos(\sin(x))) = -\sin(\sin(x)) \cdot \frac{d}{dx}(\sin(x))
\end{autobreak}
\end{align*}

\begin{center}
Один из гоблинов упал
\end{center}
\begin{align*}
\begin{autobreak}
\frac{d}{dx}(\sin(x)) = \cos(x) \cdot \frac{d}{dx}(x)
\end{autobreak}
\end{align*}

\begin{center}
Ваше заклинание свернуло невестку Боблина в шарик
\end{center}
\begin{align*}
\begin{autobreak}
\frac{d}{dx}(x) = 1
\end{autobreak}
\end{align*}

\begin{center}
Родственники Боблина продолжают лезть к вам, держите посох крепче!
\end{center}
\begin{align*}
\begin{autobreak}
\frac{d}{dx}((0 - \sin(\sin(x))) \cdot \cos(x) \cdot \cos(x)) = \frac{d}{dx}((0 - \sin(\sin(x))) \cdot \cos(x)) \cdot \cos(x) + (0 - \sin(\sin(x))) \cdot \cos(x) \cdot \frac{d}{dx}(\cos(x))
\end{autobreak}
\end{align*}

\begin{center}
Битва продолжается, не теряйте духу, они когда-то, наверное, закончаться!
\end{center}
\begin{align*}
\begin{autobreak}
\frac{d}{dx}(\cos(x)) = -\sin(x) \cdot \frac{d}{dx}(x)
\end{autobreak}
\end{align*}

\begin{center}
ААХХАХААХАХ Гоблин-Боблин
\end{center}
\begin{align*}
\begin{autobreak}
\frac{d}{dx}(x) = 1
\end{autobreak}
\end{align*}

\begin{center}
Их не становиться меньше, откуда они только лезут?!
\end{center}
\begin{align*}
\begin{autobreak}
\frac{d}{dx}((0 - \sin(\sin(x))) \cdot \cos(x)) = \frac{d}{dx}(0 - \sin(\sin(x))) \cdot \cos(x) + 0 - \sin(\sin(x)) \cdot \frac{d}{dx}(\cos(x))
\end{autobreak}
\end{align*}

\begin{center}
Вот так и рождаются легенды о герое, истребившем половину рода Боблина.
\end{center}
\begin{align*}
\begin{autobreak}
\frac{d}{dx}(\cos(x)) = -\sin(x) \cdot \frac{d}{dx}(x)
\end{autobreak}
\end{align*}

\begin{center}
Ваше вошшебство откатило деверя Боблина до младенчества
\end{center}
\begin{align*}
\begin{autobreak}
\frac{d}{dx}(x) = 1
\end{autobreak}
\end{align*}

\begin{center}
Небольшой взмах посохом — и план сражения выглядит куда приличнее.
\end{center}
\begin{align*}
\begin{autobreak}
\frac{d}{dx}(0 - \sin(\sin(x))) = \frac{d}{dx}(0) - \frac{d}{dx}(\sin(\sin(x)))
\end{autobreak}
\end{align*}

\begin{center}
Один из гоблинов упал
\end{center}
\begin{align*}
\begin{autobreak}
\frac{d}{dx}(\sin(\sin(x))) = \cos(\sin(x)) \cdot \frac{d}{dx}(\sin(x))
\end{autobreak}
\end{align*}

\begin{center}
Родственники Боблина продолжают лезть к вам, держите посох крепче!
\end{center}
\begin{align*}
\begin{autobreak}
\frac{d}{dx}(\sin(x)) = \cos(x) \cdot \frac{d}{dx}(x)
\end{autobreak}
\end{align*}

\begin{center}
Полиморф сработал отлично: зять Боблина теперь лягушка
\end{center}
\begin{align*}
\begin{autobreak}
\frac{d}{dx}(x) = 1
\end{autobreak}
\end{align*}

\begin{center}
Ваш портал небытия вежливо удалил тещу Боблина из этого измерения
\end{center}
\begin{align*}
\begin{autobreak}
\frac{d}{dx}(0) = 0
\end{autobreak}
\end{align*}

\begin{center}
Ваше заклинание дезинтегрировало брата Боблина
\end{center}
\begin{align*}
\begin{autobreak}
\frac{d}{dx}(0) = 0
\end{autobreak}
\end{align*}

\begin{center}
Битва продолжается, не теряйте духу, они когда-то, наверное, закончаться!
\end{center}
A = \begin{align*}
\begin{autobreak}
((0 - \cos(\sin(x)) \cdot \cos(x)) \cdot \cos(x) + (0 - \sin(\sin(x))) \cdot (0 - \sin(x))) \cdot \cos(x)
\end{autobreak}
\end{align*}

B = \begin{align*}
\begin{autobreak}
(0 - \sin(\sin(x))) \cdot \cos(x) \cdot (0 - \sin(x)) + \cos(\sin(x)) \cdot (0 - \cos(x))
\end{autobreak}
\end{align*}

C = \begin{align*}
\begin{autobreak}
0 - ((0 - \sin(\sin(x))) \cdot \cos(x) \cdot \cos(x) + \cos(\sin(x)) \cdot (0 - \sin(x)))
\end{autobreak}
\end{align*}

D = \begin{align*}
\begin{autobreak}
(0 - \sin(\sin(x))) \cdot \cos(x) \cdot (0 - \sin(x))
\end{autobreak}
\end{align*}

E = \begin{align*}
\begin{autobreak}
0 - \sin(x)
\end{autobreak}
\end{align*}

\begin{align*}
\begin{autobreak}
\frac{d}{dx}((0 - (A + D + B)) \cdot \cos(x) + C \cdot E) = \frac{d}{dx}((0 - (A + D + B)) \cdot \cos(x)) + \frac{d}{dx}(C \cdot E)
\end{autobreak}
\end{align*}

\begin{center}
Их не становиться меньше, откуда они только лезут?!
\end{center}
A = \begin{align*}
\begin{autobreak}
0 - ((0 - \sin(\sin(x))) \cdot \cos(x) \cdot \cos(x) + \cos(\sin(x)) \cdot (0 - \sin(x)))
\end{autobreak}
\end{align*}

\begin{align*}
\begin{autobreak}
\frac{d}{dx}(A \cdot (0 - \sin(x))) = \frac{d}{dx}(A) \cdot 0 - \sin(x) + A \cdot \frac{d}{dx}(0 - \sin(x))
\end{autobreak}
\end{align*}

\begin{center}
Вот так и рождаются легенды о герое, истребившем половину рода Боблина.
\end{center}
\begin{align*}
\begin{autobreak}
\frac{d}{dx}(0 - \sin(x)) = \frac{d}{dx}(0) - \frac{d}{dx}(\sin(x))
\end{autobreak}
\end{align*}

\begin{center}
Небольшой взмах посохом — и план сражения выглядит куда приличнее.
\end{center}
\begin{align*}
\begin{autobreak}
\frac{d}{dx}(\sin(x)) = \cos(x) \cdot \frac{d}{dx}(x)
\end{autobreak}
\end{align*}

\begin{center}
Поздравляю! От свояка Боблина осталась только полоыина
\end{center}
\begin{align*}
\begin{autobreak}
\frac{d}{dx}(x) = 1
\end{autobreak}
\end{align*}

\begin{center}
Ваше колдовство низвело сестру Боблина до атомов
\end{center}
\begin{align*}
\begin{autobreak}
\frac{d}{dx}(0) = 0
\end{autobreak}
\end{align*}

\begin{center}
Один из гоблинов упал
\end{center}
A = \begin{align*}
\begin{autobreak}
0 - ((0 - \sin(\sin(x))) \cdot \cos(x) \cdot \cos(x) + \cos(\sin(x)) \cdot (0 - \sin(x)))
\end{autobreak}
\end{align*}

\begin{align*}
\begin{autobreak}
\frac{d}{dx}(A) = \frac{d}{dx}(0) - \frac{d}{dx}((0 - \sin(\sin(x))) \cdot \cos(x) \cdot \cos(x) + \cos(\sin(x)) \cdot (0 - \sin(x)))
\end{autobreak}
\end{align*}

\begin{center}
Родственники Боблина продолжают лезть к вам, держите посох крепче!
\end{center}
A = \begin{align*}
\begin{autobreak}
(0 - \sin(\sin(x))) \cdot \cos(x) \cdot \cos(x) + \cos(\sin(x)) \cdot (0 - \sin(x))
\end{autobreak}
\end{align*}

\begin{align*}
\begin{autobreak}
\frac{d}{dx}(A) = \frac{d}{dx}((0 - \sin(\sin(x))) \cdot \cos(x) \cdot \cos(x)) + \frac{d}{dx}(\cos(\sin(x)) \cdot (0 - \sin(x)))
\end{autobreak}
\end{align*}

\begin{center}
Битва продолжается, не теряйте духу, они когда-то, наверное, закончаться!
\end{center}
\begin{align*}
\begin{autobreak}
\frac{d}{dx}(\cos(\sin(x)) \cdot (0 - \sin(x))) = \frac{d}{dx}(\cos(\sin(x))) \cdot 0 - \sin(x) + \cos(\sin(x)) \cdot \frac{d}{dx}(0 - \sin(x))
\end{autobreak}
\end{align*}

\begin{center}
Их не становиться меньше, откуда они только лезут?!
\end{center}
\begin{align*}
\begin{autobreak}
\frac{d}{dx}(0 - \sin(x)) = \frac{d}{dx}(0) - \frac{d}{dx}(\sin(x))
\end{autobreak}
\end{align*}

\begin{center}
Вот так и рождаются легенды о герое, истребившем половину рода Боблина.
\end{center}
\begin{align*}
\begin{autobreak}
\frac{d}{dx}(\sin(x)) = \cos(x) \cdot \frac{d}{dx}(x)
\end{autobreak}
\end{align*}

\begin{center}
Ваше заклинание свернуло невестку Боблина в шарик
\end{center}
\begin{align*}
\begin{autobreak}
\frac{d}{dx}(x) = 1
\end{autobreak}
\end{align*}

\begin{center}
Вы разложили дядю Боблина на молекулы
\end{center}
\begin{align*}
\begin{autobreak}
\frac{d}{dx}(0) = 0
\end{autobreak}
\end{align*}

\begin{center}
Небольшой взмах посохом — и план сражения выглядит куда приличнее.
\end{center}
\begin{align*}
\begin{autobreak}
\frac{d}{dx}(\cos(\sin(x))) = -\sin(\sin(x)) \cdot \frac{d}{dx}(\sin(x))
\end{autobreak}
\end{align*}

\begin{center}
Один из гоблинов упал
\end{center}
\begin{align*}
\begin{autobreak}
\frac{d}{dx}(\sin(x)) = \cos(x) \cdot \frac{d}{dx}(x)
\end{autobreak}
\end{align*}

\begin{center}
ААХХАХААХАХ Гоблин-Боблин
\end{center}
\begin{align*}
\begin{autobreak}
\frac{d}{dx}(x) = 1
\end{autobreak}
\end{align*}

\begin{center}
Родственники Боблина продолжают лезть к вам, держите посох крепче!
\end{center}
\begin{align*}
\begin{autobreak}
\frac{d}{dx}((0 - \sin(\sin(x))) \cdot \cos(x) \cdot \cos(x)) = \frac{d}{dx}((0 - \sin(\sin(x))) \cdot \cos(x)) \cdot \cos(x) + (0 - \sin(\sin(x))) \cdot \cos(x) \cdot \frac{d}{dx}(\cos(x))
\end{autobreak}
\end{align*}

\begin{center}
Битва продолжается, не теряйте духу, они когда-то, наверное, закончаться!
\end{center}
\begin{align*}
\begin{autobreak}
\frac{d}{dx}(\cos(x)) = -\sin(x) \cdot \frac{d}{dx}(x)
\end{autobreak}
\end{align*}

\begin{center}
Ваше вошшебство откатило деверя Боблина до младенчества
\end{center}
\begin{align*}
\begin{autobreak}
\frac{d}{dx}(x) = 1
\end{autobreak}
\end{align*}

\begin{center}
Их не становиться меньше, откуда они только лезут?!
\end{center}
\begin{align*}
\begin{autobreak}
\frac{d}{dx}((0 - \sin(\sin(x))) \cdot \cos(x)) = \frac{d}{dx}(0 - \sin(\sin(x))) \cdot \cos(x) + 0 - \sin(\sin(x)) \cdot \frac{d}{dx}(\cos(x))
\end{autobreak}
\end{align*}

\begin{center}
Вот так и рождаются легенды о герое, истребившем половину рода Боблина.
\end{center}
\begin{align*}
\begin{autobreak}
\frac{d}{dx}(\cos(x)) = -\sin(x) \cdot \frac{d}{dx}(x)
\end{autobreak}
\end{align*}

\begin{center}
Полиморф сработал отлично: зять Боблина теперь лягушка
\end{center}
\begin{align*}
\begin{autobreak}
\frac{d}{dx}(x) = 1
\end{autobreak}
\end{align*}

\begin{center}
Небольшой взмах посохом — и план сражения выглядит куда приличнее.
\end{center}
\begin{align*}
\begin{autobreak}
\frac{d}{dx}(0 - \sin(\sin(x))) = \frac{d}{dx}(0) - \frac{d}{dx}(\sin(\sin(x)))
\end{autobreak}
\end{align*}

\begin{center}
Один из гоблинов упал
\end{center}
\begin{align*}
\begin{autobreak}
\frac{d}{dx}(\sin(\sin(x))) = \cos(\sin(x)) \cdot \frac{d}{dx}(\sin(x))
\end{autobreak}
\end{align*}

\begin{center}
Родственники Боблина продолжают лезть к вам, держите посох крепче!
\end{center}
\begin{align*}
\begin{autobreak}
\frac{d}{dx}(\sin(x)) = \cos(x) \cdot \frac{d}{dx}(x)
\end{autobreak}
\end{align*}

\begin{center}
Поздравляю! От свояка Боблина осталась только полоыина
\end{center}
\begin{align*}
\begin{autobreak}
\frac{d}{dx}(x) = 1
\end{autobreak}
\end{align*}

\begin{center}
Ваше волшебство оказалось не по зубам тёте Боблина, кстати, куда она делась?
\end{center}
\begin{align*}
\begin{autobreak}
\frac{d}{dx}(0) = 0
\end{autobreak}
\end{align*}

\begin{center}
Огненный шар испарил бабушку Боблина
\end{center}
\begin{align*}
\begin{autobreak}
\frac{d}{dx}(0) = 0
\end{autobreak}
\end{align*}

\begin{center}
Битва продолжается, не теряйте духу, они когда-то, наверное, закончаться!
\end{center}
A = \begin{align*}
\begin{autobreak}
((0 - \cos(\sin(x)) \cdot \cos(x)) \cdot \cos(x) + (0 - \sin(\sin(x))) \cdot (0 - \sin(x))) \cdot \cos(x)
\end{autobreak}
\end{align*}

B = \begin{align*}
\begin{autobreak}
(0 - \sin(\sin(x))) \cdot \cos(x) \cdot (0 - \sin(x)) + \cos(\sin(x)) \cdot (0 - \cos(x))
\end{autobreak}
\end{align*}

C = \begin{align*}
\begin{autobreak}
(0 - \sin(\sin(x))) \cdot \cos(x) \cdot (0 - \sin(x))
\end{autobreak}
\end{align*}

\begin{align*}
\begin{autobreak}
\frac{d}{dx}((0 - (A + C + B)) \cdot \cos(x)) = \frac{d}{dx}(0 - (A + C + B)) \cdot \cos(x) + 0 - (A + C + B) \cdot \frac{d}{dx}(\cos(x))
\end{autobreak}
\end{align*}

\begin{center}
Их не становиться меньше, откуда они только лезут?!
\end{center}
\begin{align*}
\begin{autobreak}
\frac{d}{dx}(\cos(x)) = -\sin(x) \cdot \frac{d}{dx}(x)
\end{autobreak}
\end{align*}

\begin{center}
Ваше заклинание свернуло невестку Боблина в шарик
\end{center}
\begin{align*}
\begin{autobreak}
\frac{d}{dx}(x) = 1
\end{autobreak}
\end{align*}

\begin{center}
Вот так и рождаются легенды о герое, истребившем половину рода Боблина.
\end{center}
A = \begin{align*}
\begin{autobreak}
((0 - \cos(\sin(x)) \cdot \cos(x)) \cdot \cos(x) + (0 - \sin(\sin(x))) \cdot (0 - \sin(x))) \cdot \cos(x)
\end{autobreak}
\end{align*}

B = \begin{align*}
\begin{autobreak}
(0 - \sin(\sin(x))) \cdot \cos(x) \cdot (0 - \sin(x)) + \cos(\sin(x)) \cdot (0 - \cos(x))
\end{autobreak}
\end{align*}

C = \begin{align*}
\begin{autobreak}
(0 - \sin(\sin(x))) \cdot \cos(x) \cdot (0 - \sin(x))
\end{autobreak}
\end{align*}

\begin{align*}
\begin{autobreak}
\frac{d}{dx}(0 - (A + C + B)) = \frac{d}{dx}(0) - \frac{d}{dx}(A + C + B)
\end{autobreak}
\end{align*}

\begin{center}
Небольшой взмах посохом — и план сражения выглядит куда приличнее.
\end{center}
A = \begin{align*}
\begin{autobreak}
((0 - \cos(\sin(x)) \cdot \cos(x)) \cdot \cos(x) + (0 - \sin(\sin(x))) \cdot (0 - \sin(x))) \cdot \cos(x)
\end{autobreak}
\end{align*}

B = \begin{align*}
\begin{autobreak}
(0 - \sin(\sin(x))) \cdot \cos(x) \cdot (0 - \sin(x)) + \cos(\sin(x)) \cdot (0 - \cos(x))
\end{autobreak}
\end{align*}

C = \begin{align*}
\begin{autobreak}
(0 - \sin(\sin(x))) \cdot \cos(x) \cdot (0 - \sin(x))
\end{autobreak}
\end{align*}

\begin{align*}
\begin{autobreak}
\frac{d}{dx}(A + C + B) = \frac{d}{dx}(A + C) + \frac{d}{dx}(B)
\end{autobreak}
\end{align*}

\begin{center}
Один из гоблинов упал
\end{center}
A = \begin{align*}
\begin{autobreak}
(0 - \sin(\sin(x))) \cdot \cos(x) \cdot (0 - \sin(x)) + \cos(\sin(x)) \cdot (0 - \cos(x))
\end{autobreak}
\end{align*}

\begin{align*}
\begin{autobreak}
\frac{d}{dx}(A) = \frac{d}{dx}((0 - \sin(\sin(x))) \cdot \cos(x) \cdot (0 - \sin(x))) + \frac{d}{dx}(\cos(\sin(x)) \cdot (0 - \cos(x)))
\end{autobreak}
\end{align*}

\begin{center}
Родственники Боблина продолжают лезть к вам, держите посох крепче!
\end{center}
\begin{align*}
\begin{autobreak}
\frac{d}{dx}(\cos(\sin(x)) \cdot (0 - \cos(x))) = \frac{d}{dx}(\cos(\sin(x))) \cdot 0 - \cos(x) + \cos(\sin(x)) \cdot \frac{d}{dx}(0 - \cos(x))
\end{autobreak}
\end{align*}

\begin{center}
Битва продолжается, не теряйте духу, они когда-то, наверное, закончаться!
\end{center}
\begin{align*}
\begin{autobreak}
\frac{d}{dx}(0 - \cos(x)) = \frac{d}{dx}(0) - \frac{d}{dx}(\cos(x))
\end{autobreak}
\end{align*}

\begin{center}
Их не становиться меньше, откуда они только лезут?!
\end{center}
\begin{align*}
\begin{autobreak}
\frac{d}{dx}(\cos(x)) = -\sin(x) \cdot \frac{d}{dx}(x)
\end{autobreak}
\end{align*}

\begin{center}
ААХХАХААХАХ Гоблин-Боблин
\end{center}
\begin{align*}
\begin{autobreak}
\frac{d}{dx}(x) = 1
\end{autobreak}
\end{align*}

\begin{center}
Заклинание хаоса раскидало части племянника Боблина по разным планам
\end{center}
\begin{align*}
\begin{autobreak}
\frac{d}{dx}(0) = 0
\end{autobreak}
\end{align*}

\begin{center}
Вот так и рождаются легенды о герое, истребившем половину рода Боблина.
\end{center}
\begin{align*}
\begin{autobreak}
\frac{d}{dx}(\cos(\sin(x))) = -\sin(\sin(x)) \cdot \frac{d}{dx}(\sin(x))
\end{autobreak}
\end{align*}

\begin{center}
Небольшой взмах посохом — и план сражения выглядит куда приличнее.
\end{center}
\begin{align*}
\begin{autobreak}
\frac{d}{dx}(\sin(x)) = \cos(x) \cdot \frac{d}{dx}(x)
\end{autobreak}
\end{align*}

\begin{center}
Ваше вошшебство откатило деверя Боблина до младенчества
\end{center}
\begin{align*}
\begin{autobreak}
\frac{d}{dx}(x) = 1
\end{autobreak}
\end{align*}

\begin{center}
Один из гоблинов упал
\end{center}
A = \begin{align*}
\begin{autobreak}
(0 - \sin(\sin(x))) \cdot \cos(x) \cdot (0 - \sin(x))
\end{autobreak}
\end{align*}

\begin{align*}
\begin{autobreak}
\frac{d}{dx}(A) = \frac{d}{dx}((0 - \sin(\sin(x))) \cdot \cos(x)) \cdot 0 - \sin(x) + (0 - \sin(\sin(x))) \cdot \cos(x) \cdot \frac{d}{dx}(0 - \sin(x))
\end{autobreak}
\end{align*}

\begin{center}
Родственники Боблина продолжают лезть к вам, держите посох крепче!
\end{center}
\begin{align*}
\begin{autobreak}
\frac{d}{dx}(0 - \sin(x)) = \frac{d}{dx}(0) - \frac{d}{dx}(\sin(x))
\end{autobreak}
\end{align*}

\begin{center}
Битва продолжается, не теряйте духу, они когда-то, наверное, закончаться!
\end{center}
\begin{align*}
\begin{autobreak}
\frac{d}{dx}(\sin(x)) = \cos(x) \cdot \frac{d}{dx}(x)
\end{autobreak}
\end{align*}

\begin{center}
Полиморф сработал отлично: зять Боблина теперь лягушка
\end{center}
\begin{align*}
\begin{autobreak}
\frac{d}{dx}(x) = 1
\end{autobreak}
\end{align*}

\begin{center}
Ваш портал небытия вежливо удалил тещу Боблина из этого измерения
\end{center}
\begin{align*}
\begin{autobreak}
\frac{d}{dx}(0) = 0
\end{autobreak}
\end{align*}

\begin{center}
Их не становиться меньше, откуда они только лезут?!
\end{center}
\begin{align*}
\begin{autobreak}
\frac{d}{dx}((0 - \sin(\sin(x))) \cdot \cos(x)) = \frac{d}{dx}(0 - \sin(\sin(x))) \cdot \cos(x) + 0 - \sin(\sin(x)) \cdot \frac{d}{dx}(\cos(x))
\end{autobreak}
\end{align*}

\begin{center}
Вот так и рождаются легенды о герое, истребившем половину рода Боблина.
\end{center}
\begin{align*}
\begin{autobreak}
\frac{d}{dx}(\cos(x)) = -\sin(x) \cdot \frac{d}{dx}(x)
\end{autobreak}
\end{align*}

\begin{center}
Поздравляю! От свояка Боблина осталась только полоыина
\end{center}
\begin{align*}
\begin{autobreak}
\frac{d}{dx}(x) = 1
\end{autobreak}
\end{align*}

\begin{center}
Небольшой взмах посохом — и план сражения выглядит куда приличнее.
\end{center}
\begin{align*}
\begin{autobreak}
\frac{d}{dx}(0 - \sin(\sin(x))) = \frac{d}{dx}(0) - \frac{d}{dx}(\sin(\sin(x)))
\end{autobreak}
\end{align*}

\begin{center}
Один из гоблинов упал
\end{center}
\begin{align*}
\begin{autobreak}
\frac{d}{dx}(\sin(\sin(x))) = \cos(\sin(x)) \cdot \frac{d}{dx}(\sin(x))
\end{autobreak}
\end{align*}

\begin{center}
Родственники Боблина продолжают лезть к вам, держите посох крепче!
\end{center}
\begin{align*}
\begin{autobreak}
\frac{d}{dx}(\sin(x)) = \cos(x) \cdot \frac{d}{dx}(x)
\end{autobreak}
\end{align*}

\begin{center}
Ваше заклинание свернуло невестку Боблина в шарик
\end{center}
\begin{align*}
\begin{autobreak}
\frac{d}{dx}(x) = 1
\end{autobreak}
\end{align*}

\begin{center}
Ваше заклинание дезинтегрировало брата Боблина
\end{center}
\begin{align*}
\begin{autobreak}
\frac{d}{dx}(0) = 0
\end{autobreak}
\end{align*}

\begin{center}
Битва продолжается, не теряйте духу, они когда-то, наверное, закончаться!
\end{center}
A = \begin{align*}
\begin{autobreak}
((0 - \cos(\sin(x)) \cdot \cos(x)) \cdot \cos(x) + (0 - \sin(\sin(x))) \cdot (0 - \sin(x))) \cdot \cos(x)
\end{autobreak}
\end{align*}

B = \begin{align*}
\begin{autobreak}
(0 - \sin(\sin(x))) \cdot \cos(x) \cdot (0 - \sin(x))
\end{autobreak}
\end{align*}

\begin{align*}
\begin{autobreak}
\frac{d}{dx}(A + B) = \frac{d}{dx}(A) + \frac{d}{dx}(B)
\end{autobreak}
\end{align*}

\begin{center}
Их не становиться меньше, откуда они только лезут?!
\end{center}
A = \begin{align*}
\begin{autobreak}
(0 - \sin(\sin(x))) \cdot \cos(x) \cdot (0 - \sin(x))
\end{autobreak}
\end{align*}

\begin{align*}
\begin{autobreak}
\frac{d}{dx}(A) = \frac{d}{dx}((0 - \sin(\sin(x))) \cdot \cos(x)) \cdot 0 - \sin(x) + (0 - \sin(\sin(x))) \cdot \cos(x) \cdot \frac{d}{dx}(0 - \sin(x))
\end{autobreak}
\end{align*}

\begin{center}
Вот так и рождаются легенды о герое, истребившем половину рода Боблина.
\end{center}
\begin{align*}
\begin{autobreak}
\frac{d}{dx}(0 - \sin(x)) = \frac{d}{dx}(0) - \frac{d}{dx}(\sin(x))
\end{autobreak}
\end{align*}

\begin{center}
Небольшой взмах посохом — и план сражения выглядит куда приличнее.
\end{center}
\begin{align*}
\begin{autobreak}
\frac{d}{dx}(\sin(x)) = \cos(x) \cdot \frac{d}{dx}(x)
\end{autobreak}
\end{align*}

\begin{center}
ААХХАХААХАХ Гоблин-Боблин
\end{center}
\begin{align*}
\begin{autobreak}
\frac{d}{dx}(x) = 1
\end{autobreak}
\end{align*}

\begin{center}
Ваше колдовство низвело сестру Боблина до атомов
\end{center}
\begin{align*}
\begin{autobreak}
\frac{d}{dx}(0) = 0
\end{autobreak}
\end{align*}

\begin{center}
Один из гоблинов упал
\end{center}
\begin{align*}
\begin{autobreak}
\frac{d}{dx}((0 - \sin(\sin(x))) \cdot \cos(x)) = \frac{d}{dx}(0 - \sin(\sin(x))) \cdot \cos(x) + 0 - \sin(\sin(x)) \cdot \frac{d}{dx}(\cos(x))
\end{autobreak}
\end{align*}

\begin{center}
Родственники Боблина продолжают лезть к вам, держите посох крепче!
\end{center}
\begin{align*}
\begin{autobreak}
\frac{d}{dx}(\cos(x)) = -\sin(x) \cdot \frac{d}{dx}(x)
\end{autobreak}
\end{align*}

\begin{center}
Ваше вошшебство откатило деверя Боблина до младенчества
\end{center}
\begin{align*}
\begin{autobreak}
\frac{d}{dx}(x) = 1
\end{autobreak}
\end{align*}

\begin{center}
Битва продолжается, не теряйте духу, они когда-то, наверное, закончаться!
\end{center}
\begin{align*}
\begin{autobreak}
\frac{d}{dx}(0 - \sin(\sin(x))) = \frac{d}{dx}(0) - \frac{d}{dx}(\sin(\sin(x)))
\end{autobreak}
\end{align*}

\begin{center}
Их не становиться меньше, откуда они только лезут?!
\end{center}
\begin{align*}
\begin{autobreak}
\frac{d}{dx}(\sin(\sin(x))) = \cos(\sin(x)) \cdot \frac{d}{dx}(\sin(x))
\end{autobreak}
\end{align*}

\begin{center}
Вот так и рождаются легенды о герое, истребившем половину рода Боблина.
\end{center}
\begin{align*}
\begin{autobreak}
\frac{d}{dx}(\sin(x)) = \cos(x) \cdot \frac{d}{dx}(x)
\end{autobreak}
\end{align*}

\begin{center}
Полиморф сработал отлично: зять Боблина теперь лягушка
\end{center}
\begin{align*}
\begin{autobreak}
\frac{d}{dx}(x) = 1
\end{autobreak}
\end{align*}

\begin{center}
Вы разложили дядю Боблина на молекулы
\end{center}
\begin{align*}
\begin{autobreak}
\frac{d}{dx}(0) = 0
\end{autobreak}
\end{align*}

\begin{center}
Небольшой взмах посохом — и план сражения выглядит куда приличнее.
\end{center}
A = \begin{align*}
\begin{autobreak}
((0 - \cos(\sin(x)) \cdot \cos(x)) \cdot \cos(x) + (0 - \sin(\sin(x))) \cdot (0 - \sin(x))) \cdot \cos(x)
\end{autobreak}
\end{align*}

\begin{align*}
\begin{autobreak}
\frac{d}{dx}(A) = \frac{d}{dx}((0 - \cos(\sin(x)) \cdot \cos(x)) \cdot \cos(x) + (0 - \sin(\sin(x))) \cdot (0 - \sin(x))) \cdot \cos(x) + (0 - \cos(\sin(x)) \cdot \cos(x)) \cdot \cos(x) + (0 - \sin(\sin(x))) \cdot (0 - \sin(x)) \cdot \frac{d}{dx}(\cos(x))
\end{autobreak}
\end{align*}

\begin{center}
Один из гоблинов упал
\end{center}
\begin{align*}
\begin{autobreak}
\frac{d}{dx}(\cos(x)) = -\sin(x) \cdot \frac{d}{dx}(x)
\end{autobreak}
\end{align*}

\begin{center}
Поздравляю! От свояка Боблина осталась только полоыина
\end{center}
\begin{align*}
\begin{autobreak}
\frac{d}{dx}(x) = 1
\end{autobreak}
\end{align*}

\begin{center}
Родственники Боблина продолжают лезть к вам, держите посох крепче!
\end{center}
A = \begin{align*}
\begin{autobreak}
(0 - \cos(\sin(x)) \cdot \cos(x)) \cdot \cos(x) + (0 - \sin(\sin(x))) \cdot (0 - \sin(x))
\end{autobreak}
\end{align*}

\begin{align*}
\begin{autobreak}
\frac{d}{dx}(A) = \frac{d}{dx}((0 - \cos(\sin(x)) \cdot \cos(x)) \cdot \cos(x)) + \frac{d}{dx}((0 - \sin(\sin(x))) \cdot (0 - \sin(x)))
\end{autobreak}
\end{align*}

\begin{center}
Битва продолжается, не теряйте духу, они когда-то, наверное, закончаться!
\end{center}
\begin{align*}
\begin{autobreak}
\frac{d}{dx}((0 - \sin(\sin(x))) \cdot (0 - \sin(x))) = \frac{d}{dx}(0 - \sin(\sin(x))) \cdot 0 - \sin(x) + 0 - \sin(\sin(x)) \cdot \frac{d}{dx}(0 - \sin(x))
\end{autobreak}
\end{align*}

\begin{center}
Их не становиться меньше, откуда они только лезут?!
\end{center}
\begin{align*}
\begin{autobreak}
\frac{d}{dx}(0 - \sin(x)) = \frac{d}{dx}(0) - \frac{d}{dx}(\sin(x))
\end{autobreak}
\end{align*}

\begin{center}
Вот так и рождаются легенды о герое, истребившем половину рода Боблина.
\end{center}
\begin{align*}
\begin{autobreak}
\frac{d}{dx}(\sin(x)) = \cos(x) \cdot \frac{d}{dx}(x)
\end{autobreak}
\end{align*}

\begin{center}
Ваше заклинание свернуло невестку Боблина в шарик
\end{center}
\begin{align*}
\begin{autobreak}
\frac{d}{dx}(x) = 1
\end{autobreak}
\end{align*}

\begin{center}
Ваше волшебство оказалось не по зубам тёте Боблина, кстати, куда она делась?
\end{center}
\begin{align*}
\begin{autobreak}
\frac{d}{dx}(0) = 0
\end{autobreak}
\end{align*}

\begin{center}
Небольшой взмах посохом — и план сражения выглядит куда приличнее.
\end{center}
\begin{align*}
\begin{autobreak}
\frac{d}{dx}(0 - \sin(\sin(x))) = \frac{d}{dx}(0) - \frac{d}{dx}(\sin(\sin(x)))
\end{autobreak}
\end{align*}

\begin{center}
Один из гоблинов упал
\end{center}
\begin{align*}
\begin{autobreak}
\frac{d}{dx}(\sin(\sin(x))) = \cos(\sin(x)) \cdot \frac{d}{dx}(\sin(x))
\end{autobreak}
\end{align*}

\begin{center}
Родственники Боблина продолжают лезть к вам, держите посох крепче!
\end{center}
\begin{align*}
\begin{autobreak}
\frac{d}{dx}(\sin(x)) = \cos(x) \cdot \frac{d}{dx}(x)
\end{autobreak}
\end{align*}

\begin{center}
ААХХАХААХАХ Гоблин-Боблин
\end{center}
\begin{align*}
\begin{autobreak}
\frac{d}{dx}(x) = 1
\end{autobreak}
\end{align*}

\begin{center}
Огненный шар испарил бабушку Боблина
\end{center}
\begin{align*}
\begin{autobreak}
\frac{d}{dx}(0) = 0
\end{autobreak}
\end{align*}

\begin{center}
Битва продолжается, не теряйте духу, они когда-то, наверное, закончаться!
\end{center}
\begin{align*}
\begin{autobreak}
\frac{d}{dx}((0 - \cos(\sin(x)) \cdot \cos(x)) \cdot \cos(x)) = \frac{d}{dx}(0 - \cos(\sin(x)) \cdot \cos(x)) \cdot \cos(x) + 0 - \cos(\sin(x)) \cdot \cos(x) \cdot \frac{d}{dx}(\cos(x))
\end{autobreak}
\end{align*}

\begin{center}
Их не становиться меньше, откуда они только лезут?!
\end{center}
\begin{align*}
\begin{autobreak}
\frac{d}{dx}(\cos(x)) = -\sin(x) \cdot \frac{d}{dx}(x)
\end{autobreak}
\end{align*}

\begin{center}
Ваше вошшебство откатило деверя Боблина до младенчества
\end{center}
\begin{align*}
\begin{autobreak}
\frac{d}{dx}(x) = 1
\end{autobreak}
\end{align*}

\begin{center}
Вот так и рождаются легенды о герое, истребившем половину рода Боблина.
\end{center}
\begin{align*}
\begin{autobreak}
\frac{d}{dx}(0 - \cos(\sin(x)) \cdot \cos(x)) = \frac{d}{dx}(0) - \frac{d}{dx}(\cos(\sin(x)) \cdot \cos(x))
\end{autobreak}
\end{align*}

\begin{center}
Небольшой взмах посохом — и план сражения выглядит куда приличнее.
\end{center}
\begin{align*}
\begin{autobreak}
\frac{d}{dx}(\cos(\sin(x)) \cdot \cos(x)) = \frac{d}{dx}(\cos(\sin(x))) \cdot \cos(x) + \cos(\sin(x)) \cdot \frac{d}{dx}(\cos(x))
\end{autobreak}
\end{align*}

\begin{center}
Один из гоблинов упал
\end{center}
\begin{align*}
\begin{autobreak}
\frac{d}{dx}(\cos(x)) = -\sin(x) \cdot \frac{d}{dx}(x)
\end{autobreak}
\end{align*}

\begin{center}
Полиморф сработал отлично: зять Боблина теперь лягушка
\end{center}
\begin{align*}
\begin{autobreak}
\frac{d}{dx}(x) = 1
\end{autobreak}
\end{align*}

\begin{center}
Родственники Боблина продолжают лезть к вам, держите посох крепче!
\end{center}
\begin{align*}
\begin{autobreak}
\frac{d}{dx}(\cos(\sin(x))) = -\sin(\sin(x)) \cdot \frac{d}{dx}(\sin(x))
\end{autobreak}
\end{align*}

\begin{center}
Битва продолжается, не теряйте духу, они когда-то, наверное, закончаться!
\end{center}
\begin{align*}
\begin{autobreak}
\frac{d}{dx}(\sin(x)) = \cos(x) \cdot \frac{d}{dx}(x)
\end{autobreak}
\end{align*}

\begin{center}
Поздравляю! От свояка Боблина осталась только полоыина
\end{center}
\begin{align*}
\begin{autobreak}
\frac{d}{dx}(x) = 1
\end{autobreak}
\end{align*}

\begin{center}
Заклинание хаоса раскидало части племянника Боблина по разным планам
\end{center}
\begin{align*}
\begin{autobreak}
\frac{d}{dx}(0) = 0
\end{autobreak}
\end{align*}

\begin{center}
Ваш портал небытия вежливо удалил тещу Боблина из этого измерения
\end{center}
\begin{align*}
\begin{autobreak}
\frac{d}{dx}(0) = 0
\end{autobreak}
\end{align*}

\section{Полный план сражения с Тейлором-Боблином}
A = \begin{align*}
\begin{autobreak}
1 + \frac{0}{1} \cdot (x - 0)^{1} + \frac{180.812}{2} \cdot (x - 0)^{2} + \frac{5.61652}{6} \cdot (x - 0)^{3}
\end{autobreak}
\end{align*}

f(x) = \begin{align*}
\begin{autobreak}
A + \frac{-2.50175}{24} \cdot (x - 0)^{4} + \frac{-0.448332}{120} \cdot (x - 0)^{5}
\end{autobreak}
\end{align*}

\end{document}
